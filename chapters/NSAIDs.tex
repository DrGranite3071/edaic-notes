\documentclass[11pt,a4paper]{article}

% preamble.tex
\usepackage[T1]{fontenc}
\usepackage[utf8]{inputenc}
\usepackage[english]{babel}
\usepackage[a4paper,margin=2cm]{geometry}
\usepackage{lmodern}
\usepackage{amsmath,amssymb}
\usepackage{graphicx}
\usepackage{booktabs}
\usepackage{hyperref}

\usepackage[T1]{fontenc}
\usepackage{setspace}
\usepackage{geometry}
\usepackage{titlesec}
\usepackage{enumitem}
\usepackage{hyperref}
\usepackage{booktabs}


\begin{document}

\section*{NSAIDs and DMARDs – Integrated Pharmacology Summary}

\section{Introduction}

Non-steroidal anti-inflammatory drugs (NSAIDs) are among the most frequently used analgesic and anti-inflammatory agents in clinical medicine. They play a central role in perioperative multimodal analgesia, emergency care, and long-term management of musculoskeletal and inflammatory disorders. Their clinical benefits and risks arise directly from how they alter prostaglandin synthesis, making a clear understanding of COX physiology essential.

NSAIDs act by inhibiting cyclo-oxygenase (COX) enzymes, which generate prostaglandins involved in pain, inflammation, platelet aggregation, gastric protection, renal blood flow, and vascular homeostasis. Because these functions are distributed across multiple organ systems, NSAIDs can have both therapeutic benefits and clinically significant adverse effects.

% -------------------------------------------------------

\section{The Cyclo-Oxygenase (COX) System}

\subsection*{COX-1 (Constitutive)}

COX-1 is expressed in the stomach, platelets, and kidneys. It promotes:
\begin{itemize}
\item Gastric mucosal protection (via PGE$_2$)
\item Platelet aggregation (TXA$_2$)
\item Renal blood flow during haemodynamic stress
\end{itemize}

Inhibition of COX-1 explains the hallmark NSAID toxicities such as peptic ulceration, GI bleeding, and reduced renal perfusion.

\subsection*{COX-2 (Inducible and Constitutive)}

COX-2 is upregulated during inflammation by cytokines (IL-1, TNF-$\alpha$) and produces prostaglandins responsible for inflammatory pain. It is also constitutively expressed in the kidneys and vascular endothelium. Blocking COX-2 reduces inflammation but diminishes endothelial prostacyclin, increasing cardiovascular risk.

\subsection*{Why COX Selectivity Matters}

\begin{itemize}
\item More COX-1 inhibition → greater GI and platelet-related side effects.
\item More COX-2 inhibition → fewer GI effects but increased cardiovascular risk.
\end{itemize}

% -------------------------------------------------------


\subsection*{Table 10.7 – Key Properties of Common NSAIDs}

\begin{table}[h!]
\centering
\small
\renewcommand{\arraystretch}{1.2}
\begin{tabular}{@{}%
    p{0.13\textwidth}%
    p{0.20\textwidth}%
    p{0.09\textwidth}%
    p{0.26\textwidth}%
    p{0.26\textwidth}@{}}
\toprule
\textbf{Drug} & \textbf{COX Selectivity} & \textbf{t$_{1/2}$} & \textbf{Notes} & \textbf{Main Risks} \\
\midrule
Aspirin      & COX-1 $\gg$ COX-2            & Varies   & Irreversible platelet inhibition            & GI bleed, salicylism \\
Ibuprofen    & Non-selective                & $\sim$2 h  & Good short-term safety                      & GI irritation, renal risk (high dose) \\
Naproxen     & Non-selective                & 12--15 h & Lowest CV risk                               & GI toxicity \\
Diclofenac   & COX-1/COX-2 (lean COX-2)     & 1--2 h   & Potent anti-inflammatory                     & High CV/liver toxicity \\
Indomethacin & Non-selective                & 4--5 h   & Used in gout/PDA closure                     & CNS toxicity \\
Ketorolac    & COX-1 $>$ COX-2              & 4--6 h   & Strong analgesia                             & Renal/GI toxicity (short-term only) \\
Celecoxib    & COX-2 selective              & $\sim$11 h & Reduced GI risk                             & CV risk \\
Etoricoxib   & Highly COX-2 selective       & 22 h     & Once-daily dosing                            & Highest CV risk \\
Parecoxib    & COX-2 selective              & Prodrug  & IV perioperative use                         & CV and skin toxicity \\
\bottomrule
\end{tabular}
\end{table}
% -------------------------------------------------------

\section{Pharmacokinetics}

NSAIDs are well absorbed orally, highly protein-bound, and metabolised hepatically through CYP pathways and glucuronidation. They are eliminated renally.

\begin{itemize}
\item Food delays but does not reduce absorption.
\item High protein binding creates displacement interactions (e.g.\ warfarin).
\item Placental and breast-milk transfer occur.
\item Renal impairment increases risk of toxicity.
\end{itemize}

% -------------------------------------------------------

\section{Mechanisms of Action}

NSAIDs reduce prostaglandin synthesis, producing:
\begin{itemize}
\item \textbf{Analgesia}: reduced nociceptor sensitisation and central excitability.
\item \textbf{Anti-inflammatory}: decreased vasodilation and leukocyte activity.
\item \textbf{Antipyretic}: reduced hypothalamic PGE$_2$.
\end{itemize}

NSAIDs \textbf{do not inhibit lipoxygenase}, so leukotriene synthesis continues, contributing to NSAID-exacerbated asthma.

\subsection*{Structural Classification (adapted from Peck Table 10.6)}

\begin{itemize}
\item Salicylates (aspirin)
\item Acetic acid derivatives (diclofenac, ketorolac, indomethacin)
\item Anthranilic acids (mefenamic acid)
\item Pyrazolones (phenylbutazone)
\item Propionic acids (ibuprofen, naproxen)
\item Para-aminophenols (paracetamol)
\item Oxicams (piroxicam, tenoxicam)
\item Preferential COX-2 oxicams (meloxicam)
\item Specific COX-2 inhibitors (celecoxib, etoricoxib, parecoxib)
\end{itemize}

% -------------------------------------------------------

\section{Adverse Effects}

\subsection*{Gastrointestinal}
Loss of mucosal prostaglandins → dyspepsia, ulceration, bleeding.

\begin{itemize}
\item Highest risk: ketorolac, piroxicam  
\item Intermediate: diclofenac, naproxen  
\item Lowest: ibuprofen $<$1.2 g/day  
\end{itemize}

COX-2 inhibitors reduce but do not eliminate GI risk.

\subsection*{Renal}
COX inhibition decreases renal perfusion → reduced GFR, sodium retention, hyperkalaemia.

Additional points:
\begin{itemize}
\item Analgesic nephropathy: papillary necrosis and interstitial fibrosis.
\item Aspirin alters urate handling (low dose → retention; high dose → uricosuric).
\end{itemize}

\subsection*{Cardiovascular}
Reduced PGI$_2$ with preserved TXA$_2$ increases thrombotic risk:
\begin{itemize}
\item High-dose diclofenac, ibuprofen, and COX-2 inhibitors have similar CV risk.
\item $\sim$3 extra major coronary events per 1000 patients/year (1 fatal).
\item Higher stroke risk in younger men and patients with prior TIA/stroke.
\end{itemize}

\subsection*{Respiratory}
NSAID-exacerbated respiratory disease → excess leukotrienes.

\subsection*{Pregnancy}
Risk of premature ductus arteriosus closure and oligohydramnios.

\subsection*{CNS}
Indomethacin: headache, dizziness, confusion.

% -------------------------------------------------------

\section{Non-Selective COX Inhibitors – Drug Profiles}

\subsection*{Aspirin}
Irreversibly inhibits COX-1 in platelets.

\textbf{Overdose:} tinnitus, hyperventilation, mixed acidosis/alkalosis, hyperthermia. Treat with urinary alkalinisation or dialysis.

\subsection*{Ibuprofen}
Safe at standard doses; renal risk increases $>$2.4 g/day.

\subsection*{Naproxen}
Long half-life; lowest CV risk; higher GI risk than ibuprofen.

\subsection*{Diclofenac}
Potent, partially COX-2-selective; high CV and liver toxicity.

\subsection*{Indomethacin}
Used in gout and PDA closure; CNS toxicity limits use.

\subsection*{Ketorolac}
Opioid-level analgesia; high renal/GI risk → limit to 5 days.

\subsection*{Ketoprofen}
Similar to ibuprofen.

\subsection*{Phenylbutazone}
High-risk pyrazolone; causes aplastic anaemia and agranulocytosis.

\subsection*{Paracetamol (Acetaminophen)}
Not a true NSAID; minimal anti-inflammatory effect.

\textbf{Overdose:} NAPQI accumulation → hepatotoxicity; treat with NAC.

% -------------------------------------------------------

\section{Preferential and Selective COX-2 Inhibitors}

\subsection*{Preferential COX-2 Inhibitors}
(meloxicam, nabumetone, etodolac)

\begin{itemize}
\item Better GI safety than non-selectives.  
\item Renal and CV risks persist.
\end{itemize}

\subsection*{Selective COX-2 Inhibitors (Coxibs)}

\textbf{Celecoxib:} CYP2C9 metabolism; avoid in sulfonamide allergy.  
\textbf{Etoricoxib:} highly selective; once-daily; avoid in CVD.  
\textbf{Parecoxib/Valdecoxib:} IV perioperative use; rare SJS risk.

\textbf{Key Concepts:}
\begin{itemize}
\item GI risk reduced, renal risk unchanged.
\item No platelet inhibition.
\item CV risk rises with selectivity.
\end{itemize}

% -------------------------------------------------------

\section{Clinical Uses}

NSAIDs treat:
\begin{itemize}
\item Musculoskeletal and perioperative pain
\item Arthritis and gout
\item Fever
\item Cardiovascular prevention (aspirin)
\item PDA closure (indomethacin)
\end{itemize}

% -------------------------------------------------------

\section{Overview}

NSAID choice depends on:
\begin{itemize}
\item COX selectivity
\item Duration of action
\item Patient comorbidities (CV, GI, renal)
\end{itemize}

% -------------------------------------------------------

\section{Disease-Modifying Anti-Rheumatic Drugs (DMARDs)}

DMARDs modify disease progression in autoimmune disorders. They require weeks to take effect and involve major toxicity monitoring.

\subsection*{Methotrexate}
Weekly dosing; folate analogue; hepatotoxicity, marrow suppression, pneumonitis.

\subsection*{Sulfasalazine}
Prodrug (5-ASA + sulfapyridine); GI upset; rare agranulocytosis.

\subsection*{Hydroxychloroquine}
Retinal toxicity; rare but severe overdose (sodium channel blockade, QT
prolongation).

\subsection*{Leflunomide}
Pyrimidine synthesis inhibitor; hepatotoxicity and hypertension.

\subsection*{Biologic DMARDs}
Anti-TNF, anti-IL-6, B-cell depleters, T-cell costimulation blockers. High infection risk; withhold perioperatively.

% -------------------------------------------------------

\end{document}
