\documentclass[11pt,a4paper]{article}

% preamble.tex
\usepackage[T1]{fontenc}
\usepackage[utf8]{inputenc}
\usepackage[english]{babel}
\usepackage[a4paper,margin=2cm]{geometry}
\usepackage{lmodern}
\usepackage{amsmath,amssymb}
\usepackage{graphicx}
\usepackage{booktabs}
\usepackage{hyperref}

\usepackage[T1]{fontenc}
\usepackage{setspace}
\usepackage{geometry}
\usepackage{titlesec}
\usepackage{enumitem}
\usepackage{hyperref}
\usepackage{booktabs}


\setlength{\parindent}{0pt}
\setlength{\parskip}{4pt}

\begin{document}

\section{Local Anaesthetics -- Integrated Summary (Peck pp. 156--164 + Added Sections)}

This rewritten version preserves \textbf{all original content} but reorganises it into a clearer, more logical narrative. No information has been removed; material has been grouped and sequenced to improve conceptual flow.

\section{Foundations: Nerve Physiology Relevant to Local Anaesthetics}

(\textit{Peck pp. 156--158})

Understanding LA pharmacology begins with the physiology of the nerve membrane.

\subsection{Resting Membrane Potential}

Neurons maintain a resting potential of approximately $-80$ mV due to:
\begin{itemize}
  \item Na$^+$/K$^+$ ATPase creating a high intracellular K$^+$ and high extracellular Na$^+$ gradient.
  \item Selective permeability $\rightarrow$ outward K$^+$ leak dominates, leaving intracellular proteins unopposed $\rightarrow$ negative intracellular charge.
  \item Hypokalaemia $\rightarrow$ hyperpolarises the cell (more negative membrane potential).
  \item Extracellular Na$^+$ has minimal influence at rest due to poor permeability.
\end{itemize}

\subsection{Action Potential Generation}

\begin{itemize}
  \item Triggered when membrane reaches threshold (about $-50$ mV).
  \item Voltage-gated Na$^+$ channels open $\rightarrow$ rapid Na$^+$ influx $\rightarrow$ peak $\approx +30$ mV.
  \item Repolarisation via K$^+$ efflux; the pump restores gradients.
  \item Action potentials are brief (1--2 ms).
\end{itemize}

\subsection{Conduction and Myelination}

\begin{itemize}
  \item Myelinated fibres: conduction up to 120 m/s by saltatory conduction between nodes.
  \item Unmyelinated fibres: slower conduction.
  \item Retrograde signalling prevented by Na$^+$ channel inactivation.
\end{itemize}

\section{Local Anaesthetic Preparations and Additives}

(\textit{Peck pp. 158--160})

\subsection{Formulation}

Local anaesthetics are supplied as hydrochloride salts, which are water-soluble and chemically stable.

Multi-dose vials contain preservatives:
\begin{itemize}
  \item sodium metabisulfite
  \item methyl parahydroxybenzoate (MPHB)
\end{itemize}

Preservative-free vials are mandatory for spinal use to avoid arachnoiditis.

\subsection{Solutions and Additives}

\begin{itemize}
  \item Adrenaline and felypressin reduce systemic uptake $\rightarrow$ prolong block.
  \item Lidocaine is available in 0.5--10\% concentrations.
  \item Hyperbaric spinal formulations (e.g. bupivacaine) contain 80 mg/mL glucose.
\end{itemize}

\section{Physicochemical Characteristics of Local Anaesthetics}

(\textit{Added section; improves flow before kinetics and mechanism})

Local anaesthetic behaviour is governed by predictable physicochemical factors:
\begin{itemize}
  \item Weak bases $\leftrightarrow$ equilibrium between ionised (BH$^+$) and unionised (B) forms.
  \item Lipid solubility $\rightarrow$ primary determinant of potency and duration.
  \item pKa $\rightarrow$ proximity to physiological pH determines onset.
  \item Protein binding $\rightarrow$ correlates with duration (e.g. bupivacaine $>$ lidocaine).
  \item Intrinsic vasodilation $\rightarrow$ increases systemic absorption; exception: cocaine (vasoconstrictor).
\end{itemize}

\section{Pharmacokinetics of Local Anaesthetics}

(\textit{Rewritten to improve continuity; includes all original kinetic material})

\subsection{Absorption}

Systemic uptake varies by injection site vascularity:
\[
\text{Tracheal $>$ Intercostal $>$ Caudal/Epidural $>$ Brachial Plexus $>$ Subcutaneous}
\]

Higher vascularity $\rightarrow$ greater systemic levels $\rightarrow$ increased toxicity risk.

\subsection{Distribution (Expanded)}

Distribution depends on perfusion, lipid solubility, and protein binding.

\subsubsection{Initial Rapid Distribution Phase}

Rapid uptake into vessel-rich organs (VROs): brain, heart, liver, kidneys, lungs.
\begin{itemize}
  \item Explains rapid onset of CNS symptoms in toxicity.
  \item Peak arterial concentrations exceed venous levels due to lung first-pass uptake.
\end{itemize}

\subsubsection{Secondary Distribution Phase}

Redistribution to muscle and skin terminates clinical effect after bolus injection.

\subsubsection{Influence of Lipid Solubility}

Highly lipid-soluble agents (bupivacaine, levobupivacaine):
\begin{itemize}
  \item more potent
  \item longer duration
  \item accumulate in fat $\rightarrow$ prolonged elimination half-life
\end{itemize}

\subsubsection{Protein Binding}

Only the free fraction is active. Lower $\alpha_1$-acid glycoprotein (pregnancy, neonates, liver disease) increases free fraction $\rightarrow$ increased toxicity.

\subsubsection{Modifiers of Distribution}

\begin{itemize}
  \item Low cardiac output $\rightarrow$ slower distribution $\rightarrow$ higher peak plasma levels.
  \item Pregnancy $\rightarrow$ faster absorption; increased sensitivity.
  \item Acidosis/hypercarbia $\rightarrow$ higher ionised fraction $\rightarrow$ worsened cardiotoxicity.
\end{itemize}

\subsubsection{Continuous Infusions}

Accumulation in muscle, fat and VROs contributes to context-sensitive half-times, especially for bupivacaine/levobupivacaine.

\subsection{Metabolism}

\begin{itemize}
  \item Esters: rapid hydrolysis by pseudocholinesterase $\rightarrow$ short duration.
  \item Amides: hepatic CYP450 metabolism $\rightarrow$ prolonged in liver disease, pregnancy, low cardiac output states.
\end{itemize}

\subsection{Excretion}

Mostly renally eliminated as metabolites; minimal unchanged drug excreted.

\section{Other Physiological and Systemic Effects of Local Anaesthetics}

\subsection{5.1 Antiarrhythmic Effects (Class Ib)}

Lidocaine decreases automaticity, suppresses ventricular ectopy, shortens action potential duration, and preferentially binds inactivated Na$^+$ channels in ischaemic myocardium.

\subsection{5.2 CNS Effects}

Dose-dependent progression:
\begin{enumerate}
  \item Tinnitus, circumoral numbness
  \item Agitation $\rightarrow$ seizures
  \item CNS depression $\rightarrow$ coma
\end{enumerate}

\subsection{5.3 Cardiovascular Effects (Expanded)}

Local anaesthetics exhibit dose-dependent cardiovascular actions.

\subsubsection{Low Concentrations}

\begin{itemize}
  \item Minimal haemodynamic effect in healthy individuals.
  \item Lidocaine's antiarrhythmic benefit dominates.
\end{itemize}

\subsubsection{Moderate Concentrations}

\begin{itemize}
  \item Direct myocardial depression $\rightarrow$ decreased contractility.
  \item Peripheral vasodilation (except cocaine).
  \item Slowed cardiac conduction.
  \item Result: hypotension, bradycardia, reduced cardiac output.
\end{itemize}

\subsubsection{High / Toxic Concentrations}

\begin{itemize}
  \item Severe myocardial depression; AV block.
  \item Ventricular arrhythmias, especially with bupivacaine (slow channel dissociation).
  \item Cardiac arrest may be sudden and resistant to defibrillation.
  \item Acidosis/hypercarbia intensify toxicity.
\end{itemize}

\subsubsection{Drug-Specific Notes}

\begin{itemize}
  \item Bupivacaine: highest cardiotoxicity.
  \item Levobupivacaine/ropivacaine: safer stereoselective profile.
  \item Lidocaine: rapid dissociation $\rightarrow$ safer.
  \item Cocaine: hypertension, tachycardia, vasospasm.
\end{itemize}

\subsection{5.4 Table 11.1 -- Classification of Local Anaesthetics (Peck)}

Esters (--CO.O-- linkage):
\begin{itemize}
  \item Procaine
  \item Amethocaine (tetracaine)
  \item Cocaine
\end{itemize}

Amides (--NH.CO-- linkage):
\begin{itemize}
  \item Lidocaine
  \item Prilocaine
  \item Bupivacaine
  \item Ropivacaine
  \item Dibucaine
\end{itemize}

EDAIC-relevant points:
\begin{itemize}
  \item Esters: unstable in solution, short shelf-life, rapidly hydrolysed, higher allergy risk (PABA metabolites).
  \item Amides: stable, hepatic metabolism, more common clinically.
\end{itemize}

\subsection{5.5 Table 11.2 -- Pharmacological Properties of Common LAs (Peck)}

(Values summarised from the PDF table.)

\begin{table}[h!]
\centering
\small
\setlength{\tabcolsep}{6pt}
\renewcommand{\arraystretch}{1.2}

\begin{tabular}{l c c c c c c c}
\toprule
\textbf{Drug} &
\textbf{Rel. Pot.} &
\textbf{Onset} &
\textbf{Duration} &
\textbf{Toxic (µg/mL)} &
\textbf{pKa} &
\textbf{\% unionised} &
\textbf{Protein bind.} \\
\midrule
Amethocaine & 8        & Slow     & Long     & 8.5   & 7.0 & 75\% & 200  \\
Cocaine     & Moderate & Short    & --       & 0.5   & 8.6 & 5\%  & 95\% \\
Lidocaine   & 2        & Fast     & Moderate & >5    & 7.9 & 25\% & 70\% \\
Prilocaine  & 2        & Fast     & Moderate & >5    & 7.7 & 33\% & 55\% \\
Bupivacaine & 8        & Moderate & Long     & >1.5  & 8.1 & 15\% & 95\% \\
Ropivacaine & 8        & Moderate & Long     & --    & 8.1 & 15\% & 94\% \\
\bottomrule
\end{tabular}

\caption{Table 11.2A – Core Pharmacological Properties of Common Local Anaesthetics (Peck)}
\end{table}


\begin{table}[h!]
\centering
\small
\setlength{\tabcolsep}{10pt}
\renewcommand{\arraystretch}{1.2}

\begin{tabular}{l c c}
\toprule
\textbf{Drug} &
\textbf{Lipid Solubility} &
\textbf{$t_{1/2}$ (min)} \\
\midrule
Amethocaine & 80    & --   \\
Cocaine     & 100   & --   \\
Lidocaine   & 150   & 100  \\
Prilocaine  & 50    & 100  \\
Bupivacaine & 1000  & 160  \\
Ropivacaine & 600   & 120  \\
\bottomrule
\end{tabular}

\caption{Table 11.2B – Additional PK Properties of Local Anaesthetics (Peck)}
\end{table}




EDAIC-relevant interpretations:
\begin{itemize}
  \item High lipid solubility \& high protein binding $\rightarrow$ potent, long-acting (bupivacaine $>$ ropivacaine).
  \item Low \% unionised (high pKa) $\rightarrow$ slower onset.
  \item Toxicity thresholds differ dramatically $\rightarrow$ bupivacaine dangerous at low plasma levels.
  \item Cocaine uniquely vasoconstrictive $\rightarrow$ high protein binding, slow onset toxicity.
\end{itemize}

\section{Toxicity and LAST Management}

\subsection{6.1 CNS and Cardiovascular Toxicity}

\begin{itemize}
  \item Early CNS symptoms $\rightarrow$ seizures $\rightarrow$ coma.
  \item CVS depression $\rightarrow$ arrhythmias $\rightarrow$ cardiac arrest.
  \item Special toxicities: methaemoglobinaemia (prilocaine, benzocaine).
  \item True allergy: rare; more common with esters (PABA).
\end{itemize}

\subsection{6.2 Intralipid (Lipid Emulsion) Therapy}

Intralipid 20\% is the foundation of modern LAST treatment, especially for cardiotoxic agents like bupivacaine. Its mechanisms include:
\begin{enumerate}
  \item Lipid sink effect -- creates an expanded intravascular lipid phase that sequesters lipid-soluble LAs $\rightarrow$ reduces free active concentration.
  \item Metabolic support -- provides fatty acids that improve myocardial ATP generation.
  \item Direct inotropy -- enhances intracellular calcium handling.
  \item Membrane stabilisation -- modifies ion channel environment to counteract Na$^+$ channel blockade.
\end{enumerate}

Dosing (adult example protocol):
\begin{itemize}
  \item Bolus: 1.5 mL/kg of 20\% lipid emulsion over 1 minute.
  \item Infusion: 0.25 mL/kg/min for $\ge$10 minutes after stability.
  \item If instability persists: repeat bolus and increase infusion to 0.5 mL/kg/min.
  \item Max cumulative dose $\approx$ 10--12 mL/kg.
\end{itemize}

Key EDAIC points:
\begin{itemize}
  \item Reduce adrenaline doses ($<$1 $\mu$g/kg).
  \item Avoid vasopressin, calcium-channel blockers, $\beta$-blockers.
  \item CPR may need to be prolonged.
\end{itemize}

\section{Mechanism of Action of Local Anaesthetics}

(\textit{Peck pp. 160--162, with added secondary mechanisms})

\subsection{7.1 Sodium Channel Blockade (Primary Mechanism)}

Local anaesthetics inhibit voltage-gated Na$^+$ channels from the intracellular side.

\begin{itemize}
  \item Unionised form (B) diffuses across the lipid membrane.
  \item Ionised form (BH$^+$) binds within the Na$^+$ channel.
  \item Preferential binding to open and inactivated states $\rightarrow$ use-dependent block.
\end{itemize}

Effects:
\begin{itemize}
  \item Reduced rate of rise and amplitude of the action potential.
  \item Increased threshold for depolarisation.
  \item Failure of conduction at sufficient occupancy.
\end{itemize}

\subsection{7.2 Membrane Expansion Theory (Secondary Mechanism)}

At higher concentrations, LAs incorporate into the phospholipid bilayer $\rightarrow$ increase membrane volume $\rightarrow$ alter lipid packing $\rightarrow$ disrupt ion channel configuration.

This explains:
\begin{itemize}
  \item Non-specific depression of excitability.
  \item Additional antiarrhythmic effects.
  \item Amplification of sodium-channel blockade.
\end{itemize}

\section{Nerve Fibre Susceptibility and Differential Block}

(\textit{Peck pp. 162--163})

Sensitivity depends on:
\begin{itemize}
  \item Diameter (small fibres block first).
  \item Myelination (nodes enable easier block).
  \item Firing frequency (use dependence).
\end{itemize}

Order of block:
\begin{enumerate}
  \item Sympathetic fibres
  \item Pain and temperature (A$\delta$, C)
  \item Touch/pressure (A$\beta$)
  \item Motor (A$\alpha$)
\end{enumerate}

Clinical relevance:
\begin{itemize}
  \item Dilute solutions $\rightarrow$ analgesia with minimal motor block.
  \item Pain fibres more sensitive due to high firing rates.
\end{itemize}

\section{Factors Affecting Clinical Activity of LAs}

(\textit{Peck pp. 163--164 -- expanded to include all information on these pages})

Effective nerve block depends on drug chemistry, tissue physiology, nerve characteristics, and clinical technique. Peck emphasises that small variations in pH, vascularity, protein binding, and formulation can dramatically change block quality.

\subsection{9.1 Summary: Chemical Determinants (Expanded)}

These properties determine how quickly and how effectively an LA reaches and binds to its sodium-channel target.

\subsubsection*{pKa}

\begin{itemize}
  \item Most LAs are weak bases with pKa 7.6--9.0.
  \item The closer the pKa to physiological pH (7.4), the greater the fraction of unionised base $\rightarrow$ faster onset.
  \item Lidocaine (pKa 7.9) has rapid onset; bupivacaine/ropivacaine (pKa 8.1) have slower onset.
  \item In acidic tissues (infection/ischaemia), more LA becomes ionised $\rightarrow$ poor penetration $\rightarrow$ slower onset, less reliable block.
\end{itemize}

\subsubsection*{Lipid Solubility}

\begin{itemize}
  \item Strong determinant of potency: highly lipid-soluble drugs penetrate myelin and axonal membranes more easily.
  \item High lipid solubility also contributes to longer duration because the drug remains sequestered in lipid-rich neural and connective tissues.
  \item Bupivacaine $\gg$ lidocaine in lipid solubility $\rightarrow$ higher potency but more toxicity.
\end{itemize}

\subsubsection*{Protein Binding}

\begin{itemize}
  \item High protein binding $\rightarrow$ more drug remains in tissues and plasma in a bound form $\rightarrow$ longer duration.
  \item Lower protein binding $\rightarrow$ shorter duration.
  \item Examples:
    \begin{itemize}
      \item Bupivacaine/ropivacaine: $\approx$95\% binding (long duration).
      \item Lidocaine: $\approx$65--70\% (intermediate duration).
      \item Prilocaine: $\approx$55\% (shorter duration).
    \end{itemize}
\end{itemize}

\subsubsection*{Intrinsic Vasodilator Properties}

\begin{itemize}
  \item Most LAs cause vasodilation $\rightarrow$ increased systemic absorption, reducing duration and raising toxicity risk.
  \item Cocaine is the exception -- blocks catecholamine reuptake $\rightarrow$ vasoconstriction $\rightarrow$ prolonged local effect.
\end{itemize}

\subsection{9.2 Physiological / Tissue Factors (Expanded)}

These factors alter how much drug reaches the nerve and how long it remains in the vicinity.

\subsubsection*{Tissue pH}

\begin{itemize}
  \item Lower pH (infection, abscess, poor perfusion) $\rightarrow$ increased ionisation of drug $\rightarrow$ impaired membrane penetration $\rightarrow$ block failure or delayed onset.
\end{itemize}

\subsubsection*{Vascularity of Injection Site}

Systemic uptake varies by site and determines duration and risk of toxicity:
\[
\text{Tracheal $>$ Intercostal $>$ Caudal/Epidural $>$ Brachial Plexus $>$ Subcutaneous}
\]

\begin{itemize}
  \item Intercostal blocks $\rightarrow$ highest systemic levels.
  \item Subcutaneous infiltration $\rightarrow$ lowest systemic levels.
\end{itemize}

\subsubsection*{Presence of Vasoconstrictors (Adrenaline/Felypressin)}

\begin{itemize}
  \item Slows systemic absorption $\rightarrow$ prolongs duration, deepens block.
  \item Reduces peak plasma concentration $\rightarrow$ reduces toxicity risk.
  \item Useful marker of accidental intravascular injection.
  \item Avoid in end-arterial fields (fingers, toes, penis, pinna, nose).
\end{itemize}

\subsubsection*{Patient-Specific Physiological Factors}

\begin{itemize}
  \item Pregnancy: epidural/spinal spread increased; protein binding reduced; higher sensitivity.
  \item Elderly: decreased muscle mass alters distribution; reduced hepatic clearance.
  \item Hepatic dysfunction: amide clearance reduced $\rightarrow$ prolonged duration, increased toxicity.
  \item Low cardiac output: slower redistribution $\rightarrow$ higher peak plasma levels.
\end{itemize}

\subsubsection*{Nerve Characteristics}

Although covered in Section 8, Peck ties fibre characteristics back into clinical efficacy here:
\begin{itemize}
  \item Smaller, myelinated fibres are blocked first.
  \item Pain fibres (A$\delta$, C) blocked before motor fibres.
  \item Rapidly firing nerves exhibit use-dependent block.
\end{itemize}

\subsection{9.3 Technique-Related Factors (Peck's final notes on pp. 163--164)}

Even with optimal drug chemistry and tissue conditions, poor technique can impair block quality.

\subsubsection*{Dose and Total Mass of LA}

\begin{itemize}
  \item Most important determinant of block density and duration.
  \item Increasing dose increases spread and depth but increases toxicity.
\end{itemize}

\subsubsection*{Volume of Solution}

\begin{itemize}
  \item Highly relevant for epidural and plexus blocks.
  \item Influences distribution of LA around nerves more than onset or potency.
\end{itemize}

\subsubsection*{Concentration of LA}

\begin{itemize}
  \item Higher concentration $\rightarrow$ more profound sensory and motor block.
  \item Low concentrations (e.g. obstetrics) $\rightarrow$ analgesia with motor sparing.
\end{itemize}

\subsubsection*{Proximity to the Nerve}

\begin{itemize}
  \item Closer deposition $\rightarrow$ faster onset and more complete block.
  \item Intraneural injection $\rightarrow$ risk of nerve injury; not recommended.
\end{itemize}

\subsubsection*{Use of Additives}

\begin{itemize}
  \item Adrenaline prolongs duration, reduces toxicity.
  \item Opioids and $\alpha_2$-agonists (in neuraxial block) $\rightarrow$ synergistic analgesia.
\end{itemize}

\section{Drug-Specific Kinetic Differences}

(Preserved exactly as originally written.)

\subsection*{10.1 Lidocaine}

\begin{itemize}
  \item Rapid onset (pKa 7.9).
  \item Moderate lipid solubility; $\sim$65\% protein binding.
  \item High-extraction hepatic metabolism $\rightarrow$ clearance depends on liver blood flow.
  \item Predictable profile; safer than bupivacaine.
\end{itemize}

\subsection*{10.2 Mepivacaine}

\begin{itemize}
  \item Rapid onset.
  \item Slightly greater lipid solubility and protein binding ($\sim$75\%).
  \item Slower hepatic metabolism $\rightarrow$ longer duration.
\end{itemize}

\subsection*{10.3 Prilocaine}

\begin{itemize}
  \item Rapid onset.
  \item Lower protein binding ($\sim$55\%) $\rightarrow$ shorter duration.
  \item Extrahepatic metabolism (lung, kidney).
  \item Metabolite (o-toluidine) $\rightarrow$ methaemoglobinaemia risk.
\end{itemize}

\subsection*{10.4 Bupivacaine}

\begin{itemize}
  \item Higher pKa ($\sim$8.1) $\rightarrow$ slower onset.
  \item Very lipid-soluble, highly potent.
  \item Very high protein binding ($\sim$95\%).
  \item Low-extraction hepatic clearance $\rightarrow$ sensitive to enzyme activity, not flow.
  \item High cardiotoxicity.
\end{itemize}

\subsection*{10.5 Ropivacaine}

\begin{itemize}
  \item Slightly slower onset than bupivacaine.
  \item Lower lipid solubility $\rightarrow$ less potent.
  \item High protein binding ($\sim$94\%).
  \item Safer cardiac profile; more motor-sparing.
\end{itemize}

\subsection*{10.6 Levobupivacaine}

\begin{itemize}
  \item Pharmacokinetically similar to bupivacaine.
  \item S-enantiomer $\rightarrow$ reduced free fraction $\rightarrow$ lower cardiotoxicity.
\end{itemize}

\subsection*{10.7 Ester Agents}

\begin{itemize}
  \item Very rapid plasma hydrolysis (except tetracaine).
  \item Cocaine uniquely inhibits catecholamine reuptake $\rightarrow$ vasoconstriction.
\end{itemize}

\section{Individual Local Anaesthetic Agents (Peck pp. 163--165)}

(Structured summaries of lidocaine, bupivacaine, ropivacaine, prilocaine, cocaine and chloroprocaine.)

\subsection*{11.1 Lidocaine (Lignocaine)}

\textbf{Type:} Amide local anaesthetic.

\textbf{Onset:} Rapid (relatively low pKa $\rightarrow$ larger fraction unionised at physiological pH).

\textbf{Duration:} Moderate (intermediate protein binding $\sim$70\% gives a medium-length block).

\textbf{Uses (as highlighted in Peck):}
\begin{itemize}
  \item Infiltration anaesthesia
  \item Peripheral nerve blocks
  \item Epidural anaesthesia
  \item IV regional anaesthesia (IVRA / Bier's block)
  \item Antiarrhythmic (Class Ib)
\end{itemize}

\textbf{Pharmacological profile:}
\begin{itemize}
  \item Moderate lipid solubility $\rightarrow$ moderate potency.
  \item High hepatic extraction $\rightarrow$ clearance dependent on liver blood flow.
  \item Good safety profile compared with long-acting agents.
\end{itemize}

\textbf{Toxicity considerations:}
\begin{itemize}
  \item CNS toxicity precedes CVS toxicity.
  \item Higher toxic plasma threshold than bupivacaine ($>$5 $\mu$g/mL).
\end{itemize}

\textbf{Kinetics:}
\begin{itemize}
  \item Absorption: moderate; systemic levels depend strongly on injection site vascularity.
  \item Distribution: rapid initial uptake into vessel-rich organs; moderate lipid solubility limits deep tissue accumulation.
  \item Metabolism: high-extraction hepatic clearance via CYP450; clearance directly proportional to liver blood flow.
  \item Elimination half-life: $\sim$100 minutes.
  \item Clinical implications: predictable offset; accumulation minimal unless hepatic perfusion is reduced.
\end{itemize}

\subsection*{11.2 Bupivacaine}

\textbf{Type:} Amide local anaesthetic; highly potent and long acting.

\textbf{Onset:} Moderate (higher pKa $\sim$8.1 $\rightarrow$ less unionised fraction $\rightarrow$ slower onset vs lidocaine).

\textbf{Duration:} Long (extremely high protein binding $\sim$95\% and lipid solubility prolong clinical effect).

\textbf{Uses (Peck emphasis):}
\begin{itemize}
  \item Epidural and spinal anaesthesia
  \item Peripheral nerve blocks
  \item Obstetric anaesthesia/analgesia (low concentration forms)
  \item Postoperative analgesia (infusions)
\end{itemize}

\textbf{Pharmacological profile:}
\begin{itemize}
  \item Highly lipid-soluble $\rightarrow$ potent and long-acting.
  \item Low hepatic extraction $\rightarrow$ clearance determined by liver enzyme activity rather than perfusion.
\end{itemize}

\textbf{Key toxicity warnings (as emphasised in Peck):}
\begin{itemize}
  \item Most cardiotoxic of commonly used amide LAs.
  \item Toxicity threshold markedly lower ($>$1.5 $\mu$g/mL).
  \item Causes severe ventricular arrhythmias due to slow dissociation from myocardial Na$^+$ channels.
  \item Cardiac arrest may be sudden and refractory.
\end{itemize}

\textbf{Clinical consequences:}
\begin{itemize}
  \item Avoid in IV regional anaesthesia (IVRA).
  \item Requires meticulous aspiration and incremental dosing.
\end{itemize}

\textbf{Kinetics:}
\begin{itemize}
  \item Absorption: high potency and lipid solubility increase tissue uptake; systemic absorption depends on vascularity of site.
  \item Distribution: extensive distribution into lipid-rich tissues $\rightarrow$ long context-sensitive half-time.
  \item Metabolism: low-extraction hepatic metabolism $\rightarrow$ clearance limited by intrinsic enzyme activity (not blood flow).
  \item Elimination half-life: $\sim$160 minutes.
  \item Clinical implications: high accumulation risk during infusions; prolonged recovery; toxicity risk rises disproportionately.
\end{itemize}

\subsection*{11.3 Ropivacaine}

\textbf{Type:} Amide LA; pure S-enantiomer.

\textbf{Onset:} Moderate (similar pKa to bupivacaine (8.1) $\rightarrow$ similar onset).

\textbf{Duration:} Long (high protein binding $\sim$94\%; less lipid-soluble than bupivacaine $\rightarrow$ slightly shorter duration).

\textbf{Uses (per Peck):}
\begin{itemize}
  \item Epidural anaesthesia/analgesia
  \item Peripheral nerve blocks
  \item Postoperative analgesia (infusions)
\end{itemize}

\textbf{Pharmacological profile:}
\begin{itemize}
  \item Lower lipid solubility $\rightarrow$ reduced potency but also reduced risk of systemic toxicity.
  \item Produces relatively greater sensory than motor block $\rightarrow$ clinically useful in obstetrics and postoperative analgesia.
\end{itemize}

\textbf{Toxicity profile:}
\begin{itemize}
  \item Less cardiotoxic than bupivacaine due to stereoselective binding with faster off-rate from cardiac Na$^+$ channels.
  \item Still capable of causing CNS toxicity if overdosed.
\end{itemize}

\textbf{Kinetics:}
\begin{itemize}
  \item Absorption: slightly slower systemic uptake than bupivacaine due to lower lipid solubility.
  \item Distribution: high protein binding ($\sim$94\%) ensures long duration with somewhat reduced lipid accumulation.
  \item Metabolism: hepatic metabolism via CYP1A2 and CYP3A4; moderate clearance.
  \item Elimination half-life: $\sim$120 minutes.
  \item Clinical implications: less accumulation and safer systemic profile than bupivacaine, especially for prolonged infusions.
\end{itemize}

\subsection*{11.4 Prilocaine (Expanded Full Entry)}

\textbf{Type:} Amide local anaesthetic.

\textbf{Onset:} Rapid (low pKa $\rightarrow$ good proportion unionised at physiological pH).

\textbf{Duration:} Moderate (lower protein binding $\sim$55\% produces shorter duration vs lidocaine).

\textbf{Uses (from Peck):}
\begin{itemize}
  \item Infiltration anaesthesia
  \item Peripheral nerve blocks
  \item IV regional anaesthesia (IVRA)
  \item Component of EMLA cream (with lidocaine)
\end{itemize}

\textbf{Pharmacological profile:}
\begin{itemize}
  \item Lower lipid solubility $\rightarrow$ lower potency.
  \item Less vasodilation compared with lidocaine.
  \item Preferred for IVRA because it produces good sensory block with rapid recovery.
\end{itemize}

\textbf{Toxicity considerations:}
\begin{itemize}
  \item Unique risk of methaemoglobinaemia, especially with high doses or prolonged administration.
  \item Infants and patients with G6PD deficiency at higher risk.
  \item Toxic metabolite: o-toluidine.
\end{itemize}

\textbf{Kinetics:}
\begin{itemize}
  \item Absorption: follows typical amide pattern; faster from vascular sites.
  \item Distribution: lower lipid solubility reduces deep tissue sequestration.
  \item Metabolism: significant extrahepatic metabolism (lung, kidney) $\rightarrow$ lower accumulation risk.
  \item Elimination half-life: $\sim$100 minutes.
  \item Clinical implications: rapid clearance but limited by methaemoglobin formation at higher doses.
\end{itemize}

\subsection*{11.5 Cocaine}

\textbf{Type:} Ester local anaesthetic.

\textbf{Onset:} Rapid--moderate (moderately lipid-soluble; rapid mucosal absorption).

\textbf{Duration:} Moderate (prolonged by intrinsic vasoconstrictive properties).

\textbf{Uses (from Peck):}
\begin{itemize}
  \item Topical anaesthesia, especially in ENT procedures (nasal, nasopharyngeal).
  \item Provides both anaesthesia and haemostasis.
\end{itemize}

\textbf{Pharmacological profile:}
\begin{itemize}
  \item Only LA causing vasoconstriction -- due to inhibition of catecholamine reuptake.
  \item High protein binding ($\sim$95\%).
  \item Useful in areas where bleeding control is desired.
\end{itemize}

\textbf{Toxicity considerations:}
\begin{itemize}
  \item High risk of hypertension, tachycardia, arrhythmias, and myocardial ischaemia.
  \item Lower toxic plasma concentration ($\sim$0.5 $\mu$g/mL) vs other agents.
  \item CNS stimulation, agitation, seizures.
\end{itemize}

\textbf{Kinetics:}
\begin{itemize}
  \item Absorption: rapid through mucosa; slower dermal uptake.
  \item Distribution: wide distribution with high affinity for CNS and sympathetic tissues.
  \item Metabolism: primarily hepatic and plasma cholinesterase hydrolysis; slower than other esters.
  \item Elimination half-life: variable; typically $\sim$60 minutes.
  \item Clinical implications: vasoconstriction prolongs duration but increases cardiovascular risk.
\end{itemize}

\subsection*{11.6 Chloroprocaine (2-Chloroprocaine)}

\textbf{Type:} Ester local anaesthetic.

\textbf{Onset:} Very rapid (pKa favourable; extremely fast onset in clinical use).

\textbf{Duration:} Very short (fastest offset of all clinically used LAs due to rapid ester hydrolysis).

\textbf{Uses (Peck emphasis \& modern practice):}
\begin{itemize}
  \item Epidural anaesthesia when rapid onset and rapid recovery are required (e.g. urgent obstetric procedures).
  \item Short peripheral blocks.
  \item Not used for spinal anaesthesia due to neurotoxicity concerns with older formulations; modern preservative-free preparations safer.
\end{itemize}

\textbf{Pharmacological profile:}
\begin{itemize}
  \item Very low lipid solubility.
  \item Potency lower than lidocaine.
  \item Minimal systemic accumulation.
\end{itemize}

\textbf{Toxicity considerations:}
\begin{itemize}
  \item Very low systemic toxicity due to extremely rapid metabolism.
  \item Older formulations contained sodium bisulfite $\rightarrow$ neurotoxicity risk; modern formulations improved.
\end{itemize}

\textbf{Kinetics:}
\begin{itemize}
  \item Absorption: limited by low potency but rapid when used epidurally.
  \item Distribution: minimal redistribution due to extremely fast metabolism.
  \item Metabolism: fastest pseudocholinesterase breakdown of all esters.
  \item Elimination half-life: seconds in plasma.
  \item Clinical implications: excellent safety for epidural top-ups; ideal when rapid recovery is needed.
\end{itemize}

\end{document}
