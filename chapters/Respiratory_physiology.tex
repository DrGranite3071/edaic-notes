\documentclass[11pt,a4paper]{article}

% --- pdfLaTeX / TeX Live 2024 safe preamble ---
\usepackage[utf8]{inputenc}
\usepackage[T1]{fontenc}
\usepackage{lmodern}
\usepackage[a4paper,margin=2.2cm]{geometry}
\usepackage{microtype}
\usepackage{amsmath,amssymb}
\usepackage{booktabs}
\usepackage{tabularx}
\usepackage{enumitem}
\usepackage{hyperref}
\hypersetup{hidelinks}

\title{Respiratory System Study Notes\\Chapters 15--16 (Structured rewrite)}
\author{}
\date{\today}

\begin{document}
\maketitle

\noindent\textit{Source: Kam \& Power --- Principles of Physiology for the Anaesthetist, Chapters 15.}

\tableofcontents
\newpage

% =========================================================
\section{Chapter 15 --- Functions of the Respiratory System}

\subsection{Big picture}

\subsubsection{Primary function}
\begin{itemize}[leftmargin=*,nosep]
  \item \textbf{Gas exchange}: transfer of O$_2$ into blood and CO$_2$ out of blood across the alveolar--capillary interface.
\end{itemize}

\subsubsection{Non-respiratory functions of the lung}
\begin{itemize}[leftmargin=*,nosep]
  \item \textbf{Blood filter}: traps small clots and detached cells, limiting systemic embolization.
  \item \textbf{Blood reservoir (high capacitance)}
  \begin{itemize}[leftmargin=*,nosep]
    \item \textasciitilde16\% of total blood volume in the \textbf{supine} position.
    \item \textasciitilde9\% of total blood volume in the \textbf{erect} position.
    \item Can redistribute blood to vital organs in \textbf{hypovolaemic shock}.
  \end{itemize}
  \item \textbf{Metabolism / handling of bioactive substances}
  \begin{itemize}[leftmargin=*,nosep]
    \item Converts \textbf{angiotensin I $\rightarrow$ angiotensin II}.
    \item Synthesis and breakdown of \textbf{bradykinin}.
    \item Storage/release of \textbf{serotonin} and \textbf{histamine}.
    \item Synthesis of peptides (e.g.\ \textbf{substance P}), \textbf{prostaglandins}, \textbf{surfactants}, \textbf{immunoglobulins}.
    \item Inactivation of \textbf{adrenaline} and \textbf{noradrenaline}.
    \item Presence of \textbf{cytochrome P-450 isoenzymes}.
  \end{itemize}
  \item \textbf{Acid--base regulation}: ventilation changes arterial $P_{\mathrm{CO_2}}$.
  \item \textbf{Phonation}: CNS control of respiratory muscles generates airflow through vocal cords.
  \item \textbf{Pulmonary defence}
  \begin{itemize}[leftmargin=*,nosep]
    \item Secretion of \textbf{IgA} (innate immunity).
    \item Removal of airborne particles by \textbf{phagocytosis} and \textbf{mucociliary action}.
    \item \textbf{Lymphoid tissue with T lymphocytes} provides first-line defence against the external environment.
  \end{itemize}
\end{itemize}

\paragraph{Exam pitfalls}
\begin{itemize}[leftmargin=*,nosep]
  \item ``Lungs only do gas exchange'' is false: know key \textbf{non-respiratory} roles (filter/reservoir/metabolic/defence).
\end{itemize}

\subsection{Functional anatomy: airway tree, respiratory zone, alveolar--capillary unit}

\subsubsection{Conducting airways vs respiratory zone}

\paragraph{Conducting airways}
\begin{itemize}[leftmargin=*,nosep]
  \item Function: \textbf{bulk flow} to/from respiratory zone + \textbf{warm, humidify, filter} inspired air.
  \item Anatomical extent: \textbf{trachea (generation 1)} $\rightarrow$ \textbf{terminal bronchioles (generation 16)}.
  \item Conducting zone volume \textasciitilde \textbf{150 mL}.
  \item \textbf{Cartilage disappears from the 11th generation}; beyond this, airway diameter is mainly determined by \textbf{lung volume}.
  \item Airway wall \textbf{smooth muscle}: dilates with \textbf{sympathetic} stimulation; constricts with \textbf{parasympathetic} stimulation.
\end{itemize}

\paragraph{Respiratory zone}
\begin{itemize}[leftmargin=*,nosep]
  \item Begins at \textbf{respiratory bronchioles (generations 17--19)} (first airways with alveoli in their walls).
  \item Continues through \textbf{alveolar ducts (generations 20--22)} to \textbf{alveolar sacs (generation 23)}.
  \item Respiratory zone volume \textasciitilde \textbf{3000 mL}.
  \item Gas exchange occurs by \textbf{diffusion} (not bulk flow).
\end{itemize}

\subsubsection{Pulmonary lobule and alveolar stability}
\begin{itemize}[leftmargin=*,nosep]
  \item Parenchyma is an interconnected network of alveolar walls and interstitial tissues.
  \item \textbf{Alveolar interdependence}: a collapsing region tends to be pulled open by surrounding stretched tissue.
  \item With \textbf{surfactant} and \textbf{collateral ventilation via pores of Kohn} $\rightarrow$ helps prevent alveolar collapse.
  \item \textbf{Pulmonary lobule}: airways and alveoli distal to a single terminal bronchiole.
\end{itemize}

\subsubsection{Alveolar--capillary unit: key numbers and structure}
\begin{itemize}[leftmargin=*,nosep]
  \item \textbf{Alveoli}
  \begin{itemize}[leftmargin=*,nosep]
    \item \textbf{200--600 million} (average \textasciitilde\textbf{300 million}).
    \item Mean diameter at \textbf{FRC}: \textbf{0.2 mm}.
    \item \textbf{Polyhedral} (not spherical) because septae are flat.
    \item \textbf{Type I} cells: flat squamous epithelium (major gas-exchanging surface).
    \item \textbf{Type II} cells: cuboidal, produce \textbf{surfactant}.
  \end{itemize}
  \item \textbf{Pulmonary capillaries}
  \begin{itemize}[leftmargin=*,nosep]
    \item Diameter \textasciitilde\textbf{10 $\mu$m}.
    \item Endothelial thickness \textasciitilde\textbf{0.1 $\mu$m}.
    \item RBC transit through capillary network (and 2--3 alveoli): \textasciitilde\textbf{0.75 s}.
  \end{itemize}
  \item \textbf{Diffusion barrier thickness} (alveolar gas $\rightarrow$ capillary blood): \textasciitilde\textbf{0.3 $\mu$m} (RBC diameter \textasciitilde\textbf{7 $\mu$m}).
  \item \textbf{Surface area}: \textasciitilde\textbf{50--100 m$^2$}.
\end{itemize}

\paragraph{Exam pitfalls}
\begin{itemize}[leftmargin=*,nosep]
  \item Conducting zone (bulk flow; \textasciitilde150 mL) vs respiratory zone (diffusion; \textasciitilde3000 mL).
  \item Type I (exchange) vs Type II (surfactant).
  \item Generation 11 (cartilage disappears) vs generation 16 (end of conducting zone).
\end{itemize}

\subsection{Muscles of ventilation}

\subsubsection{Mechanics overview}
\begin{itemize}[leftmargin=*,nosep]
  \item Thorax separated from abdomen by the \textbf{diaphragm}.
  \item Diaphragm contraction increases vertical chest dimension by pushing abdominal contents down.
  \item Ribs move laterally/anteriorly to increase thoracic cross-sectional area.
  \item \textbf{Inspiration}: active. \quad \textbf{Expiration}: passive during quiet breathing; becomes active when ventilation increases.
\end{itemize}

\subsubsection{Inspiratory muscles}
\begin{itemize}[leftmargin=*,nosep]
  \item \textbf{Diaphragm}
  \begin{itemize}[leftmargin=*,nosep]
    \item Innervation: \textbf{phrenic nerve (C3--5)}.
    \item Descent: \textbf{1--2 cm} in quiet breathing; up to \textbf{10 cm} in forced inspiration.
  \end{itemize}
  \item \textbf{External intercostals}: fibres slope down and anteriorly; ribs move \textbf{upwards and forwards}.
  \item \textbf{Scalenes}: active even during quiet breathing; elevate ribcage.
  \item \textbf{Sternocleidomastoid}: recruited when breathing increases; elevates ribcage.
\end{itemize}

\subsubsection{Expiratory muscles}
\begin{itemize}[leftmargin=*,nosep]
  \item Quiet expiration is \textbf{passive}.
  \item Active expiration occurs with increased ventilation (exercise, speech, coughing/sneezing) and in pathology (e.g.\ COPD).
  \item Main expiratory muscles: \textbf{abdominal wall} (rectus abdominis, internal/external obliques, transversalis) and \textbf{internal intercostals}.
\end{itemize}

\paragraph{Exam pitfalls}
\begin{itemize}[leftmargin=*,nosep]
  \item Quiet expiration is passive; \textbf{forced} expiration recruits abdominal wall + internal intercostals.
\end{itemize}

\subsection{Lung--chest wall equilibrium and pressure definitions}

\subsubsection{Pleural arrangement}
\begin{itemize}[leftmargin=*,nosep]
  \item Lungs covered by \textbf{visceral pleura}; chest wall lined by \textbf{parietal pleura}.
  \item Between them is a potential \textbf{intrapleural space}.
  \item Diaphragm separates lungs from abdominal contents.
\end{itemize}

\subsubsection{Resting equilibrium (end of normal expiration)}
\begin{itemize}[leftmargin=*,nosep]
  \item Elastic forces balance:
  \begin{itemize}[leftmargin=*,nosep]
    \item Lung tends to collapse inward.
    \item Chest wall tends to expand outward (plus contribution from diaphragmatic tone).
  \end{itemize}
  \item This produces a \textbf{negative intrapleural pressure}.
\end{itemize}

\subsubsection{Transpulmonary pressure}
\begin{itemize}[leftmargin=*,nosep]
  \item \textbf{Transpulmonary pressure} $=$ alveolar pressure $-$ intrapleural pressure.
  \item Distending (transmural) pressure across alveoli.
  \item At end of normal expiration, forces balance at \textbf{FRC}.
\end{itemize}

\paragraph{Exam pitfalls}
\begin{itemize}[leftmargin=*,nosep]
  \item Transpulmonary pressure is a \textbf{difference}, not the same as intrapleural pressure.
\end{itemize}

\subsection{Events during a normal tidal breath (Table 15.1)}

\subsubsection{Summary table (spontaneous breathing)}
\begin{table}[h!]
\centering
\begin{tabularx}{\textwidth}{@{}lXX@{}}
\toprule
\textbf{Step} & \textbf{Inspiration (active)} & \textbf{Expiration (passive, quiet breathing)}\\
\midrule
Neural drive & Inspiratory centre activated $\rightarrow$ impulses to inspiratory muscles & Inspiratory centre activity ceases\\
Muscle activity & Diaphragm contracts ($\pm$ external intercostals) & Inspiratory muscles relax\\
Thoracic volume & Increases & Decreases toward resting level\\
Intrapleural pressure & Becomes \textbf{more negative} & Becomes \textbf{less negative} (returns toward baseline)\\
Transpulmonary pressure & Increases $\rightarrow$ alveoli distend; elastic recoil increases & Decreases as elastic recoil empties lungs\\
Alveolar pressure (vs atmosphere) & Falls slightly below 0 $\rightarrow$ airflow \textbf{into} alveoli & Rises slightly above 0 $\rightarrow$ airflow \textbf{out}\\
End of phase & Flow stops when alveolar pressure returns to 0 & Flow stops when alveolar pressure returns to 0\\
\bottomrule
\end{tabularx}
\end{table}

\subsubsection{Key takeaways (MCQ-friendly)}
\begin{itemize}[leftmargin=*,nosep]
  \item Flow occurs only when alveolar pressure $\neq$ atmospheric pressure.
  \item End-inspiration and end-expiration: alveolar pressure $=$ atmospheric (no flow).
  \item The pressure that keeps alveoli open is \textbf{transpulmonary pressure} (alveolar $-$ intrapleural).
\end{itemize}

\subsection{Pressures, flow, and volume during the breathing cycle}

\subsubsection{At rest (end expiration)}
\begin{itemize}[leftmargin=*,nosep]
  \item Alveolar pressure = mouth pressure = 0 (relative to atmosphere).
  \item Intrapleural pressure \textasciitilde $-5$ cmH$_2$O.
  \item Lung volume = FRC.
  \item No flow.
\end{itemize}

\subsubsection{Inspiration}
\begin{itemize}[leftmargin=*,nosep]
  \item Inspiratory muscle activity expands chest wall $\rightarrow$ intrapleural pressure falls.
  \item Alveolar pressure falls to \textasciitilde $-1$ cmH$_2$O $\rightarrow$ airflow into lungs.
  \item End-inspiration:
  \begin{itemize}[leftmargin=*,nosep]
    \item Intrapleural pressure \textasciitilde $-8$ cmH$_2$O.
    \item Alveolar pressure returns to 0 (no flow).
    \item Lung volume increases by \textasciitilde 500 mL (tidal volume).
  \end{itemize}
\end{itemize}

\subsubsection{Expiration}
\begin{itemize}[leftmargin=*,nosep]
  \item Inspiratory drive ceases $\rightarrow$ system returns toward resting equilibrium.
  \item Intrapleural pressure becomes less negative.
  \item Alveolar pressure becomes \textasciitilde $+1$ cmH$_2$O $\rightarrow$ airflow out.
  \item End-expiration:
  \begin{itemize}[leftmargin=*,nosep]
    \item Intrapleural pressure returns to \textasciitilde $-5$ cmH$_2$O.
    \item Alveolar pressure returns to 0; flow stops; volume returns to FRC.
  \end{itemize}
  \item Quiet expiration is passive (elastic recoil); can become active with expiratory muscle recruitment.
\end{itemize}

\paragraph{Exam pitfalls}
\begin{itemize}[leftmargin=*,nosep]
  \item In spontaneous breathing, alveolar pressure is 0 at end-inspiration and end-expiration; flow is driven by small swings (e.g.\ to $-1$ and $+1$ cmH$_2$O).
\end{itemize}

\subsection{Pressure waveforms during intermittent positive pressure ventilation (IPPV)}
\begin{itemize}[leftmargin=*,nosep]
  \item During \textbf{inspiration} in IPPV:
  \begin{itemize}[leftmargin=*,nosep]
    \item Alveolar pressure rises from baseline to positive values.
    \item Intrapleural pressure rises from \textasciitilde $-5$ cmH$_2$O to about $+2$ to $+3$ cmH$_2$O at end inspiration.
  \end{itemize}
  \item During \textbf{expiration}: intrapleural pressure returns to \textasciitilde $-5$ cmH$_2$O.
\end{itemize}

\paragraph{Exam pitfalls}
\begin{itemize}[leftmargin=*,nosep]
  \item Spontaneous inspiration: intrapleural pressure becomes \emph{more negative}.
  \item IPPV inspiration: intrapleural pressure becomes \emph{less negative and may become positive}.
\end{itemize}

% =========================================================

\section{Chapter 16 --- Mechanical Properties of the Lungs}
\textit{Primary source: Kam \& Power — \emph{Principles of Physiology for the Anaesthetist}, Chapter 16.}

\subsection{What this chapter is about}
Lung mechanics describes how pressure generates:
\begin{itemize}
  \item \textbf{Flow} (air movement),
  \item \textbf{Volume change} (inflation/deflation),
  \item \textbf{Stability} of alveoli and small airways (surfactant, interdependence, closing capacity).
\end{itemize}

\subsection{Pressures and reference points}
\subsubsection{Key pressures}
\begin{itemize}
  \item \textbf{Alveolar pressure} ($P_A$)
  \item \textbf{Pleural (intrapleural) pressure} ($P_{pl}$)
  \item \textbf{Transpulmonary pressure} ($P_L$): distending pressure of the lung
\end{itemize}
\[
P_L = P_A - P_{pl}
\]

\subsubsection{Elastic equilibrium at end-expiration (FRC)}
At \textbf{FRC}, inward recoil of the lung balances outward recoil of the chest wall. The chapter states loss of diaphragmatic tone reduces FRC by \textasciitilde 400 mL.

\subsection{Compliance, elastance, and measurement}
\subsubsection{Definitions}
\[
C = \frac{\Delta V}{\Delta P}, \qquad E = \frac{1}{C}
\]
\begin{itemize}
  \item High compliance: reminder ``easy to inflate''.
  \item Low compliance: ``stiff''.
\end{itemize}

\subsubsection{Static compliance}
Measured when \textbf{flow has ceased} (pause/breath-hold), allowing slow units to fill; typically $C_{static} > C_{dynamic}$.

\subsubsection{Dynamic compliance}
Measured during ongoing breathing; reduced when units have different time constants.

\subsubsection{Frequency dependence}
Dynamic compliance falls as respiratory rate rises in heterogeneous lungs; used as an early marker of small airway closure.

\subsubsection{Specific compliance}
\[
C_{\text{spec}} = \frac{C}{FRC}
\]

\subsection{Determinants of compliance}
Broad determinants:
\begin{enumerate}
  \item \textbf{Elastic recoil of tissue} (elastin/collagen)
  \item \textbf{Surface tension} at the alveolar air--fluid interface
\end{enumerate}

Additional factors listed in the chapter (selected):
\begin{itemize}
  \item Lung volume; age; posture; obesity; pulmonary blood volume; bronchial smooth muscle tone.
  \item Disease:
  \begin{itemize}
    \item Fibrosis: stiffer lungs.
    \item Emphysema: increased compliance from loss of septal tissue opposing expansion.
  \end{itemize}
\end{itemize}

\subsection{Thoracic cage compliance and total respiratory system compliance}
Lung and chest wall compliances combine ``in series'':
\[
\frac{1}{C_{total}} = \frac{1}{C_{lung}} + \frac{1}{C_{chest}}
\]
Chapter values: $C_{lung} \approx 200$ mL/cmH$_2$O, $C_{chest} \approx 200$ mL/cmH$_2$O, so $C_{total} \approx 100$ mL/cmH$_2$O.

\subsection{Elastic recoil: tissue + surface tension}
The chapter states surface tension contributes \textasciitilde 70\% of total elastic forces in the normal lung.

\subsubsection{Laplace and stability}
If surface tension were constant, smaller units would have higher pressure and tend to empty into larger units, promoting collapse.

\subsection{Surfactant}
\subsubsection{Origin and turnover}
Produced by type II alveolar epithelial cells; stored in lamellated bodies. Half-life stated as \textasciitilde 15--30 h; most components recycled by type II cells.

\subsubsection{Composition (chapter description)}
\begin{itemize}
  \item \textasciitilde 90\% lipid (mainly phospholipid; also cholesterol)
  \begin{itemize}
    \item principal: dipalmitoyl phosphatidylcholine
    \item significant: phosphatidylglycerol
  \end{itemize}
  \item proteins \textasciitilde 2--8\%: SP-A, SP-B, SP-C, SP-D
\end{itemize}

\subsubsection{Functions}
\begin{itemize}
  \item Reduces surface tension $\rightarrow$ increases distensibility, reduces elastic recoil.
  \item Reduces work of breathing.
  \item Stabilises alveoli (preferential reduction of surface tension in small alveoli, especially during expiration).
  \item Reduces atelectasis; reduces tendency to alveolar oedema.
  \item Produces hysteresis.
\end{itemize}

\subsection{Hysteresis}
Inflation and deflation pressure--volume curves differ. Contributors described:
\begin{itemize}
  \item surfactant-related changes in surface tension,
  \item recruitment of collapsed units early in inspiration,
  \item stress relaxation at sustained high lung volumes.
\end{itemize}

\subsection{Alveolar interdependence}
Neighbouring alveoli share walls; traction from adjacent units tends to prevent collapse of a single unit (stabilising mechanism alongside surfactant).

\subsection{Lung volumes and FRC}
\subsubsection{Definition}
\[
FRC = ERV + RV
\]
Occurs where outward recoil of chest wall balances inward recoil of lung.

\subsubsection{Functions of FRC (chapter list)}
\begin{itemize}
  \item Oxygen store; reduces PaO$_2$ swings.
  \item Helps prevent airway collapse/atelectasis.
  \item Minimises pulmonary vascular resistance.
  \item Maintains airway patency; optimises compliance; minimises work of breathing.
\end{itemize}

\subsubsection{Factors affecting FRC (chapter list)}
\begin{itemize}
  \item Height (direct relationship); females \textasciitilde 10\% less than males; not correlated with age (per chapter).
  \item Posture: supine reduces FRC by up to \textasciitilde 1000 mL.
  \item Pregnancy (3rd trimester \textasciitilde 20\% reduction) and obesity reduce FRC.
  \item Increased elastic recoil (e.g.\ fibrosis) reduces FRC.
  \item Closing capacity rises with age and may exceed FRC (chapter notes upright \textasciitilde 70 y; supine \textasciitilde 44 y).
\end{itemize}

\subsubsection{Measurement methods}
Nitrogen washout; helium dilution; body plethysmography.

\subsection{Closing capacity}
\subsubsection{Definition}
\[
\text{Closing capacity} = RV + \text{Closing volume}
\]
Normally, $FRC >$ closing capacity. Small airways ($<2$ mm) begin to collapse during expiration at closing capacity.

\subsubsection{Increases with}
Age; smoking; lung disease (reduced elastic recoil and reduced radial traction).

\subsubsection{Consequences and posture/age}
Airway closure can leave perfused units unventilated $\rightarrow$ impaired oxygenation. Chapter statements include:
\begin{itemize}
  \item by \textasciitilde 65 y, closing capacity exceeds standing FRC,
  \item supine airway closure during normal tidal breathing may occur by \textasciitilde 45 y.
\end{itemize}

\subsection{Gas flow patterns and airway resistance}
\subsubsection{Laminar, turbulent, transitional}
\begin{itemize}
  \item Laminar: parabolic profile; strong radius dependence ($r^4$); viscosity is key gas property.
  \item Turbulent: eddies; higher driving pressure; $\Delta P \propto \text{flow}^2$; density is key gas property; strong radius dependence (Fanning equation context).
  \item Transitional: common at airway branches; mixed dependence on flow and flow$^2$.
\end{itemize}

\subsubsection{Reynolds number}
Turbulence tends to occur when $Re \gtrsim 2000$ in smooth tubes; promoted by high velocity, larger diameter, higher density, lower viscosity gas.

\subsubsection{Distribution in the lung}
Trachea/larynx tend to be turbulent at high velocities; laminar likely only in very small airways; much of bronchial tree is transitional.

\subsubsection{Airway resistance (Raw)}
Chapter values: $R_{aw}$ \textasciitilde 2 cmH$_2$O/L/s; normal tidal airflow requires a mouth--alveolar pressure difference \textasciitilde 1 cmH$_2$O.
Determinants: lung volume; bronchial smooth muscle tone; dynamic airway compression.

\subsection{Bronchial smooth muscle tone}
\begin{itemize}
  \item Parasympathetic (vagal) ACh at muscarinic receptors $\rightarrow$ bronchoconstriction.
  \item Sympathetic innervation less important; circulating adrenaline at $\beta_2$ receptors $\rightarrow$ bronchodilation.
  \item Non-cholinergic, non-adrenergic pathways can bronchodilate.
  \item Mediators: histamine causes constriction + mucosal swelling; leukotrienes and some prostaglandins can contribute to bronchospasm.
\end{itemize}

\subsection{Dynamic compression and the equal pressure point}
During forced expiration, pleural pressure becomes positive; airway pressure declines toward the mouth due to resistive pressure drop. The \textbf{equal pressure point} is where intraluminal pressure equals pleural pressure; beyond this point (toward mouth), airways can collapse.
Flow becomes effort-independent once dynamic compression limits further increase. Collapse is more likely at low lung volumes, high airway resistance, or reduced elastic recoil (e.g.\ emphysema).

\subsection{Work of ventilation}
\subsubsection{Components}
Inspiratory muscles do work against elastic and non-elastic (resistive) forces; expiration is normally passive.

\subsubsection{Recall values (chapter)}
\begin{itemize}
  \item Resistive work: \textasciitilde 35\% (dissipated as heat); tissue resistive work contributes \textasciitilde 10\% of resistance work.
  \item Elastic work: \textasciitilde 65\% (stored as potential energy).
\end{itemize}

\subsubsection{When expiration becomes active}
If airway resistance or expiratory flow demand rises, expiratory resistive work may exceed stored elastic energy $\rightarrow$ expiratory muscle recruitment.

\subsubsection{Oxygen cost of breathing}
Chapter value: respiratory muscle oxygen requirement \textasciitilde 3 mL/min at rest; rises with increased ventilation and in lung disease.

\subsection{Exam-oriented pitfalls and high-yield checks}
\begin{itemize}
  \item Compliance is $\Delta V/\Delta P$; elastance is the inverse.
  \item Static compliance (no flow) $>$ dynamic compliance.
  \item Closing capacity $= RV +$ closing volume.
  \item Laminar: viscosity; turbulent: density; much of bronchial tree is transitional.
  \item Emphysema: increased compliance but reduced elastic recoil $\rightarrow$ air trapping and dynamic compression.
\end{itemize}

\subsection{Additions and useful nuances from other core texts (not emphasised here)}
\subsubsection{Compliance is volume-dependent (P--V curve shape)}
Other texts emphasise compliance is only approximately linear around the normal working range near FRC:
\begin{itemize}
  \item High lung volumes: compliance falls as elastic elements approach their limit.
  \item Low lung volumes: compliance falls due to airway/alveolar closure and opening pressure requirement.
\end{itemize}
\textit{Sources:} \emph{Fundamentals of Anaesthesia} (4th ed), Ch.17 (P--V curve; compliance at extremes of lung volume).

\subsubsection{Pendelluft and dynamic compliance}
Pendelluft (gas redistribution between lung regions during dynamic conditions) can reduce apparent dynamic compliance.
\textit{Sources:} \emph{Morgan \& Mikhail's Clinical Anesthesiology} (5th ed), Ch.23 (pendelluft/gas redistribution concept).

\subsubsection{Measuring pleural pressure clinically}
Pleural pressure can be measured by intrapleural catheter or estimated using a mid-oesophageal balloon.
\textit{Sources:} \emph{Fundamentals of Anaesthesia} (4th ed), Ch.17 (pleural pressure measurement; oesophageal balloon).

\subsubsection{Typical pressure values during quiet breathing}
Illustrative values commonly quoted:
\begin{itemize}
  \item End-expiration: $P_{pl}$ about $-5$ cmH$_2$O; $P_A \approx 0$ $\Rightarrow P_L \approx +5$ cmH$_2$O.
  \item Inspiration: $P_{pl}$ becomes more negative; $P_A$ slightly negative to drive flow; at end-inspiration flow is zero and $P_A$ returns to 0 while $P_{pl}$ remains more negative.
\end{itemize}
\textit{Sources:} \emph{Morgan \& Mikhail} (5th ed), Ch.23 (quiet breathing pressure changes); similar concepts also described (often in kPa) in \emph{Fundamentals of Anaesthesia} (4th ed), Ch.17.

\subsubsection{Tissue resistance}
Some texts explicitly highlight tissue (viscoelastic) resistance as an important non-elastic component.
\textit{Sources:} \emph{Morgan \& Mikhail} (5th ed), Ch.23 (section on tissue resistance).

\subsubsection{Work of breathing: alternative breakdowns}
Resistive work may be subdivided into airway resistance work and tissue viscous (inelastic) work; proportional contributions vary by source.
\textit{Sources:} \emph{Ganong's Review of Medical Physiology} (Barrett et al.), pulmonary mechanics/work of breathing section; \emph{Morgan \& Mikhail} (5th ed), Ch.23.

\subsubsection{Proximal airway pressure may not equal alveolar pressure}
During dynamic conditions, proximal airway pressure (ventilator) may not reflect distal/alveolar pressure when resistance is high and/or tubing is compliant.
\textit{Sources:} \emph{Morgan \& Mikhail} (5th ed), Ch.4 (mechanical ventilation: airway vs alveolar pressure; resistance/compliance).

\subsubsection{Ventilator/circuit compliance and ``lost'' tidal volume}
In volume-controlled ventilation, some set tidal volume is lost to circuit expansion (compressible volume); exhaled tidal volume is best measured near the airway.
\textit{Sources:} \emph{Morgan \& Mikhail} (5th ed), Ch.3 (breathing systems) and Ch.4 (ventilation: circuit compliance/compressible volume).

\subsubsection{Time constant heuristic}
Some FRCA-style resources explicitly frame lung filling/emptying with the time constant:
\[
\tau = R \times C
\]
High $R$ and/or high $C$ (e.g.\ obstructive disease) $\rightarrow$ prolonged emptying, need for longer expiratory time to avoid gas trapping.
\textit{Sources:} \emph{Primary FRCA in a Box} (2nd ed), time constants/exponential relationships section; \emph{Morgan \& Mikhail} (5th ed), Ch.23 (air trapping/auto-PEEP concepts).

\subsubsection{Surfactant proteins: roles beyond surface tension}
SP-B and SP-C help form the surface film; SP-A and SP-D (collectins) contribute to innate immunity and surfactant turnover/regulation.
\textit{Sources:} \emph{Ganong} (Barrett et al.), pulmonary surfactant section.

\subsubsection{Spirometry as surrogate markers of airway mechanics}
FEV$_1$/FVC as obstruction index; FEF$_{25\text{--}75\%}$ (MMEF) described as relatively effort-independent and sometimes abnormal earlier in obstructive disease.
\textit{Sources:} \emph{Morgan \& Mikhail} (5th ed), Ch.23--24 (spirometric indices); \emph{Fundamentals of Anaesthesia} (4th ed), pulmonary function testing section.

\subsubsection{Positive pressure breathing systems: practical mechanics complications}
\begin{itemize}
  \item CPAP/PEEP can improve oxygenation but may reduce cardiac output (raised intrathoracic pressure effects).
  \item High-frequency jet ventilation can generate intrinsic PEEP at high frequencies; delivered tidal volume and FiO$_2$ may be uncertain due to entrainment.
\end{itemize}
\textit{Sources:} \emph{Essentials of Equipment in Anaesthesia} (Al-Shaikh \& Stacey, 2023), Ch.13 (CPAP complications) and Ch.8 (HF jet ventilation: entrainment/intrinsic PEEP); \emph{Morgan \& Mikhail} (5th ed), Ch.57 (PEEP/CPAP adverse nonpulmonary effects incl. reduced cardiac output).

% Fragment for \input{} / \include{} (no preamble)

% =========================================================

\section{Chapter 17 -- Gas Exchange in the Lungs }

\subsection{What ``gas exchange'' means (exam framing)}
Gas exchange is the \textbf{net transfer} of \textbf{O$_2$} from alveoli to blood and \textbf{CO$_2$} from blood to alveoli.

\paragraph{Determinants of arterial blood gases}
\begin{itemize}
  \item \textbf{Input gas}: inspired partial pressures (humidification, barometric pressure).
  \item \textbf{Bulk flow}: alveolar ventilation and pulmonary blood flow.
  \item \textbf{Matching}: ventilation--perfusion distribution \((\dot V_A/\dot Q)\).
  \item \textbf{Membrane transfer}: diffusion across the blood--gas barrier.
  \item \textbf{Mixing}: venous admixture/shunt.
\end{itemize}

\paragraph{Practical organisation of problems}
\begin{itemize}
  \item \textbf{Hypoventilation} (global \(\downarrow \dot V_A\)) \(\rightarrow\) \(\uparrow\)PaCO$_2$, \(\downarrow\)PAO$_2$.
  \item \textbf{V/Q inequality} \(\rightarrow\) impaired O$_2$ transfer; CO$_2$ often buffered by ventilation.
  \item \textbf{Shunt / venous admixture} \(\rightarrow\) lower arterial O$_2$.
  \item \textbf{Diffusion limitation} \(\rightarrow\) widened A--a difference and impaired transfer.
\end{itemize}

\subsection{Partial pressures along the circulation (anchor values)}
Typical partial pressures (breathing air):
\begin{itemize}
  \item \textbf{Inspired (dry):} PIO$_2$ \(\approx\) 160 mmHg (21.3 kPa), PICO$_2$ \(\approx\) 0.
  \item \textbf{Alveolar gas:} PO$_2$ \(\approx\) 105 mmHg (14 kPa), PCO$_2$ \(\approx\) 40 mmHg (5.3 kPa).
  \item \textbf{Arterial blood:} PaO$_2$ \(\approx\) 100 mmHg (13.3 kPa), PaCO$_2$ \(\approx\) 40 mmHg (5.3 kPa).
  \item \textbf{Mixed venous blood:} PvO$_2$ \(\approx\) 40 mmHg (5.3 kPa), PvCO$_2$ \(\approx\) 46 mmHg (6.1 kPa).
\end{itemize}

\subsection{Alveolar ventilation and dead space}
\subsubsection{Definitions}
Minute alveolar ventilation:
\[
\dot V_A = f\,(V_T - V_D)
\]
Dead space = the ``wasted'' portion of tidal volume not contributing to exchange.

\subsubsection{Types of dead space}
\begin{itemize}
  \item \textbf{Anatomical dead space}: conducting airways.
  \item \textbf{Alveolar dead space}: ventilated alveoli with no perfusion.
  \item \textbf{Physiological dead space}: anatomical + alveolar dead space.
\end{itemize}

\subsubsection{CO$_2$ as a marker of ventilation}
With stable metabolism, alveolar CO$_2$ is inversely related to \(\dot V_A\).
Clinically, PaCO$_2$ tracks PACO$_2$ closely and is a practical index of alveolar ventilation when CO$_2$ production is stable.

\subsubsection{Measuring dead space (core equations)}
\begin{itemize}
  \item \textbf{Anatomical dead space}: Fowler method.
  \item \textbf{Physiological dead space}: Bohr equation.
\end{itemize}
\[
\frac{V_D}{V_T} = \frac{P_{a\mathrm{CO}_2} - P_{\bar E\mathrm{CO}_2}}{P_{a\mathrm{CO}_2}}
\]
where \(P_{\bar E\mathrm{CO}_2}\) is mixed expired CO$_2$ partial pressure.

\subsection{Oxygen in alveolar gas and the alveolar air equation}
\subsubsection{Why calculate ``ideal'' PAO$_2$?}
Ideal alveolar gas cannot be sampled directly: end-expired gas may include dead-space contributions, and shunt/VQ mismatch affects arterial PO$_2$ more than arterial PCO$_2$.

\subsubsection{Alveolar air equation}
Assumption used: ideal alveolar PCO$_2$ \(\approx\) PACO$_2$ \(\approx\) PaCO$_2$.
\[
R = \frac{\dot V_{\mathrm{CO}_2}}{\dot V_{\mathrm{O}_2}} \approx 0.8
\]
Conceptual form:
\[
P_{A\mathrm{O}_2} = P_{I\mathrm{O}_2} - \frac{P_{a\mathrm{CO}_2}}{R}
\]
Humidification:
\[
P_{I\mathrm{O}_2} = F_{I\mathrm{O}_2}\,(P_B - P_{H_2O})
\]

\subsection{Ventilation--perfusion (V/Q) inequality}
\subsubsection{Definitions and extremes}
Whole lung average in normal lungs: \(\dot V_A/\dot Q \approx 0.8\).
\begin{itemize}
  \item High \(\dot V_A/\dot Q\): \(\uparrow\) PO$_2$, \(\downarrow\) PCO$_2$ (dead-space effect if perfusion low).
  \item Low \(\dot V_A/\dot Q\): \(\downarrow\) PO$_2$, \(\uparrow\) PCO$_2$ (venous admixture effect).
  \item \textbf{Dead space}: \(\dot V_A/\dot Q \to \infty\).
  \item \textbf{Shunt}: \(\dot V_A/\dot Q \to 0\).
\end{itemize}

\subsubsection{Gravity and regional V/Q}
Perfusion falls more than ventilation towards the apex \(\Rightarrow\) \(\dot V_A/\dot Q\) higher at the top and lower at the bottom.
Representative apex values: \(\dot V_A/\dot Q \sim 3.3\), PO$_2 \sim 130\) mmHg, PCO$_2 \sim 28\) mmHg.

\subsubsection{Assessing V/Q mismatch}
\begin{itemize}
  \item Qualitative: compare PaO$_2$ with calculated ideal PAO$_2$; reminder that ABGs, imaging and nuclear methods can support diagnosis.
  \item Quantitative: shunt estimation (shunt equation, iso-shunt concepts), dead space (Bohr), and MIGET (research).
\end{itemize}

\subsection{Venous admixture (physiological shunt)}
\subsubsection{Normal sources}
\begin{itemize}
  \item Bronchial venous drainage into pulmonary veins: \(<1\%\) of cardiac output.
  \item Thebesian veins into the left heart: \(\sim 0.3\%\) of cardiac output.
\end{itemize}

\subsubsection{Shunt equation (oxygen content form)}
\[
\frac{\dot Q_S}{\dot Q_T} = \frac{C'_{c\mathrm{O}_2} - C_{a\mathrm{O}_2}}{C'_{c\mathrm{O}_2} - C_{v\mathrm{O}_2}}
\]
where \(C'_{c\mathrm{O}_2}\) is end-capillary O$_2$ content (derived from ideal PAO$_2$ and the dissociation curve). Assumption: shunted blood has the same O$_2$ content as mixed venous blood.

\subsection{Oxygen cascade}
O$_2$ moves down a sequence of partial pressure gradients from atmosphere to mitochondria.
Key steps: humidification in the trachea (\(P_{H_2O}\) at \(37^\circ\)C \(\approx 47\) mmHg), alveolar setting of PAO$_2$, small further fall to PaO$_2$ due to shunt/VQ mismatch/diffusion effects, then tissue extraction to low mitochondrial PO$_2$.

\paragraph{Factors that shift the cascade}
Inspired O$_2$ concentration; barometric pressure; alveolar ventilation; O$_2$ consumption; scatter of \(\dot V_A/\dot Q\); venous admixture; blood flow; haemoglobin concentration.

\subsection{Diffusion across the alveolar--capillary barrier}
\subsubsection{Fick's law (conceptual)}
Transfer \(\propto\) area \(\times\) partial pressure difference \(\times\) diffusion constant, and \(\propto 1/\)thickness.

\subsubsection{Why CO$_2$ diffuses easily}
CO$_2$ has a diffusion constant \(\sim 20\times\) that of O$_2$ (greater solubility), so despite a small gradient it transfers efficiently.

\subsubsection{Perfusion-limited transfer (normal)}
At rest, equilibration of blood with alveolar gases occurs rapidly (\(\sim 0.25\) s, about one-third of capillary transit time) \(\Rightarrow\) O$_2$ and CO$_2$ transfer are typically perfusion-limited.

\subsubsection{Diffusing capacity (DLCO)}
DLCO assesses diffusion properties:
\begin{itemize}
  \item Normal DLCO \(\sim 25\) mL/min/mmHg at rest; increases up to \(\sim 3\times\) during exercise (capillary recruitment/dilation).
  \item Decreases in emphysema (\(\downarrow\) surface area), fibrosis and pulmonary oedema (\(\uparrow\) thickness).
\end{itemize}

\subsection{High-yield consolidation (what to be able to do)}
\begin{itemize}
  \item Define anatomical vs alveolar vs physiological dead space.
  \item Use \(\dot V_A = f(V_T - V_D)\) and interpret effects on PaCO$_2$.
  \item Recall Bohr equation for \(V_D/V_T\) and the shunt equation for \(\dot Q_S/\dot Q_T\).
  \item Use the alveolar air equation conceptually to estimate ideal PAO$_2$.
  \item Explain V/Q inequality and see dead space/shunt as extremes of \(\dot V_A/\dot Q\).
  \item Describe the oxygen cascade and what lowers PaO$_2$ between atmosphere and mitochondria.
  \item Explain why CO$_2$ is rarely diffusion-limited; state what DLCO measures and how it changes in disease/exercise.
\end{itemize}

\subsection{Additions highlighted in other core texts (complements)}
\subsubsection{Absolute vs relative shunt and response to oxygen}
\begin{itemize}
  \item \textbf{Absolute shunt}: anatomic shunt + lung units with \(\dot V_A/\dot Q = 0\).
  \item \textbf{Relative shunt}: low \(\dot V_A/\dot Q\) units.
\end{itemize}
Hypoxaemia from low \(\dot V_A/\dot Q\) is usually partially correctable with increased FiO$_2$; hypoxaemia from true shunt corrects poorly.

\subsubsection{Three-compartment model}
Dead space compartment (\(\dot V_A/\dot Q \to \infty\)), normal exchange compartment, and shunt compartment (\(\dot V_A/\dot Q = 0\)); helpful for viva-style explanations.

\subsubsection{Effects of anaesthesia on gas exchange (checklist)}
Hypoventilation; increased dead space; increased intrapulmonary shunting; increased scatter of \(\dot V_A/\dot Q\) ratios.

\subsubsection{Closing capacity, FRC, atelectasis risk and PEEP}
If closing capacity approaches/exceeds FRC, dependent airway closure during tidal breathing promotes atelectasis, low V/Q and shunt; PEEP may help by maintaining lung volume and reducing airway closure. Obesity is a common applied example (reduced FRC/compliance, increased V/Q mismatch).

\subsubsection{Oxygen dissociation curve (ODC) framing}
ODC/P50 links PaO$_2$ to SaO$_2$ and oxygen content; classic modifiers include CO$_2$/H$^+$ (Bohr), temperature, 2,3-DPG, and dyshemoglobins (CO, methaemoglobin) that reduce effective oxygen carriage.

\subsubsection{Practical dead space points}
Some texts emphasise the Bohr derivation, the practical assumption PACO$_2 \approx\) PaCO$_2$, and that anatomical dead space varies with head/neck position, intubation, and flow pattern.

\subsubsection{West zones / Starling resistor}
West zones and Starling resistor concepts are used to explain regional perfusion and how this may vary with pulmonary arterial/alveolar pressures; reminder linkage to altitude physiology is often used as an integrative example.

\subsubsection{High altitude as a gas-exchange ``stress test''}
Lower inspired PO$_2$ drives acute hyperventilation; chronic responses include increased 2,3-DPG and polycythaemia; hypoxic pulmonary vasoconstriction raises pulmonary arterial pressures and may strain the RV over time.

\subsection*{Sources used for the additions}
\begin{itemize}
  \item Kam \& Power, \emph{Principles of Physiology for the Anaesthetist}, Ch 17 (Gas exchange in the lungs).
  \item Morgan \& Mikhail, \emph{Clinical Anesthesiology} (5th ed.), Ch 23 (Respiratory physiology \& anesthesia): shunt vs V/Q response to FiO$_2$, three-compartment model, effects of anesthesia.
  \item \emph{Fundamentals of Anaesthesia} (4th ed.), Ch 17: Fowler/Bohr details, anatomical dead space variability, West zones/Starling resistor, altitude physiology.
  \item \emph{Primary FRCA in a Box} (2nd ed.): CC vs FRC/PEEP and obesity respiratory physiology; ODC/P50 and dyshemoglobins.
  \item \emph{1,000 Practice MTF} (Physiology answers): high-altitude responses (HPV and chronic RV effects).
\end{itemize}

% =========================================================

\section{Chapter 18 -- Carriage of Oxygen in Blood}

\subsection{Haemoglobin: the carrier}
\begin{itemize}
  \item Erythrocyte Hb content: \(\sim 200\text{--}300\) million Hb molecules per red cell.
  \item Hb structure (conceptual): 4 subunits, each with a haem (protoporphyrin ring + central iron in ferrous state, Fe\(^{2+}\)) \(\rightarrow\) binds up to 4 O\(_2\) molecules per Hb molecule.
\end{itemize}

\paragraph{Key point}
Only \emph{dissolved} O\(_2\) contributes to blood \(P\mathrm{O}_2\). Hb-bound O\(_2\) increases O\(_2\) \emph{content} substantially but does not directly raise \(P\mathrm{O}_2\).

\subsection{Oxygen in blood: dissolved vs haemoglobin-bound}

\subsubsection{Dissolved oxygen}
\begin{itemize}
  \item Dissolved O\(_2\) is directly proportional to \(P\mathrm{O}_2\).
  \item Solubility coefficient (37\(^\circ\)C): \(0.003~\mathrm{mL~O_2}/100~\mathrm{mL~blood/mmHg}\).
  \item At \(P\mathrm{a}O_2 \sim 100~\mathrm{mmHg}\) (\(13.3~\mathrm{kPa}\)): dissolved O\(_2 \approx 0.3~\mathrm{mL~O_2}/100~\mathrm{mL~blood}\).
\end{itemize}

\subsubsection{Oxygen carriage by haemoglobin}
\begin{itemize}
  \item At normal atmospheric pressure, about 98\% of blood O\(_2\) is carried by Hb.
  \item O\(_2\) combining capacity (fully saturated Hb):
  \begin{itemize}
    \item Adult blood: \(1.306~\mathrm{mL~O_2/g~Hb}\)
    \item Fetal blood: \(1.312~\mathrm{mL~O_2/g~Hb}\)
  \end{itemize}
  \item Increasing \(P\mathrm{O}_2\) increases Hb saturation (SaO\(_2\)) in a sigmoid pattern due to sequential binding across 4 subunits.
\end{itemize}

\paragraph{Definition}
Oxygen capacity of blood: the maximum amount of O\(_2\) that can be carried by Hb.

\subsection{Oxygen--haemoglobin dissociation curve (ODC)}

\subsubsection{Shape and physiological meaning}
\begin{itemize}
  \item Plot: \(P\mathrm{O}_2\) vs \% Hb saturation (and optionally O\(_2\) content).
  \item Steep portion: \(P\mathrm{O}_2 \sim 10\text{--}60~\mathrm{mmHg}\) (\(1.3\text{--}8~\mathrm{kPa}\)) \(\rightarrow\) small \(P\mathrm{O}_2\) changes cause large saturation changes.
  \item At \(P\mathrm{O}_2 \sim 60~\mathrm{mmHg}\) (\(8~\mathrm{kPa}\)), saturation is \(\sim 90\%\).
  \item Plateau portion: provides buffering such that modest falls in alveolar/arterial \(P\mathrm{O}_2\) produce only modest falls in SaO\(_2\).
\end{itemize}

\subsubsection{Typical arterial and mixed venous points}
\begin{itemize}
  \item Arterial (normal): \(P\mathrm{O}_2 \sim 100~\mathrm{mmHg}\) (\(13.3~\mathrm{kPa}\)); SaO\(_2 \sim 100\%\); O\(_2\) content \(\sim 20~\mathrm{mL}/100~\mathrm{mL}\) (Hb-bound).
  \item Normal tissue extraction: \(\sim 5~\mathrm{mL~O_2}/100~\mathrm{mL~blood}\) removed during capillary transit.
  \item Mixed venous (normal): \(P\mathrm{O}_2 \sim 40~\mathrm{mmHg}\) (\(5.3~\mathrm{kPa}\)); SvO\(_2 \sim 75\%\); O\(_2\) content \(\sim 15~\mathrm{mL}/100~\mathrm{mL}\) (Hb-bound).
\end{itemize}

\subsubsection{\texorpdfstring{\(P_{50}\)}{P50}}
\begin{itemize}
  \item \(P_{50}\): \(P\mathrm{O}_2\) at which Hb is 50\% saturated.
  \item Normal: \(\sim 26~\mathrm{mmHg}\) (\(\approx 3.5~\mathrm{kPa}\)).
  \item Use: compact descriptor of left/right shift of the ODC.
\end{itemize}

\subsection{Shifts of the ODC and the Bohr effect}

\subsubsection{Factors shifting the curve}
\begin{itemize}
  \item Right shift (decreased affinity, increased unloading): increased \(P\mathrm{CO}_2\), increased \([\mathrm{H}^+]\), increased temperature, increased 2,3-DPG.
  \item Left shift (increased affinity, reduced unloading): decreased \(P\mathrm{CO}_2\), decreased \([\mathrm{H}^+]\), decreased temperature, decreased 2,3-DPG, HbF.
\end{itemize}

\subsubsection{Bohr effect (mechanism and purpose)}
\begin{itemize}
  \item When \(P\mathrm{CO}_2\), \([\mathrm{H}^+]\), and temperature rise (e.g.\ working muscle), Hb affinity for O\(_2\) falls \(\rightarrow\) right shift \(\rightarrow\) O\(_2\) released more easily.
  \item In pulmonary capillaries, \(P\mathrm{CO}_2\) and \([\mathrm{H}^+]\) fall \(\rightarrow\) affinity rises \(\rightarrow\) facilitates O\(_2\) uptake.
  \item Mechanistic notes: \(\mathrm{H}^+\) binds to \(\alpha\)-amino and imidazole groups on Hb; CO\(_2\) binds to N-terminal amino groups \(\rightarrow\) reduced O\(_2\) affinity.
\end{itemize}

\subsubsection{2,3-DPG (2,3-diphosphoglycerate)}
\begin{itemize}
  \item Binds to \(\beta\) chains of one tetramer of deoxyHb \(\rightarrow\) conformational change \(\rightarrow\) reduced O\(_2\) affinity \(\rightarrow\) right shift.
  \item Produced via a side shunt from glycolysis; present in large amounts in erythrocytes.
  \item Increases with anaemia and high altitude exposure.
\end{itemize}

\subsubsection{Venous point concept}
\begin{itemize}
  \item If \(5~\mathrm{mL~O_2}/100~\mathrm{mL}\) is removed from arterial blood, the resulting \(P\mathrm{O}_2\) (normally \(\sim 40~\mathrm{mmHg}\)) represents the driving gradient for O\(_2\) movement into tissues.
  \item Right shift increases the venous-point \(P\mathrm{O}_2\); left shift reduces it.
\end{itemize}

\subsection{Oxygen content, delivery and consumption}

\subsubsection{Oxygen delivery (\(D\!O_2\)) / oxygen flux}
\begin{itemize}
  \item Oxygen delivery per minute depends on arterial oxygen content (\(C\mathrm{a}O_2\)) and cardiac output (CO).
  \item Typical resting values: \(C\mathrm{a}O_2 \approx 20~\mathrm{mL~O_2}/100~\mathrm{mL}\), CO \(\approx 5~\mathrm{L/min}\) \(\Rightarrow D\!O_2 \approx 1000~\mathrm{mL~O_2/min}\).
  \item Tissue oxygen usage (\(\dot V\!O_2\)): \(\sim 250~\mathrm{mL/min}\).
\end{itemize}

\subsubsection{Expanded delivery equation (including dissolved O\(_2\))}
\[
D\!O_2 =
\Bigl[\bigl(\mathrm{Hb}\times 1.306\times \mathrm{SaO_2}/100\bigr) + \bigl(0.003\times P\mathrm{a}O_2\bigr)\Bigr]\times \mathrm{CO}\times 10
\]
\noindent where Hb is in \(\mathrm{g}/100~\mathrm{mL}\), \(P\mathrm{a}O_2\) in \(\mathrm{mmHg}\), CO in \(\mathrm{L/min}\), and \(C\mathrm{a}O_2\) in \(\mathrm{mL~O_2}/100~\mathrm{mL}\). The dissolved O\(_2\) term is small but included for completeness.

\subsubsection{Determinants of \(C\mathrm{a}O_2\) and \(\dot V\!O_2\)}
\begin{itemize}
  \item \(C\mathrm{a}O_2\): influenced by \(F\mathrm{i}O_2\), alveolar ventilation, \(V/Q\) mismatch, Hb concentration, and ODC position (left/right shift).
  \item \(\dot V\!O_2\): influenced by metabolic rate; increases with exercise/surgical stress and decreases with sleep/anaesthesia.
\end{itemize}

\subsubsection{Supply dependence}
As cardiac output decreases, \(\dot V\!O_2\) can become supply dependent and may fall with CO.

\subsection{Mixed venous oxygen content/saturation (SvO\(_2\))}

\subsubsection{Determinants}
Mixed venous oxygen content (\(C\mathrm{v}O_2\)) is determined by \(C\mathrm{a}O_2\), \(\dot V\!O_2\), and flow (CO).

\subsubsection{Estimating SvO\(_2\)}
If \(C\mathrm{v}O_2\) is known, SvO\(_2\) can be derived using Hb concentration and the ODC (the text notes Hb O\(_2\) binding capacity values around \(\sim 1.31\text{--}1.34~\mathrm{mL/g}\)).

\subsubsection{Factors affecting SvO\(_2\)}
\paragraph{High SvO\(_2\)}
\begin{itemize}
  \item Increased O\(_2\) delivery: increased inspired O\(_2\) concentration
  \item Decreased O\(_2\) demand: hypothermia
  \item Other: histotoxic hypoxia (e.g.\ cyanide poisoning); sepsis (altered regional blood flow/distribution)
\end{itemize}

\paragraph{Low SvO\(_2\)}
\begin{itemize}
  \item Decreased O\(_2\) delivery: decreased Hb concentration; decreased arterial O\(_2\) saturation; decreased CO (e.g.\ hypovolaemia)
  \item Increased O\(_2\) demand: pain, shivering, hyperthermia, seizures
\end{itemize}

\subsection{Exam-focused pitfalls}
\begin{itemize}
  \item \(P\mathrm{O}_2\) is set by dissolved O\(_2\), not Hb-bound O\(_2\).
  \item Plateau vs steep portion: plateau buffers SaO\(_2\); steep portion supports efficient unloading.
  \item Right shift: improved unloading but reduced affinity; left shift: improved loading but impaired unloading.
  \item \(P_{50}\) as a single-number descriptor: increased \(P_{50}\) = right shift; decreased \(P_{50}\) = left shift.
  \item SvO\(_2\) reflects delivery, demand, and distribution (e.g.\ sepsis), not just ``oxygenation''.
\end{itemize}

\paragraph{Source}
\textit{Kam \& Power — Principles of Physiology for the Anaesthetist, Chapter 18 (Carriage of oxygen in blood).}

\section{Additional high-yield points emphasised in other core texts (not explicit in Kam \& Power Ch.\ 18)}

\subsection{Mechanistic basis of cooperativity (T \(\leftrightarrow\) R transition; Adair concept)}
\begin{itemize}
  \item Hb exists in tense (T) and relaxed (R) conformations; O\(_2\) binding promotes the R state, increasing affinity at remaining sites (positive cooperativity).
  \item Some texts explicitly name the Adair equation as the descriptive model for stepwise O\(_2\) binding.
\end{itemize}
\textit{Sources: Principles of Physiology for the Anaesthetist (2021), Hb structure/function and cooperativity; Ganong’s Review of Medical Physiology, ODC and T--R interconversion; Cross \& Plunkett (2008), Oxygen delivery and transport.}

\subsection{Dyshemoglobins and abnormal Hb: effects on O\(_2\) carriage and the ODC}
\begin{itemize}
  \item Carbon monoxide (CO): reduces O\(_2\)-binding capacity (high-affinity Hb binding); classically alters the ODC (often taught as a left shift of remaining sites).
  \item Methaemoglobin (Fe\(^{3+}\)): cannot bind O\(_2\); reduces total O\(_2\)-carrying capacity; some texts state left shift and impaired O\(_2\) release.
  \item Haemoglobin variants (e.g.\ HbS): different O\(_2\)-saturation characteristics; clinical consequences often considered in low \(P\mathrm{O}_2\)/low pH states.
\end{itemize}
\textit{Sources: Primary FRCA in a Box (2019), respiration/ODC notes and Hb variants; Morgan \& Mikhail (5th ed.), respiratory physiology/monitoring; Cross \& Plunkett (2008), oxygen delivery and transport.}

\subsection{Hüfner constant: why different numbers appear across books}
\begin{itemize}
  \item Distinction between a theoretical maximum O\(_2\) binding per gram Hb (often \(\sim 1.39~\mathrm{mL~O_2/g~Hb}\)) and a lower practical/clinical constant (often \(\sim 1.31~\mathrm{mL~O_2/g~Hb}\)).
\end{itemize}
\textit{Sources: Morgan \& Mikhail (5th ed.), respiratory physiology; Cross \& Plunkett (2008), oxygen delivery/transport.}

\subsection{Anaemia and polycythaemia: capacity vs saturation vs \(P\mathrm{O}_2\)}
\begin{itemize}
  \item Anaemia: decreases O\(_2\)-carrying capacity/content without independently altering \(P_{50}\).
  \item Polycythaemia: increases O\(_2\)-carrying capacity/content without altering \(P_{50}\).
\end{itemize}
\textit{Source: Primary FRCA in a Box (2019), respiration/ODC notes.}

\subsection{O\(_2\) extraction ratio and the Fick principle}
\begin{itemize}
  \item Fick relationship: \(\dot V\!O_2 = \mathrm{CO}\times (C\mathrm{a}O_2 - C\mathrm{v}O_2)\).
  \item Normal extraction fraction often expressed as \(\sim 25\%\) at rest: \((C\mathrm{a}O_2 - C\mathrm{v}O_2)/C\mathrm{a}O_2\).
\end{itemize}
\textit{Sources: Morgan \& Mikhail (5th ed.), respiratory physiology; Principles of Physiology for the Anaesthetist (2021), cardiopulmonary physiology; 1,000 Practice MTF MCQs for the Primary and Final FRCA (2019), physiology answers (Fick/CO calculations).}

\subsection{Critical \(D\!O_2\) framing of supply dependence}
\begin{itemize}
  \item Some exam texts define a critical oxygen delivery below which \(\dot V\!O_2\) becomes supply dependent, giving an approximate threshold around \(300~\mathrm{mL/min}\).
\end{itemize}
\textit{Source: Cross \& Plunkett (2008), oxygen delivery and transport.}

\subsection{Oxygen stores and preoxygenation (apnoea time concept)}
\begin{itemize}
  \item Emphasis on body O\(_2\) stores, especially O\(_2\) reservoir in FRC; increasing \(F\mathrm{i}O_2\) enlarges this store and delays hypoxaemia during apnoea.
\end{itemize}
\textit{Sources: Morgan \& Mikhail (5th ed.), respiratory physiology (FRC O\(_2\) store/preoxygenation); Primary FRCA in a Box (2019), oxygen stores/transport overview.}

\subsection{Monitoring linkage: pulse oximetry limitations in abnormal Hb states}
\begin{itemize}
  \item COHb: two-wavelength pulse oximeters may display falsely high SpO\(_2\).
  \item MetHb: displayed saturation tends toward \(\sim 85\%\) (can appear falsely low or high depending on true saturation).
  \item Averaging/response time and probe/site effects can introduce delays/artefacts.
\end{itemize}
\textit{Sources: Morgan \& Mikhail (5th ed.), monitoring; Al-Shaikh \& Stacey (2023), pulse oximetry limitations/averaging; Primary FRCA in a Box (2019), monitoring artefacts and dyshemoglobins.}

\subsection{CO\(_2\) carriage interaction: the Haldane effect}
\begin{itemize}
  \item Deoxygenated Hb carries more CO\(_2\) than oxygenated Hb (Haldane effect), linking O\(_2\) unloading to CO\(_2\) uptake in tissues and the reverse in lungs.
\end{itemize}
\textit{Sources: Morgan \& Mikhail (5th ed.), CO\(_2\) transport; Primary FRCA in a Box (2019), CO\(_2\) transport; Fundamentals of Anaesthesia (4th ed.), pregnancy physiology (Haldane/double Haldane effect).}

% =========================================================
% Fragment mode (for \input{} / \include{}). No preamble.

\section{Chapter 19 -- Carbon Dioxide Carriage in Blood}
\textit{Primary source: Kam \& Power, Chapter 19 (Carbon Dioxide Carriage in Blood).}

\subsection{Big picture}
\begin{itemize}
  \item \(\mathrm{CO_2}\) is carried in blood in three forms:
  \begin{enumerate}
    \item Dissolved in physical solution
    \item As bicarbonate (\(\mathrm{HCO_3^-}\)) (via carbonic acid)
    \item As carbamino compounds (bound to proteins, mainly haemoglobin)
  \end{enumerate}
  \item Venous blood contains more total \(\mathrm{CO_2}\) than arterial blood (systemic uptake from tissues).
  \item Lower \(\mathrm{O_2}\) content increases \(\mathrm{CO_2}\) carriage capacity (Haldane effect).
\end{itemize}

\subsection{Quantitative split (arterial vs transfer across tissues)}
\subsubsection{Arterial blood (typical proportions)}
\begin{itemize}
  \item \(\sim 90\%\) as bicarbonate
  \item \(\sim 5\%\) dissolved
  \item \(\sim 5\%\) as carbamino compounds
\end{itemize}

\subsubsection{Of the \(\mathrm{CO_2}\) transferred from tissues and eliminated in the lung}
\begin{itemize}
  \item \(\sim 60\%\) transferred as bicarbonate
  \item \(\sim 30\%\) transferred as carbamino compounds
  \item \(\sim 10\%\) transferred as dissolved \(\mathrm{CO_2}\)
\end{itemize}

\subsection{The three forms in more detail}
\subsubsection{Dissolved \(\mathrm{CO_2}\)}
\begin{itemize}
  \item \(\mathrm{CO_2}\) is more soluble than \(\mathrm{O_2}\).
  \item Dissolved \(\mathrm{CO_2}\) accounts for \(\sim 10\%\) of the \(\mathrm{CO_2}\) evolved in the lungs.
  \item Solubility in plasma at \(37^\circ\mathrm{C}\): \(0.231~\mathrm{mmol\,L^{-1}\,kPa^{-1}}\) (or \(0.0308~\mathrm{mmol\,L^{-1}\,mmHg^{-1}}\)).
\end{itemize}

\subsubsection{Bicarbonate / carbonic acid}
\begin{itemize}
  \item Conceptual sequence:
  \[
    \mathrm{CO_2 + H_2O \rightleftharpoons H_2CO_3 \rightleftharpoons HCO_3^- + H^+}
  \]
  \item Hydration is slow but catalysed by carbonic anhydrase.
  \item Carbonic anhydrase is present in erythrocytes and endothelium; not present in plasma.
\end{itemize}

\subsubsection{Carbamino compounds}
\begin{itemize}
  \item \(\mathrm{CO_2}\) combines with amino groups in proteins to form carbamates.
  \item In blood this is mainly with haemoglobin.
\end{itemize}

\subsection{Haldane effect}
\subsubsection{Definition}
Deoxygenated haemoglobin carries more \(\mathrm{CO_2}\) than oxygenated haemoglobin.

\subsubsection{Mechanisms described}
\begin{itemize}
  \item Oxygenation of haemoglobin reduces its capacity to carry \(\mathrm{CO_2}\) as carbamino compounds (via changes in ionization of nitrogen groups).
  \item Deoxyhaemoglobin is more basic and buffers \(\mathrm{H^+}\) produced during bicarbonate formation.
\end{itemize}

\subsubsection{Consequences}
\begin{itemize}
  \item Venous (deoxygenated) blood takes up and transports more \(\mathrm{CO_2}\) than oxygenated arterial blood.
  \item Carbamino compounds account for \(\sim 1/3\) of the arteriovenous difference in \(\mathrm{CO_2}\) carried.
\end{itemize}

\subsection{\(\mathrm{CO_2}\) dissociation curve (content vs \(P_{\mathrm{CO_2}}\))}
\begin{itemize}
  \item Relationship between \(\mathrm{CO_2}\) content and \(P_{\mathrm{CO_2}}\) is more linear than the oxygen dissociation curve.
  \item Different curves occur at different oxygen saturations (Haldane effect).
\end{itemize}
\paragraph{Physiological reference points}
\begin{itemize}
  \item Arterial: \(P_{\mathrm{CO_2}} \approx 40~\mathrm{mmHg}~(5.3~\mathrm{kPa})\), \(\mathrm{O_2}\) saturation \(\approx 100\%\).
  \item Mixed venous: \(P_{\mathrm{CO_2}} \approx 46~\mathrm{mmHg}~(6.1~\mathrm{kPa})\), \(\mathrm{O_2}\) saturation \(\approx 75\%\).
\end{itemize}

\subsection{Systemic capillaries: \(\mathrm{CO_2}\) transfer into blood}
\subsubsection{Driving gradient}
\(\mathrm{CO_2}\) is produced in mitochondria and diffuses cell \(\rightarrow\) interstitial fluid \(\rightarrow\) capillary \(\rightarrow\) plasma \(\rightarrow\) erythrocyte.

\subsubsection{What happens to \(\mathrm{CO_2}\) in blood}
\begin{itemize}
  \item Some dissolves in plasma and erythrocyte.
  \item Most becomes carbamino compounds and/or bicarbonate (carbonic anhydrase-dependent within RBC).
\end{itemize}

\subsubsection{Coupling to \(\mathrm{O_2}\) unloading}
\begin{itemize}
  \item As \(\mathrm{O_2}\) leaves the RBC, haemoglobin becomes deoxygenated (\(\mathrm{HHb}\)).
  \item \(\mathrm{HHb}\) buffers \(\mathrm{H^+}\) and promotes bicarbonate and carbaminohaemoglobin formation.
\end{itemize}

\subsubsection{Chloride shift (Hamburger effect)}
\begin{itemize}
  \item \(\mathrm{HCO_3^-}\) diffuses out of RBC into plasma; \(\mathrm{H^+}\) is buffered inside RBC.
  \item To maintain electroneutrality, \(\mathrm{Cl^-}\) diffuses into the RBC; water also enters.
\end{itemize}

\subsection{Pulmonary capillaries: \(\mathrm{CO_2}\) transfer out of blood}
\subsubsection{Driving gradient}
Venous \(P_{\mathrm{CO_2}} \approx 46~\mathrm{mmHg}~(6.1~\mathrm{kPa})\) exceeds alveolar \(P_{\mathrm{CO_2}} \approx 40~\mathrm{mmHg}~(5.3~\mathrm{kPa})\); \(\mathrm{CO_2}\) diffuses from blood to alveolus.

\subsubsection{Coupling to \(\mathrm{O_2}\) uptake}
In pulmonary capillaries, \(\mathrm{O_2}\) diffuses into RBC and oxygenation facilitates \(\mathrm{CO_2}\) unloading (Haldane effect).

\subsubsection{Determinant of alveolar \(P_{\mathrm{CO_2}}\)}
Alveolar \(P_{\mathrm{CO_2}}\) depends on balance between \(\mathrm{CO_2}\) delivery/output and alveolar ventilation. Fractional alveolar \(\mathrm{CO_2}\) concentration is described as:
\[
  \frac{\text{rate of } \mathrm{CO_2}\ \text{output (normally }200~\mathrm{mL\,min^{-1}}\text{)}}{\text{minute alveolar ventilation}}
\]

\subsubsection{End-capillary equilibration}
End pulmonary capillary blood \(P_{\mathrm{CO_2}}\) is very close to alveolar gas \(P_{\mathrm{CO_2}}\).

\subsection{FRCA/EDAIC traps (aligned to chapter)}
\begin{itemize}
  \item Haldane (effect of \(\mathrm{O_2}\) on \(\mathrm{CO_2}\) carriage) vs Bohr (effect of \(\mathrm{CO_2/H^+}\) on \(\mathrm{O_2}\) affinity).
  \item \(\mathrm{CO_2}\) content--\(P_{\mathrm{CO_2}}\) is more linear than the \(\mathrm{O_2}\) dissociation curve.
  \item Carbonic anhydrase location: RBCs and endothelium, not plasma.
  \item Chloride shift direction in systemic capillaries: \(\mathrm{HCO_3^-}\) out, \(\mathrm{Cl^-}\) in (water follows).
\end{itemize}

\section{Additions / extensions from other core texts (not explicit in Kam \& Power Ch 19)}
\subsection{Chloride shift: transporter, speed, and osmotic consequence}
\begin{itemize}
  \item Mechanism via RBC anion exchanger (AE1 / Band 3): \(\mathrm{HCO_3^-}\) leaves RBC in exchange for \(\mathrm{Cl^-}\).
  \item Time course: essentially complete within \(\sim 1~\mathrm{s}\).
  \item Osmotic effect: \(\mathrm{CO_2}\) addition increases osmotically active particles; water enters RBC \(\rightarrow\) swelling.
  \item Venous haematocrit described as \(\sim 3\%\) greater than arterial (RBC swelling plus small fluid handling effects).
\end{itemize}
\textit{Source: Ganong’s Review of Medical Physiology, section/chapter on transport of \(\mathrm{O_2}\) and \(\mathrm{CO_2}\) in blood.}

\subsection{Haemoglobin buffering: what groups buffer \(\mathrm{H^+}\)}
\begin{itemize}
  \item \(\mathrm{H^+}\) generated during \(\mathrm{CO_2}\) hydration is buffered mainly by haemoglobin, especially imidazole groups on histidine residues.
\end{itemize}
\textit{Source: Primary FRCA in a Box, Respiration: carbon dioxide stores/transport section.}

\subsection{A--V \(\mathrm{CO_2}\) differences and respiratory quotient linkage}
\begin{itemize}
  \item A--V \(P_{\mathrm{CO_2}}\) difference: \(\sim 0.7~\mathrm{kPa}\) (about \(6~\mathrm{mmHg}\)).
  \item Linked statement: corresponds to \(\sim 4~\mathrm{mL~CO_2}\) per \(100~\mathrm{mL}\) blood (noted to depend on respiratory quotient, \(RQ\)).
  \item \(RQ = \frac{\mathrm{CO_2\ produced}}{\mathrm{O_2\ consumed}}\) (steady state).
\end{itemize}
\textit{Source: Primary FRCA in a Box, Respiration: carbon dioxide stores/transport section.}

\subsection{Plasma vs whole-blood framing and plasma hydration point}
\begin{itemize}
  \item Distinguish reported \(\mathrm{CO_2}\) content in plasma vs whole blood.
  \item In plasma, less than \(1\%\) of dissolved \(\mathrm{CO_2}\) undergoes hydration (contrast with rapid RBC/endothelial catalysis).
\end{itemize}
\textit{Source: Morgan \& Mikhail’s Clinical Anesthesiology, respiratory physiology section on \(\mathrm{CO_2}\) transport and tables.}

\subsection{Lung-side linkage equation (oxygenation driving \(\mathrm{CO_2}\) unloading)}
\[
  \mathrm{O_2 + HCO_3^- + HbH^+ \rightarrow H_2O + CO_2 + HbO_2}
\]
\textit{Source: Morgan \& Mikhail’s Clinical Anesthesiology, respiratory physiology section on haemoglobin buffering and \(\mathrm{CO_2}\) transport.}

\subsection{Carbonic anhydrase inhibition (pharmacological hook)}
\begin{itemize}
  \item Acetazolamide (carbonic anhydrase inhibitor) described as impairing \(\mathrm{CO_2}\) transport between tissues and alveoli.
\end{itemize}
\textit{Source: Morgan \& Mikhail’s Clinical Anesthesiology, respiratory physiology section (bicarbonate/\(\mathrm{CO_2}\) transport discussion).}

\subsection{CO\(_2\) dissociation curves: diagram technique}
\begin{itemize}
  \item Include a dissolved \(\mathrm{CO_2}\) line through the origin.
  \item Deoxygenated curve lies above oxygenated curve (Haldane effect).
  \item Vertical gap between dissolved line and total \(\mathrm{CO_2}\) curve represents bicarbonate carriage.
\end{itemize}
\textit{Source: Cross \& Plunkett, Physics, Pharmacology and Physiology for Anaesthetists: section on carriage of \(\mathrm{CO_2}\) and dissociation curves.}

\subsection{Pregnancy/placenta: double Haldane effect}
\begin{itemize}
  \item Placental \(\mathrm{CO_2}\) transfer enhanced by maternal deoxygenation (increased maternal \(\mathrm{CO_2}\) uptake) and fetal oxygenation (increased fetal \(\mathrm{CO_2}\) release): ``double Haldane effect''.
  \item Quantitative statement given: may account for \(\sim 46\%\) of transplacental \(\mathrm{CO_2}\) transfer.
\end{itemize}
\textit{Source: Fundamentals of Anaesthesia, physiology of pregnancy section on placental gas transfer/\(\mathrm{CO_2}\).}

\subsection{Placental \(\mathrm{CO_2}\): diffusibility and forms}
\begin{itemize}
  \item Placenta described as highly permeable to \(\mathrm{CO_2}\); \(\mathrm{CO_2}\) stated to be \(\sim 20\times\) more diffusible than \(\mathrm{O_2}\).
  \item Approximate forms in fetal/maternal blood: dissolved \(\sim 8\%\), bicarbonate \(\sim 62\%\), carbaminohaemoglobin \(\sim 30\%\), with very small amounts as carbonic acid/carbonate.
  \item Dissolved \(\mathrm{CO_2}\) is the form that crosses; bicarbonate acts as a reservoir via equilibrium.
\end{itemize}
\textit{Source: Fundamentals of Anaesthesia, physiology of pregnancy section on \(\mathrm{CO_2}\) and placental transfer.}

\subsection{CO\(_2\) measurement: electrode (blood gas analyser)}
\begin{itemize}
  \item Severinghaus-type electrode: \(\mathrm{CO_2}\) diffuses across a CO\(_2\)-permeable membrane into a bicarbonate film; pH change reflects \(\mathrm{CO_2}\).
  \item Quantitative statement: \(\sim 0.01\) pH units per \(0.1~\mathrm{kPa}\) change in \(\mathrm{CO_2}\).
  \item Response time limited by diffusion; one source states \(2\)--\(3\) minutes.
\end{itemize}
\textit{Sources: Fundamentals of Anaesthesia (CO\(_2\) measurement section); Essentials of Equipment in Anaesthesia, Critical Care and Perioperative Medicine (blood gas analyser electrodes/response time).}

\subsection{End-tidal CO\(_2\) (capnography): principle and waveform cues}
\begin{itemize}
  \item Infrared absorption principle: \(\mathrm{CO_2}\) absorbs IR (noted around \(4.3~\mu\mathrm{m}\)); absorption proportional to \(\mathrm{CO_2}\) partial pressure.
  \item Rebreathing: baseline fails to return to zero during inspiration.
  \item Obstructive disease: sloping alveolar plateau compared with normal.
\end{itemize}
\textit{Source: Essentials of Equipment in Anaesthesia, Critical Care and Perioperative Medicine (CO\(_2\) analysers/capnography section).}

\subsection{Central chemoreceptors: why \(\mathrm{CO_2}\) changes CSF pH}
\begin{itemize}
  \item Central chemoreceptors respond to CSF pH.
  \item Plasma \(\mathrm{H^+}\) does not cross the blood--brain barrier, whereas \(\mathrm{CO_2}\) diffuses readily and alters CSF pH via hydration.
\end{itemize}
\textit{Source: 1,000 Practice MTF (physiology: control of ventilation/chemoreceptors content).}

\subsection{Additional quantitative statements on carbamino binding / Haldane}
\begin{itemize}
  \item Deoxyhaemoglobin described as having \(\sim 3.5\times\) greater affinity for \(\mathrm{CO_2}\) than oxyhaemoglobin.
  \item Within physiological ranges, \(P_{\mathrm{CO_2}}\) stated to have little effect on the fraction of \(\mathrm{CO_2}\) carried as carbaminohaemoglobin.
  \item Plasma tables: carbamino \(\mathrm{CO_2}\) in plasma described as negligible (carbamino carriage is primarily haemoglobin-based).
\end{itemize}
\textit{Source: Morgan \& Mikhail’s Clinical Anesthesiology, respiratory physiology section on \(\mathrm{CO_2}\) transport (text + tables).}


\end{document}
