\documentclass[11pt,a4paper]{article}

% --- pdfLaTeX / TeX Live 2024 safe preamble ---
\usepackage[utf8]{inputenc}
\usepackage[T1]{fontenc}
\usepackage{lmodern}
\usepackage[a4paper,margin=2.2cm]{geometry}
\usepackage{microtype}
\usepackage{amsmath,amssymb}
\usepackage{booktabs}
\usepackage{tabularx}
\usepackage{enumitem}
\usepackage{hyperref}
\hypersetup{hidelinks}

\title{Respiratory System Study Notes\\Chapters 15--16 (Structured rewrite)}
\author{}
\date{\today}

\begin{document}
\maketitle

\noindent\textit{Source: Kam \& Power --- Principles of Physiology for the Anaesthetist, Chapters 15.}

\tableofcontents
\newpage

% =========================================================
\section{Chapter 15 --- Functions of the Respiratory System}

\subsection{Big picture}

\subsubsection{Primary function}
\begin{itemize}[leftmargin=*,nosep]
  \item \textbf{Gas exchange}: transfer of O$_2$ into blood and CO$_2$ out of blood across the alveolar--capillary interface.
\end{itemize}

\subsubsection{Non-respiratory functions of the lung}
\begin{itemize}[leftmargin=*,nosep]
  \item \textbf{Blood filter}: traps small clots and detached cells, limiting systemic embolization.
  \item \textbf{Blood reservoir (high capacitance)}
  \begin{itemize}[leftmargin=*,nosep]
    \item \textasciitilde16\% of total blood volume in the \textbf{supine} position.
    \item \textasciitilde9\% of total blood volume in the \textbf{erect} position.
    \item Can redistribute blood to vital organs in \textbf{hypovolaemic shock}.
  \end{itemize}
  \item \textbf{Metabolism / handling of bioactive substances}
  \begin{itemize}[leftmargin=*,nosep]
    \item Converts \textbf{angiotensin I $\rightarrow$ angiotensin II}.
    \item Synthesis and breakdown of \textbf{bradykinin}.
    \item Storage/release of \textbf{serotonin} and \textbf{histamine}.
    \item Synthesis of peptides (e.g.\ \textbf{substance P}), \textbf{prostaglandins}, \textbf{surfactants}, \textbf{immunoglobulins}.
    \item Inactivation of \textbf{adrenaline} and \textbf{noradrenaline}.
    \item Presence of \textbf{cytochrome P-450 isoenzymes}.
  \end{itemize}
  \item \textbf{Acid--base regulation}: ventilation changes arterial $P_{\mathrm{CO_2}}$.
  \item \textbf{Phonation}: CNS control of respiratory muscles generates airflow through vocal cords.
  \item \textbf{Pulmonary defence}
  \begin{itemize}[leftmargin=*,nosep]
    \item Secretion of \textbf{IgA} (innate immunity).
    \item Removal of airborne particles by \textbf{phagocytosis} and \textbf{mucociliary action}.
    \item \textbf{Lymphoid tissue with T lymphocytes} provides first-line defence against the external environment.
  \end{itemize}
\end{itemize}

\paragraph{Exam pitfalls}
\begin{itemize}[leftmargin=*,nosep]
  \item ``Lungs only do gas exchange'' is false: know key \textbf{non-respiratory} roles (filter/reservoir/metabolic/defence).
\end{itemize}

\subsection{Functional anatomy: airway tree, respiratory zone, alveolar--capillary unit}

\subsubsection{Conducting airways vs respiratory zone}

\paragraph{Conducting airways}
\begin{itemize}[leftmargin=*,nosep]
  \item Function: \textbf{bulk flow} to/from respiratory zone + \textbf{warm, humidify, filter} inspired air.
  \item Anatomical extent: \textbf{trachea (generation 1)} $\rightarrow$ \textbf{terminal bronchioles (generation 16)}.
  \item Conducting zone volume \textasciitilde \textbf{150 mL}.
  \item \textbf{Cartilage disappears from the 11th generation}; beyond this, airway diameter is mainly determined by \textbf{lung volume}.
  \item Airway wall \textbf{smooth muscle}: dilates with \textbf{sympathetic} stimulation; constricts with \textbf{parasympathetic} stimulation.
\end{itemize}

\paragraph{Respiratory zone}
\begin{itemize}[leftmargin=*,nosep]
  \item Begins at \textbf{respiratory bronchioles (generations 17--19)} (first airways with alveoli in their walls).
  \item Continues through \textbf{alveolar ducts (generations 20--22)} to \textbf{alveolar sacs (generation 23)}.
  \item Respiratory zone volume \textasciitilde \textbf{3000 mL}.
  \item Gas exchange occurs by \textbf{diffusion} (not bulk flow).
\end{itemize}

\subsubsection{Pulmonary lobule and alveolar stability}
\begin{itemize}[leftmargin=*,nosep]
  \item Parenchyma is an interconnected network of alveolar walls and interstitial tissues.
  \item \textbf{Alveolar interdependence}: a collapsing region tends to be pulled open by surrounding stretched tissue.
  \item With \textbf{surfactant} and \textbf{collateral ventilation via pores of Kohn} $\rightarrow$ helps prevent alveolar collapse.
  \item \textbf{Pulmonary lobule}: airways and alveoli distal to a single terminal bronchiole.
\end{itemize}

\subsubsection{Alveolar--capillary unit: key numbers and structure}
\begin{itemize}[leftmargin=*,nosep]
  \item \textbf{Alveoli}
  \begin{itemize}[leftmargin=*,nosep]
    \item \textbf{200--600 million} (average \textasciitilde\textbf{300 million}).
    \item Mean diameter at \textbf{FRC}: \textbf{0.2 mm}.
    \item \textbf{Polyhedral} (not spherical) because septae are flat.
    \item \textbf{Type I} cells: flat squamous epithelium (major gas-exchanging surface).
    \item \textbf{Type II} cells: cuboidal, produce \textbf{surfactant}.
  \end{itemize}
  \item \textbf{Pulmonary capillaries}
  \begin{itemize}[leftmargin=*,nosep]
    \item Diameter \textasciitilde\textbf{10 $\mu$m}.
    \item Endothelial thickness \textasciitilde\textbf{0.1 $\mu$m}.
    \item RBC transit through capillary network (and 2--3 alveoli): \textasciitilde\textbf{0.75 s}.
  \end{itemize}
  \item \textbf{Diffusion barrier thickness} (alveolar gas $\rightarrow$ capillary blood): \textasciitilde\textbf{0.3 $\mu$m} (RBC diameter \textasciitilde\textbf{7 $\mu$m}).
  \item \textbf{Surface area}: \textasciitilde\textbf{50--100 m$^2$}.
\end{itemize}

\paragraph{Exam pitfalls}
\begin{itemize}[leftmargin=*,nosep]
  \item Conducting zone (bulk flow; \textasciitilde150 mL) vs respiratory zone (diffusion; \textasciitilde3000 mL).
  \item Type I (exchange) vs Type II (surfactant).
  \item Generation 11 (cartilage disappears) vs generation 16 (end of conducting zone).
\end{itemize}

\subsection{Muscles of ventilation}

\subsubsection{Mechanics overview}
\begin{itemize}[leftmargin=*,nosep]
  \item Thorax separated from abdomen by the \textbf{diaphragm}.
  \item Diaphragm contraction increases vertical chest dimension by pushing abdominal contents down.
  \item Ribs move laterally/anteriorly to increase thoracic cross-sectional area.
  \item \textbf{Inspiration}: active. \quad \textbf{Expiration}: passive during quiet breathing; becomes active when ventilation increases.
\end{itemize}

\subsubsection{Inspiratory muscles}
\begin{itemize}[leftmargin=*,nosep]
  \item \textbf{Diaphragm}
  \begin{itemize}[leftmargin=*,nosep]
    \item Innervation: \textbf{phrenic nerve (C3--5)}.
    \item Descent: \textbf{1--2 cm} in quiet breathing; up to \textbf{10 cm} in forced inspiration.
  \end{itemize}
  \item \textbf{External intercostals}: fibres slope down and anteriorly; ribs move \textbf{upwards and forwards}.
  \item \textbf{Scalenes}: active even during quiet breathing; elevate ribcage.
  \item \textbf{Sternocleidomastoid}: recruited when breathing increases; elevates ribcage.
\end{itemize}

\subsubsection{Expiratory muscles}
\begin{itemize}[leftmargin=*,nosep]
  \item Quiet expiration is \textbf{passive}.
  \item Active expiration occurs with increased ventilation (exercise, speech, coughing/sneezing) and in pathology (e.g.\ COPD).
  \item Main expiratory muscles: \textbf{abdominal wall} (rectus abdominis, internal/external obliques, transversalis) and \textbf{internal intercostals}.
\end{itemize}

\paragraph{Exam pitfalls}
\begin{itemize}[leftmargin=*,nosep]
  \item Quiet expiration is passive; \textbf{forced} expiration recruits abdominal wall + internal intercostals.
\end{itemize}

\subsection{Lung--chest wall equilibrium and pressure definitions}

\subsubsection{Pleural arrangement}
\begin{itemize}[leftmargin=*,nosep]
  \item Lungs covered by \textbf{visceral pleura}; chest wall lined by \textbf{parietal pleura}.
  \item Between them is a potential \textbf{intrapleural space}.
  \item Diaphragm separates lungs from abdominal contents.
\end{itemize}

\subsubsection{Resting equilibrium (end of normal expiration)}
\begin{itemize}[leftmargin=*,nosep]
  \item Elastic forces balance:
  \begin{itemize}[leftmargin=*,nosep]
    \item Lung tends to collapse inward.
    \item Chest wall tends to expand outward (plus contribution from diaphragmatic tone).
  \end{itemize}
  \item This produces a \textbf{negative intrapleural pressure}.
\end{itemize}

\subsubsection{Transpulmonary pressure}
\begin{itemize}[leftmargin=*,nosep]
  \item \textbf{Transpulmonary pressure} $=$ alveolar pressure $-$ intrapleural pressure.
  \item Distending (transmural) pressure across alveoli.
  \item At end of normal expiration, forces balance at \textbf{FRC}.
\end{itemize}

\paragraph{Exam pitfalls}
\begin{itemize}[leftmargin=*,nosep]
  \item Transpulmonary pressure is a \textbf{difference}, not the same as intrapleural pressure.
\end{itemize}

\subsection{Events during a normal tidal breath (Table 15.1)}

\subsubsection{Summary table (spontaneous breathing)}
\begin{table}[h!]
\centering
\begin{tabularx}{\textwidth}{@{}lXX@{}}
\toprule
\textbf{Step} & \textbf{Inspiration (active)} & \textbf{Expiration (passive, quiet breathing)}\\
\midrule
Neural drive & Inspiratory centre activated $\rightarrow$ impulses to inspiratory muscles & Inspiratory centre activity ceases\\
Muscle activity & Diaphragm contracts ($\pm$ external intercostals) & Inspiratory muscles relax\\
Thoracic volume & Increases & Decreases toward resting level\\
Intrapleural pressure & Becomes \textbf{more negative} & Becomes \textbf{less negative} (returns toward baseline)\\
Transpulmonary pressure & Increases $\rightarrow$ alveoli distend; elastic recoil increases & Decreases as elastic recoil empties lungs\\
Alveolar pressure (vs atmosphere) & Falls slightly below 0 $\rightarrow$ airflow \textbf{into} alveoli & Rises slightly above 0 $\rightarrow$ airflow \textbf{out}\\
End of phase & Flow stops when alveolar pressure returns to 0 & Flow stops when alveolar pressure returns to 0\\
\bottomrule
\end{tabularx}
\end{table}

\subsubsection{Key takeaways (MCQ-friendly)}
\begin{itemize}[leftmargin=*,nosep]
  \item Flow occurs only when alveolar pressure $\neq$ atmospheric pressure.
  \item End-inspiration and end-expiration: alveolar pressure $=$ atmospheric (no flow).
  \item The pressure that keeps alveoli open is \textbf{transpulmonary pressure} (alveolar $-$ intrapleural).
\end{itemize}

\subsection{Pressures, flow, and volume during the breathing cycle}

\subsubsection{At rest (end expiration)}
\begin{itemize}[leftmargin=*,nosep]
  \item Alveolar pressure = mouth pressure = 0 (relative to atmosphere).
  \item Intrapleural pressure \textasciitilde $-5$ cmH$_2$O.
  \item Lung volume = FRC.
  \item No flow.
\end{itemize}

\subsubsection{Inspiration}
\begin{itemize}[leftmargin=*,nosep]
  \item Inspiratory muscle activity expands chest wall $\rightarrow$ intrapleural pressure falls.
  \item Alveolar pressure falls to \textasciitilde $-1$ cmH$_2$O $\rightarrow$ airflow into lungs.
  \item End-inspiration:
  \begin{itemize}[leftmargin=*,nosep]
    \item Intrapleural pressure \textasciitilde $-8$ cmH$_2$O.
    \item Alveolar pressure returns to 0 (no flow).
    \item Lung volume increases by \textasciitilde 500 mL (tidal volume).
  \end{itemize}
\end{itemize}

\subsubsection{Expiration}
\begin{itemize}[leftmargin=*,nosep]
  \item Inspiratory drive ceases $\rightarrow$ system returns toward resting equilibrium.
  \item Intrapleural pressure becomes less negative.
  \item Alveolar pressure becomes \textasciitilde $+1$ cmH$_2$O $\rightarrow$ airflow out.
  \item End-expiration:
  \begin{itemize}[leftmargin=*,nosep]
    \item Intrapleural pressure returns to \textasciitilde $-5$ cmH$_2$O.
    \item Alveolar pressure returns to 0; flow stops; volume returns to FRC.
  \end{itemize}
  \item Quiet expiration is passive (elastic recoil); can become active with expiratory muscle recruitment.
\end{itemize}

\paragraph{Exam pitfalls}
\begin{itemize}[leftmargin=*,nosep]
  \item In spontaneous breathing, alveolar pressure is 0 at end-inspiration and end-expiration; flow is driven by small swings (e.g.\ to $-1$ and $+1$ cmH$_2$O).
\end{itemize}

\subsection{Pressure waveforms during intermittent positive pressure ventilation (IPPV)}
\begin{itemize}[leftmargin=*,nosep]
  \item During \textbf{inspiration} in IPPV:
  \begin{itemize}[leftmargin=*,nosep]
    \item Alveolar pressure rises from baseline to positive values.
    \item Intrapleural pressure rises from \textasciitilde $-5$ cmH$_2$O to about $+2$ to $+3$ cmH$_2$O at end inspiration.
  \end{itemize}
  \item During \textbf{expiration}: intrapleural pressure returns to \textasciitilde $-5$ cmH$_2$O.
\end{itemize}

\paragraph{Exam pitfalls}
\begin{itemize}[leftmargin=*,nosep]
  \item Spontaneous inspiration: intrapleural pressure becomes \emph{more negative}.
  \item IPPV inspiration: intrapleural pressure becomes \emph{less negative and may become positive}.
\end{itemize}

% =========================================================

\section{Chapter 16 --- Mechanical Properties of the Lungs}
\textit{Primary source: Kam \& Power — \emph{Principles of Physiology for the Anaesthetist}, Chapter 16.}

\subsection{What this chapter is about}
Lung mechanics describes how pressure generates:
\begin{itemize}
  \item \textbf{Flow} (air movement),
  \item \textbf{Volume change} (inflation/deflation),
  \item \textbf{Stability} of alveoli and small airways (surfactant, interdependence, closing capacity).
\end{itemize}

\subsection{Pressures and reference points}
\subsubsection{Key pressures}
\begin{itemize}
  \item \textbf{Alveolar pressure} ($P_A$)
  \item \textbf{Pleural (intrapleural) pressure} ($P_{pl}$)
  \item \textbf{Transpulmonary pressure} ($P_L$): distending pressure of the lung
\end{itemize}
\[
P_L = P_A - P_{pl}
\]

\subsubsection{Elastic equilibrium at end-expiration (FRC)}
At \textbf{FRC}, inward recoil of the lung balances outward recoil of the chest wall. The chapter states loss of diaphragmatic tone reduces FRC by \textasciitilde 400 mL.

\subsection{Compliance, elastance, and measurement}
\subsubsection{Definitions}
\[
C = \frac{\Delta V}{\Delta P}, \qquad E = \frac{1}{C}
\]
\begin{itemize}
  \item High compliance: reminder ``easy to inflate''.
  \item Low compliance: ``stiff''.
\end{itemize}

\subsubsection{Static compliance}
Measured when \textbf{flow has ceased} (pause/breath-hold), allowing slow units to fill; typically $C_{static} > C_{dynamic}$.

\subsubsection{Dynamic compliance}
Measured during ongoing breathing; reduced when units have different time constants.

\subsubsection{Frequency dependence}
Dynamic compliance falls as respiratory rate rises in heterogeneous lungs; used as an early marker of small airway closure.

\subsubsection{Specific compliance}
\[
C_{\text{spec}} = \frac{C}{FRC}
\]

\subsection{Determinants of compliance}
Broad determinants:
\begin{enumerate}
  \item \textbf{Elastic recoil of tissue} (elastin/collagen)
  \item \textbf{Surface tension} at the alveolar air--fluid interface
\end{enumerate}

Additional factors listed in the chapter (selected):
\begin{itemize}
  \item Lung volume; age; posture; obesity; pulmonary blood volume; bronchial smooth muscle tone.
  \item Disease:
  \begin{itemize}
    \item Fibrosis: stiffer lungs.
    \item Emphysema: increased compliance from loss of septal tissue opposing expansion.
  \end{itemize}
\end{itemize}

\subsection{Thoracic cage compliance and total respiratory system compliance}
Lung and chest wall compliances combine ``in series'':
\[
\frac{1}{C_{total}} = \frac{1}{C_{lung}} + \frac{1}{C_{chest}}
\]
Chapter values: $C_{lung} \approx 200$ mL/cmH$_2$O, $C_{chest} \approx 200$ mL/cmH$_2$O, so $C_{total} \approx 100$ mL/cmH$_2$O.

\subsection{Elastic recoil: tissue + surface tension}
The chapter states surface tension contributes \textasciitilde 70\% of total elastic forces in the normal lung.

\subsubsection{Laplace and stability}
If surface tension were constant, smaller units would have higher pressure and tend to empty into larger units, promoting collapse.

\subsection{Surfactant}
\subsubsection{Origin and turnover}
Produced by type II alveolar epithelial cells; stored in lamellated bodies. Half-life stated as \textasciitilde 15--30 h; most components recycled by type II cells.

\subsubsection{Composition (chapter description)}
\begin{itemize}
  \item \textasciitilde 90\% lipid (mainly phospholipid; also cholesterol)
  \begin{itemize}
    \item principal: dipalmitoyl phosphatidylcholine
    \item significant: phosphatidylglycerol
  \end{itemize}
  \item proteins \textasciitilde 2--8\%: SP-A, SP-B, SP-C, SP-D
\end{itemize}

\subsubsection{Functions}
\begin{itemize}
  \item Reduces surface tension $\rightarrow$ increases distensibility, reduces elastic recoil.
  \item Reduces work of breathing.
  \item Stabilises alveoli (preferential reduction of surface tension in small alveoli, especially during expiration).
  \item Reduces atelectasis; reduces tendency to alveolar oedema.
  \item Produces hysteresis.
\end{itemize}

\subsection{Hysteresis}
Inflation and deflation pressure--volume curves differ. Contributors described:
\begin{itemize}
  \item surfactant-related changes in surface tension,
  \item recruitment of collapsed units early in inspiration,
  \item stress relaxation at sustained high lung volumes.
\end{itemize}

\subsection{Alveolar interdependence}
Neighbouring alveoli share walls; traction from adjacent units tends to prevent collapse of a single unit (stabilising mechanism alongside surfactant).

\subsection{Lung volumes and FRC}
\subsubsection{Definition}
\[
FRC = ERV + RV
\]
Occurs where outward recoil of chest wall balances inward recoil of lung.

\subsubsection{Functions of FRC (chapter list)}
\begin{itemize}
  \item Oxygen store; reduces PaO$_2$ swings.
  \item Helps prevent airway collapse/atelectasis.
  \item Minimises pulmonary vascular resistance.
  \item Maintains airway patency; optimises compliance; minimises work of breathing.
\end{itemize}

\subsubsection{Factors affecting FRC (chapter list)}
\begin{itemize}
  \item Height (direct relationship); females \textasciitilde 10\% less than males; not correlated with age (per chapter).
  \item Posture: supine reduces FRC by up to \textasciitilde 1000 mL.
  \item Pregnancy (3rd trimester \textasciitilde 20\% reduction) and obesity reduce FRC.
  \item Increased elastic recoil (e.g.\ fibrosis) reduces FRC.
  \item Closing capacity rises with age and may exceed FRC (chapter notes upright \textasciitilde 70 y; supine \textasciitilde 44 y).
\end{itemize}

\subsubsection{Measurement methods}
Nitrogen washout; helium dilution; body plethysmography.

\subsection{Closing capacity}
\subsubsection{Definition}
\[
\text{Closing capacity} = RV + \text{Closing volume}
\]
Normally, $FRC >$ closing capacity. Small airways ($<2$ mm) begin to collapse during expiration at closing capacity.

\subsubsection{Increases with}
Age; smoking; lung disease (reduced elastic recoil and reduced radial traction).

\subsubsection{Consequences and posture/age}
Airway closure can leave perfused units unventilated $\rightarrow$ impaired oxygenation. Chapter statements include:
\begin{itemize}
  \item by \textasciitilde 65 y, closing capacity exceeds standing FRC,
  \item supine airway closure during normal tidal breathing may occur by \textasciitilde 45 y.
\end{itemize}

\subsection{Gas flow patterns and airway resistance}
\subsubsection{Laminar, turbulent, transitional}
\begin{itemize}
  \item Laminar: parabolic profile; strong radius dependence ($r^4$); viscosity is key gas property.
  \item Turbulent: eddies; higher driving pressure; $\Delta P \propto \text{flow}^2$; density is key gas property; strong radius dependence (Fanning equation context).
  \item Transitional: common at airway branches; mixed dependence on flow and flow$^2$.
\end{itemize}

\subsubsection{Reynolds number}
Turbulence tends to occur when $Re \gtrsim 2000$ in smooth tubes; promoted by high velocity, larger diameter, higher density, lower viscosity gas.

\subsubsection{Distribution in the lung}
Trachea/larynx tend to be turbulent at high velocities; laminar likely only in very small airways; much of bronchial tree is transitional.

\subsubsection{Airway resistance (Raw)}
Chapter values: $R_{aw}$ \textasciitilde 2 cmH$_2$O/L/s; normal tidal airflow requires a mouth--alveolar pressure difference \textasciitilde 1 cmH$_2$O.
Determinants: lung volume; bronchial smooth muscle tone; dynamic airway compression.

\subsection{Bronchial smooth muscle tone}
\begin{itemize}
  \item Parasympathetic (vagal) ACh at muscarinic receptors $\rightarrow$ bronchoconstriction.
  \item Sympathetic innervation less important; circulating adrenaline at $\beta_2$ receptors $\rightarrow$ bronchodilation.
  \item Non-cholinergic, non-adrenergic pathways can bronchodilate.
  \item Mediators: histamine causes constriction + mucosal swelling; leukotrienes and some prostaglandins can contribute to bronchospasm.
\end{itemize}

\subsection{Dynamic compression and the equal pressure point}
During forced expiration, pleural pressure becomes positive; airway pressure declines toward the mouth due to resistive pressure drop. The \textbf{equal pressure point} is where intraluminal pressure equals pleural pressure; beyond this point (toward mouth), airways can collapse.
Flow becomes effort-independent once dynamic compression limits further increase. Collapse is more likely at low lung volumes, high airway resistance, or reduced elastic recoil (e.g.\ emphysema).

\subsection{Work of ventilation}
\subsubsection{Components}
Inspiratory muscles do work against elastic and non-elastic (resistive) forces; expiration is normally passive.

\subsubsection{Recall values (chapter)}
\begin{itemize}
  \item Resistive work: \textasciitilde 35\% (dissipated as heat); tissue resistive work contributes \textasciitilde 10\% of resistance work.
  \item Elastic work: \textasciitilde 65\% (stored as potential energy).
\end{itemize}

\subsubsection{When expiration becomes active}
If airway resistance or expiratory flow demand rises, expiratory resistive work may exceed stored elastic energy $\rightarrow$ expiratory muscle recruitment.

\subsubsection{Oxygen cost of breathing}
Chapter value: respiratory muscle oxygen requirement \textasciitilde 3 mL/min at rest; rises with increased ventilation and in lung disease.

\subsection{Exam-oriented pitfalls and high-yield checks}
\begin{itemize}
  \item Compliance is $\Delta V/\Delta P$; elastance is the inverse.
  \item Static compliance (no flow) $>$ dynamic compliance.
  \item Closing capacity $= RV +$ closing volume.
  \item Laminar: viscosity; turbulent: density; much of bronchial tree is transitional.
  \item Emphysema: increased compliance but reduced elastic recoil $\rightarrow$ air trapping and dynamic compression.
\end{itemize}

\subsection{Additions and useful nuances from other core texts (not emphasised here)}
\subsubsection{Compliance is volume-dependent (P--V curve shape)}
Other texts emphasise compliance is only approximately linear around the normal working range near FRC:
\begin{itemize}
  \item High lung volumes: compliance falls as elastic elements approach their limit.
  \item Low lung volumes: compliance falls due to airway/alveolar closure and opening pressure requirement.
\end{itemize}
\textit{Sources:} \emph{Fundamentals of Anaesthesia} (4th ed), Ch.17 (P--V curve; compliance at extremes of lung volume).

\subsubsection{Pendelluft and dynamic compliance}
Pendelluft (gas redistribution between lung regions during dynamic conditions) can reduce apparent dynamic compliance.
\textit{Sources:} \emph{Morgan \& Mikhail's Clinical Anesthesiology} (5th ed), Ch.23 (pendelluft/gas redistribution concept).

\subsubsection{Measuring pleural pressure clinically}
Pleural pressure can be measured by intrapleural catheter or estimated using a mid-oesophageal balloon.
\textit{Sources:} \emph{Fundamentals of Anaesthesia} (4th ed), Ch.17 (pleural pressure measurement; oesophageal balloon).

\subsubsection{Typical pressure values during quiet breathing}
Illustrative values commonly quoted:
\begin{itemize}
  \item End-expiration: $P_{pl}$ about $-5$ cmH$_2$O; $P_A \approx 0$ $\Rightarrow P_L \approx +5$ cmH$_2$O.
  \item Inspiration: $P_{pl}$ becomes more negative; $P_A$ slightly negative to drive flow; at end-inspiration flow is zero and $P_A$ returns to 0 while $P_{pl}$ remains more negative.
\end{itemize}
\textit{Sources:} \emph{Morgan \& Mikhail} (5th ed), Ch.23 (quiet breathing pressure changes); similar concepts also described (often in kPa) in \emph{Fundamentals of Anaesthesia} (4th ed), Ch.17.

\subsubsection{Tissue resistance}
Some texts explicitly highlight tissue (viscoelastic) resistance as an important non-elastic component.
\textit{Sources:} \emph{Morgan \& Mikhail} (5th ed), Ch.23 (section on tissue resistance).

\subsubsection{Work of breathing: alternative breakdowns}
Resistive work may be subdivided into airway resistance work and tissue viscous (inelastic) work; proportional contributions vary by source.
\textit{Sources:} \emph{Ganong's Review of Medical Physiology} (Barrett et al.), pulmonary mechanics/work of breathing section; \emph{Morgan \& Mikhail} (5th ed), Ch.23.

\subsubsection{Proximal airway pressure may not equal alveolar pressure}
During dynamic conditions, proximal airway pressure (ventilator) may not reflect distal/alveolar pressure when resistance is high and/or tubing is compliant.
\textit{Sources:} \emph{Morgan \& Mikhail} (5th ed), Ch.4 (mechanical ventilation: airway vs alveolar pressure; resistance/compliance).

\subsubsection{Ventilator/circuit compliance and ``lost'' tidal volume}
In volume-controlled ventilation, some set tidal volume is lost to circuit expansion (compressible volume); exhaled tidal volume is best measured near the airway.
\textit{Sources:} \emph{Morgan \& Mikhail} (5th ed), Ch.3 (breathing systems) and Ch.4 (ventilation: circuit compliance/compressible volume).

\subsubsection{Time constant heuristic}
Some FRCA-style resources explicitly frame lung filling/emptying with the time constant:
\[
\tau = R \times C
\]
High $R$ and/or high $C$ (e.g.\ obstructive disease) $\rightarrow$ prolonged emptying, need for longer expiratory time to avoid gas trapping.
\textit{Sources:} \emph{Primary FRCA in a Box} (2nd ed), time constants/exponential relationships section; \emph{Morgan \& Mikhail} (5th ed), Ch.23 (air trapping/auto-PEEP concepts).

\subsubsection{Surfactant proteins: roles beyond surface tension}
SP-B and SP-C help form the surface film; SP-A and SP-D (collectins) contribute to innate immunity and surfactant turnover/regulation.
\textit{Sources:} \emph{Ganong} (Barrett et al.), pulmonary surfactant section.

\subsubsection{Spirometry as surrogate markers of airway mechanics}
FEV$_1$/FVC as obstruction index; FEF$_{25\text{--}75\%}$ (MMEF) described as relatively effort-independent and sometimes abnormal earlier in obstructive disease.
\textit{Sources:} \emph{Morgan \& Mikhail} (5th ed), Ch.23--24 (spirometric indices); \emph{Fundamentals of Anaesthesia} (4th ed), pulmonary function testing section.

\subsubsection{Positive pressure breathing systems: practical mechanics complications}
\begin{itemize}
  \item CPAP/PEEP can improve oxygenation but may reduce cardiac output (raised intrathoracic pressure effects).
  \item High-frequency jet ventilation can generate intrinsic PEEP at high frequencies; delivered tidal volume and FiO$_2$ may be uncertain due to entrainment.
\end{itemize}
\textit{Sources:} \emph{Essentials of Equipment in Anaesthesia} (Al-Shaikh \& Stacey, 2023), Ch.13 (CPAP complications) and Ch.8 (HF jet ventilation: entrainment/intrinsic PEEP); \emph{Morgan \& Mikhail} (5th ed), Ch.57 (PEEP/CPAP adverse nonpulmonary effects incl. reduced cardiac output).

% Fragment for \input{} / \include{} (no preamble)

% =========================================================

\section{Chapter 17 -- Gas Exchange in the Lungs }

\subsection{What ``gas exchange'' means (exam framing)}
Gas exchange is the \textbf{net transfer} of \textbf{O$_2$} from alveoli to blood and \textbf{CO$_2$} from blood to alveoli.

\paragraph{Determinants of arterial blood gases}
\begin{itemize}
  \item \textbf{Input gas}: inspired partial pressures (humidification, barometric pressure).
  \item \textbf{Bulk flow}: alveolar ventilation and pulmonary blood flow.
  \item \textbf{Matching}: ventilation--perfusion distribution \((\dot V_A/\dot Q)\).
  \item \textbf{Membrane transfer}: diffusion across the blood--gas barrier.
  \item \textbf{Mixing}: venous admixture/shunt.
\end{itemize}

\paragraph{Practical organisation of problems}
\begin{itemize}
  \item \textbf{Hypoventilation} (global \(\downarrow \dot V_A\)) \(\rightarrow\) \(\uparrow\)PaCO$_2$, \(\downarrow\)PAO$_2$.
  \item \textbf{V/Q inequality} \(\rightarrow\) impaired O$_2$ transfer; CO$_2$ often buffered by ventilation.
  \item \textbf{Shunt / venous admixture} \(\rightarrow\) lower arterial O$_2$.
  \item \textbf{Diffusion limitation} \(\rightarrow\) widened A--a difference and impaired transfer.
\end{itemize}

\subsection{Partial pressures along the circulation (anchor values)}
Typical partial pressures (breathing air):
\begin{itemize}
  \item \textbf{Inspired (dry):} PIO$_2$ \(\approx\) 160 mmHg (21.3 kPa), PICO$_2$ \(\approx\) 0.
  \item \textbf{Alveolar gas:} PO$_2$ \(\approx\) 105 mmHg (14 kPa), PCO$_2$ \(\approx\) 40 mmHg (5.3 kPa).
  \item \textbf{Arterial blood:} PaO$_2$ \(\approx\) 100 mmHg (13.3 kPa), PaCO$_2$ \(\approx\) 40 mmHg (5.3 kPa).
  \item \textbf{Mixed venous blood:} PvO$_2$ \(\approx\) 40 mmHg (5.3 kPa), PvCO$_2$ \(\approx\) 46 mmHg (6.1 kPa).
\end{itemize}

\subsection{Alveolar ventilation and dead space}
\subsubsection{Definitions}
Minute alveolar ventilation:
\[
\dot V_A = f\,(V_T - V_D)
\]
Dead space = the ``wasted'' portion of tidal volume not contributing to exchange.

\subsubsection{Types of dead space}
\begin{itemize}
  \item \textbf{Anatomical dead space}: conducting airways.
  \item \textbf{Alveolar dead space}: ventilated alveoli with no perfusion.
  \item \textbf{Physiological dead space}: anatomical + alveolar dead space.
\end{itemize}

\subsubsection{CO$_2$ as a marker of ventilation}
With stable metabolism, alveolar CO$_2$ is inversely related to \(\dot V_A\).
Clinically, PaCO$_2$ tracks PACO$_2$ closely and is a practical index of alveolar ventilation when CO$_2$ production is stable.

\subsubsection{Measuring dead space (core equations)}
\begin{itemize}
  \item \textbf{Anatomical dead space}: Fowler method.
  \item \textbf{Physiological dead space}: Bohr equation.
\end{itemize}
\[
\frac{V_D}{V_T} = \frac{P_{a\mathrm{CO}_2} - P_{\bar E\mathrm{CO}_2}}{P_{a\mathrm{CO}_2}}
\]
where \(P_{\bar E\mathrm{CO}_2}\) is mixed expired CO$_2$ partial pressure.

\subsection{Oxygen in alveolar gas and the alveolar air equation}
\subsubsection{Why calculate ``ideal'' PAO$_2$?}
Ideal alveolar gas cannot be sampled directly: end-expired gas may include dead-space contributions, and shunt/VQ mismatch affects arterial PO$_2$ more than arterial PCO$_2$.

\subsubsection{Alveolar air equation}
Assumption used: ideal alveolar PCO$_2$ \(\approx\) PACO$_2$ \(\approx\) PaCO$_2$.
\[
R = \frac{\dot V_{\mathrm{CO}_2}}{\dot V_{\mathrm{O}_2}} \approx 0.8
\]
Conceptual form:
\[
P_{A\mathrm{O}_2} = P_{I\mathrm{O}_2} - \frac{P_{a\mathrm{CO}_2}}{R}
\]
Humidification:
\[
P_{I\mathrm{O}_2} = F_{I\mathrm{O}_2}\,(P_B - P_{H_2O})
\]

\subsection{Ventilation--perfusion (V/Q) inequality}
\subsubsection{Definitions and extremes}
Whole lung average in normal lungs: \(\dot V_A/\dot Q \approx 0.8\).
\begin{itemize}
  \item High \(\dot V_A/\dot Q\): \(\uparrow\) PO$_2$, \(\downarrow\) PCO$_2$ (dead-space effect if perfusion low).
  \item Low \(\dot V_A/\dot Q\): \(\downarrow\) PO$_2$, \(\uparrow\) PCO$_2$ (venous admixture effect).
  \item \textbf{Dead space}: \(\dot V_A/\dot Q \to \infty\).
  \item \textbf{Shunt}: \(\dot V_A/\dot Q \to 0\).
\end{itemize}

\subsubsection{Gravity and regional V/Q}
Perfusion falls more than ventilation towards the apex \(\Rightarrow\) \(\dot V_A/\dot Q\) higher at the top and lower at the bottom.
Representative apex values: \(\dot V_A/\dot Q \sim 3.3\), PO$_2 \sim 130\) mmHg, PCO$_2 \sim 28\) mmHg.

\subsubsection{Assessing V/Q mismatch}
\begin{itemize}
  \item Qualitative: compare PaO$_2$ with calculated ideal PAO$_2$; reminder that ABGs, imaging and nuclear methods can support diagnosis.
  \item Quantitative: shunt estimation (shunt equation, iso-shunt concepts), dead space (Bohr), and MIGET (research).
\end{itemize}

\subsection{Venous admixture (physiological shunt)}
\subsubsection{Normal sources}
\begin{itemize}
  \item Bronchial venous drainage into pulmonary veins: \(<1\%\) of cardiac output.
  \item Thebesian veins into the left heart: \(\sim 0.3\%\) of cardiac output.
\end{itemize}

\subsubsection{Shunt equation (oxygen content form)}
\[
\frac{\dot Q_S}{\dot Q_T} = \frac{C'_{c\mathrm{O}_2} - C_{a\mathrm{O}_2}}{C'_{c\mathrm{O}_2} - C_{v\mathrm{O}_2}}
\]
where \(C'_{c\mathrm{O}_2}\) is end-capillary O$_2$ content (derived from ideal PAO$_2$ and the dissociation curve). Assumption: shunted blood has the same O$_2$ content as mixed venous blood.

\subsection{Oxygen cascade}
O$_2$ moves down a sequence of partial pressure gradients from atmosphere to mitochondria.
Key steps: humidification in the trachea (\(P_{H_2O}\) at \(37^\circ\)C \(\approx 47\) mmHg), alveolar setting of PAO$_2$, small further fall to PaO$_2$ due to shunt/VQ mismatch/diffusion effects, then tissue extraction to low mitochondrial PO$_2$.

\paragraph{Factors that shift the cascade}
Inspired O$_2$ concentration; barometric pressure; alveolar ventilation; O$_2$ consumption; scatter of \(\dot V_A/\dot Q\); venous admixture; blood flow; haemoglobin concentration.

\subsection{Diffusion across the alveolar--capillary barrier}
\subsubsection{Fick's law (conceptual)}
Transfer \(\propto\) area \(\times\) partial pressure difference \(\times\) diffusion constant, and \(\propto 1/\)thickness.

\subsubsection{Why CO$_2$ diffuses easily}
CO$_2$ has a diffusion constant \(\sim 20\times\) that of O$_2$ (greater solubility), so despite a small gradient it transfers efficiently.

\subsubsection{Perfusion-limited transfer (normal)}
At rest, equilibration of blood with alveolar gases occurs rapidly (\(\sim 0.25\) s, about one-third of capillary transit time) \(\Rightarrow\) O$_2$ and CO$_2$ transfer are typically perfusion-limited.

\subsubsection{Diffusing capacity (DLCO)}
DLCO assesses diffusion properties:
\begin{itemize}
  \item Normal DLCO \(\sim 25\) mL/min/mmHg at rest; increases up to \(\sim 3\times\) during exercise (capillary recruitment/dilation).
  \item Decreases in emphysema (\(\downarrow\) surface area), fibrosis and pulmonary oedema (\(\uparrow\) thickness).
\end{itemize}

\subsection{High-yield consolidation (what to be able to do)}
\begin{itemize}
  \item Define anatomical vs alveolar vs physiological dead space.
  \item Use \(\dot V_A = f(V_T - V_D)\) and interpret effects on PaCO$_2$.
  \item Recall Bohr equation for \(V_D/V_T\) and the shunt equation for \(\dot Q_S/\dot Q_T\).
  \item Use the alveolar air equation conceptually to estimate ideal PAO$_2$.
  \item Explain V/Q inequality and see dead space/shunt as extremes of \(\dot V_A/\dot Q\).
  \item Describe the oxygen cascade and what lowers PaO$_2$ between atmosphere and mitochondria.
  \item Explain why CO$_2$ is rarely diffusion-limited; state what DLCO measures and how it changes in disease/exercise.
\end{itemize}

\subsection{Additions highlighted in other core texts (complements)}
\subsubsection{Absolute vs relative shunt and response to oxygen}
\begin{itemize}
  \item \textbf{Absolute shunt}: anatomic shunt + lung units with \(\dot V_A/\dot Q = 0\).
  \item \textbf{Relative shunt}: low \(\dot V_A/\dot Q\) units.
\end{itemize}
Hypoxaemia from low \(\dot V_A/\dot Q\) is usually partially correctable with increased FiO$_2$; hypoxaemia from true shunt corrects poorly.

\subsubsection{Three-compartment model}
Dead space compartment (\(\dot V_A/\dot Q \to \infty\)), normal exchange compartment, and shunt compartment (\(\dot V_A/\dot Q = 0\)); helpful for viva-style explanations.

\subsubsection{Effects of anaesthesia on gas exchange (checklist)}
Hypoventilation; increased dead space; increased intrapulmonary shunting; increased scatter of \(\dot V_A/\dot Q\) ratios.

\subsubsection{Closing capacity, FRC, atelectasis risk and PEEP}
If closing capacity approaches/exceeds FRC, dependent airway closure during tidal breathing promotes atelectasis, low V/Q and shunt; PEEP may help by maintaining lung volume and reducing airway closure. Obesity is a common applied example (reduced FRC/compliance, increased V/Q mismatch).

\subsubsection{Oxygen dissociation curve (ODC) framing}
ODC/P50 links PaO$_2$ to SaO$_2$ and oxygen content; classic modifiers include CO$_2$/H$^+$ (Bohr), temperature, 2,3-DPG, and dyshemoglobins (CO, methaemoglobin) that reduce effective oxygen carriage.

\subsubsection{Practical dead space points}
Some texts emphasise the Bohr derivation, the practical assumption PACO$_2 \approx\) PaCO$_2$, and that anatomical dead space varies with head/neck position, intubation, and flow pattern.

\subsubsection{West zones / Starling resistor}
West zones and Starling resistor concepts are used to explain regional perfusion and how this may vary with pulmonary arterial/alveolar pressures; reminder linkage to altitude physiology is often used as an integrative example.

\subsubsection{High altitude as a gas-exchange ``stress test''}
Lower inspired PO$_2$ drives acute hyperventilation; chronic responses include increased 2,3-DPG and polycythaemia; hypoxic pulmonary vasoconstriction raises pulmonary arterial pressures and may strain the RV over time.

\subsection*{Sources used for the additions}
\begin{itemize}
  \item Kam \& Power, \emph{Principles of Physiology for the Anaesthetist}, Ch 17 (Gas exchange in the lungs).
  \item Morgan \& Mikhail, \emph{Clinical Anesthesiology} (5th ed.), Ch 23 (Respiratory physiology \& anesthesia): shunt vs V/Q response to FiO$_2$, three-compartment model, effects of anesthesia.
  \item \emph{Fundamentals of Anaesthesia} (4th ed.), Ch 17: Fowler/Bohr details, anatomical dead space variability, West zones/Starling resistor, altitude physiology.
  \item \emph{Primary FRCA in a Box} (2nd ed.): CC vs FRC/PEEP and obesity respiratory physiology; ODC/P50 and dyshemoglobins.
  \item \emph{1,000 Practice MTF} (Physiology answers): high-altitude responses (HPV and chronic RV effects).
\end{itemize}

% =========================================================

\section{Chapter 18 -- Carriage of Oxygen in Blood}

\subsection{Haemoglobin: the carrier}
\begin{itemize}
  \item Erythrocyte Hb content: \(\sim 200\text{--}300\) million Hb molecules per red cell.
  \item Hb structure (conceptual): 4 subunits, each with a haem (protoporphyrin ring + central iron in ferrous state, Fe\(^{2+}\)) \(\rightarrow\) binds up to 4 O\(_2\) molecules per Hb molecule.
\end{itemize}

\paragraph{Key point}
Only \emph{dissolved} O\(_2\) contributes to blood \(P\mathrm{O}_2\). Hb-bound O\(_2\) increases O\(_2\) \emph{content} substantially but does not directly raise \(P\mathrm{O}_2\).

\subsection{Oxygen in blood: dissolved vs haemoglobin-bound}

\subsubsection{Dissolved oxygen}
\begin{itemize}
  \item Dissolved O\(_2\) is directly proportional to \(P\mathrm{O}_2\).
  \item Solubility coefficient (37\(^\circ\)C): \(0.003~\mathrm{mL~O_2}/100~\mathrm{mL~blood/mmHg}\).
  \item At \(P\mathrm{a}O_2 \sim 100~\mathrm{mmHg}\) (\(13.3~\mathrm{kPa}\)): dissolved O\(_2 \approx 0.3~\mathrm{mL~O_2}/100~\mathrm{mL~blood}\).
\end{itemize}

\subsubsection{Oxygen carriage by haemoglobin}
\begin{itemize}
  \item At normal atmospheric pressure, about 98\% of blood O\(_2\) is carried by Hb.
  \item O\(_2\) combining capacity (fully saturated Hb):
  \begin{itemize}
    \item Adult blood: \(1.306~\mathrm{mL~O_2/g~Hb}\)
    \item Fetal blood: \(1.312~\mathrm{mL~O_2/g~Hb}\)
  \end{itemize}
  \item Increasing \(P\mathrm{O}_2\) increases Hb saturation (SaO\(_2\)) in a sigmoid pattern due to sequential binding across 4 subunits.
\end{itemize}

\paragraph{Definition}
Oxygen capacity of blood: the maximum amount of O\(_2\) that can be carried by Hb.

\subsection{Oxygen--haemoglobin dissociation curve (ODC)}

\subsubsection{Shape and physiological meaning}
\begin{itemize}
  \item Plot: \(P\mathrm{O}_2\) vs \% Hb saturation (and optionally O\(_2\) content).
  \item Steep portion: \(P\mathrm{O}_2 \sim 10\text{--}60~\mathrm{mmHg}\) (\(1.3\text{--}8~\mathrm{kPa}\)) \(\rightarrow\) small \(P\mathrm{O}_2\) changes cause large saturation changes.
  \item At \(P\mathrm{O}_2 \sim 60~\mathrm{mmHg}\) (\(8~\mathrm{kPa}\)), saturation is \(\sim 90\%\).
  \item Plateau portion: provides buffering such that modest falls in alveolar/arterial \(P\mathrm{O}_2\) produce only modest falls in SaO\(_2\).
\end{itemize}

\subsubsection{Typical arterial and mixed venous points}
\begin{itemize}
  \item Arterial (normal): \(P\mathrm{O}_2 \sim 100~\mathrm{mmHg}\) (\(13.3~\mathrm{kPa}\)); SaO\(_2 \sim 100\%\); O\(_2\) content \(\sim 20~\mathrm{mL}/100~\mathrm{mL}\) (Hb-bound).
  \item Normal tissue extraction: \(\sim 5~\mathrm{mL~O_2}/100~\mathrm{mL~blood}\) removed during capillary transit.
  \item Mixed venous (normal): \(P\mathrm{O}_2 \sim 40~\mathrm{mmHg}\) (\(5.3~\mathrm{kPa}\)); SvO\(_2 \sim 75\%\); O\(_2\) content \(\sim 15~\mathrm{mL}/100~\mathrm{mL}\) (Hb-bound).
\end{itemize}

\subsubsection{\texorpdfstring{\(P_{50}\)}{P50}}
\begin{itemize}
  \item \(P_{50}\): \(P\mathrm{O}_2\) at which Hb is 50\% saturated.
  \item Normal: \(\sim 26~\mathrm{mmHg}\) (\(\approx 3.5~\mathrm{kPa}\)).
  \item Use: compact descriptor of left/right shift of the ODC.
\end{itemize}

\subsection{Shifts of the ODC and the Bohr effect}

\subsubsection{Factors shifting the curve}
\begin{itemize}
  \item Right shift (decreased affinity, increased unloading): increased \(P\mathrm{CO}_2\), increased \([\mathrm{H}^+]\), increased temperature, increased 2,3-DPG.
  \item Left shift (increased affinity, reduced unloading): decreased \(P\mathrm{CO}_2\), decreased \([\mathrm{H}^+]\), decreased temperature, decreased 2,3-DPG, HbF.
\end{itemize}

\subsubsection{Bohr effect (mechanism and purpose)}
\begin{itemize}
  \item When \(P\mathrm{CO}_2\), \([\mathrm{H}^+]\), and temperature rise (e.g.\ working muscle), Hb affinity for O\(_2\) falls \(\rightarrow\) right shift \(\rightarrow\) O\(_2\) released more easily.
  \item In pulmonary capillaries, \(P\mathrm{CO}_2\) and \([\mathrm{H}^+]\) fall \(\rightarrow\) affinity rises \(\rightarrow\) facilitates O\(_2\) uptake.
  \item Mechanistic notes: \(\mathrm{H}^+\) binds to \(\alpha\)-amino and imidazole groups on Hb; CO\(_2\) binds to N-terminal amino groups \(\rightarrow\) reduced O\(_2\) affinity.
\end{itemize}

\subsubsection{2,3-DPG (2,3-diphosphoglycerate)}
\begin{itemize}
  \item Binds to \(\beta\) chains of one tetramer of deoxyHb \(\rightarrow\) conformational change \(\rightarrow\) reduced O\(_2\) affinity \(\rightarrow\) right shift.
  \item Produced via a side shunt from glycolysis; present in large amounts in erythrocytes.
  \item Increases with anaemia and high altitude exposure.
\end{itemize}

\subsubsection{Venous point concept}
\begin{itemize}
  \item If \(5~\mathrm{mL~O_2}/100~\mathrm{mL}\) is removed from arterial blood, the resulting \(P\mathrm{O}_2\) (normally \(\sim 40~\mathrm{mmHg}\)) represents the driving gradient for O\(_2\) movement into tissues.
  \item Right shift increases the venous-point \(P\mathrm{O}_2\); left shift reduces it.
\end{itemize}

\subsection{Oxygen content, delivery and consumption}

\subsubsection{Oxygen delivery (\(D\!O_2\)) / oxygen flux}
\begin{itemize}
  \item Oxygen delivery per minute depends on arterial oxygen content (\(C\mathrm{a}O_2\)) and cardiac output (CO).
  \item Typical resting values: \(C\mathrm{a}O_2 \approx 20~\mathrm{mL~O_2}/100~\mathrm{mL}\), CO \(\approx 5~\mathrm{L/min}\) \(\Rightarrow D\!O_2 \approx 1000~\mathrm{mL~O_2/min}\).
  \item Tissue oxygen usage (\(\dot V\!O_2\)): \(\sim 250~\mathrm{mL/min}\).
\end{itemize}

\subsubsection{Expanded delivery equation (including dissolved O\(_2\))}
\[
D\!O_2 =
\Bigl[\bigl(\mathrm{Hb}\times 1.306\times \mathrm{SaO_2}/100\bigr) + \bigl(0.003\times P\mathrm{a}O_2\bigr)\Bigr]\times \mathrm{CO}\times 10
\]
\noindent where Hb is in \(\mathrm{g}/100~\mathrm{mL}\), \(P\mathrm{a}O_2\) in \(\mathrm{mmHg}\), CO in \(\mathrm{L/min}\), and \(C\mathrm{a}O_2\) in \(\mathrm{mL~O_2}/100~\mathrm{mL}\). The dissolved O\(_2\) term is small but included for completeness.

\subsubsection{Determinants of \(C\mathrm{a}O_2\) and \(\dot V\!O_2\)}
\begin{itemize}
  \item \(C\mathrm{a}O_2\): influenced by \(F\mathrm{i}O_2\), alveolar ventilation, \(V/Q\) mismatch, Hb concentration, and ODC position (left/right shift).
  \item \(\dot V\!O_2\): influenced by metabolic rate; increases with exercise/surgical stress and decreases with sleep/anaesthesia.
\end{itemize}

\subsubsection{Supply dependence}
As cardiac output decreases, \(\dot V\!O_2\) can become supply dependent and may fall with CO.

\subsection{Mixed venous oxygen content/saturation (SvO\(_2\))}

\subsubsection{Determinants}
Mixed venous oxygen content (\(C\mathrm{v}O_2\)) is determined by \(C\mathrm{a}O_2\), \(\dot V\!O_2\), and flow (CO).

\subsubsection{Estimating SvO\(_2\)}
If \(C\mathrm{v}O_2\) is known, SvO\(_2\) can be derived using Hb concentration and the ODC (the text notes Hb O\(_2\) binding capacity values around \(\sim 1.31\text{--}1.34~\mathrm{mL/g}\)).

\subsubsection{Factors affecting SvO\(_2\)}
\paragraph{High SvO\(_2\)}
\begin{itemize}
  \item Increased O\(_2\) delivery: increased inspired O\(_2\) concentration
  \item Decreased O\(_2\) demand: hypothermia
  \item Other: histotoxic hypoxia (e.g.\ cyanide poisoning); sepsis (altered regional blood flow/distribution)
\end{itemize}

\paragraph{Low SvO\(_2\)}
\begin{itemize}
  \item Decreased O\(_2\) delivery: decreased Hb concentration; decreased arterial O\(_2\) saturation; decreased CO (e.g.\ hypovolaemia)
  \item Increased O\(_2\) demand: pain, shivering, hyperthermia, seizures
\end{itemize}

\subsection{Exam-focused pitfalls}
\begin{itemize}
  \item \(P\mathrm{O}_2\) is set by dissolved O\(_2\), not Hb-bound O\(_2\).
  \item Plateau vs steep portion: plateau buffers SaO\(_2\); steep portion supports efficient unloading.
  \item Right shift: improved unloading but reduced affinity; left shift: improved loading but impaired unloading.
  \item \(P_{50}\) as a single-number descriptor: increased \(P_{50}\) = right shift; decreased \(P_{50}\) = left shift.
  \item SvO\(_2\) reflects delivery, demand, and distribution (e.g.\ sepsis), not just ``oxygenation''.
\end{itemize}

\paragraph{Source}
\textit{Kam \& Power — Principles of Physiology for the Anaesthetist, Chapter 18 (Carriage of oxygen in blood).}

\subsection{Additional high-yield points emphasised in other core texts (not explicit in Kam \& Power Ch.\ 18)}

\subsubsection{Mechanistic basis of cooperativity (T \(\leftrightarrow\) R transition; Adair concept)}
\begin{itemize}
  \item Hb exists in tense (T) and relaxed (R) conformations; O\(_2\) binding promotes the R state, increasing affinity at remaining sites (positive cooperativity).
  \item Some texts explicitly name the Adair equation as the descriptive model for stepwise O\(_2\) binding.
\end{itemize}
\textit{Sources: Principles of Physiology for the Anaesthetist (2021), Hb structure/function and cooperativity; Ganong’s Review of Medical Physiology, ODC and T--R interconversion; Cross \& Plunkett (2008), Oxygen delivery and transport.}

\subsubsection{Dyshemoglobins and abnormal Hb: effects on O\(_2\) carriage and the ODC}
\begin{itemize}
  \item Carbon monoxide (CO): reduces O\(_2\)-binding capacity (high-affinity Hb binding); classically alters the ODC (often taught as a left shift of remaining sites).
  \item Methaemoglobin (Fe\(^{3+}\)): cannot bind O\(_2\); reduces total O\(_2\)-carrying capacity; some texts state left shift and impaired O\(_2\) release.
  \item Haemoglobin variants (e.g.\ HbS): different O\(_2\)-saturation characteristics; clinical consequences often considered in low \(P\mathrm{O}_2\)/low pH states.
\end{itemize}
\textit{Sources: Primary FRCA in a Box (2019), respiration/ODC notes and Hb variants; Morgan \& Mikhail (5th ed.), respiratory physiology/monitoring; Cross \& Plunkett (2008), oxygen delivery and transport.}

\subsubsection{Hüfner constant: why different numbers appear across books}
\begin{itemize}
  \item Distinction between a theoretical maximum O\(_2\) binding per gram Hb (often \(\sim 1.39~\mathrm{mL~O_2/g~Hb}\)) and a lower practical/clinical constant (often \(\sim 1.31~\mathrm{mL~O_2/g~Hb}\)).
\end{itemize}
\textit{Sources: Morgan \& Mikhail (5th ed.), respiratory physiology; Cross \& Plunkett (2008), oxygen delivery/transport.}

\subsubsection{Anaemia and polycythaemia: capacity vs saturation vs \(P\mathrm{O}_2\)}
\begin{itemize}
  \item Anaemia: decreases O\(_2\)-carrying capacity/content without independently altering \(P_{50}\).
  \item Polycythaemia: increases O\(_2\)-carrying capacity/content without altering \(P_{50}\).
\end{itemize}
\textit{Source: Primary FRCA in a Box (2019), respiration/ODC notes.}

\subsubsection{O\(_2\) extraction ratio and the Fick principle}
\begin{itemize}
  \item Fick relationship: \(\dot V\!O_2 = \mathrm{CO}\times (C\mathrm{a}O_2 - C\mathrm{v}O_2)\).
  \item Normal extraction fraction often expressed as \(\sim 25\%\) at rest: \((C\mathrm{a}O_2 - C\mathrm{v}O_2)/C\mathrm{a}O_2\).
\end{itemize}
\textit{Sources: Morgan \& Mikhail (5th ed.), respiratory physiology; Principles of Physiology for the Anaesthetist (2021), cardiopulmonary physiology; 1,000 Practice MTF MCQs for the Primary and Final FRCA (2019), physiology answers (Fick/CO calculations).}

\subsubsection{Critical \(D\!O_2\) framing of supply dependence}
\begin{itemize}
  \item Some exam texts define a critical oxygen delivery below which \(\dot V\!O_2\) becomes supply dependent, giving an approximate threshold around \(300~\mathrm{mL/min}\).
\end{itemize}
\textit{Source: Cross \& Plunkett (2008), oxygen delivery and transport.}

\subsubsection{Oxygen stores and preoxygenation (apnoea time concept)}
\begin{itemize}
  \item Emphasis on body O\(_2\) stores, especially O\(_2\) reservoir in FRC; increasing \(F\mathrm{i}O_2\) enlarges this store and delays hypoxaemia during apnoea.
\end{itemize}
\textit{Sources: Morgan \& Mikhail (5th ed.), respiratory physiology (FRC O\(_2\) store/preoxygenation); Primary FRCA in a Box (2019), oxygen stores/transport overview.}

\subsubsection{Monitoring linkage: pulse oximetry limitations in abnormal Hb states}
\begin{itemize}
  \item COHb: two-wavelength pulse oximeters may display falsely high SpO\(_2\).
  \item MetHb: displayed saturation tends toward \(\sim 85\%\) (can appear falsely low or high depending on true saturation).
  \item Averaging/response time and probe/site effects can introduce delays/artefacts.
\end{itemize}
\textit{Sources: Morgan \& Mikhail (5th ed.), monitoring; Al-Shaikh \& Stacey (2023), pulse oximetry limitations/averaging; Primary FRCA in a Box (2019), monitoring artefacts and dyshemoglobins.}

\subsubsection{CO\(_2\) carriage interaction: the Haldane effect}
\begin{itemize}
  \item Deoxygenated Hb carries more CO\(_2\) than oxygenated Hb (Haldane effect), linking O\(_2\) unloading to CO\(_2\) uptake in tissues and the reverse in lungs.
\end{itemize}
\textit{Sources: Morgan \& Mikhail (5th ed.), CO\(_2\) transport; Primary FRCA in a Box (2019), CO\(_2\) transport; Fundamentals of Anaesthesia (4th ed.), pregnancy physiology (Haldane/double Haldane effect).}

% =========================================================
% Fragment mode (for \input{} / \include{}). No preamble.

\section{Chapter 19 -- Carbon Dioxide Carriage in Blood}
\textit{Primary source: Kam \& Power, Chapter 19 (Carbon Dioxide Carriage in Blood).}

\subsection{Big picture}
\begin{itemize}
  \item \(\mathrm{CO_2}\) is carried in blood in three forms:
  \begin{enumerate}
    \item Dissolved in physical solution
    \item As bicarbonate (\(\mathrm{HCO_3^-}\)) (via carbonic acid)
    \item As carbamino compounds (bound to proteins, mainly haemoglobin)
  \end{enumerate}
  \item Venous blood contains more total \(\mathrm{CO_2}\) than arterial blood (systemic uptake from tissues).
  \item Lower \(\mathrm{O_2}\) content increases \(\mathrm{CO_2}\) carriage capacity (Haldane effect).
\end{itemize}

\subsection{Quantitative split (arterial vs transfer across tissues)}
\subsubsection{Arterial blood (typical proportions)}
\begin{itemize}
  \item \(\sim 90\%\) as bicarbonate
  \item \(\sim 5\%\) dissolved
  \item \(\sim 5\%\) as carbamino compounds
\end{itemize}

\subsubsection{Of the \(\mathrm{CO_2}\) transferred from tissues and eliminated in the lung}
\begin{itemize}
  \item \(\sim 60\%\) transferred as bicarbonate
  \item \(\sim 30\%\) transferred as carbamino compounds
  \item \(\sim 10\%\) transferred as dissolved \(\mathrm{CO_2}\)
\end{itemize}

\subsection{The three forms in more detail}
\subsubsection{Dissolved \(\mathrm{CO_2}\)}
\begin{itemize}
  \item \(\mathrm{CO_2}\) is more soluble than \(\mathrm{O_2}\).
  \item Dissolved \(\mathrm{CO_2}\) accounts for \(\sim 10\%\) of the \(\mathrm{CO_2}\) evolved in the lungs.
  \item Solubility in plasma at \(37^\circ\mathrm{C}\): \(0.231~\mathrm{mmol\,L^{-1}\,kPa^{-1}}\) (or \(0.0308~\mathrm{mmol\,L^{-1}\,mmHg^{-1}}\)).
\end{itemize}

\subsubsection{Bicarbonate / carbonic acid}
\begin{itemize}
  \item Conceptual sequence:
  \[
    \mathrm{CO_2 + H_2O \rightleftharpoons H_2CO_3 \rightleftharpoons HCO_3^- + H^+}
  \]
  \item Hydration is slow but catalysed by carbonic anhydrase.
  \item Carbonic anhydrase is present in erythrocytes and endothelium; not present in plasma.
\end{itemize}

\subsubsection{Carbamino compounds}
\begin{itemize}
  \item \(\mathrm{CO_2}\) combines with amino groups in proteins to form carbamates.
  \item In blood this is mainly with haemoglobin.
\end{itemize}

\subsection{Haldane effect}
\subsubsection{Definition}
Deoxygenated haemoglobin carries more \(\mathrm{CO_2}\) than oxygenated haemoglobin.

\subsubsection{Mechanisms described}
\begin{itemize}
  \item Oxygenation of haemoglobin reduces its capacity to carry \(\mathrm{CO_2}\) as carbamino compounds (via changes in ionization of nitrogen groups).
  \item Deoxyhaemoglobin is more basic and buffers \(\mathrm{H^+}\) produced during bicarbonate formation.
\end{itemize}

\subsubsection{Consequences}
\begin{itemize}
  \item Venous (deoxygenated) blood takes up and transports more \(\mathrm{CO_2}\) than oxygenated arterial blood.
  \item Carbamino compounds account for \(\sim 1/3\) of the arteriovenous difference in \(\mathrm{CO_2}\) carried.
\end{itemize}

\subsection{\(\mathrm{CO_2}\) dissociation curve (content vs \(P_{\mathrm{CO_2}}\))}
\begin{itemize}
  \item Relationship between \(\mathrm{CO_2}\) content and \(P_{\mathrm{CO_2}}\) is more linear than the oxygen dissociation curve.
  \item Different curves occur at different oxygen saturations (Haldane effect).
\end{itemize}
\paragraph{Physiological reference points}
\begin{itemize}
  \item Arterial: \(P_{\mathrm{CO_2}} \approx 40~\mathrm{mmHg}~(5.3~\mathrm{kPa})\), \(\mathrm{O_2}\) saturation \(\approx 100\%\).
  \item Mixed venous: \(P_{\mathrm{CO_2}} \approx 46~\mathrm{mmHg}~(6.1~\mathrm{kPa})\), \(\mathrm{O_2}\) saturation \(\approx 75\%\).
\end{itemize}

\subsection{Systemic capillaries: \(\mathrm{CO_2}\) transfer into blood}
\subsubsection{Driving gradient}
\(\mathrm{CO_2}\) is produced in mitochondria and diffuses cell \(\rightarrow\) interstitial fluid \(\rightarrow\) capillary \(\rightarrow\) plasma \(\rightarrow\) erythrocyte.

\subsubsection{What happens to \(\mathrm{CO_2}\) in blood}
\begin{itemize}
  \item Some dissolves in plasma and erythrocyte.
  \item Most becomes carbamino compounds and/or bicarbonate (carbonic anhydrase-dependent within RBC).
\end{itemize}

\subsubsection{Coupling to \(\mathrm{O_2}\) unloading}
\begin{itemize}
  \item As \(\mathrm{O_2}\) leaves the RBC, haemoglobin becomes deoxygenated (\(\mathrm{HHb}\)).
  \item \(\mathrm{HHb}\) buffers \(\mathrm{H^+}\) and promotes bicarbonate and carbaminohaemoglobin formation.
\end{itemize}

\subsubsection{Chloride shift (Hamburger effect)}
\begin{itemize}
  \item \(\mathrm{HCO_3^-}\) diffuses out of RBC into plasma; \(\mathrm{H^+}\) is buffered inside RBC.
  \item To maintain electroneutrality, \(\mathrm{Cl^-}\) diffuses into the RBC; water also enters.
\end{itemize}

\subsection{Pulmonary capillaries: \(\mathrm{CO_2}\) transfer out of blood}
\subsubsection{Driving gradient}
Venous \(P_{\mathrm{CO_2}} \approx 46~\mathrm{mmHg}~(6.1~\mathrm{kPa})\) exceeds alveolar \(P_{\mathrm{CO_2}} \approx 40~\mathrm{mmHg}~(5.3~\mathrm{kPa})\); \(\mathrm{CO_2}\) diffuses from blood to alveolus.

\subsubsection{Coupling to \(\mathrm{O_2}\) uptake}
In pulmonary capillaries, \(\mathrm{O_2}\) diffuses into RBC and oxygenation facilitates \(\mathrm{CO_2}\) unloading (Haldane effect).

\subsubsection{Determinant of alveolar \(P_{\mathrm{CO_2}}\)}
Alveolar \(P_{\mathrm{CO_2}}\) depends on balance between \(\mathrm{CO_2}\) delivery/output and alveolar ventilation. Fractional alveolar \(\mathrm{CO_2}\) concentration is described as:
\[
  \frac{\text{rate of } \mathrm{CO_2}\ \text{output (normally }200~\mathrm{mL\,min^{-1}}\text{)}}{\text{minute alveolar ventilation}}
\]

\subsubsection{End-capillary equilibration}
End pulmonary capillary blood \(P_{\mathrm{CO_2}}\) is very close to alveolar gas \(P_{\mathrm{CO_2}}\).

\subsection{FRCA/EDAIC traps (aligned to chapter)}
\begin{itemize}
  \item Haldane (effect of \(\mathrm{O_2}\) on \(\mathrm{CO_2}\) carriage) vs Bohr (effect of \(\mathrm{CO_2/H^+}\) on \(\mathrm{O_2}\) affinity).
  \item \(\mathrm{CO_2}\) content--\(P_{\mathrm{CO_2}}\) is more linear than the \(\mathrm{O_2}\) dissociation curve.
  \item Carbonic anhydrase location: RBCs and endothelium, not plasma.
  \item Chloride shift direction in systemic capillaries: \(\mathrm{HCO_3^-}\) out, \(\mathrm{Cl^-}\) in (water follows).
\end{itemize}

\subsection{Additions / extensions from other core texts (not explicit in Kam \& Power Ch 19)}
\subsubsection{Chloride shift: transporter, speed, and osmotic consequence}
\begin{itemize}
  \item Mechanism via RBC anion exchanger (AE1 / Band 3): \(\mathrm{HCO_3^-}\) leaves RBC in exchange for \(\mathrm{Cl^-}\).
  \item Time course: essentially complete within \(\sim 1~\mathrm{s}\).
  \item Osmotic effect: \(\mathrm{CO_2}\) addition increases osmotically active particles; water enters RBC \(\rightarrow\) swelling.
  \item Venous haematocrit described as \(\sim 3\%\) greater than arterial (RBC swelling plus small fluid handling effects).
\end{itemize}
\textit{Source: Ganong’s Review of Medical Physiology, section/chapter on transport of \(\mathrm{O_2}\) and \(\mathrm{CO_2}\) in blood.}

\subsubsection{Haemoglobin buffering: what groups buffer \(\mathrm{H^+}\)}
\begin{itemize}
  \item \(\mathrm{H^+}\) generated during \(\mathrm{CO_2}\) hydration is buffered mainly by haemoglobin, especially imidazole groups on histidine residues.
\end{itemize}
\textit{Source: Primary FRCA in a Box, Respiration: carbon dioxide stores/transport section.}

\subsubsection{A--V \(\mathrm{CO_2}\) differences and respiratory quotient linkage}
\begin{itemize}
  \item A--V \(P_{\mathrm{CO_2}}\) difference: \(\sim 0.7~\mathrm{kPa}\) (about \(6~\mathrm{mmHg}\)).
  \item Linked statement: corresponds to \(\sim 4~\mathrm{mL~CO_2}\) per \(100~\mathrm{mL}\) blood (noted to depend on respiratory quotient, \(RQ\)).
  \item \(RQ = \frac{\mathrm{CO_2\ produced}}{\mathrm{O_2\ consumed}}\) (steady state).
\end{itemize}
\textit{Source: Primary FRCA in a Box, Respiration: carbon dioxide stores/transport section.}

\subsubsection{Plasma vs whole-blood framing and plasma hydration point}
\begin{itemize}
  \item Distinguish reported \(\mathrm{CO_2}\) content in plasma vs whole blood.
  \item In plasma, less than \(1\%\) of dissolved \(\mathrm{CO_2}\) undergoes hydration (contrast with rapid RBC/endothelial catalysis).
\end{itemize}
\textit{Source: Morgan \& Mikhail’s Clinical Anesthesiology, respiratory physiology section on \(\mathrm{CO_2}\) transport and tables.}

\subsubsection{Lung-side linkage equation (oxygenation driving \(\mathrm{CO_2}\) unloading)}
\[
  \mathrm{O_2 + HCO_3^- + HbH^+ \rightarrow H_2O + CO_2 + HbO_2}
\]
\textit{Source: Morgan \& Mikhail’s Clinical Anesthesiology, respiratory physiology section on haemoglobin buffering and \(\mathrm{CO_2}\) transport.}

\subsubsection{Carbonic anhydrase inhibition (pharmacological hook)}
\begin{itemize}
  \item Acetazolamide (carbonic anhydrase inhibitor) described as impairing \(\mathrm{CO_2}\) transport between tissues and alveoli.
\end{itemize}
\textit{Source: Morgan \& Mikhail’s Clinical Anesthesiology, respiratory physiology section (bicarbonate/\(\mathrm{CO_2}\) transport discussion).}

\subsubsection{CO\(_2\) dissociation curves: diagram technique}
\begin{itemize}
  \item Include a dissolved \(\mathrm{CO_2}\) line through the origin.
  \item Deoxygenated curve lies above oxygenated curve (Haldane effect).
  \item Vertical gap between dissolved line and total \(\mathrm{CO_2}\) curve represents bicarbonate carriage.
\end{itemize}
\textit{Source: Cross \& Plunkett, Physics, Pharmacology and Physiology for Anaesthetists: section on carriage of \(\mathrm{CO_2}\) and dissociation curves.}

\subsubsection{Pregnancy/placenta: double Haldane effect}
\begin{itemize}
  \item Placental \(\mathrm{CO_2}\) transfer enhanced by maternal deoxygenation (increased maternal \(\mathrm{CO_2}\) uptake) and fetal oxygenation (increased fetal \(\mathrm{CO_2}\) release): ``double Haldane effect''.
  \item Quantitative statement given: may account for \(\sim 46\%\) of transplacental \(\mathrm{CO_2}\) transfer.
\end{itemize}
\textit{Source: Fundamentals of Anaesthesia, physiology of pregnancy section on placental gas transfer/\(\mathrm{CO_2}\).}

\subsubsection{Placental \(\mathrm{CO_2}\): diffusibility and forms}
\begin{itemize}
  \item Placenta described as highly permeable to \(\mathrm{CO_2}\); \(\mathrm{CO_2}\) stated to be \(\sim 20\times\) more diffusible than \(\mathrm{O_2}\).
  \item Approximate forms in fetal/maternal blood: dissolved \(\sim 8\%\), bicarbonate \(\sim 62\%\), carbaminohaemoglobin \(\sim 30\%\), with very small amounts as carbonic acid/carbonate.
  \item Dissolved \(\mathrm{CO_2}\) is the form that crosses; bicarbonate acts as a reservoir via equilibrium.
\end{itemize}
\textit{Source: Fundamentals of Anaesthesia, physiology of pregnancy section on \(\mathrm{CO_2}\) and placental transfer.}

\subsubsection{CO\(_2\) measurement: electrode (blood gas analyser)}
\begin{itemize}
  \item Severinghaus-type electrode: \(\mathrm{CO_2}\) diffuses across a CO\(_2\)-permeable membrane into a bicarbonate film; pH change reflects \(\mathrm{CO_2}\).
  \item Quantitative statement: \(\sim 0.01\) pH units per \(0.1~\mathrm{kPa}\) change in \(\mathrm{CO_2}\).
  \item Response time limited by diffusion; one source states \(2\)--\(3\) minutes.
\end{itemize}
\textit{Sources: Fundamentals of Anaesthesia (CO\(_2\) measurement section); Essentials of Equipment in Anaesthesia, Critical Care and Perioperative Medicine (blood gas analyser electrodes/response time).}

\subsubsection{End-tidal CO\(_2\) (capnography): principle and waveform cues}
\begin{itemize}
  \item Infrared absorption principle: \(\mathrm{CO_2}\) absorbs IR (noted around \(4.3~\mu\mathrm{m}\)); absorption proportional to \(\mathrm{CO_2}\) partial pressure.
  \item Rebreathing: baseline fails to return to zero during inspiration.
  \item Obstructive disease: sloping alveolar plateau compared with normal.
\end{itemize}
\textit{Source: Essentials of Equipment in Anaesthesia, Critical Care and Perioperative Medicine (CO\(_2\) analysers/capnography section).}

\subsubsection{Central chemoreceptors: why \(\mathrm{CO_2}\) changes CSF pH}
\begin{itemize}
  \item Central chemoreceptors respond to CSF pH.
  \item Plasma \(\mathrm{H^+}\) does not cross the blood--brain barrier, whereas \(\mathrm{CO_2}\) diffuses readily and alters CSF pH via hydration.
\end{itemize}
\textit{Source: 1,000 Practice MTF (physiology: control of ventilation/chemoreceptors content).}

\subsubsection{Additional quantitative statements on carbamino binding / Haldane}
\begin{itemize}
  \item Deoxyhaemoglobin described as having \(\sim 3.5\times\) greater affinity for \(\mathrm{CO_2}\) than oxyhaemoglobin.
  \item Within physiological ranges, \(P_{\mathrm{CO_2}}\) stated to have little effect on the fraction of \(\mathrm{CO_2}\) carried as carbaminohaemoglobin.
  \item Plasma tables: carbamino \(\mathrm{CO_2}\) in plasma described as negligible (carbamino carriage is primarily haemoglobin-based).
\end{itemize}
\textit{Source: Morgan \& Mikhail’s Clinical Anesthesiology, respiratory physiology section on \(\mathrm{CO_2}\) transport (text + tables).}

% =========================================================

% Fragment mode (for \input{} / \include{}). No preamble included.

\section{Chapter 20 -- Pulmonary Circulation}

\subsection{Big picture}
\begin{itemize}
  \item Low-pressure, low-resistance circulation in series with the right ventricle.
  \item Because pressures are low, hydrostatic (gravitational) effects are large in the upright lung \(\rightarrow\) regional differences in perfusion and potential \(V/Q\) mismatch.
\end{itemize}

\subsection{Functions of the pulmonary circulation}
\subsubsection{Primary function}
\begin{itemize}
  \item Gas exchange: deliver mixed venous blood to pulmonary capillaries for contact with alveoli.
\end{itemize}

\subsubsection{Secondary functions}
\begin{enumerate}
  \item Filter: traps emboli (gas, fat, blood clots) within small pulmonary vessels/capillaries.
  \item Reservoir (high capacitance): stores \(\sim 450\text{--}500\,\mathrm{mL}\) blood (arteries \(\sim 150\,\mathrm{mL}\); capillaries \(\sim 80\,\mathrm{mL}\); veins \(\sim 250\,\mathrm{mL}\)).
  \item Metabolic organ: pulmonary endothelium expresses ACE (Ang I \(\rightarrow\) Ang II) and inactivates bradykinin, serotonin, noradrenaline, and prostaglandins E \& F.
\end{enumerate}

\subsection{Vessels and architecture}
\subsubsection{Pulmonary arteries \(\rightarrow\) capillaries \(\rightarrow\) veins}
\begin{itemize}
  \item Pulmonary artery: shorter, thinner-walled, more distensible than the aorta.
  \item Pulmonary arterioles: relatively little smooth muscle, but sufficient for effective vasoconstriction.
  \item Pulmonary venules/veins: thin-walled, almost devoid of smooth muscle, very distensible.
  \item Distensibility permits full RV output with little rise in pressure.
\end{itemize}

\subsubsection{Capillaries and relation to alveoli}
\begin{itemize}
  \item Capillaries arranged around small airways; blood flows in thin sheets over alveolar walls.
  \item Large gas-exchange interface (\(\sim 50\text{--}70\,\mathrm{m^2}\)).
  \item Capillaries applied directly to alveolar walls \(\rightarrow\) flow depends on vascular pressures and alveolar pressure.
\end{itemize}

\subsubsection{Recruitment/distension and vessel groups}
\begin{itemize}
  \item Increased flow/pressure is accommodated by distension of patent vessels and recruitment of previously closed vessels \(\rightarrow\) PVR falls.
  \item Extra-alveolar vessels: expanded by elastic pull of surrounding tissues; lung expansion increases their diameter (tends to decrease PVR); low lung volumes reduce radial traction \(\rightarrow\) increased resistance.
  \item Alveolar vessels: compressed by high alveolar pressures.
\end{itemize}

\subsection{Bronchial circulation}
\begin{itemize}
  \item Supplies airways down to terminal bronchioles via bronchial arteries (from thoracic aorta); flow \(\sim 1\%\) of CO.
  \item Venous drainage: \(\sim 1/3\) to systemic veins (azygous/hemiazygous), \(\sim 2/3\) to pulmonary veins \(\rightarrow\) anatomical shunt (venous admixture).
  \item Small venous admixture contributes to slightly lower arterial \(P_{O_2}\); LV output slightly exceeds RV output.
\end{itemize}

\subsection{Pressures in the pulmonary circulation}
\begin{itemize}
  \item Pulmonary pressures are \(\sim 20\%\) of systemic because resistance is much lower.
  \item Typical values: PAP \(25/10\,\mathrm{mmHg}\); mean PAP \(\sim 15\,\mathrm{mmHg}\); LA pressure \(\sim 5\,\mathrm{mmHg}\); mean drop across pulmonary circulation \(\sim 10\,\mathrm{mmHg}\); mean pulmonary capillary pressure \(\sim 10\,\mathrm{mmHg}\) (estimated).
\end{itemize}

\subsection{Distribution of blood flow in the upright lung (West zones)}
\subsubsection{Why distribution varies}
\begin{itemize}
  \item Gravity alters vascular pressures relative to heart level: apex \(\sim 10\,\mathrm{mmHg}\) lower; base \(\sim 10\,\mathrm{mmHg}\) higher.
  \item If alveolar pressure exceeds capillary pressure, vessels close \(\rightarrow\) flow stops.
  \item Normal breathing: alveolar pressure varies around atmospheric by \(\pm 1\,\mathrm{cmH_2O}\); can be much higher with positive-pressure ventilation.
\end{itemize}

\subsubsection{Zones and defining pressure relationships}
\noindent Key: \(P_a\) = pulmonary arterial pressure; \(P_v\) = pulmonary venous pressure; \(P_A\) = alveolar pressure.
\begin{itemize}
  \item Zone 1 (no flow; alveolar dead space): \(P_A > P_a\).
  \item Zone 2 (intermittent flow; ``waterfall''): \(P_a > P_A > P_v\); flow depends mainly on \(P_a - P_A\).
  \item Zone 3 (continuous flow): \(P_a > P_v > P_A\); flow depends on \(P_a - P_v\).
  \item Zone 4 (reduced flow in most dependent lung): reduced flow due to increased interstitial pressure exceeding alveolar and pulmonary venous pressure.
\end{itemize}

\subsection{Pulmonary vascular resistance (PVR)}
\subsubsection{Core features}
\begin{itemize}
  \item Pulmonary capillary flow is markedly pulsatile (low arteriolar resistance does not damp the RV pressure waveform).
  \item Normal PVR: \(150\text{--}200\,\mathrm{dyn\cdot s/cm^5}\) (\(\approx 2\) Wood units) \(\approx\) one-tenth of SVR.
  \item Resistance distribution: PVR roughly equally divided among arteries, capillaries, and veins (contrast: systemic resistance mainly arteriolar).
\end{itemize}

\subsubsection{Where PVR is controlled}
\begin{itemize}
  \item Extra-alveolar pulmonary arteries/arterioles: main site of active control (neural, humoral, gaseous).
  \item Pulmonary capillary resistance influenced by alveolar volume and pressure.
\end{itemize}

\subsection{Factors affecting PVR}
\subsubsection{Passive factors}
\begin{enumerate}
  \item Distension/recruitment: increased flow decreases PVR via passive distension (and recruitment).
  \item Lung volume: PVR is lowest at FRC; below FRC extra-alveolar vessel narrowing increases PVR; at high lung volumes alveolar capillaries compressed \(\rightarrow\) PVR increases.
  \item Gravity: regional pressure relationships produce West zones.
\end{enumerate}

\subsubsection{Active factors}
\begin{enumerate}
  \item Autonomic: \(\alpha\) vasoconstriction; \(\beta\) vasodilatation; vagal ACh (muscarinic) vasodilatation. Overall role limited as vessels normally maximally dilated; local factors dominate.
  \item HPV: see below.
  \item Alveolar hypercapnia: elevated alveolar \(P_{CO_2}\) causes pulmonary vasoconstriction.
  \item Humoral: epinephrine (predominantly constriction), thromboxane/leukotrienes (constrict), prostacyclin \(PGI_2\) (dilate), serotonin (potent constrictor).
  \item Drugs: inhaled nitric oxide (potent pulmonary vasodilator); nebulized prostacyclin; PDE inhibitors (milrinone, levosimendan, sildenafil); volatile agents vasodilate pulmonary vasculature.
\end{enumerate}

\subsection{Hypoxic pulmonary vasoconstriction (HPV)}
\subsubsection{Trigger}
\begin{itemize}
  \item Trigger is alveolar hypoxia (low alveolar \(P_{O_2}\)) or atelectasis.
  \item Key: stimulus is low alveolar \(P_{O_2}\), not low arterial \(P_{O_2}\).
  \item Mediated mainly by low alveolar \(P_{O_2}\) and, to a lesser extent, mixed venous (pulmonary arterial) \(P_{O_2}\).
\end{itemize}

\subsubsection{Site}
\begin{itemize}
  \item Predominantly at arterial precapillary vessels close to the alveoli.
\end{itemize}

\subsubsection{Purpose}
\begin{itemize}
  \item Diverts perfusion from poorly ventilated regions toward better ventilated alveoli.
\end{itemize}

\subsubsection{Time course}
\begin{itemize}
  \item Biphasic: initial rapid phase plateau after \(\sim 5\) minutes; second phase begins \(\sim 40\) minutes later; maximal at \(\sim 2\text{--}4\) hours.
\end{itemize}

\subsubsection{Mechanistic mediators highlighted}
\begin{itemize}
  \item First phase: inhibition of voltage-gated potassium (Kv) channels (reduced outward \(K^+\) current).
  \item Second phase: likely mediated by endothelin.
  \item Nitric oxide synthesis reduced when alveolar \(P_{O_2}\) falls below \(\sim 70\,\mathrm{mmHg}\) (\(9.3\,\mathrm{kPa}\)).
\end{itemize}

\subsubsection{Modifiers}
\begin{itemize}
  \item Both metabolic and respiratory acidosis augment HPV.
\end{itemize}

\subsection{FRCA/EDAIC exam hooks and pitfalls}
\begin{itemize}
  \item Zone 1 usually absent in healthy upright lung; appears with low PAP and/or high alveolar pressure (e.g.\ PPV).
  \item Zone 2 gradient: \(P_a - P_A\) (waterfall); venous pressure relatively unimportant.
  \item PVR lowest at FRC (U-shaped vs lung volume).
  \item PVR shared across arteries/capillaries/veins (not ``all arteriolar'').
  \item HPV depends on alveolar \(P_{O_2}\) (not arterial).
  \item Bronchial venous drainage to pulmonary veins contributes to venous admixture; LV output slightly exceeds RV output.
\end{itemize}

\subsection{Additions from other core texts (topics not emphasised in Kam \& Power)}

\subsubsection{Pulmonary artery catheter (PAC), wedge pressure, and derived variables}
\begin{itemize}
  \item A correctly wedged PAC distal lumen is exposed to pulmonary capillary pressure which, in the absence of high airway pressures or pulmonary vascular disease, approximates left atrial pressure.
  \item PAOP is an indirect estimate of LVEDP (and LV end-diastolic volume) but may be inaccurate when ventricular compliance is abnormal.
  \item Limitations as preload surrogates: altered compliance, mitral disease, positive intrathoracic pressure (e.g.\ PEEP) increasing measured PAOP, hydrostatic effects in dependent lung, pulmonary hypertension affecting relationships.
  \item Derived PVR (clinical): \((\\text{mean PAP} - \\text{PAOP}) \\times 80 / \\text{CO}\).
\end{itemize}
\noindent\textit{Sources: Morgan \& Mikhail’s Clinical Anesthesiology (PAC/PAOP principles; derived variables). Fundamentals of Anaesthesia (limitations of CVP/PCWP as preload measures).}

\subsubsection{Pulmonary hypertension (PH): causes and pathophysiology}
\begin{itemize}
  \item Mechanisms/examples: intracardiac shunt (ASD/VSD); raised LVEDP (mitral stenosis, constrictive pericarditis); obliteration of vascular bed (pulmonary fibrosis); obstruction (pulmonary embolism); vasoconstriction (sleep apnoea, high altitude).
  \item Chronic PH/PAH framing: sustained vasoconstriction, vascular remodeling, in situ thrombosis and increased vascular wall stiffness increase PVR/PAP \(\rightarrow\) RV afterload and eventual RV failure.
\end{itemize}
\noindent\textit{Sources: Fundamentals of Anaesthesia (mechanisms/examples). Ganong’s Review of Medical Physiology (remodeling/thrombosis/wall stiffness contributors).}

\subsubsection{HPV and PVR: drug modifiers (often tested)}
\begin{itemize}
  \item Attenuate HPV: volatile agents, nitrates, nitroprusside, calcium channel blockers, bronchodilators.
  \item Potentiate HPV: cyclo-oxygenase inhibitors, propranolol, almitrine.
  \item Pharmacology hook: calcium channel blockers (e.g.\ nifedipine) can inhibit HPV.
\end{itemize}
\noindent\textit{Sources: Fundamentals of Anaesthesia (drug modifiers of HPV). Primary FRCA in a Box (nifedipine inhibits HPV).}

\subsubsection{Fetal/neonatal pulmonary circulation and transition at birth}
\begin{itemize}
  \item In utero: high PVR supports right-to-left shunting (ductus arteriosus/foramen ovale).
  \item At birth: increased pulmonary blood flow and left-sided pressures contribute to functional closure of foramen ovale; increased arterial oxygen tension contributes to ductus arteriosus constriction/functional closure.
  \item Hypoxia/acidosis can prevent/reverse transition \(\rightarrow\) persistent pulmonary hypertension of the newborn (PPHN) with right-to-left shunt and worsening hypoxaemia/acidosis.
\end{itemize}
\noindent\textit{Source: Morgan \& Mikhail’s Clinical Anesthesiology (fetal circulation/transition at birth).}

\subsubsection{Pulmonary oedema and Starling forces}
\begin{itemize}
  \item Pulmonary oedema: movement of fluid from pulmonary capillaries \(\rightarrow\) interstitium \(\rightarrow\) alveoli.
  \item Starling equation framing: balance of hydrostatic vs oncotic pressures; filtration coefficient \(K\) and reflection coefficient \(\sigma\).
  \item Gravity influences pulmonary capillary hydrostatic pressure; balance of forces usually prevents alveolar flooding.
\end{itemize}
\noindent\textit{Source: Morgan \& Mikhail’s Clinical Anesthesiology (pulmonary oedema pathophysiology; Starling forces).}

\subsubsection{Targeted PH/PAH pharmacology (mechanism-focused)}
\begin{itemize}
  \item PDE5 inhibitor (sildenafil): increases cGMP \(\rightarrow\) pulmonary vascular smooth muscle relaxation.
  \item Endothelin receptor antagonist (bosentan): reduces pulmonary (and systemic) vascular resistance.
\end{itemize}
\noindent\textit{Source: Peck \& Harris, Pharmacology for Anaesthesia and Intensive Care (vasodilators / PAH drugs sections).}

\subsubsection{Primary source note}
\noindent\textit{Primary rewrite source: Kam \& Power, Principles of Physiology for the Anaesthetist, Chapter 20 (Pulmonary Circulation).}

%==========================================
% Fragment for \input{} / \include{}
% NOTE: Hierarchy adjusted as requested:
% canvas "Chapter" -> \section
% canvas "Section" -> \subsection
% canvas "Subsection" -> \subsubsection
% deeper headings -> \paragraph

\section{Chapter 21 -- Control of Ventilation }
\textbf{Sources:} Kam \& Power, \textit{Principles of Physiology for the Anaesthetist}, Ch 21 (Control of ventilation).\\
Additions cross-checked with: \textit{Ganong’s Review of Medical Physiology} (respiratory control / periodic breathing / altitude); \textit{Fundamentals of Anaesthesia} 4e (respiratory physiology; anaesthetic gases \& vapours; analgesic drugs); \textit{Primary FRCA in a Box} 2e (Respiration: Control of breathing).

\subsection{Big picture: what is being controlled?}
Ventilation is adjusted by \textbf{tidal volume (VT)} and \textbf{respiratory frequency} to match metabolic needs.
Control is via:
\begin{itemize}
  \item \textbf{Voluntary control}: cortex $\rightarrow$ corticospinal tracts $\rightarrow$ respiratory motor neurones.
  \item \textbf{Involuntary (automatic) control}: medullary rhythm generation shaped by pontine networks and feedback.
\end{itemize}

\subsection{Core architecture of ventilatory control}
Four components:
\begin{enumerate}
  \item Brainstem centres (medulla + pons)
  \item Chemoreceptors (CO$_2$, O$_2$, and [H$^+$])
  \item Mechanoreceptors (lung + chest wall / muscles / joints)
  \item Effectors (respiratory muscles via spinal motor neurones)
\end{enumerate}
Inputs integrated by the medullary respiratory centre include cortex/pons/hypothalamus, lung/upper airway afferents, baroreceptors, joint/muscle receptors, and central/peripheral chemoreceptor feedback.

\subsection{Effectors and reciprocal inhibition}
\subsubsection{Spinal motor outflow}
\begin{itemize}
  \item Cervical motor neurones $\rightarrow$ diaphragm (phrenic nerve)
  \item Thoracic motor neurones $\rightarrow$ intercostals and other expiratory muscles
\end{itemize}

\subsubsection{Reciprocal inhibition}
Inspiratory and expiratory motor neurones inhibit each other to generate alternating phases. A brief post-inspiratory phrenic activity can help ``brake'' lung elastic recoil and smooth breathing.

\subsection{Medullary centres: rhythm generation and pattern}
\subsubsection{Dorsal respiratory group (DRG)}
Near the nucleus tractus solitarius (termination of IX and X afferents). Predominantly inspiratory neurones; controls normal tidal inspiration (diaphragm) and contributes to timing.

\subsubsection{Ventral respiratory group (VRG)}
Contains inspiratory and expiratory neurones. Includes regions associated with airway patency and active expiration (e.g.\ nucleus ambiguus, nucleus retroambigualis; Botzinger and pre-Botzinger complexes). Largely recruited when ventilation increases (e.g.\ exercise/active expiration).

\subsubsection{Phases of respiration (neural pattern)}
\begin{enumerate}
  \item \textbf{Inspiratory phase}: onset with a rising ``ramp'' of inspiratory activity.
  \item \textbf{Early expiratory phase}: declining inspiratory tone (post-inspiratory activity).
  \item \textbf{Late expiratory phase}: inspiratory activity inactive; expiratory muscles recruited mainly when demand rises.
\end{enumerate}

\subsection{Pontine centres: shaping inspiratory timing and VT}
\subsubsection{Apneustic centre (lower pons)}
Associated with prolongation of the inspiratory ramp; lesions can produce apneustic breathing patterns.

\subsubsection{Pneumotaxic centre (upper pons)}
Inhibits inspiratory ramp activity: reduces VT and increases respiratory rate.

\subsection{Chemoreceptors}
\subsubsection{Central chemoreceptors}
Located in the medulla (beneath ventral surface). Primary stimulus is \textbf{CSF [H$^+$]}, driven acutely by arterial PCO$_2$ (CO$_2$ diffuses readily across BBB; generates H$^+$ in CSF). In chronic hypercapnia, CSF bicarbonate adjustment reduces the CSF [H$^+$] stimulus (``reset'' of CO$_2$ drive).

\subsubsection{Peripheral chemoreceptors}
\paragraph{Locations and afferents.}
\begin{itemize}
  \item \textbf{Carotid bodies}: bifurcation of common carotid; afferent via IX (dominant in humans).
  \item \textbf{Aortic bodies}: aortic arch; afferent via X.
\end{itemize}

\paragraph{What they sense.}
They sense \textbf{arterial PO$_2$} (high flow; dissolved oxygen reflects arterial PO$_2$). Glomus (type 1) cell transduction: K$^+$ channel inhibition $\rightarrow$ depolarisation $\rightarrow$ Ca$^{2+}$ influx $\rightarrow$ transmitter release.

\paragraph{Stimuli (exam-relevant).}
\begin{itemize}
  \item $\downarrow$PaO$_2$: carotid and aortic bodies stimulated
  \item $\uparrow$PaCO$_2$: carotid and aortic bodies stimulated
  \item $\downarrow$pH (metabolic acidaemia): carotid bodies stimulated (aortic body contribution commonly taught as minimal/absent)
\end{itemize}

\paragraph{Pitfall.}
Respond to \textbf{O$_2$ tension not O$_2$ content} (anaemia/COHb may have normal PaO$_2$).

\subsection{Mechanoreceptors and other non-chemical inputs}
\subsubsection{Lung receptors}
\begin{enumerate}
  \item \textbf{Slow-adapting stretch receptors}: inflation prolongs expiration (Hering--Breuer-related reflexes).
  \item \textbf{Rapidly adapting receptors}: respond to rate of change; promote rapid shallow breathing; airway irritants; may contribute to sighs; Head's paradoxical reflex described in neonates/anaesthesia contexts.
  \item \textbf{C-fibre (J) receptors}: near pulmonary capillaries; respond to interstitial oedema and mediators; can cause apnoea followed by rapid shallow breathing plus bradycardia/hypotension.
\end{enumerate}

\subsubsection{Muscle/joint receptors, baroreceptors, temperature, hormones}
Limb movement and muscle spindle feedback can stimulate ventilation; baroreceptor-mediated changes link BP to ventilation; increased temperature stimulates respiration; catecholamines stimulate ventilation.

\subsection{Reflex ventilatory responses}
\subsubsection{Raised arterial PCO$_2$}
PCO$_2$ is the key controlled variable under normal conditions; CO$_2$ increases ventilation via central and peripheral chemoreceptors (central predominant). CO$_2$ response is often described as linear over a broad physiological range with an extrapolated apnoeic threshold; opioids shift the curve rightwards and reduce slope; extreme hypercapnia can depress respiration.

\subsubsection{Falling arterial PO$_2$ (hypoxic ventilatory response)}
Driven by peripheral chemoreceptors; ventilatory increase becomes prominent at lower PaO$_2$ (commonly taught threshold around 60\,mmHg / 8\,kPa). The curve is classically hyperbolic and is potentiated by higher PaCO$_2$; low PaO$_2$ potentiates CO$_2$ responsiveness.

\subsubsection{Raised arterial [H$^+$] (acidaemia)}
Carotid bodies respond to increased [H$^+$] (from CO$_2$-derived or metabolic acids) and increase ventilation; BBB limits direct H$^+$ effects on CSF acutely.

\subsection{Exam-focused synthesis}
Structure a viva/SAQ answer by: controlled variables (VT, rate) $\rightarrow$ controllers (medulla/pons/cortex) $\rightarrow$ sensors (central + peripheral chemoreceptors; mechanoreceptors) $\rightarrow$ effectors (respiratory muscles) $\rightarrow$ key curves/interactions (CO$_2$, O$_2$, pH; drug effects).

\subsection{Common pitfalls}
\begin{itemize}
  \item O$_2$ \textbf{tension} vs content: carotid bodies respond to PaO$_2$, not anaemia/COHb content changes.
  \item Central chemoreceptors sense CO$_2$ predominantly via CSF [H$^+$], not PaO$_2$ directly.
  \item Chronic hypercapnia: CO$_2$ drive can reset; hypoxic drive becomes proportionally more important.
  \item Opioids: blunt CO$_2$ responsiveness and increase apnoeic risk.
\end{itemize}

\subsection{Additions from other core texts (not emphasised in Kam \& Power Ch 21)}
\subsubsection{Sleep-related control and periodic breathing}
\paragraph{Sleep apnoea.}
Sleep apnoea is classified into \textbf{central} (loss of drive) and \textbf{obstructive} (upper airway obstruction).
\paragraph{Cheyne--Stokes / periodic breathing.}
Cheyne--Stokes respiration / periodic breathing is described in heart failure and uraemia; mechanisms include increased CO$_2$ sensitivity and/or delayed feedback due to prolonged circulation time.
\paragraph{Source.}
Ganong’s Review of Medical Physiology (Clinical Box on periodic breathing/sleep apnoea).

\subsubsection{Time course of CO$_2$ response adaptation}
Ventilatory response to elevated CO$_2$ is described as greatest early, then declining over about 48\,h as bicarbonate-related compensation reduces CSF [H$^+$] stimulus.
\paragraph{Source.}
Fundamentals of Anaesthesia 4e, Respiratory physiology (ventilatory response to CO$_2$).

\subsubsection{Isocapnic hypoxic response and CO$_2$ interaction (extra nuance)}
Hypoxic ventilatory curves are often considered under \textbf{isocapnic} conditions; fixing PaCO$_2$ at low values can blunt hypoxic stimulation until lower PO$_2$ levels. Hypoxia increases the \textbf{slope} of the CO$_2$ response without necessarily changing the intersection threshold described in teaching figures.
\paragraph{Sources.}
Ganong’s Review of Medical Physiology (figures on O$_2$/CO$_2$ interaction); Fundamentals of Anaesthesia 4e (O$_2$ response discussion).

\subsubsection{Altitude-related control}
Acclimatisation is explained partly by changes in CSF acid--base over days, which permits sustained hyperventilation at altitude; periodic breathing can occur, especially during sleep.
\paragraph{Sources.}
Ganong’s Review of Medical Physiology (acclimatisation section); Fundamentals of Anaesthesia 4e (periodic respiration noted with hypoxaemia).

\subsubsection{Anaesthetic/drug effects beyond opioids}
\begin{itemize}
  \item Volatile agents reduce ventilatory drive; hypoxic ventilatory response may be blunted at subanaesthetic concentrations.
  \item Worked teaching example: halothane reduces VT, increases RR, and shifts the CO$_2$ response curve rightwards.
  \item Opioid summaries often include reduced carotid body chemoreception/hypoxic drive and preserved voluntary control (in addition to right shift/reduced slope).
\end{itemize}
\paragraph{Source.}
Fundamentals of Anaesthesia 4e (respiratory physiology; anaesthetic vapours; analgesic drugs summary figure).

\subsubsection{Fast viva structure prompt}
A condensed receptor/centre/effector framework is presented for exam answers.
\paragraph{Source.}
Primary FRCA in a Box 2e (Respiration: control of breathing).

%======================================

\section{Chapter 22 — Applied Respiratory Physiology (Kam \& Power)}

\subsection{Physiological effects of intermittent positive pressure ventilation (IPPV)}

\subsubsection{Core idea}
IPPV raises intrathoracic pressure during inspiration. This changes \textbf{lung mechanics} and \textbf{heart--lung interactions}, with downstream \textbf{renal/hepatic/endocrine} effects.

\subsubsection{Respiratory effects}

\paragraph{A) Functional residual capacity (FRC).}
\begin{itemize}
  \item IPPV can \textbf{increase FRC} by recruiting alveoli.
  \item However, in anaesthesia the net effect on FRC depends on: baseline lung volume, compliance, airway closure/atelectasis, and whether PEEP is applied.
\end{itemize}

\paragraph{B) Alveolar dead space.}
\begin{itemize}
  \item IPPV can \textbf{increase alveolar dead space}.
\end{itemize}

\paragraph{C) Ventilation distribution and gas exchange (conceptual).}
\begin{itemize}
  \item Positive pressure preferentially expands more compliant lung units.
  \item Heterogeneity of compliance/resistance can lead to uneven ventilation.
\end{itemize}

\subsubsection{Cardiovascular effects}

\paragraph{Big picture.}
During IPPV, the dominant haemodynamic effect is usually \textbf{reduced right and left ventricular performance}, largely via reduced venous return and altered ventricular loading.

Key determinants:
\begin{itemize}
  \item \textbf{Magnitude of mean airway/intrapleural pressure} ($\uparrow$ with higher airway pressures and with PEEP).
  \item \textbf{Intravascular volume status}.
  \item \textbf{Ventricular function} and pulmonary vascular resistance.
\end{itemize}

\paragraph{Mechanism: venous return and right heart (RV).}
\textbf{Main mechanism:} increased intrathoracic pressure $\rightarrow$ \textbf{$\downarrow$ pressure gradient for venous return} $\rightarrow$ \textbf{$\downarrow$ RV preload} $\rightarrow$ \textbf{$\downarrow$ RV stroke volume}.

Additional RV mechanisms:
\begin{itemize}
  \item \textbf{$\uparrow$ RV afterload} if pulmonary vascular resistance rises (e.g.\ overdistension/compression of pulmonary vessels at high lung volumes).
\end{itemize}

\paragraph{Left ventricle (LV) and systemic circulation.}
Mechanisms affecting LV and systemic circulation include:
\begin{itemize}
  \item \textbf{$\downarrow$ LV preload} due to lower pulmonary venous pressure.
  \item \textbf{$\downarrow$ LV afterload} ($\downarrow$ LV end-systolic transmural pressure; $\uparrow$ pressure gradient between intrathoracic aorta and extrathoracic systemic circuit).
  \item Net effect commonly: \textbf{$\downarrow$ LV stroke volume}.
\end{itemize}

\textbf{Ventricular interdependence}
\begin{itemize}
  \item \textbf{$\uparrow$ RV volume} can reduce LV filling.
  \item Septal shift \textbf{towards LV} contributes to reduced LV volume.
\end{itemize}

\textbf{Important nuance (exam trap): early inspiratory LV output can transiently rise}
\begin{itemize}
  \item A less appreciated effect is an \textbf{initial increase in LV stroke volume during early inspiration}, attributed to:
  \begin{itemize}
    \item LV compression from increased intrathoracic pressure.
    \item Reduced LV afterload via increased intra- vs extra-thoracic pressure gradient.
  \end{itemize}
  \item This may improve output in some patients with \textbf{poor LV reserve}.
\end{itemize}

\textbf{Overall pattern:} initial transient $\uparrow$ LV output $\rightarrow$ followed by \textbf{prolonged $\downarrow$ LV output}.

\paragraph{Effects on organ perfusion gradients (renal/hepatic).}
\begin{itemize}
  \item Higher central venous pressures reduce the perfusion pressure gradient across renal and hepatic circulations.
  \item With prolonged IPPV: \textbf{$\downarrow$ renal blood flow} and \textbf{hepatic congestion} may occur.
\end{itemize}

\subsubsection{Endocrine effects}
Driven mainly by reduced venous return and reduced cardiac output:
\begin{itemize}
  \item \textbf{$\downarrow$ atrial stretch} (low-pressure receptors) $\rightarrow$ \textbf{$\downarrow$ atrial natriuretic peptide (ANP)}.
  \item Reflex \textbf{$\uparrow$ antidiuretic hormone (ADH)}.
  \item Reduced renal blood flow $\rightarrow$ \textbf{$\uparrow$ angiotensin and aldosterone}.
  \item Reduced cardiac output $\rightarrow$ activation of high-pressure baroreceptor reflexes $\rightarrow$ \textbf{$\uparrow$ sympathetic activity}.
\end{itemize}

\subsection{Breath holding}

\subsubsection{What happens to gases during breath holding?}
\begin{itemize}
  \item After breath holding on air, alveolar gas reaches equilibrium with mixed venous blood within minutes.
  \item \textbf{PaCO$_2$ rises} at $\sim$\textbf{3--6 mmHg/min} (0.4--0.8 kPa/min).
\end{itemize}

\subsubsection{Airway patency and ambient gas determine the course}

\paragraph{A) Airway occluded.}
\begin{itemize}
  \item Alveolar \textbf{PO$_2$ falls towards mixed venous PO$_2$ within $\sim$1 min}.
\end{itemize}

\paragraph{B) Airway patent, ambient air.}
\begin{itemize}
  \item Ambient air is drawn in to replace volume loss during apnoea.
  \item Alveolar \textbf{nitrogen accumulates}, and \textbf{hypoxia occurs after $\sim$2 min}.
\end{itemize}

\paragraph{C) Airway patent, ambient oxygen.}
\begin{itemize}
  \item Oxygen removed from alveoli is replaced by oxygen drawn in via mass movement.
  \item With no nitrogen added, alveolar PO$_2$ falls roughly as fast as PCO$_2$ rises ($\sim$3--6 mmHg; 0.4--0.8 kPa).
  \item Serious hypoxia is delayed for several minutes.
\end{itemize}

\subsubsection{Break point (BP)}
\begin{itemize}
  \item BP = when urge to breathe overwhelms voluntary breath holding.
  \item After breathing air, BP occurs at \textbf{PCO$_2$ $\sim$50 mmHg (6.7 kPa)}.
  \item BP is not purely CO$_2$-driven; concomitant hypoxia is likely important.
\end{itemize}

\textbf{Factors prolonging breath holding}
\begin{itemize}
  \item Preliminary oxygen breathing (delays hypoxia).
  \item Larger lung volume at onset ($\uparrow$ O$_2$ stores; plus afferent effects).
  \item Hyperventilation + preoxygenation.
\end{itemize}

\subsection{Preoxygenation (denitrogenation) — breathing 100\% oxygen}

\subsubsection{Aim}
\begin{itemize}
  \item Increase oxygen stores (primarily in \textbf{FRC}) to extend safe apnoea time.
  \item Mechanistically: \textbf{denitrogenation} of FRC.
\end{itemize}

\subsubsection{Wash-in / washout behaviour}
\begin{itemize}
  \item Nitrogen washout is \textbf{exponential}.
  \item Exhaled N$_2$ concentration can reach $\sim$\textbf{5\% after 2--3 min} of breathing 100\% O$_2$.
\end{itemize}

\subsubsection{3.3 Physiological basis (numbers from the text)}

\paragraph{Breathing air (example adult 70 kg).}
\begin{itemize}
  \item FRC $\sim$2100 mL; FAO$_2$ $\sim$13\% $\rightarrow$ O$_2$ in FRC $\sim$270 mL.
  \item With VO$_2$ $\sim$250 mL/min, FRC alone would supply O$_2$ for $\sim$70 s.
  \item But not all O$_2$ can be extracted: desaturation occurs when alveolar PO$_2$ $<$ \textbf{6 kPa (45 mmHg)}.
  \item Therefore only $\sim$\textbf{150 mL O$_2$} can be taken into blood $\rightarrow$ desaturation in \textbf{$<$1 min}.
\end{itemize}

\paragraph{Breathing 100\% O$_2$ (wash-in kinetics).}
\begin{itemize}
  \item Rate depends directly on \textbf{alveolar ventilation} and inversely on \textbf{FRC}.
  \item If alveolar ventilation is 3000 mL/min, half-time for wash-in:
  \[
    2100 \times 0.693 / 3000 = \textbf{0.48 min}.
  \]
  \item To achieve $\sim$95\% exchange requires \textbf{five time constants}:
  \[
    5 \times 0.48 = \textbf{2.4 min}.
  \]
  \item This can achieve alveolar PO$_2$ $\sim$\textbf{660 mmHg} ($\sim$85\% O$_2$ in alveolus), i.e.\ $\sim$\textbf{1800 mL O$_2$} in FRC.
  \item This is $\sim$\textbf{8 min} worth of oxygen consumption.
\end{itemize}

\subsubsection{Oxygen stores: air vs 100\% oxygen}
\begin{center}
\begin{tabular}{lrr}
\hline
\textbf{Compartment} & \textbf{Breathing air (mL)} & \textbf{Breathing 100\% O$_2$ (mL)} \\
\hline
Lungs (FRC) & 450 & 3000 \\
Blood & 850 & 950 \\
Dissolved in tissue fluids & 50 & $\sim$100 \\
Myoglobin & $\sim$200 & $\sim$200 \\
\textbf{Total} & \textbf{1550} & \textbf{4250} \\
\hline
\end{tabular}
\end{center}

\subsubsection{Effects on mixed venous oxygen}
\begin{itemize}
  \item With 100\% O$_2$, dissolved plasma O$_2$ increases from \textbf{0.3 mL/100 mL} to about \textbf{2 mL/100 mL} (Hb already fully saturated).
  \item Total arterial O$_2$ content increases from \textbf{20 mL/100 mL} to \textbf{22 mL/100 mL}.
  \item With A--V O$_2$ difference 5 mL/100 mL:
  \begin{itemize}
    \item Mixed venous O$_2$ content increases from \textbf{15} to about \textbf{18 mL/100 mL}.
    \item Mixed venous PO$_2$ $\approx$ \textbf{60 mmHg}.
  \end{itemize}
\end{itemize}

\textbf{Alveolar / arterial / mixed venous gases (breathing 100\% O$_2$; table values):}
\begin{center}
\begin{tabular}{lrrr}
\hline
 & \textbf{Alveolar (mmHg)} & \textbf{Arterial (mmHg)} & \textbf{Mixed venous (mmHg)} \\
\hline
PO$_2$ & 673 & 640 & 60 \\
PCO$_2$ & 40 & 40 & 46 \\
PH$_2$O & 47 & 47 & 47 \\
PN$_2$ & 0 & 0 & 0 \\
\textbf{Total} & 760 & 727 & 153 \\
\hline
\end{tabular}
\end{center}

\subsection{Apnoeic oxygenation}

\subsubsection{Initial equilibration during apnoea}
\begin{itemize}
  \item Alveolar gas equilibrates with mixed venous blood within minutes.
  \item Alveolar PCO$_2$ rises from \textbf{5.5 $\rightarrow$ 6.1 kPa (40--46 mmHg)}.
  \item PaO$_2$ decreases from \textbf{14 $\rightarrow$ 5.3 kPa (100--40 mmHg)}.
  \item Alveolar CO$_2$ equilibrates within \textbf{$\sim$10 s}; alveolar PO$_2$ takes about \textbf{$\sim$1 min}.
\end{itemize}

\textbf{Net gas flux concept (numbers in text):}
\begin{itemize}
  \item O$_2$ movement alveoli $\rightarrow$ blood: $\sim$\textbf{250 mL/min}.
  \item CO$_2$ movement blood $\rightarrow$ alveoli: $\sim$\textbf{20 mL/min}.
  \item $\sim$\textbf{180 mL/min} CO$_2$ buffered in blood.
  \item Net gas exchange: \textbf{$\sim$209 mL/min}, causing subatmospheric alveolar pressure and \textbf{mass inflow} from pharynx to alveoli if airway is patent.
\end{itemize}

\subsubsection{Changes with obstructed airway}
\begin{itemize}
  \item \textbf{PCO$_2$:} after $\sim$10 s equilibration, arterial/alveolar/mixed venous PCO$_2$ increase together at \textbf{3--6 mmHg/min} (0.4--0.9 kPa/min).
  \begin{itemize}
    \item Myocardial depression at PaCO$_2$ $\sim$\textbf{9--10 kPa}.
    \item Cerebral blood flow rises linearly to a maximum at PaCO$_2$ $\sim$\textbf{10.5 kPa}.
    \item CO$_2$ narcosis at PaCO$_2$ $\sim$\textbf{12 kPa}.
  \end{itemize}
  \item \textbf{PO$_2$:} PaO$_2$ steady for $\sim$2 min if breathing air, then falls at \textbf{4--6 mmHg/min} (0.5--0.9 kPa/min).
  \begin{itemize}
    \item If effectively preoxygenated (denitrogenated), \textbf{8--10 min} may elapse before desaturation becomes evident.
  \end{itemize}
\end{itemize}

\subsubsection{Changes with patent (non-obstructed) airway}

\paragraph{A) Ambient gas = air.}
\begin{itemize}
  \item Alveolar N$_2$ rises as O$_2$ is removed.
  \item After $\sim$2 min: alveolar N$_2$ reaches $\sim$90\% (with alveolar CO$_2$ $\sim$8\%).
  \item Hypoxia/desaturation supervenes after $\sim$2 min.
  \item PaCO$_2$ rises \textbf{3--6 mmHg/min} throughout.
\end{itemize}

\paragraph{B) Ambient gas = oxygen.}
\begin{itemize}
  \item O$_2$ removed from alveoli is replaced by mass movement of O$_2$.
  \item PaCO$_2$ rises \textbf{3--6 mmHg/min}; PaO$_2$ falls at a similar rate.
  \item If effective denitrogenation with 100\% O$_2$ and airway patent:
  \begin{itemize}
    \item Theoretical non-desaturation time $\sim$\textbf{100 min}.
  \end{itemize}
\end{itemize}

\paragraph{THRIVE (high-flow nasal oxygen).}
\begin{itemize}
  \item High flows \textbf{20--70 L/min} to prolong apnoeic mass oxygenation.
  \item Up to \textbf{55 min} apnoea time reported with maintained patency.
  \item CO$_2$ rises less than classical apnoeic oxygenation; proposed mechanism: interaction between turbulent supraglottic vortices and cardiogenic oscillations.
\end{itemize}

\subsection{Hypoxia}

\subsubsection{Definition and critical values}
\begin{itemize}
  \item Hypoxia = inadequate tissue oxygenation.
  \item Critical PO$_2$:
  \begin{itemize}
    \item Capillary PO$_2$ $\sim$\textbf{6--7 kPa (40 mmHg)}.
    \item Mitochondrial PO$_2$ $\sim$\textbf{1--5 kPa (4--22 mmHg)}.
  \end{itemize}
\end{itemize}

\subsubsection{Responses to hypoxaemia}

\paragraph{Cellular response.}
\begin{itemize}
  \item Below PO$_2$ $\sim$\textbf{50 mmHg}, aerobic metabolism is impaired.
\end{itemize}

\paragraph{Organ-level responses.}
\begin{itemize}
  \item CNS: \textbf{headache, impaired judgement and coordination}.
  \item CVS: \textbf{tachycardia}, with risk of ventricular arrhythmias.
\end{itemize}

\subsubsection{Classification of hypoxia}

\paragraph{A) Hypoxic (hypoxaemic) hypoxia.}
\begin{itemize}
  \item Low arterial PO$_2$ due to:
  \begin{itemize}
    \item Low inspired PO$_2$
    \item Hypoventilation
    \item Diffusion impairment
    \item V/Q mismatch
    \item Shunt
  \end{itemize}
\end{itemize}

\paragraph{B) Anaemic hypoxia.}
\begin{itemize}
  \item Reduced O$_2$ carrying capacity despite PaO$_2$ possibly normal.
  \item Examples:
  \begin{itemize}
    \item Low haemoglobin
    \item Carbon monoxide poisoning
  \end{itemize}
\end{itemize}

\paragraph{C) Stagnant (ischaemic) hypoxia.}
\begin{itemize}
  \item Inadequate tissue blood flow.
  \item Examples:
  \begin{itemize}
    \item Circulatory failure (e.g.\ cardiogenic shock)
    \item Arterial insufficiency
  \end{itemize}
\end{itemize}

\paragraph{D) Histotoxic hypoxia.}
\begin{itemize}
  \item Inadequate cellular utilization of O$_2$.
  \item Example:
  \begin{itemize}
    \item Cyanide poisoning
  \end{itemize}
\end{itemize}

\textbf{Clinical note embedded in the chapter:}
\begin{itemize}
  \item Low tissue PO$_2$ triggers anaerobic metabolism $\rightarrow$ lactic acid generation.
  \item Only a few organs (e.g.\ brain) require oxygen constantly and rely on a continuous supply.
\end{itemize}

\subsection{Additional applied issues emphasised in other core texts (not explicit / not developed in Kam \& Power Ch.22)}

\subsubsection{Preoxygenation: practical technique, endpoints, and common failure modes}
\begin{itemize}
  \item \textbf{How to do it (pragmatic):} 3--5 minutes of tidal ventilation at high fresh gas flows via a close-fitting facemask is typically sufficient for denitrogenation; \textbf{four vital capacity breaths} may be similarly effective.
  \item \textbf{How to know it’s working:} avoid leaks (suggested indicators include an absent/abnormal capnograph trace and an empty reservoir bag). Target an \textbf{end-tidal oxygen concentration as close to 100\% as possible}; $\sim$90\% is acceptable.
  \item \textbf{When it’s hard:} effective preoxygenation can be difficult in agitated patients (e.g.\ head injury), so you should actively pursue opportunities to deliver supplemental oxygen throughout airway management.
\end{itemize}

\subsubsection{Why some patients desaturate rapidly despite ``preoxygenation'': FRC and oxygen consumption}
\begin{itemize}
  \item The rate of desaturation during apnoea relates to \textbf{oxygen stores ($\approx$ FRC)} and \textbf{oxygen consumption}.
  \item \textbf{High-risk groups for rapid desaturation} highlighted in other texts:
  \begin{itemize}
    \item \textbf{Pregnancy:} ERV, RV and FRC \textbf{decrease by $\sim$20\% at term}; preoxygenation before anaesthesia is essential.
    \item \textbf{Obesity:} reduced FRC and compliance; \textbf{closing capacity may approach or exceed FRC}, contributing to airway closure and hypoxaemia; apnoea hypoxaemia may occur rapidly, reinforcing preoxygenation.
    \item \textbf{Older age:} increased V/Q mismatch and venous admixture; \textbf{closing capacity may enter the tidal ventilation range when supine}; reduced central response to hypoxaemia/hypercarbia.
  \end{itemize}
\end{itemize}

\subsubsection{Anaesthesia, reduced FRC, V/Q mismatch, and atelectasis as key mechanisms of perioperative hypoxaemia}
\begin{itemize}
  \item Other texts explicitly state that after induction of anaesthesia there can be a \textbf{$\sim$20\% reduction in FRC}, predisposing to dependent airway closure.
  \item This promotes \textbf{V/Q mismatch} (and venous admixture) and contributes to perioperative hypoxaemia.
  \item V/Q mismatch causes listed include \textbf{atelectasis}, bronchopneumonia, aspiration, pulmonary oedema, and pneumothorax.
  \item \textbf{Atelectasis} may develop within $\sim$15 minutes of induction and can persist postoperatively; \textbf{absorption atelectasis} is implicated as a contributing mechanism.
\end{itemize}

\subsubsection{Oxygen is not harmless: hazards of high inspired oxygen fractions}

\paragraph{A) Absorption atelectasis (mechanism and consequence).}
\begin{itemize}
  \item High oxygen concentrations can promote atelectasis in regions with low V/Q.
  \item Mechanism described elsewhere: nitrogen washout lowers alveolar gas tension; increased uptake of alveolar gas leads to \textbf{absorption atelectasis}, increasing intrapulmonary shunt and widening the A--a gradient.
\end{itemize}

\paragraph{B) Oxygen toxicity (time- and dose-dependent).}
\begin{itemize}
  \item Prolonged high oxygen concentrations may cause pulmonary toxicity; toxicity is described as dependent on both inspired PO$_2$ and duration.
  \item Other texts emphasise alveolar PO$_2$ (rather than arterial PO$_2$) as the key determinant.
\end{itemize}

\paragraph{C) Hypoventilation in chronic CO$_2$ retainers + monitoring trap.}
\begin{itemize}
  \item High-flow oxygen in COPD with chronic CO$_2$ retention may worsen CO$_2$ retention and respiratory depression.
  \item High FiO$_2$ can also mask hypoventilation if SpO$_2$ is used as the sole monitor (SpO$_2$ may remain ``reassuring'' while CO$_2$ rises).
\end{itemize}

\paragraph{D) Fire risk.}
\begin{itemize}
  \item Oxygen supports combustion and increases fire risk in the perioperative environment.
\end{itemize}

\subsubsection{Shunt physiology: why some hypoxaemia responds poorly to FiO$_2$}
\begin{itemize}
  \item Shunt is described as mixed venous blood bypassing oxygenation (intrapulmonary or anatomical).
  \item Clinical implication highlighted: \textbf{hypoxaemia due to shunt responds poorly to increases in FiO$_2$}, because end-capillary oxygen content is already near maximal on the flat portion of the O$_2$ dissociation curve (100\% O$_2$ mainly increases dissolved O$_2$).
\end{itemize}

\subsubsection{PEEP as an applied tool for atelectasis-related hypoxia}
\begin{itemize}
  \item Other texts more explicitly frame PEEP as a recruitment-maintaining tool: appropriate PEEP can help prevent atelectasis, maintain oxygenation, and keep the lung on a mechanically advantageous portion of the pressure--volume relationship.
  \item PEEP requires individual optimisation; \textbf{5 cmH$_2$O} is suggested as a reasonable starting point.
\end{itemize}

\subsection{Sources (for the added material in Section 6)}
\begin{itemize}
  \item \textbf{Fundamentals of Anaesthesia (4th ed.)}: Chapter 2 \emph{Conduct of anaesthesia} (Preoxygenation section); Chapter 4 \emph{Postoperative management} (V/Q mismatch; atelectasis; causes of postoperative hypoxaemia); Section 1 special circumstances (ageing table including respiratory changes).
  \item \textbf{Morgan \& Mikhail’s Clinical Anesthesiology (5th ed.)}: Oxygen therapy hazards (absorption atelectasis; pulmonary toxicity; hypoventilation in COPD).
  \item \textbf{Primary FRCA in a Box (2nd ed.)}: Cards on \emph{Pregnancy -- respiratory changes}; \emph{Obesity -- respiratory physiology/anaesthesia implications}; \emph{Shunt and shunt equation / clinical nugget on response to FiO$_2$}; \emph{Dangers of oxygen therapy}.
\end{itemize}


\end{document}
