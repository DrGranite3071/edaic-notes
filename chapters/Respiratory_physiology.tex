\documentclass[11pt,a4paper]{article}

% --- pdfLaTeX / TeX Live 2024 safe preamble ---
\usepackage[utf8]{inputenc}
\usepackage[T1]{fontenc}
\usepackage{lmodern}
\usepackage[a4paper,margin=2.2cm]{geometry}
\usepackage{microtype}
\usepackage{amsmath,amssymb}
\usepackage{booktabs}
\usepackage{tabularx}
\usepackage{enumitem}
\usepackage{hyperref}
\hypersetup{hidelinks}

\title{Respiratory System Study Notes\\Chapters 15--16 (Structured rewrite)}
\author{}
\date{\today}

\begin{document}
\maketitle

\noindent\textit{Source: Kam \& Power --- Principles of Physiology for the Anaesthetist, Chapters 15.}

\tableofcontents
\newpage

% =========================================================
\section{Chapter 15 --- Functions of the Respiratory System}

\subsection{Big picture}

\subsubsection{Primary function}
\begin{itemize}[leftmargin=*,nosep]
  \item \textbf{Gas exchange}: transfer of O$_2$ into blood and CO$_2$ out of blood across the alveolar--capillary interface.
\end{itemize}

\subsubsection{Non-respiratory functions of the lung}
\begin{itemize}[leftmargin=*,nosep]
  \item \textbf{Blood filter}: traps small clots and detached cells, limiting systemic embolization.
  \item \textbf{Blood reservoir (high capacitance)}
  \begin{itemize}[leftmargin=*,nosep]
    \item \textasciitilde16\% of total blood volume in the \textbf{supine} position.
    \item \textasciitilde9\% of total blood volume in the \textbf{erect} position.
    \item Can redistribute blood to vital organs in \textbf{hypovolaemic shock}.
  \end{itemize}
  \item \textbf{Metabolism / handling of bioactive substances}
  \begin{itemize}[leftmargin=*,nosep]
    \item Converts \textbf{angiotensin I $\rightarrow$ angiotensin II}.
    \item Synthesis and breakdown of \textbf{bradykinin}.
    \item Storage/release of \textbf{serotonin} and \textbf{histamine}.
    \item Synthesis of peptides (e.g.\ \textbf{substance P}), \textbf{prostaglandins}, \textbf{surfactants}, \textbf{immunoglobulins}.
    \item Inactivation of \textbf{adrenaline} and \textbf{noradrenaline}.
    \item Presence of \textbf{cytochrome P-450 isoenzymes}.
  \end{itemize}
  \item \textbf{Acid--base regulation}: ventilation changes arterial $P_{\mathrm{CO_2}}$.
  \item \textbf{Phonation}: CNS control of respiratory muscles generates airflow through vocal cords.
  \item \textbf{Pulmonary defence}
  \begin{itemize}[leftmargin=*,nosep]
    \item Secretion of \textbf{IgA} (innate immunity).
    \item Removal of airborne particles by \textbf{phagocytosis} and \textbf{mucociliary action}.
    \item \textbf{Lymphoid tissue with T lymphocytes} provides first-line defence against the external environment.
  \end{itemize}
\end{itemize}

\paragraph{Exam pitfalls}
\begin{itemize}[leftmargin=*,nosep]
  \item ``Lungs only do gas exchange'' is false: know key \textbf{non-respiratory} roles (filter/reservoir/metabolic/defence).
\end{itemize}

\subsection{Functional anatomy: airway tree, respiratory zone, alveolar--capillary unit}

\subsubsection{Conducting airways vs respiratory zone}

\paragraph{Conducting airways}
\begin{itemize}[leftmargin=*,nosep]
  \item Function: \textbf{bulk flow} to/from respiratory zone + \textbf{warm, humidify, filter} inspired air.
  \item Anatomical extent: \textbf{trachea (generation 1)} $\rightarrow$ \textbf{terminal bronchioles (generation 16)}.
  \item Conducting zone volume \textasciitilde \textbf{150 mL}.
  \item \textbf{Cartilage disappears from the 11th generation}; beyond this, airway diameter is mainly determined by \textbf{lung volume}.
  \item Airway wall \textbf{smooth muscle}: dilates with \textbf{sympathetic} stimulation; constricts with \textbf{parasympathetic} stimulation.
\end{itemize}

\paragraph{Respiratory zone}
\begin{itemize}[leftmargin=*,nosep]
  \item Begins at \textbf{respiratory bronchioles (generations 17--19)} (first airways with alveoli in their walls).
  \item Continues through \textbf{alveolar ducts (generations 20--22)} to \textbf{alveolar sacs (generation 23)}.
  \item Respiratory zone volume \textasciitilde \textbf{3000 mL}.
  \item Gas exchange occurs by \textbf{diffusion} (not bulk flow).
\end{itemize}

\subsubsection{Pulmonary lobule and alveolar stability}
\begin{itemize}[leftmargin=*,nosep]
  \item Parenchyma is an interconnected network of alveolar walls and interstitial tissues.
  \item \textbf{Alveolar interdependence}: a collapsing region tends to be pulled open by surrounding stretched tissue.
  \item With \textbf{surfactant} and \textbf{collateral ventilation via pores of Kohn} $\rightarrow$ helps prevent alveolar collapse.
  \item \textbf{Pulmonary lobule}: airways and alveoli distal to a single terminal bronchiole.
\end{itemize}

\subsubsection{Alveolar--capillary unit: key numbers and structure}
\begin{itemize}[leftmargin=*,nosep]
  \item \textbf{Alveoli}
  \begin{itemize}[leftmargin=*,nosep]
    \item \textbf{200--600 million} (average \textasciitilde\textbf{300 million}).
    \item Mean diameter at \textbf{FRC}: \textbf{0.2 mm}.
    \item \textbf{Polyhedral} (not spherical) because septae are flat.
    \item \textbf{Type I} cells: flat squamous epithelium (major gas-exchanging surface).
    \item \textbf{Type II} cells: cuboidal, produce \textbf{surfactant}.
  \end{itemize}
  \item \textbf{Pulmonary capillaries}
  \begin{itemize}[leftmargin=*,nosep]
    \item Diameter \textasciitilde\textbf{10 $\mu$m}.
    \item Endothelial thickness \textasciitilde\textbf{0.1 $\mu$m}.
    \item RBC transit through capillary network (and 2--3 alveoli): \textasciitilde\textbf{0.75 s}.
  \end{itemize}
  \item \textbf{Diffusion barrier thickness} (alveolar gas $\rightarrow$ capillary blood): \textasciitilde\textbf{0.3 $\mu$m} (RBC diameter \textasciitilde\textbf{7 $\mu$m}).
  \item \textbf{Surface area}: \textasciitilde\textbf{50--100 m$^2$}.
\end{itemize}

\paragraph{Exam pitfalls}
\begin{itemize}[leftmargin=*,nosep]
  \item Conducting zone (bulk flow; \textasciitilde150 mL) vs respiratory zone (diffusion; \textasciitilde3000 mL).
  \item Type I (exchange) vs Type II (surfactant).
  \item Generation 11 (cartilage disappears) vs generation 16 (end of conducting zone).
\end{itemize}

\subsection{Muscles of ventilation}

\subsubsection{Mechanics overview}
\begin{itemize}[leftmargin=*,nosep]
  \item Thorax separated from abdomen by the \textbf{diaphragm}.
  \item Diaphragm contraction increases vertical chest dimension by pushing abdominal contents down.
  \item Ribs move laterally/anteriorly to increase thoracic cross-sectional area.
  \item \textbf{Inspiration}: active. \quad \textbf{Expiration}: passive during quiet breathing; becomes active when ventilation increases.
\end{itemize}

\subsubsection{Inspiratory muscles}
\begin{itemize}[leftmargin=*,nosep]
  \item \textbf{Diaphragm}
  \begin{itemize}[leftmargin=*,nosep]
    \item Innervation: \textbf{phrenic nerve (C3--5)}.
    \item Descent: \textbf{1--2 cm} in quiet breathing; up to \textbf{10 cm} in forced inspiration.
  \end{itemize}
  \item \textbf{External intercostals}: fibres slope down and anteriorly; ribs move \textbf{upwards and forwards}.
  \item \textbf{Scalenes}: active even during quiet breathing; elevate ribcage.
  \item \textbf{Sternocleidomastoid}: recruited when breathing increases; elevates ribcage.
\end{itemize}

\subsubsection{Expiratory muscles}
\begin{itemize}[leftmargin=*,nosep]
  \item Quiet expiration is \textbf{passive}.
  \item Active expiration occurs with increased ventilation (exercise, speech, coughing/sneezing) and in pathology (e.g.\ COPD).
  \item Main expiratory muscles: \textbf{abdominal wall} (rectus abdominis, internal/external obliques, transversalis) and \textbf{internal intercostals}.
\end{itemize}

\paragraph{Exam pitfalls}
\begin{itemize}[leftmargin=*,nosep]
  \item Quiet expiration is passive; \textbf{forced} expiration recruits abdominal wall + internal intercostals.
\end{itemize}

\subsection{Lung--chest wall equilibrium and pressure definitions}

\subsubsection{Pleural arrangement}
\begin{itemize}[leftmargin=*,nosep]
  \item Lungs covered by \textbf{visceral pleura}; chest wall lined by \textbf{parietal pleura}.
  \item Between them is a potential \textbf{intrapleural space}.
  \item Diaphragm separates lungs from abdominal contents.
\end{itemize}

\subsubsection{Resting equilibrium (end of normal expiration)}
\begin{itemize}[leftmargin=*,nosep]
  \item Elastic forces balance:
  \begin{itemize}[leftmargin=*,nosep]
    \item Lung tends to collapse inward.
    \item Chest wall tends to expand outward (plus contribution from diaphragmatic tone).
  \end{itemize}
  \item This produces a \textbf{negative intrapleural pressure}.
\end{itemize}

\subsubsection{Transpulmonary pressure}
\begin{itemize}[leftmargin=*,nosep]
  \item \textbf{Transpulmonary pressure} $=$ alveolar pressure $-$ intrapleural pressure.
  \item Distending (transmural) pressure across alveoli.
  \item At end of normal expiration, forces balance at \textbf{FRC}.
\end{itemize}

\paragraph{Exam pitfalls}
\begin{itemize}[leftmargin=*,nosep]
  \item Transpulmonary pressure is a \textbf{difference}, not the same as intrapleural pressure.
\end{itemize}

\subsection{Events during a normal tidal breath (Table 15.1)}

\subsubsection{Summary table (spontaneous breathing)}
\begin{table}[h!]
\centering
\begin{tabularx}{\textwidth}{@{}lXX@{}}
\toprule
\textbf{Step} & \textbf{Inspiration (active)} & \textbf{Expiration (passive, quiet breathing)}\\
\midrule
Neural drive & Inspiratory centre activated $\rightarrow$ impulses to inspiratory muscles & Inspiratory centre activity ceases\\
Muscle activity & Diaphragm contracts ($\pm$ external intercostals) & Inspiratory muscles relax\\
Thoracic volume & Increases & Decreases toward resting level\\
Intrapleural pressure & Becomes \textbf{more negative} & Becomes \textbf{less negative} (returns toward baseline)\\
Transpulmonary pressure & Increases $\rightarrow$ alveoli distend; elastic recoil increases & Decreases as elastic recoil empties lungs\\
Alveolar pressure (vs atmosphere) & Falls slightly below 0 $\rightarrow$ airflow \textbf{into} alveoli & Rises slightly above 0 $\rightarrow$ airflow \textbf{out}\\
End of phase & Flow stops when alveolar pressure returns to 0 & Flow stops when alveolar pressure returns to 0\\
\bottomrule
\end{tabularx}
\end{table}

\subsubsection{Key takeaways (MCQ-friendly)}
\begin{itemize}[leftmargin=*,nosep]
  \item Flow occurs only when alveolar pressure $\neq$ atmospheric pressure.
  \item End-inspiration and end-expiration: alveolar pressure $=$ atmospheric (no flow).
  \item The pressure that keeps alveoli open is \textbf{transpulmonary pressure} (alveolar $-$ intrapleural).
\end{itemize}

\subsection{Pressures, flow, and volume during the breathing cycle}

\subsubsection{At rest (end expiration)}
\begin{itemize}[leftmargin=*,nosep]
  \item Alveolar pressure = mouth pressure = 0 (relative to atmosphere).
  \item Intrapleural pressure \textasciitilde $-5$ cmH$_2$O.
  \item Lung volume = FRC.
  \item No flow.
\end{itemize}

\subsubsection{Inspiration}
\begin{itemize}[leftmargin=*,nosep]
  \item Inspiratory muscle activity expands chest wall $\rightarrow$ intrapleural pressure falls.
  \item Alveolar pressure falls to \textasciitilde $-1$ cmH$_2$O $\rightarrow$ airflow into lungs.
  \item End-inspiration:
  \begin{itemize}[leftmargin=*,nosep]
    \item Intrapleural pressure \textasciitilde $-8$ cmH$_2$O.
    \item Alveolar pressure returns to 0 (no flow).
    \item Lung volume increases by \textasciitilde 500 mL (tidal volume).
  \end{itemize}
\end{itemize}

\subsubsection{Expiration}
\begin{itemize}[leftmargin=*,nosep]
  \item Inspiratory drive ceases $\rightarrow$ system returns toward resting equilibrium.
  \item Intrapleural pressure becomes less negative.
  \item Alveolar pressure becomes \textasciitilde $+1$ cmH$_2$O $\rightarrow$ airflow out.
  \item End-expiration:
  \begin{itemize}[leftmargin=*,nosep]
    \item Intrapleural pressure returns to \textasciitilde $-5$ cmH$_2$O.
    \item Alveolar pressure returns to 0; flow stops; volume returns to FRC.
  \end{itemize}
  \item Quiet expiration is passive (elastic recoil); can become active with expiratory muscle recruitment.
\end{itemize}

\paragraph{Exam pitfalls}
\begin{itemize}[leftmargin=*,nosep]
  \item In spontaneous breathing, alveolar pressure is 0 at end-inspiration and end-expiration; flow is driven by small swings (e.g.\ to $-1$ and $+1$ cmH$_2$O).
\end{itemize}

\subsection{Pressure waveforms during intermittent positive pressure ventilation (IPPV)}
\begin{itemize}[leftmargin=*,nosep]
  \item During \textbf{inspiration} in IPPV:
  \begin{itemize}[leftmargin=*,nosep]
    \item Alveolar pressure rises from baseline to positive values.
    \item Intrapleural pressure rises from \textasciitilde $-5$ cmH$_2$O to about $+2$ to $+3$ cmH$_2$O at end inspiration.
  \end{itemize}
  \item During \textbf{expiration}: intrapleural pressure returns to \textasciitilde $-5$ cmH$_2$O.
\end{itemize}

\paragraph{Exam pitfalls}
\begin{itemize}[leftmargin=*,nosep]
  \item Spontaneous inspiration: intrapleural pressure becomes \emph{more negative}.
  \item IPPV inspiration: intrapleural pressure becomes \emph{less negative and may become positive}.
\end{itemize}

% =========================================================

\section{Mechanical Properties of the Lungs (Kam \& Power, Ch.16)}
\textit{Primary source: Kam \& Power — \emph{Principles of Physiology for the Anaesthetist}, Chapter 16.}

\subsection{What this chapter is about}
Lung mechanics describes how pressure generates:
\begin{itemize}
  \item \textbf{Flow} (air movement),
  \item \textbf{Volume change} (inflation/deflation),
  \item \textbf{Stability} of alveoli and small airways (surfactant, interdependence, closing capacity).
\end{itemize}

\subsection{Pressures and reference points}
\subsubsection{Key pressures}
\begin{itemize}
  \item \textbf{Alveolar pressure} ($P_A$)
  \item \textbf{Pleural (intrapleural) pressure} ($P_{pl}$)
  \item \textbf{Transpulmonary pressure} ($P_L$): distending pressure of the lung
\end{itemize}
\[
P_L = P_A - P_{pl}
\]

\subsubsection{Elastic equilibrium at end-expiration (FRC)}
At \textbf{FRC}, inward recoil of the lung balances outward recoil of the chest wall. The chapter states loss of diaphragmatic tone reduces FRC by \textasciitilde 400 mL.

\subsection{Compliance, elastance, and measurement}
\subsubsection{Definitions}
\[
C = \frac{\Delta V}{\Delta P}, \qquad E = \frac{1}{C}
\]
\begin{itemize}
  \item High compliance: reminder ``easy to inflate''.
  \item Low compliance: ``stiff''.
\end{itemize}

\subsubsection{Static compliance}
Measured when \textbf{flow has ceased} (pause/breath-hold), allowing slow units to fill; typically $C_{static} > C_{dynamic}$.

\subsubsection{Dynamic compliance}
Measured during ongoing breathing; reduced when units have different time constants.

\subsubsection{Frequency dependence}
Dynamic compliance falls as respiratory rate rises in heterogeneous lungs; used as an early marker of small airway closure.

\subsubsection{Specific compliance}
\[
C_{\text{spec}} = \frac{C}{FRC}
\]

\subsection{Determinants of compliance}
Broad determinants:
\begin{enumerate}
  \item \textbf{Elastic recoil of tissue} (elastin/collagen)
  \item \textbf{Surface tension} at the alveolar air--fluid interface
\end{enumerate}

Additional factors listed in the chapter (selected):
\begin{itemize}
  \item Lung volume; age; posture; obesity; pulmonary blood volume; bronchial smooth muscle tone.
  \item Disease:
  \begin{itemize}
    \item Fibrosis: stiffer lungs.
    \item Emphysema: increased compliance from loss of septal tissue opposing expansion.
  \end{itemize}
\end{itemize}

\subsection{Thoracic cage compliance and total respiratory system compliance}
Lung and chest wall compliances combine ``in series'':
\[
\frac{1}{C_{total}} = \frac{1}{C_{lung}} + \frac{1}{C_{chest}}
\]
Chapter values: $C_{lung} \approx 200$ mL/cmH$_2$O, $C_{chest} \approx 200$ mL/cmH$_2$O, so $C_{total} \approx 100$ mL/cmH$_2$O.

\subsection{Elastic recoil: tissue + surface tension}
The chapter states surface tension contributes \textasciitilde 70\% of total elastic forces in the normal lung.

\subsubsection{Laplace and stability}
If surface tension were constant, smaller units would have higher pressure and tend to empty into larger units, promoting collapse.

\subsection{Surfactant}
\subsubsection{Origin and turnover}
Produced by type II alveolar epithelial cells; stored in lamellated bodies. Half-life stated as \textasciitilde 15--30 h; most components recycled by type II cells.

\subsubsection{Composition (chapter description)}
\begin{itemize}
  \item \textasciitilde 90\% lipid (mainly phospholipid; also cholesterol)
  \begin{itemize}
    \item principal: dipalmitoyl phosphatidylcholine
    \item significant: phosphatidylglycerol
  \end{itemize}
  \item proteins \textasciitilde 2--8\%: SP-A, SP-B, SP-C, SP-D
\end{itemize}

\subsubsection{Functions}
\begin{itemize}
  \item Reduces surface tension $\rightarrow$ increases distensibility, reduces elastic recoil.
  \item Reduces work of breathing.
  \item Stabilises alveoli (preferential reduction of surface tension in small alveoli, especially during expiration).
  \item Reduces atelectasis; reduces tendency to alveolar oedema.
  \item Produces hysteresis.
\end{itemize}

\subsection{Hysteresis}
Inflation and deflation pressure--volume curves differ. Contributors described:
\begin{itemize}
  \item surfactant-related changes in surface tension,
  \item recruitment of collapsed units early in inspiration,
  \item stress relaxation at sustained high lung volumes.
\end{itemize}

\subsection{Alveolar interdependence}
Neighbouring alveoli share walls; traction from adjacent units tends to prevent collapse of a single unit (stabilising mechanism alongside surfactant).

\subsection{Lung volumes and FRC}
\subsubsection{Definition}
\[
FRC = ERV + RV
\]
Occurs where outward recoil of chest wall balances inward recoil of lung.

\subsubsection{Functions of FRC (chapter list)}
\begin{itemize}
  \item Oxygen store; reduces PaO$_2$ swings.
  \item Helps prevent airway collapse/atelectasis.
  \item Minimises pulmonary vascular resistance.
  \item Maintains airway patency; optimises compliance; minimises work of breathing.
\end{itemize}

\subsubsection{Factors affecting FRC (chapter list)}
\begin{itemize}
  \item Height (direct relationship); females \textasciitilde 10\% less than males; not correlated with age (per chapter).
  \item Posture: supine reduces FRC by up to \textasciitilde 1000 mL.
  \item Pregnancy (3rd trimester \textasciitilde 20\% reduction) and obesity reduce FRC.
  \item Increased elastic recoil (e.g.\ fibrosis) reduces FRC.
  \item Closing capacity rises with age and may exceed FRC (chapter notes upright \textasciitilde 70 y; supine \textasciitilde 44 y).
\end{itemize}

\subsubsection{Measurement methods}
Nitrogen washout; helium dilution; body plethysmography.

\subsection{Closing capacity}
\subsubsection{Definition}
\[
\text{Closing capacity} = RV + \text{Closing volume}
\]
Normally, $FRC >$ closing capacity. Small airways ($<2$ mm) begin to collapse during expiration at closing capacity.

\subsubsection{Increases with}
Age; smoking; lung disease (reduced elastic recoil and reduced radial traction).

\subsubsection{Consequences and posture/age}
Airway closure can leave perfused units unventilated $\rightarrow$ impaired oxygenation. Chapter statements include:
\begin{itemize}
  \item by \textasciitilde 65 y, closing capacity exceeds standing FRC,
  \item supine airway closure during normal tidal breathing may occur by \textasciitilde 45 y.
\end{itemize}

\subsection{Gas flow patterns and airway resistance}
\subsubsection{Laminar, turbulent, transitional}
\begin{itemize}
  \item Laminar: parabolic profile; strong radius dependence ($r^4$); viscosity is key gas property.
  \item Turbulent: eddies; higher driving pressure; $\Delta P \propto \text{flow}^2$; density is key gas property; strong radius dependence (Fanning equation context).
  \item Transitional: common at airway branches; mixed dependence on flow and flow$^2$.
\end{itemize}

\subsubsection{Reynolds number}
Turbulence tends to occur when $Re \gtrsim 2000$ in smooth tubes; promoted by high velocity, larger diameter, higher density, lower viscosity gas.

\subsubsection{Distribution in the lung}
Trachea/larynx tend to be turbulent at high velocities; laminar likely only in very small airways; much of bronchial tree is transitional.

\subsubsection{Airway resistance (Raw)}
Chapter values: $R_{aw}$ \textasciitilde 2 cmH$_2$O/L/s; normal tidal airflow requires a mouth--alveolar pressure difference \textasciitilde 1 cmH$_2$O.
Determinants: lung volume; bronchial smooth muscle tone; dynamic airway compression.

\subsection{Bronchial smooth muscle tone}
\begin{itemize}
  \item Parasympathetic (vagal) ACh at muscarinic receptors $\rightarrow$ bronchoconstriction.
  \item Sympathetic innervation less important; circulating adrenaline at $\beta_2$ receptors $\rightarrow$ bronchodilation.
  \item Non-cholinergic, non-adrenergic pathways can bronchodilate.
  \item Mediators: histamine causes constriction + mucosal swelling; leukotrienes and some prostaglandins can contribute to bronchospasm.
\end{itemize}

\subsection{Dynamic compression and the equal pressure point}
During forced expiration, pleural pressure becomes positive; airway pressure declines toward the mouth due to resistive pressure drop. The \textbf{equal pressure point} is where intraluminal pressure equals pleural pressure; beyond this point (toward mouth), airways can collapse.
Flow becomes effort-independent once dynamic compression limits further increase. Collapse is more likely at low lung volumes, high airway resistance, or reduced elastic recoil (e.g.\ emphysema).

\subsection{Work of ventilation}
\subsubsection{Components}
Inspiratory muscles do work against elastic and non-elastic (resistive) forces; expiration is normally passive.

\subsubsection{Recall values (chapter)}
\begin{itemize}
  \item Resistive work: \textasciitilde 35\% (dissipated as heat); tissue resistive work contributes \textasciitilde 10\% of resistance work.
  \item Elastic work: \textasciitilde 65\% (stored as potential energy).
\end{itemize}

\subsubsection{When expiration becomes active}
If airway resistance or expiratory flow demand rises, expiratory resistive work may exceed stored elastic energy $\rightarrow$ expiratory muscle recruitment.

\subsubsection{Oxygen cost of breathing}
Chapter value: respiratory muscle oxygen requirement \textasciitilde 3 mL/min at rest; rises with increased ventilation and in lung disease.

\subsection{Exam-oriented pitfalls and high-yield checks}
\begin{itemize}
  \item Compliance is $\Delta V/\Delta P$; elastance is the inverse.
  \item Static compliance (no flow) $>$ dynamic compliance.
  \item Closing capacity $= RV +$ closing volume.
  \item Laminar: viscosity; turbulent: density; much of bronchial tree is transitional.
  \item Emphysema: increased compliance but reduced elastic recoil $\rightarrow$ air trapping and dynamic compression.
\end{itemize}

\subsection{Additions and useful nuances from other core texts (not emphasised here)}
\subsubsection{Compliance is volume-dependent (P--V curve shape)}
Other texts emphasise compliance is only approximately linear around the normal working range near FRC:
\begin{itemize}
  \item High lung volumes: compliance falls as elastic elements approach their limit.
  \item Low lung volumes: compliance falls due to airway/alveolar closure and opening pressure requirement.
\end{itemize}
\textit{Sources:} \emph{Fundamentals of Anaesthesia} (4th ed), Ch.17 (P--V curve; compliance at extremes of lung volume).

\subsubsection{Pendelluft and dynamic compliance}
Pendelluft (gas redistribution between lung regions during dynamic conditions) can reduce apparent dynamic compliance.
\textit{Sources:} \emph{Morgan \& Mikhail's Clinical Anesthesiology} (5th ed), Ch.23 (pendelluft/gas redistribution concept).

\subsubsection{Measuring pleural pressure clinically}
Pleural pressure can be measured by intrapleural catheter or estimated using a mid-oesophageal balloon.
\textit{Sources:} \emph{Fundamentals of Anaesthesia} (4th ed), Ch.17 (pleural pressure measurement; oesophageal balloon).

\subsubsection{Typical pressure values during quiet breathing}
Illustrative values commonly quoted:
\begin{itemize}
  \item End-expiration: $P_{pl}$ about $-5$ cmH$_2$O; $P_A \approx 0$ $\Rightarrow P_L \approx +5$ cmH$_2$O.
  \item Inspiration: $P_{pl}$ becomes more negative; $P_A$ slightly negative to drive flow; at end-inspiration flow is zero and $P_A$ returns to 0 while $P_{pl}$ remains more negative.
\end{itemize}
\textit{Sources:} \emph{Morgan \& Mikhail} (5th ed), Ch.23 (quiet breathing pressure changes); similar concepts also described (often in kPa) in \emph{Fundamentals of Anaesthesia} (4th ed), Ch.17.

\subsubsection{Tissue resistance}
Some texts explicitly highlight tissue (viscoelastic) resistance as an important non-elastic component.
\textit{Sources:} \emph{Morgan \& Mikhail} (5th ed), Ch.23 (section on tissue resistance).

\subsubsection{Work of breathing: alternative breakdowns}
Resistive work may be subdivided into airway resistance work and tissue viscous (inelastic) work; proportional contributions vary by source.
\textit{Sources:} \emph{Ganong's Review of Medical Physiology} (Barrett et al.), pulmonary mechanics/work of breathing section; \emph{Morgan \& Mikhail} (5th ed), Ch.23.

\subsubsection{Proximal airway pressure may not equal alveolar pressure}
During dynamic conditions, proximal airway pressure (ventilator) may not reflect distal/alveolar pressure when resistance is high and/or tubing is compliant.
\textit{Sources:} \emph{Morgan \& Mikhail} (5th ed), Ch.4 (mechanical ventilation: airway vs alveolar pressure; resistance/compliance).

\subsubsection{Ventilator/circuit compliance and ``lost'' tidal volume}
In volume-controlled ventilation, some set tidal volume is lost to circuit expansion (compressible volume); exhaled tidal volume is best measured near the airway.
\textit{Sources:} \emph{Morgan \& Mikhail} (5th ed), Ch.3 (breathing systems) and Ch.4 (ventilation: circuit compliance/compressible volume).

\subsubsection{Time constant heuristic}
Some FRCA-style resources explicitly frame lung filling/emptying with the time constant:
\[
\tau = R \times C
\]
High $R$ and/or high $C$ (e.g.\ obstructive disease) $\rightarrow$ prolonged emptying, need for longer expiratory time to avoid gas trapping.
\textit{Sources:} \emph{Primary FRCA in a Box} (2nd ed), time constants/exponential relationships section; \emph{Morgan \& Mikhail} (5th ed), Ch.23 (air trapping/auto-PEEP concepts).

\subsubsection{Surfactant proteins: roles beyond surface tension}
SP-B and SP-C help form the surface film; SP-A and SP-D (collectins) contribute to innate immunity and surfactant turnover/regulation.
\textit{Sources:} \emph{Ganong} (Barrett et al.), pulmonary surfactant section.

\subsubsection{Spirometry as surrogate markers of airway mechanics}
FEV$_1$/FVC as obstruction index; FEF$_{25\text{--}75\%}$ (MMEF) described as relatively effort-independent and sometimes abnormal earlier in obstructive disease.
\textit{Sources:} \emph{Morgan \& Mikhail} (5th ed), Ch.23--24 (spirometric indices); \emph{Fundamentals of Anaesthesia} (4th ed), pulmonary function testing section.

\subsubsection{Positive pressure breathing systems: practical mechanics complications}
\begin{itemize}
  \item CPAP/PEEP can improve oxygenation but may reduce cardiac output (raised intrathoracic pressure effects).
  \item High-frequency jet ventilation can generate intrinsic PEEP at high frequencies; delivered tidal volume and FiO$_2$ may be uncertain due to entrainment.
\end{itemize}
\textit{Sources:} \emph{Essentials of Equipment in Anaesthesia} (Al-Shaikh \& Stacey, 2023), Ch.13 (CPAP complications) and Ch.8 (HF jet ventilation: entrainment/intrinsic PEEP); \emph{Morgan \& Mikhail} (5th ed), Ch.57 (PEEP/CPAP adverse nonpulmonary effects incl. reduced cardiac output).


\end{document}
