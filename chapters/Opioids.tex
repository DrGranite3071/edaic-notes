\documentclass[11pt,a4paper]{article}

% preamble.tex
\usepackage[T1]{fontenc}
\usepackage[utf8]{inputenc}
\usepackage[english]{babel}
\usepackage[a4paper,margin=2cm]{geometry}
\usepackage{lmodern}
\usepackage{amsmath,amssymb}
\usepackage{graphicx}
\usepackage{booktabs}
\usepackage{hyperref}



\title{Opioid Analgesia Summary\\[4pt]
\large (Based on Peck \& Harris, Chapter 10 -- Analgesics)}
\author{}
\date{}

\begin{document}
\maketitle

\section*{Expanded Summary with Peck \& Harris Structure and Tables 10.1--10.4}

\subsection*{SECTION INTRO -- Alignment with Peck \& Harris}

This expanded chapter now follows the structure used in \emph{Peck \& Harris: Pharmacology for Anaesthesia and Intensive Care}, Chapter 10 (Analgesics). It introduces:
\begin{itemize}
  \item Physiology of pain
  \item Opioid receptor pharmacology
  \item Clinical pharmacology of opioids (drug-by-drug)
  \item Comparative tables (Tables 10.1--10.4 reproduced and expanded)
\end{itemize}

\section{TABLE 10.1 -- Classification of Opioid Analgesics}

(Adapted and expanded from Peck \& Harris Table 10.1)

\subsection*{A. Pure Agonists ($\mu$-agonists)}
\begin{itemize}
  \item Morphine
  \item Diamorphine
  \item Fentanyl
  \item Alfentanil
  \item Remifentanil
  \item Sufentanil
  \item Pethidine
  \item Codeine (weak agonist)
  \item Methadone
\end{itemize}

\subsection*{B. Partial Agonists / Mixed Agonist--Antagonists}
\begin{itemize}
  \item Buprenorphine (partial $\mu$-agonist)
  \item Nalbuphine ($\kappa$-agonist, $\mu$-antagonist)
  \item Butorphanol ($\kappa$-agonist)
\end{itemize}

\subsection*{C. Antagonists}
\begin{itemize}
  \item Naloxone
  \item Naltrexone
  \item Nalmefene
\end{itemize}

\section{TABLE 10.2 -- Opioid Receptor Effects}

(Expanded from Peck \& Harris)

\begin{table}[h]
\centering
\begin{tabularx}{\textwidth}{l l X X}
\toprule
\textbf{Receptor} & \textbf{Site} & \textbf{Desired Effects} & \textbf{Unwanted Effects} \\
\midrule
$\mathbf{\mu_1}$ & Supraspinal (PAG, thalamus) & Analgesia & --- \\
$\mathbf{\mu_2}$ & Brainstem, spinal cord & Respiratory depression, spinal analgesia & Constipation, dependence, pruritus, urinary retention \\
$\mathbf{\kappa}$ & Spinal cord, limbic & Spinal analgesia, sedation & Dysphoria, hallucinations, diuresis \\
$\mathbf{\delta}$ & Limbic system & Analgesia, mood modulation & Seizure risk, unclear clinical relevance \\
\bottomrule
\end{tabularx}
\end{table}

Mechanisms: $\downarrow$ cAMP, $\uparrow$ K$^+$ efflux, $\downarrow$ Ca$^{2+}$ influx $\rightarrow$ $\downarrow$ neurotransmission.

\section{TABLE 10.3 -- Comparative Potency of Opioids (Adapted)}

\begin{table}[h]
\centering
\begin{tabular}{l l l}
\toprule
\textbf{Drug} & \textbf{Relative Potency (Morphine = 1)} & \textbf{Notes} \\
\midrule
Morphine     & 1       & Standard reference \\
Diamorphine  & 2--3    & Rapid CNS entry; active metabolites \\
Codeine      & 0.1     & Prodrug; CYP2D6-dependent \\
Fentanyl     & 50--100 & Very lipid soluble \\
Alfentanil   & 10--20  & Low pKa; rapid onset \\
Remifentanil & 15--20  & Metabolised by esterases; CSHT $\sim$3 min \\
Sufentanil   & 500--1000 & Most potent in clinical use \\
Pethidine    & 0.1--0.2 & Toxic metabolite norpethidine \\
Methadone    & 1--2    & NMDA antagonist; long half-life \\
\bottomrule
\end{tabular}
\end{table}

\section{TABLE 10.4 -- Pharmacokinetic/Pharmacodynamic Comparison}

(Highly expanded version of Peck \& Harris Table 10.4)

\begin{table}[h!]
\centering
\small
\caption*{\textbf{TABLE 10.4 – Pharmacokinetic/Pharmacodynamic Comparison}\\
(Expanded version of Peck \& Harris Table 10.4)}
\begin{tabularx}{\textwidth}{l *{4}{X}}
\toprule
\textbf{Feature} & \textbf{Morphine} & \textbf{Diamorphine} & \textbf{Fentanyl} & \textbf{Alfentanil} \\
\midrule
Lipid solubility & Low & Very high & Very high & High \\
Onset            & Slow & Very fast & Fast & Very fast \\
Duration         & Long & Moderate & Moderate $\rightarrow$ long (infusions) & Short \\
Clearance        & Hepatic (glucuronidation) & Hepatic $\rightarrow$ 6-MAM $\rightarrow$ M6G & Hepatic & Hepatic \\
Active metabolites & M6G, M3G & 6-MAM, M6G & No & No \\
Context-sensitive half-time & Long & Moderate & Long with infusion & Low--moderate \\
Histamine release & Yes & No & No & No \\
Special notes & Accumulates in renal failure & Potent \& rapid & Cardiac-stable & Great for RSI \\
\bottomrule
\end{tabularx}

\vspace{1em}

\begin{tabularx}{\textwidth}{l *{4}{X}}
\toprule
\textbf{Feature} & \textbf{Remifentanil} & \textbf{Sufentanil} & \textbf{Pethidine} & \textbf{Methadone} \\
\midrule
Lipid solubility & High & Very high & Moderate & Moderate \\
Onset            & Ultra-fast & Fast & Moderate & Slow \\
Duration         & Ultra-short & Moderate & Short & Very long \\
Clearance        & Esterases & Hepatic & Hepatic & Hepatic \\
Active metabolites & No & No & Norpethidine & No \\
Context-sensitive half-time & Constant (3--5 min) & Moderate & Short & Very long \\
Histamine release & No & No & Yes & No \\
Special notes & Predictable offset & High-potency epidural & Antimuscarinic & NMDA antagonism \\
\bottomrule
\end{tabularx}
\end{table}


\section{FULL EXPANDED SUMMARY (Integrated with Peck \& Harris)}

\subsection{SECTION 1 -- Physiology of Pain (Peck structure aligned)}

Pain physiology is divided into:
\begin{enumerate}
  \item Transduction
  \item Transmission
  \item Modulation
  \item Perception
\end{enumerate}

\subsubsection*{Peripheral Transduction}

Nociceptors are free nerve endings responding to heat, mechanical injury, and inflammatory mediators.
\begin{itemize}
  \item \textbf{Key channels:} TRPV1, TRPA1, ASIC, P2X3, Nav1.7--1.9.
  \item \textbf{Inflammatory mediators:} bradykinin, prostaglandins, histamine, NGF.
  \item \textbf{Result:} lowered activation threshold (primary hyperalgesia).
\end{itemize}

\subsubsection*{Spinal Transmission}

Occurs at dorsal horn laminae I, II, and V.
\begin{itemize}
  \item \textbf{Neurotransmitters:} glutamate (fast), substance P (slow), CGRP.
  \item \textbf{Receptors:} AMPA $\rightarrow$ NMDA (requires Mg$^{2+}$ disinhibition).
\end{itemize}

\subsubsection*{Central Sensitisation (``wind-up'')}

NMDA-dependent Ca$^{2+}$ influx $\rightarrow$ PKC, CAMKII activation $\rightarrow$ transcriptional changes. This increases dorsal horn neuron excitability.

\subsubsection*{Ascending Pathways}

\begin{itemize}
  \item \textbf{Spinothalamic:} discriminative pain.
  \item \textbf{Spinoreticular:} autonomic/emotional pain.
  \item \textbf{Spinomesencephalic:} PAG activation.
\end{itemize}

\subsubsection*{Descending Modulation}

PAG $\rightarrow$ RVM $\rightarrow$ spinal dorsal horn interneurons.\\
Neurotransmitters: 5-HT, noradrenaline, enkephalins.\\
Mechanisms: $\downarrow$ substance P release, $\uparrow$ K$^+$ efflux.

\subsection{SECTION 2 -- Opioid Pharmacology (Peck Section Structure)}

Opioids mimic endogenous peptides (endorphins, enkephalins, dynorphins).

\subsubsection*{Receptors}

\begin{itemize}
  \item $\mu_1$: supraspinal analgesia
  \item $\mu_2$: spinal analgesia, respiratory depression, dependence
  \item $\kappa$: dysphoria, diuresis, spinal analgesia
  \item $\delta$: chronic pain modulation
\end{itemize}

\subsubsection*{Cellular Mechanisms}

\begin{itemize}
  \item Gi/Go $\rightarrow$ $\downarrow$ adenylyl cyclase $\rightarrow$ $\downarrow$ cAMP
  \item $\downarrow$ Ca$^{2+}$ entry presynaptically $\rightarrow$ $\downarrow$ glutamate/substance P
  \item $\uparrow$ K$^+$ efflux postsynaptically $\rightarrow$ hyperpolarisation
\end{itemize}

\subsubsection*{Systemic Effects}

\textbf{Respiratory:} $\downarrow$ response to CO$_2$ and O$_2$.

\textbf{CNS:} miosis, analgesia, euphoria ($\mu$), dysphoria ($\kappa$), chest wall rigidity (high-dose fentanyl).

\textbf{GI:} constipation, delayed gastric emptying.

\textbf{CVS:} bradycardia, histamine-mediated hypotension.

\textbf{Endocrine:} $\downarrow$ cortisol; $\downarrow$ testosterone; amenorrhoea.

\textbf{Immune:} NK cell suppression.

\section{SECTION 3 -- Opioids in Clinical Anaesthesia}

\subsection*{Dosage Summary for Common Opioids}

(Values are standard adult perioperative doses; adjust for age, frailty, comorbidity.)

\begin{description}
  \item[Morphine] IV: 1--2 mg boluses, titrated (usual total 5--10 mg); IM/SC: 10 mg; Epidural: 2--4 mg; Intrathecal: 0.1--0.3 mg.
  \item[Diamorphine] IV: 0.5--1 mg boluses; Epidural: 1--3 mg; Intrathecal: 0.1--0.3 mg.
  \item[Fentanyl] Induction: 1--3 $\mu$g/kg; Intraoperative boluses: 25--100 $\mu$g; Infusion: 1--3 $\mu$g/kg/h.
  \item[Alfentanil] Bolus for intubation: 10--20 $\mu$g/kg; Infusion: 0.5--2 $\mu$g/kg/min.
  \item[Remifentanil] Induction/maintenance: 0.05--0.3 $\mu$g/kg/min; Bolus generally avoided; if used: 0.5--1 $\mu$g/kg.
  \item[Sufentanil] Induction (cardiac): 0.5--1 $\mu$g/kg; Epidural: 10--20 $\mu$g.
  \item[Pethidine] Analgesia: 25--50 mg IV or 50--100 mg IM; Shivering: 12.5--25 mg IV.
  \item[Codeine] PO: 30--60 mg, max 240 mg/day.
  \item[Tramadol] IV/PO: 50--100 mg, max 400 mg/day.
  \item[Methadone] Analgesia: 2.5--10 mg PO every 8--12 h; Acute perioperative (specialist): 0.1--0.2 mg/kg IV.
\end{description}

\subsection{Morphine}

\textbf{Naturally occurring phenanthrene opioid; prototype $\mu$-agonist and reference standard for potency and PK comparisons.}

\subsubsection*{Presentation \& Uses}
\begin{itemize}
  \item IV/IM/SC/PO; epidural/intrathecal.
  \item Moderate--severe acute and chronic pain.
  \item Long-acting neuraxial analgesia (intrathecal/epidural).
\end{itemize}

\subsubsection*{Kinetics}
\begin{itemize}
  \item pKa 7.9 $\rightarrow$ $\sim$10--15\% unionised at physiological pH $\rightarrow$ slow BBB penetration.
  \item Low lipid solubility $\rightarrow$ delayed CNS entry and slower onset.
  \item Oral bioavailability: $\sim$20--40\% (significant first-pass metabolism).
  \item Metabolism: hepatic glucuronidation (UGT2B7) $\rightarrow$
    \begin{itemize}
      \item M3G: neuro-excitatory, no analgesic effect.
      \item M6G: potent $\mu$-agonist (up to 13$\times$ morphine) $\rightarrow$ contributes to analgesia \& respiratory depression.
    \end{itemize}
  \item Elimination: renal; metabolites accumulate in renal impairment.
  \item Half-life: 2--3 h (prolonged in neonates, elderly, renal failure).
  \item Neuraxial kinetics: hydrophilic $\rightarrow$ slow CSF spread, long duration, delayed respiratory depression.
  \item Placental transfer: yes $\rightarrow$ neonatal respiratory depression risk.
\end{itemize}

\subsubsection*{Effects}
\begin{itemize}
  \item Histamine release $\rightarrow$ vasodilation, hypotension, pruritus.
  \item Respiratory depression ($\mu_2$): $\downarrow$ CO$_2$ sensitivity.
  \item CNS: analgesia, sedation, euphoria.
  \item Cough suppression (medullary).
  \item Miosis (no tolerance develops).
  \item GI: nausea/vomiting, severe constipation.
  \item Smooth muscle:
    \begin{itemize}
      \item $\uparrow$ sphincter of Oddi tone $\rightarrow$ biliary colic.
      \item $\uparrow$ bladder sphincter tone $\rightarrow$ urinary retention.
      \item $\downarrow$ gastric emptying $\rightarrow$ aspiration risk.
    \end{itemize}
  \item Endocrine: $\downarrow$ cortisol, $\downarrow$ testosterone.
  \item Immune: NK cell suppression.
  \item No tolerance to constipation or miosis.
\end{itemize}

\subsection{Diamorphine}

\textbf{Semi-synthetic di-acetylated morphine; more lipid-soluble and $\sim$2--3$\times$ more potent than morphine with faster CNS penetration.}

\subsubsection*{Presentation \& Uses}
\begin{itemize}
  \item IV/IM/SC; neuraxial (epidural/intrathecal).
  \item Severe postoperative pain, trauma, pulmonary oedema.
  \item Palliative care: high potency $\rightarrow$ small injection volumes.
\end{itemize}

\textbf{Potency:} 2--3$\times$ morphine.

\subsubsection*{Kinetics}
\begin{itemize}
  \item pKa 7.9: higher unionised fraction $\rightarrow$ rapid CNS entry.
  \item Lipid solubility $\sim$280$\times$ morphine.
  \item Oral bioavailability 50--65\%.
  \item Metabolism: diamorphine $\rightarrow$ 6-MAM $\rightarrow$ morphine $\rightarrow$ M6G/M3G.
  \item Half-life: 2--3 h.
  \item Renal excretion: metabolites accumulate.
  \item CSF: fast onset, shorter duration vs morphine.
\end{itemize}

\subsubsection*{Effects}
\begin{itemize}
  \item Rapid profound analgesia.
  \item Strong euphoria.
  \item Less histamine release than morphine.
  \item Rapid onset; strong reinforcement/euphoria.
\end{itemize}

\subsection{Fentanyl}

\textbf{Synthetic phenylpiperidine; very potent, highly lipid-soluble $\mu$-agonist with rapid onset and short redistribution half-life.}

\subsubsection*{Presentation \& Uses}
\begin{itemize}
  \item IV, transdermal patches, buccal/nasal formulations.
  \item Induction, intraoperative analgesia, cardiac anaesthesia.
\end{itemize}

\subsubsection*{Kinetics}
\begin{itemize}
  \item pKa 8.4: primarily ionised but extremely lipid-soluble $\rightarrow$ fast CNS entry.
  \item Lipid solubility $\sim$600$\times$ morphine.
  \item Large Vd; sequestration in fat.
  \item Metabolism: CYP3A4 $\rightarrow$ inactive metabolites.
  \item Bioavailability: 50\% buccal, 70--80\% intranasal.
  \item CSHT: prolonged with long infusions.
\end{itemize}

\subsubsection*{Effects}
\begin{itemize}
  \item Stable haemodynamics.
  \item Strong analgesia.
  \item Chest wall rigidity (especially rapid bolus).
  \item No histamine release.
\end{itemize}

\subsection{Alfentanil}

\textbf{Short-acting fentanyl analogue with low pKa and very rapid onset; ideal for brief, intense noxious stimuli.}

\subsubsection*{Presentation \& Uses}
\begin{itemize}
  \item IV bolus or infusion; excellent for brief painful stimuli.
\end{itemize}

\subsubsection*{Kinetics}
\begin{itemize}
  \item pKa 6.5: $\sim$90\% unionised at physiological pH $\rightarrow$ extremely rapid onset.
  \item High protein binding (90\%) $\rightarrow$ small Vd.
  \item Metabolism: CYP3A4.
  \item CSHT: low--moderate.
\end{itemize}

\subsubsection*{Effects}
\begin{itemize}
  \item Rapid titratability.
  \item Exaggerated effects in elderly and liver disease.
  \item Marked respiratory depression despite short duration.
\end{itemize}

\subsection{Remifentanil}

\textbf{Ultra-short-acting $\mu$-agonist ester opioid; uniquely metabolised by non-specific esterases with context-insensitive kinetics.}

\subsubsection*{Presentation \& Uses}
\begin{itemize}
  \item IV infusion; ideal for TIVA, neuro/cardiac/bariatric procedures.
\end{itemize}

\subsubsection*{Kinetics}
\begin{itemize}
  \item pKa 7.1.
  \item Metabolism by plasma/tissue esterases $\rightarrow$ organ-independent.
  \item Half-life: 6--12 min.
  \item CSHT: constant (3--5 min) regardless of infusion length.
  \item No accumulation.
\end{itemize}

\subsubsection*{Effects}
\begin{itemize}
  \item Profound analgesia.
  \item Bradycardia, hypotension.
  \item Opioid-induced hyperalgesia after abrupt stop.
\end{itemize}

\subsection{Sufentanil}

\textbf{Highly potent fentanyl analogue (500--1000$\times$ morphine) with extreme lipid solubility and excellent haemodynamic stability.}

\subsubsection*{Presentation \& Uses}
\begin{itemize}
  \item IV and epidural; used in cardiac surgery and labour analgesia.
\end{itemize}

\subsubsection*{Kinetics}
\begin{itemize}
  \item pKa 8.0.
  \item Extremely high lipid solubility (even $>$ fentanyl).
  \item Potency 500--1000$\times$ morphine.
  \item Metabolism: hepatic $\rightarrow$ inactive metabolites.
  \item CSHT: shorter than fentanyl; longer than remifentanil.
  \item Rapid neuraxial onset due to lipophilicity.
\end{itemize}

\subsubsection*{Effects}
\begin{itemize}
  \item Profound analgesia.
  \item Stable haemodynamics.
  \item Preferred epidural adjunct.
\end{itemize}

\subsection{Codeine}

\textbf{Methylmorphine; weak $\mu$-agonist prodrug requiring CYP2D6 conversion to morphine for most of its analgesic effect.}

\subsubsection*{Presentation \& Uses}
\begin{itemize}
  \item Oral tablets, capsules, and syrups.
  \item Used for mild to moderate acute pain, often in combination with paracetamol/NSAIDs.
  \item Also used as an antitussive in some formulations.
\end{itemize}

\subsubsection*{Kinetics}
\begin{itemize}
  \item Prodrug: very weak intrinsic $\mu$-agonist.
  \item Activation via CYP2D6 $\rightarrow$ O-demethylation to morphine, which mediates most of the analgesic effect.
  \item Genetic polymorphism of CYP2D6:
    \begin{itemize}
      \item Poor metabolisers (PM) ($\sim$5--10\% of Caucasians): little or no conversion $\rightarrow$ poor analgesia.
      \item Ultra-rapid metabolisers (UM) ($\sim$1--2\% in some populations, higher in North Africans/Ethiopians): excessive morphine generation $\rightarrow$ risk of respiratory depression even at standard doses.
    \end{itemize}
  \item Oral bioavailability: $\sim$50--60\%.
  \item Onset: 30--60 minutes; peak effect 1--2 h.
  \item Half-life: 3--4 h (prolonged in liver dysfunction).
  \item Elimination: hepatic metabolism (CYP2D6, CYP3A4) with renal excretion of metabolites.
\end{itemize}

\subsubsection*{Effects}
\begin{itemize}
  \item Analgesia limited by the need for metabolic activation and by a ceiling effect (analgesia plateaus at higher doses but side-effects increase).
  \item Typical opioid side effects: nausea, vomiting, constipation, mild sedation.
  \item Risk of toxicity in breastfed infants of UM mothers (excess morphine in breast milk).
  \item Not ideal in severe pain (insufficient potency, PK variability).
\end{itemize}

\subsection{Tramadol}

\textbf{Atypical synthetic opioid combining weak $\mu$-agonism with serotonin and noradrenaline reuptake inhibition (SNRI-like).}

\subsubsection*{Presentation \& Uses}
\begin{itemize}
  \item Available as oral tablets/capsules, slow-release preparations, and IV formulation.
  \item Used for moderate acute and chronic pain, particularly when NSAIDs are contraindicated or inadequate.
\end{itemize}

\subsubsection*{Kinetics}
\begin{itemize}
  \item Mechanism is dual:
    \begin{itemize}
      \item Weak $\mu$-opioid receptor agonist.
      \item Inhibits neuronal reuptake of noradrenaline and serotonin (5-HT) (SNRI-like effect).
    \end{itemize}
  \item Prodrug component: O-demethylation by CYP2D6 produces O-desmethyltramadol (M1), which has much higher $\mu$-receptor affinity than the parent compound.
  \item Oral bioavailability: $\sim$70\% (first-pass metabolism but active metabolite contributes to effect).
  \item Onset: 30--60 minutes; peak 2--3 h.
  \item Half-life: parent $\sim$6 h; metabolite $\sim$7--8 h (prolonged in renal/hepatic impairment).
  \item Elimination: mainly renal (parent + metabolites).
\end{itemize}

\subsubsection*{Effects}
\begin{itemize}
  \item Analgesia with less respiratory depression than equianalgesic doses of morphine (due to weaker $\mu$-agonism) but not negligible.
  \item Serotonergic/NAdrenergic effects:
    \begin{itemize}
      \item Useful in some neuropathic and mixed pain states.
      \item Risk of serotonin syndrome when combined with SSRIs, SNRIs, MAOIs, TCAs, or other serotonergic drugs.
    \end{itemize}
  \item Seizure threshold is lowered, especially at high doses or in patients with epilepsy or on interacting medications (e.g. antidepressants, antipsychotics).
  \item Typical opioid side effects: nausea, dizziness, sweating, constipation.
  \item Abrupt cessation after prolonged use can cause a withdrawal syndrome with both opioid-like and SNRI-like features (anxiety, agitation, dysphoria).
\end{itemize}

\subsection{Methadone}

\textbf{Synthetic diphenylheptane opioid with long and variable half-life and additional NMDA-antagonist properties.}

\subsubsection*{Presentation \& Uses}
\begin{itemize}
  \item Oral/IV; chronic pain; opioid substitution; useful in tolerance.
\end{itemize}

\subsubsection*{Kinetics}
\begin{itemize}
  \item pKa 8.3.
  \item Bioavailability 80--90\%.
  \item Half-life 12--60+ h (highly variable; risk of accumulation).
  \item Metabolism: CYP3A4, CYP2B6, CYP2C19.
  \item Stereoisomers: R-methadone ($\mu$ agonist), S-methadone (NMDA antagonist).
  \item Large Vd; long time to steady state.
\end{itemize}

\subsubsection*{Effects}
\begin{itemize}
  \item Potent analgesia.
  \item NMDA antagonism $\rightarrow$ helps neuropathic pain and opioid tolerance.
  \item QT prolongation $\rightarrow$ torsades risk.
  \item Less sedation in stable chronic use.
\end{itemize}

\subsection{Oxycodone}

\textbf{Semi-synthetic thebaine-derived opioid; moderate-to-strong $\mu$-agonist often used as an oral alternative to morphine with better bioavailability.}

\subsubsection*{Presentation \& Uses}
\begin{itemize}
  \item Oral immediate-release and controlled-release formulations.
  \item Used for moderate--severe acute and chronic pain.
  \item Often combined with paracetamol in fixed-dose preparations.
\end{itemize}

\subsubsection*{Kinetics}
\begin{itemize}
  \item Oral bioavailability: $\sim$60--70\% (higher than morphine).
  \item Metabolism: hepatic via CYP3A4 $\rightarrow$ noroxycodone (weak), and CYP2D6 $\rightarrow$ oxymorphone (potent $\mu$-agonist).
  \item Onset: 30--60 minutes; peak $\sim$1 hour.
  \item Half-life: 3--4 hours (IR); 8--12 hours (CR formulations).
  \item Elimination: renal.
\end{itemize}

\subsubsection*{Effects}
\begin{itemize}
  \item Similar $\mu$-opioid effects as morphine.
  \item Less histamine release.
  \item CYP2D6 polymorphism influences analgesic response.
  \item Risk of accumulation in renal impairment.
\end{itemize}

\subsection{Pethidine}

\textbf{Synthetic phenylpiperidine opioid with antimuscarinic properties and a neurotoxic metabolite (norpethidine).}

\subsubsection*{Presentation \& Uses}
\begin{itemize}
  \item IV/IM/SC; labour analgesia; postoperative shivering.
\end{itemize}

\subsubsection*{Kinetics}
\begin{itemize}
  \item pKa 8.6: highly ionised $\rightarrow$ slower CNS entry.
  \item Moderate lipid solubility.
  \item Oral bioavailability $\sim$50\%.
  \item Metabolism: hepatic $\rightarrow$ norpethidine (neurotoxic; half-life 15--30 h).
  \item Elimination: renal; metabolite accumulates in renal failure.
\end{itemize}

\subsubsection*{Effects}
\begin{itemize}
  \item Antimuscarinic: tachycardia, dry mouth.
  \item Seizure risk from norpethidine.
  \item Histamine release.
  \item Serotonin syndrome with MAOIs/SSRIs.
\end{itemize}

\subsection{Tapentadol}

\textbf{Atypical centrally acting analgesic combining $\mu$-agonism with noradrenaline reuptake inhibition (no serotonergic activity; closer to tramadol’s concept but more $\mu$-potent).}

\subsubsection*{Presentation \& Uses}
\begin{itemize}
  \item Oral immediate-release and prolonged-release tablets.
  \item Moderate--severe acute postoperative pain; chronic musculoskeletal and neuropathic pain.
\end{itemize}

\subsubsection*{Kinetics}
\begin{itemize}
  \item Dual mechanism: $\mu$-agonist + NA-reuptake inhibitor.
  \item Oral bioavailability: $\sim$32\%.
  \item Metabolism: extensive hepatic metabolism (mostly phase-2 glucuronidation; minor CYP involvement $\rightarrow$ fewer drug interactions).
  \item Half-life: 4--6 hours.
  \item Elimination: renal.
\end{itemize}

\subsubsection*{Effects}
\begin{itemize}
  \item Less nausea/vomiting than equianalgesic tramadol or codeine.
  \item Minimal serotonergic effect $\rightarrow$ lower risk of serotonin syndrome compared with tramadol.
  \item Lower seizure risk than tramadol.
  \item Respiratory depression still possible at high doses.
\end{itemize}

\section{SECTION 4 -- Adverse Effects (Expanded)}

\begin{itemize}
  \item Respiratory depression
  \item Constipation (no tolerance)
  \item Chest wall rigidity
  \item Histamine release
  \item Immunosuppression
  \item Endocrine suppression
\end{itemize}

\section{SECTION 5 -- Tolerance, Dependence, Withdrawal (Expanded)}

\begin{itemize}
  \item Tolerance due to receptor desensitisation, downregulation, NMDA activation.
  \item No tolerance: miosis, constipation.
  \item Withdrawal: mydriasis, diarrhoea, sweating, cramps.
\end{itemize}

\section{SECTION 6 -- Partial Agonists \& Antagonists}

\subsection*{Opioid Antagonists (Full Expansion)}

Opioid antagonists competitively displace opioids from $\mu$, $\kappa$, and $\delta$ receptors. They have high affinity, minimal intrinsic activity, and are essential in toxicity management, reversal of anaesthetic opioid effects, and treatment of dependence.

\subsection{Naloxone}

\textbf{Short-acting competitive opioid antagonist at $\mu$, $\kappa$ and $\delta$ receptors; first-line drug for acute opioid toxicity.}

\subsubsection*{Presentation \& Uses}
\begin{itemize}
  \item IV/IM/SC/IN formulations.
  \item First-line treatment for opioid overdose.
  \item Used intraoperatively to reverse excessive respiratory depression.
  \item Can treat opioid-induced pruritus and urinary retention (small doses).
\end{itemize}

\subsubsection*{Kinetics}
\begin{itemize}
  \item Half-life: 20--60 minutes (much shorter than most opioids $\rightarrow$ recurrence of respiratory depression possible).
  \item Onset: 1--2 minutes IV; 2--5 minutes IM/SC.
  \item Bioavailability: poor orally ($<$3\%); intranasal $\sim$50\%.
  \item Metabolism: hepatic glucuronidation; renal excretion.
\end{itemize}

\subsubsection*{Effects}
\begin{itemize}
  \item Rapid reversal of respiratory/CNS depression.
  \item Can precipitate acute withdrawal in dependent individuals (hypertension, tachycardia, agitation, vomiting).
  \item May blunt analgesia completely.
\end{itemize}

\subsection{Buprenorphine}

\textbf{Partial $\mu$-agonist and $\kappa$-antagonist with extremely high receptor affinity and slow dissociation.}

\subsubsection*{Key Points (Peck \& Harris)}
\begin{itemize}
  \item $\mu$-mediated analgesia with a ceiling effect (especially for respiratory depression).
  \item Very high receptor affinity $\rightarrow$ displaces full agonists $\rightarrow$ can precipitate acute withdrawal.
  \item Difficult to reverse with naloxone (requires high, repeated doses).
  \item Used in chronic pain and opioid-dependence maintenance therapy.
\end{itemize}

\subsubsection*{Doses}
\begin{itemize}
  \item Sublingual: 200--400 micrograms every 6--8 hours for analgesia.
  \item IV/IM: 300 micrograms every 6--8 hours.
  \item Transdermal patches: 5--20 micrograms/hour (long-acting analgesia).
  \item Substitution therapy: higher-dose formulations (not typically anaesthetic practice).
\end{itemize}

\subsection{Nalbuphine}

\textbf{$\kappa$-agonist and $\mu$-antagonist with limited analgesic ceiling.}

\subsubsection*{Key Points (Peck \& Harris)}
\begin{itemize}
  \item Produces spinal analgesia with minimal respiratory depression.
  \item $\kappa$-agonism $\rightarrow$ dysphoria, unpleasant psychotomimetic effects.
  \item Very effective for opioid-induced pruritus (via $\mu$-antagonism).
  \item Can precipitate withdrawal in opioid-dependent patients.
\end{itemize}

\subsubsection*{Doses}
\begin{itemize}
  \item IV/IM/SC: 10--20 mg every 3--6 hours.
  \item Maximum recommended daily dose: 160 mg.
\end{itemize}

\subsection{Butorphanol}

\textbf{Predominantly $\kappa$-agonist with weak partial $\mu$-agonist activity.}

\subsubsection*{Key Points (Peck \& Harris)}
\begin{itemize}
  \item Produces moderate analgesia with a ceiling effect.
  \item $\kappa$-agonism $\rightarrow$ dysphoria, unpleasant psychological effects.
  \item Less respiratory depression than full $\mu$-agonists.
  \item Can cause hypertension and tachycardia at higher doses.
\end{itemize}

\subsubsection*{Doses}
\begin{itemize}
  \item IV/IM: 1--2 mg, repeat every 3--4 hours as needed.
  \item Intranasal (for migraine): 1 mg in each nostril, may repeat once after 60--90 minutes.
\end{itemize}

\end{document}
