\documentclass[11pt,a4paper]{article}

\usepackage[a4paper,margin=2.2cm]{geometry}
\usepackage[T1]{fontenc}
\usepackage[utf8]{inputenc} % pdfLaTeX-safe
\usepackage{lmodern}
\usepackage{microtype}
\usepackage{hyperref}
\usepackage{booktabs}
\usepackage{tabularx}
\usepackage{array}
\usepackage{enumitem}
\usepackage{amsmath}


\begin{document}

\section{Adrenoceptor Antagonists}

\subsection{High-yield map}
\begin{itemize}
  \item \textbf{$\alpha$-antagonists}: non-selective ($\alpha_1 + \alpha_2$), selective $\alpha_1$ (prazosin), selective $\alpha_2$ (yohimbine).
  \item \textbf{$\beta$-antagonists ($\beta$-blockers)}: competitive antagonists; key differentiators are \textbf{$\beta_1$-selectivity}, \textbf{intrinsic sympathomimetic activity (ISA)}, and \textbf{membrane-stabilising activity}.
  \item \textbf{Combined $\alpha + \beta$}: labetalol ($\alpha_1$ + non-selective $\beta$).
\end{itemize}

\subsection{$\alpha$-adrenoceptor antagonists}

\subsubsection{Table 14.1 --- Actions of specific $\alpha$-adrenoceptor stimulation}
\begin{center}
\begin{tabular}{|p{2.2cm}|p{2.2cm}|p{8.2cm}|}
\hline
\textbf{Site} & \textbf{Receptor} & \textbf{Actions} \\
\hline
Postsynaptic & $\alpha_1$ & Vasoconstriction; mydriasis; contraction of bladder sphincter \\
\hline
Postsynaptic & $\alpha_2$ & Platelet aggregation; hyperpolarisation of some CNS neurones \\
\hline
Presynaptic & $\alpha_2$ & Inhibits noradrenaline release \\
\hline
\end{tabular}
\end{center}

\subsubsection{Non-selective $\alpha$-blockade}

\paragraph{Phentolamine (competitive, non-selective).}
\begin{itemize}
  \item \textbf{Affinity}: $\alpha_1$ affinity $\approx 3\times \alpha_2$.
  \item \textbf{Presentation}: 10 mg phentolamine mesylate in 1 mL (clear pale-yellow solution).
  \item \textbf{Dose}: \textbf{IV 1--5 mg}, titrate to effect. \textbf{Onset 1--2 min}, \textbf{duration 5--20 min}.
  \item \textbf{Uses}: hypertensive crises due to excessive sympathomimetics; MAOI + tyramine reactions; phaeochromocytoma (especially during tumour manipulation); assessment of sympathetically mediated chronic pain; previously used for pulmonary hypertension; intracavernosal use for erectile failure.
  \item \textbf{Effects}:
    \begin{itemize}
      \item \textbf{Cardiovascular}: $\alpha_1$ blockade $\rightarrow$ vasodilatation + hypotension; $\alpha_2$ blockade $\rightarrow$ facilitates noradrenaline release $\rightarrow$ tachycardia + increased cardiac output; reduced pulmonary artery pressure; nasal congestion (nasal mucosal vasodilatation).
      \item \textbf{Respiratory}: sulphites may trigger hypersensitivity $\rightarrow$ acute bronchospasm in susceptible asthmatics.
      \item \textbf{Gastrointestinal}: increased GI secretions and motility.
      \item \textbf{Metabolic}: may precipitate hypoglycaemia (increased insulin secretion).
    \end{itemize}
  \item \textbf{Kinetics}: oral route rarely used (bioavailability $\sim 20\%$); $\sim 50\%$ protein-bound; extensively metabolised; $\sim 10\%$ excreted unchanged in urine; elimination $t_{1/2} \sim 20$ min.
\end{itemize}

\paragraph{Phenoxybenzamine (long-acting, non-selective).}
\begin{itemize}
  \item \textbf{Presentation}: 10 mg capsules; injection 100 mg/2 mL (with ethyl alcohol, hydrochloric acid and propylene glycol).
  \item \textbf{Uses}: pre-operative phaeochromocytoma (to facilitate intravascular expansion); peri-operative management of some neonates undergoing cardiac surgery; hypertensive crises; occasionally as an adjunct in severe shock.
    \begin{itemize}
      \item \textbf{Oral}: start 10 mg, increase daily; typical \textbf{1--2 mg\,kg$^{-1}$\,day$^{-1}$}.
      \item \textbf{IV}: via central cannula; \textbf{$\sim$1 mg\,kg$^{-1}$\,day$^{-1}$} as a slow infusion in $\geq 200$ mL 0.9\% saline.
      \item $\beta$-blockade may be needed to limit reflex tachycardia.
    \end{itemize}
  \item \textbf{Mechanism}: reactive intermediate forms a \textbf{covalent bond} to the $\alpha$-receptor $\rightarrow$ \textbf{irreversible} blockade; also inhibits neuronal and extra-neuronal catecholamine uptake.
  \item \textbf{Effects}:
    \begin{itemize}
      \item \textbf{Cardiovascular}: hypotension (may be orthostatic) + reflex tachycardia. \textbf{Overdose: treat with noradrenaline.} Adrenaline can produce \textbf{unopposed $\beta$ effects} $\rightarrow$ worsened hypotension/tachycardia.
      \item \textbf{Perfusion}: increased cardiac output and increased blood flow to skin/viscera/nasal mucosa $\rightarrow$ nasal congestion.
      \item \textbf{CNS}: marked sedation; convulsions reported after rapid IV infusion; miosis.
      \item \textbf{Miscellaneous}: impotence; contact dermatitis.
    \end{itemize}
  \item \textbf{Kinetics}: oral bioavailability $\sim 25\%$ (incomplete/variable absorption); maximum effect $\sim 1$ h after IV dose; plasma $t_{1/2} \sim 24$ h; effects may persist $\sim 3$ days (while new receptors are synthesised); hepatic metabolism; excreted in urine and bile.
\end{itemize}

\subsubsection{Selective $\alpha_1$-blockade}

\paragraph{Prazosin (quinazoline; highly selective $\alpha_1$ antagonist).}
\begin{itemize}
  \item \textbf{Formulation}: 0.5--2 mg tablets.
  \item \textbf{Uses}: essential hypertension; congestive heart failure; Raynaud's syndrome; benign prostatic hypertrophy.
  \item \textbf{Dose}: start \textbf{0.5 mg tds}, increase up to \textbf{20 mg/day}.
  \item \textbf{Effects}:
    \begin{itemize}
      \item \textbf{Cardiovascular}: arterial and venous dilatation $\rightarrow$ reduced SVR with little/no reflex tachycardia; diastolic pressure falls most; \textbf{first-dose} severe postural hypotension + syncope possible; cardiac output may increase in HF (reduced filling pressures).
      \item \textbf{Urinary}: relaxes bladder trigone and sphincter $\rightarrow$ improved urine flow in BPH; impotence/priapism reported.
    \end{itemize}
  \item \textbf{Kinetics}: peak $\sim 90$ min after oral dose; oral bioavailability 50--80\% (variable); highly protein-bound (mainly albumin); hepatic metabolism (demethylation and conjugation) with some active metabolites; plasma $t_{1/2} \sim 3$ h; excreted largely in bile $\rightarrow$ may be used safely in renal impairment.
\end{itemize}

\subsubsection{Selective $\alpha_2$-blockade}

\paragraph{Yohimbine.}
\begin{itemize}
  \item Alkaloid from yohimbe tree (as hydrochloride); used in impotence.
  \item Variable cardiovascular effects: may raise HR and BP, but can precipitate orthostatic hypotension.
  \item In vitro blocks hypotensive responses of clonidine.
  \item Antidiuretic; may cause anxiety and manic reactions.
  \item Contraindicated in renal or hepatic disease.
\end{itemize}

\subsection{$\beta$-adrenoceptor antagonists ($\beta$-blockers)}

\subsubsection{Core differentiators}
\begin{itemize}
  \item Competitive antagonists with varying receptor selectivity.
  \item Some have \textbf{ISA} (partial agonism); some show \textbf{membrane-stabilising} activity.
  \item Prolonged administration may increase the number of $\beta$-adrenoceptors.
\end{itemize}

\subsubsection{Receptor selectivity}
\begin{itemize}
  \item Desired effects are largely via \textbf{$\beta_1$ antagonism}; \textbf{$\beta_2$ antagonism} drives many unwanted effects.
  \item \textbf{Cardioselective ($\beta_1$)}: atenolol, esmolol, metoprolol (selectivity reduces at high doses).
  \item Use with extreme caution in poor ventricular function (may precipitate serious cardiac failure).
\end{itemize}

\subsubsection{ISA - Intrinsec Sympathomimetic Activity - (partial agonist activity)}
\begin{itemize}
  \item Partial agonists may produce sympathomimetic effects when catecholamines are low, and antagonism when sympathetic tone is high.
  \item In mild cardiac failure: theoretically less bradycardia/HF; avoid in more severe HF.
\end{itemize}

\subsubsection{Membrane-stabilising activity}
\begin{itemize}
  \item Probably little clinical significance (requires higher doses than those achieved in vivo).
\end{itemize}

\subsubsection{Table 14.2 --- $\beta$-blocker qualitative pharmacology}
\begin{center}
\begin{tabular}{|p{3.0cm}|p{2.0cm}|p{1.5cm}|p{2.6cm}|}
\hline
\textbf{Drug} & \textbf{$\beta_1$ selectivity} & \textbf{ISA} & \textbf{Membrane-stabilising} \\
\hline
Acebutolol & $+$ & $+$ & $+$ \\
\hline
Atenolol & $++$ & $-$ & $-$ \\
\hline
Esmolol & $++$ & $-$ & $-$ \\
\hline
Metoprolol & $++$ & $-$ & $+$ \\
\hline
Pindolol & $-$ & $++$ & $+$ \\
\hline
Propranolol & $-$ & $-$ & $++$ \\
\hline
Sotalol & $-$ & $-$ & $-$ \\
\hline
Timolol & $-$ & $+$ & $+$ \\
\hline
Labetalol & $-$ & $\pm$ & $+$ \\
\hline
\end{tabular}
\end{center}

\subsubsection{System effects}
\begin{itemize}
  \item \textbf{Cardiac}: negative chronotropy/inotropy; reduced SA automaticity; prolonged AV conduction time $\rightarrow$ bradycardia; improved O$_2$ supply/demand balance underpins use in angina and peri-MI; class II anti-arrhythmics (especially catecholamine-associated arrhythmias). Can precipitate cardiac failure in poor LV function.
  \item \textbf{Circulatory (BP)}: reduction via reduced HR/CO and inhibition of renin--angiotensin (via $\beta_1$ at JGA $\rightarrow$ reduced renin). Peripheral $\beta_2$ antagonism can cause vasoconstriction $\rightarrow$ cold hands/poor peripheral circulation.
  \item \textbf{Respiratory}: in sufficient dose, all can precipitate bronchospasm via $\beta_2$ antagonism; cardioselective agents are preferred but still require extreme caution in asthma.
  \item \textbf{Metabolic}: complicates blood sugar control; may mask symptoms of hypoglycaemia; alters lipids (increased triglycerides, reduced HDL).
  \item \textbf{CNS}: lipid-soluble agents (metoprolol, propranolol) more likely to cause depression, hallucinations, nightmares, paranoia, fatigue.
  \item \textbf{Ocular}: reduced intra-ocular pressure (likely reduced aqueous humour production).
  \item \textbf{Gastrointestinal}: dry mouth and GI disturbances.
\end{itemize}

\subsubsection{Kinetics (unifying concept)}
\begin{itemize}
  \item Main kinetic differences relate to \textbf{lipid solubility}:
    \begin{itemize}
      \item \textbf{Low lipid solubility} (e.g. atenolol): poorer gut absorption, little hepatic metabolism, renal excretion largely unchanged.
      \item \textbf{High lipid solubility}: good absorption, extensive hepatic metabolism, shorter $t_{1/2}$; crosses BBB $\rightarrow$ sedation/nightmares.
      \item Protein binding is variable.
    \end{itemize}
\end{itemize}

\subsubsection{Table 14.3 --- Selected $\beta$-blocker pharmacokinetics (Part A)}
\begin{center}
\small
\begin{tabular}{|p{2.6cm}|p{1.6cm}|p{1.6cm}|p{1.8cm}|p{1.8cm}|p{1.8cm}|}
\hline
\textbf{Drug} & \textbf{Lipid solubility} & \textbf{Absorption (\%)} & \textbf{Bioavailability (\%)} & \textbf{Protein binding (\%)} & \textbf{Elimination $t_{1/2}$ (h)} \\
\hline
Acebutolol & $++$ & 90 & 40 & 25 & 6 \\
\hline
Atenolol & $+$ & 45 & 45 & 5 & 7 \\
\hline
Esmolol & $+++$ & n/a & n/a & 60 & 0.15 \\
\hline
Metoprolol* & $+++$ & 95 & 50 & 20 & 3--7* \\
\hline
Oxprenolol & $+++$ & 80 & 40 & 80 & 2 \\
\hline
Pindolol & $++$ & 90 & 90 & 50 & 4 \\
\hline
Propranolol & $+++$ & 90 & 30 & 90 & 4 \\
\hline
Sotalol & $+$ & 85 & 85 & 0 & 15 \\
\hline
Timolol & $+++$ & 90 & 50 & 10 & 4 \\
\hline
Labetalol & $+++$ & 70 & 25 & 50 & 5 \\
\hline
\end{tabular}
\end{center}
\noindent\small *Metoprolol half-life depends on genetic polymorphism (fast/slow hydroxylators).\normalsize

\subsubsection{Table 14.3 --- Selected $\beta$-blocker pharmacokinetics (Part B)}
\begin{center}
\small
\begin{tabular}{|p{2.6cm}|p{8.2cm}|p{2.2cm}|}
\hline
\textbf{Drug} & \textbf{Clearance} & \textbf{Active metabolites} \\
\hline
Acebutolol & Hepatic metabolism and renal excretion & Yes \\
\hline
Atenolol & Renal & No \\
\hline
Esmolol & Plasma hydrolysis & No \\
\hline
Metoprolol & Hepatic metabolism & No \\
\hline
Oxprenolol & Hepatic metabolism & No \\
\hline
Pindolol & Hepatic metabolism & No \\
\hline
Propranolol & Hepatic metabolism & Yes \\
\hline
Sotalol & Renal & No \\
\hline
Timolol & Hepatic metabolism and renal excretion & No \\
\hline
Labetalol & Hepatic metabolism & No \\
\hline
\end{tabular}
\end{center}
\normalsize

\subsubsection{Individual $\beta$-blockers described in the chapter}

\paragraph{Acebutolol.}
Relatively $\beta_1$-selective; oral only; limited ISA; some membrane-stabilising activity. Dose 400 mg bd (up to 1.2 g/day).
Kinetics: oral bioavailability $\sim 40\%$ (first pass); minimal BBB penetration; hepatic metabolism to active metabolite diacetol (longer $t_{1/2}$, less cardioselective); excreted in bile (possible enterohepatic recycling) and urine (reduce dose in renal impairment).

\paragraph{Atenolol.}
Relatively $\beta_1$-selective. Presentation: tablets 25--100 mg; syrup 5 mg/mL; IV 5 mg/10 mL.
Dose: oral 50--100 mg/day; IV 2.5 mg slowly, repeat to max 10 mg (may follow with an infusion).
Kinetics: incomplete absorption; little metabolism; oral bioavailability $\sim 45\%$; $\sim 5\%$ protein-bound; excreted unchanged in urine (reduce dose in renal impairment); elimination $t_{1/2} \sim 7$ h.

\paragraph{Esmolol.}
Highly lipophilic, $\beta_1$-selective; rapid onset/offset. Presentation: 2.5 g or 100 mg in 10 mL.
Use: short-term peri-operative tachycardia/hypertension; acute SVT. Dose: infusion 50--200 $\mu$g\,kg$^{-1}$\,min$^{-1}$ (2.5 g vial diluted); 10 mg boluses (100 mg vial).
Kinetics: IV only; $\sim 60\%$ protein-bound; Vd $\sim 3.5$ L/kg; metabolised by red cell esterases to inactive acid metabolite and methanol; $t_{1/2} \sim 10$ min; distinct from plasma cholinesterase (does not prolong suxamethonium). Notes: irritant---extravasation may cause tissue necrosis.

\paragraph{Metoprolol.}
Relatively $\beta_1$-selective; no ISA. Uses: hypertension; adjunct in thyrotoxicosis; migraine prophylaxis; early use in MI reduces infarct size and VF.
Dose: 50--200 mg/day; up to 5 mg IV (arrhythmias/MI).
Kinetics: oral bioavailability $\sim 50\%$ (first pass) but increases with continuous administration and with food; genetic polymorphism ($t_{1/2}$ profiles $\sim 3$ and 7 h); high lipid solubility crosses BBB and into breast milk; $\sim 20\%$ protein-bound.

\paragraph{Propranolol.}
Non-selective; no ISA; racemic mixture: S-isomer confers most $\beta$-blockade; R-isomer inhibits peripheral conversion of T$_4$ to T$_3$.
Uses: hypertension; angina; essential tremor; migraine prophylaxis; $\beta$-blocker of choice in thyrotoxicosis.
Dose: IV 0.5 mg (up to 10 mg) titrate; oral 160--320 mg/day (higher may be required in thyrotoxicosis).
Kinetics: high lipid solubility; oral bioavailability $\sim 30\%$ (first pass); highly protein-bound (may be reduced by heparin); hepatic metabolism yields active metabolite (4-hydroxypropranolol); duration longer than $t_{1/2} \sim 4$ h suggests.

\paragraph{Sotalol.}
Non-selective; no ISA; also class III anti-arrhythmic (racemic: D-isomer class III; L-isomer class III + $\beta$-blockade).
Uses: ventricular tachyarrhythmias; prophylaxis of paroxysmal SVT after DC cardioversion; rate control if AF recurs. (CSM: not for angina, hypertension, thyrotoxicosis, peri-MI.)
Dose: oral 80--160 mg bd; IV 50--100 mg over 20 min.
Key adverse effect: torsades de pointes (rare; $<2\%$ in sustained VT/VF treatment), risk increases with higher dose, prolonged QT, electrolyte imbalance; may precipitate HF.
Kinetics: oral bioavailability $>90\%$; not protein-bound; not metabolised; $\sim 90\%$ excreted unchanged in urine; renal impairment markedly reduces clearance.

\subsection{Combined $\alpha$- and $\beta$-adrenoceptor antagonists}

\subsubsection{Labetalol}
\begin{itemize}
  \item \textbf{Receptors}: $\alpha_1$ blockade + non-selective $\beta$ blockade.
  \item \textbf{Stereoisomers}: two asymmetric centres $\rightarrow$ four stereoisomers (equal proportions); (SR) probably responsible for $\alpha_1$ effects; (RR) probably confers $\beta$ blockade.
  \item \textbf{$\alpha_1:\beta$ effect ratio}: route-dependent --- oral 1:3, IV 1:7.
  \item \textbf{Presentation}: tablets 50--400 mg; colourless solution 5 mg/mL.
  \item \textbf{Uses}: hypertensive crises; facilitation of hypotension during anaesthesia; oral use for hypertension associated with angina and during pregnancy.
  \item \textbf{Doses}: IV 5--20 mg titrated (maximum 200 mg); oral 100--800 mg bd (maximum 2.4 g/day).
  \item \textbf{Mechanism}: $\alpha_1$ blockade causes peripheral vasodilatation; $\beta$ blockade prevents reflex tachycardia; reduces afterload and myocardial O$_2$ demand.
  \item \textbf{Kinetics}: well absorbed but extensive first pass; oral bioavailability $\sim 25\%$ (may increase with age and with food); $\sim 50\%$ protein-bound; hepatic metabolism produces inactive conjugates.
\end{itemize}

\subsection{Exam-focused synthesis (common traps)}
\begin{itemize}
  \item \textbf{$\alpha_1$-selective vs non-selective $\alpha$-blockade}: $\alpha_1$-selective (prazosin) gives little/no reflex tachycardia; non-selective $\alpha$-blockade removes $\alpha_2$ inhibition and may increase tachycardia.
  \item \textbf{Phenoxybenzamine + adrenaline}: adrenaline may worsen hypotension/tachycardia (unopposed $\beta$); overdose management uses noradrenaline.
  \item \textbf{$\beta_1$-selectivity is dose-dependent}: cardioselective agents can still precipitate bronchospasm at higher doses.
  \item \textbf{Lipid solubility}: predicts BBB penetration (CNS effects) and hepatic vs renal handling.
  \item \textbf{Esmolol}: red cell esterase metabolism $\rightarrow$ very short $t_{1/2}$; distinct from plasma cholinesterase (does not prolong suxamethonium).
  \item \textbf{Sotalol}: remember torsades risk, renal clearance, and class III properties.
\end{itemize}

\subsection{Additions from other references (complements to Peck)}
\subsubsection{Extra $\alpha$-antagonists and selectivity}
\begin{itemize}
  \item Other $\alpha_1$-antagonists used clinically: alfuzosin, doxazosin, tamsulosin. \textit{(Fundamentals of Anaesthesia 4e, Ch 35, p.685)}
  \item $\alpha_{1A}$-selective (“uroselective”) agents: tamsulosin and alfuzosin (BPH). \textit{(Fundamentals 4e, Ch 35, p.685)}
\end{itemize}

\subsubsection{Phenoxybenzamine: additional mechanistic detail}
\begin{itemize}
  \item Haloalkylamine; N-chloroethyl group binds covalently $\rightarrow$ “competitive irreversible”; reported recovery half-life $\sim 24$ h. \textit{(Fundamentals 4e, Ch 35, p.685)}
  \item Also antagonises cholinergic, 5-HT, and histamine receptors. \textit{(Fundamentals 4e, Ch 35, p.685)}
\end{itemize}

\subsubsection{Phentolamine: practical uses}
\begin{itemize}
  \item Useful for hypertension due to excessive $\alpha$-stimulation including clonidine withdrawal. \textit{(Morgan \& Mikhail 5e, Ch 14, p.249)}
  \item Extravasation management: infiltrate phentolamine 5--10 mg in 10 mL normal saline locally after $\alpha$-agonist extravasation. \textit{(Morgan \& Mikhail 5e, Ch 14, p.249)}
\end{itemize}

\subsubsection{Labetalol: additional details}
\begin{itemize}
  \item Route-dependent predominance: $\alpha$ blockade more prominent IV; $\beta$ blockade more prominent orally. \textit{(Fundamentals 4e, Ch 35, p.685)}
  \item Post-op hypertension: labetalol 5 mg IV boluses act within $\sim 5$ min and last up to $\sim 1$ h. \textit{(Fundamentals 4e, Ch 4, p.64)}
  \item IV dosing schema for hypertension: 2.5--10 mg IV over 2 min, repeat (up to double) every 10 min; infusion 0.5--2 mg/min; peak effect $\sim 5$ min; prolonged infusions not recommended (elimination half-life $>5$ h). \textit{(Morgan \& Mikhail 5e, Ch 14, p.249--250)}
\end{itemize}

\subsubsection{$\beta$-blockers: interaction and perioperative handling}
\begin{itemize}
  \item Verapamil + $\beta$-blocker: avoid (synergistic depression of HR, contractility, AV conduction). \textit{(Fundamentals 4e, Ch 36, p.707; Morgan \& Mikhail 5e, Ch 14, p.251)}
  \item Perioperative continuation: continue established $\beta$-blockers; routine high-dose initiation without titration may be harmful. \textit{(Morgan \& Mikhail 5e, Ch 14, p.252; Primary FRCA in a Box 2e, p.205)}
  \item Withdrawal: stopping for 24--48 h may trigger rebound hypertension, tachycardia, angina (attributed to $\beta$-receptor up-regulation). \textit{(Morgan \& Mikhail 5e, Ch 14, p.252)}
\end{itemize}

\subsubsection{Newer $\beta$-blockers (not in Peck Ch 14)}
\begin{itemize}
  \item Nebivolol: high affinity for $\beta_1$; direct vasodilation via endothelial NO synthase stimulation; oral only (5--40 mg daily stated). \textit{(Morgan \& Mikhail 5e, Ch 14, p.251)}
  \item Carvedilol: mixed $\beta$ and $\alpha$ blocker used in chronic HF, post-MI LV dysfunction, and hypertension; titrated up to 25 mg twice daily as tolerated. \textit{(Morgan \& Mikhail 5e, Ch 14, p.251)}
\end{itemize}

\end{document}
