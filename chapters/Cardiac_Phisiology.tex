\documentclass[11pt,a4paper]{article}

% ---------- ENCODING & LANGUAGE ----------
\usepackage[T1]{fontenc}
\usepackage[utf8]{inputenc}
\usepackage[english]{babel}

% ---------- MATH & LAYOUT ----------
\usepackage{lmodern}
\usepackage{amsmath,amssymb}
\usepackage{geometry}
\geometry{margin=2cm}

\setlength{\parindent}{0pt}
\setlength{\parskip}{6pt}

\begin{document}

%%%%%%%%%%%%%%%%%%%%%%%%%%%%%%%%%%%%%%%%%%%%%%%%
% Chapter 23 – Function of the Cardiovascular System
%%%%%%%%%%%%%%%%%%%%%%%%%%%%%%%%%%%%%%%%%%%%%%%%

\section{Chapter 23 – Function of the Cardiovascular System}

\subsection{Introduction}

The cardiovascular system ensures that every tissue receives an adequate supply of blood in order to meet its constantly changing metabolic requirements. This involves delivering oxygen and nutrients, removing metabolic waste, redistributing flow between organs and maintaining a stable internal environment.

The heart, vasculature and regulatory systems must cooperate to generate sufficient pressure, produce appropriate flow and distribute blood according to local needs. Kam \& Powers emphasise that the circulation must simultaneously provide both \emph{adequate arterial pressure} and \emph{appropriate flow distribution}.

%----------------------------------------------
\subsection{Core Functions of the Cardiovascular System}

\subsubsection{Transport Functions}

The circulation is the body's main transport network:
\begin{itemize}
  \item Delivery of oxygen, glucose, fatty acids and other nutrients to support cellular metabolism.
  \item Removal of carbon dioxide, lactate and other metabolic by-products.
  \item Distribution of hormones and signalling molecules to distant targets.
  \item Transport of heat from metabolically active tissues to the skin for thermal regulation.
  \item Facilitation of immune surveillance by circulating leukocytes and antibodies.
\end{itemize}

\subsubsection{Pressure Generation}

\begin{itemize}
  \item The heart generates arterial pressure, the driving force for blood flow through systemic and pulmonary circulations.
  \item Pressure gradients between arteries and veins are required for organ perfusion.
  \item Adequate arterial pressure is also necessary for capillary filtration and reabsorption and for maintaining coronary perfusion.
\end{itemize}

\subsubsection{Homeostatic Roles}

In addition to transport and pressure generation, the cardiovascular system contributes to homeostasis:
\begin{itemize}
  \item Regulation of extracellular fluid composition by ensuring appropriate renal perfusion.
  \item Maintenance of body temperature via controlled changes in skin blood flow.
  \item Stabilisation of arterial pressure despite changes in posture, activity and environment.
\end{itemize}

%----------------------------------------------
\subsection{Pressure, Flow and Resistance}

\subsubsection{Fundamental Relationship}

Flow in the circulation follows an Ohm-like relationship:
\[
Q = \frac{\Delta P}{R},
\]
where $Q$ is flow, $\Delta P$ is the pressure difference between arterial and venous ends, and $R$ is vascular resistance (dominated by arterioles).

\subsubsection{Systemic vs. Pulmonary Circulation}

\begin{itemize}
  \item \textbf{Systemic circulation:} high pressure, high resistance; designed to perfuse tissues at varying distances and heights from the heart.
  \item \textbf{Pulmonary circulation:} low pressure, low resistance; allows efficient gas exchange without damaging the delicate alveolar--capillary interface.
\end{itemize}

\subsubsection{Importance of Maintaining Arterial Pressure}

Adequate mean arterial pressure (MAP) is essential to:
\begin{itemize}
  \item Perfuse organs at or above heart level, especially the brain.
  \item Maintain appropriate Starling forces for controlled capillary fluid exchange.
  \item Preserve coronary perfusion, which largely depends on aortic diastolic pressure.
\end{itemize}

%----------------------------------------------
\subsection{Heart Anatomy and Coronary Circulation}

\subsubsection{Basic Anatomical Organisation}

\begin{itemize}
  \item The heart consists of four chambers: two atria and two ventricles.
  \item The right heart receives systemic venous blood and pumps it into the low-pressure pulmonary circulation.
  \item The left heart receives oxygenated pulmonary venous blood and pumps it into the high-pressure systemic circulation.
  \item Unidirectional flow is maintained by the tricuspid, pulmonary, mitral and aortic valves.
  \item The interventricular septum is largely muscular and contributes importantly to the mechanical function of both ventricles.
\end{itemize}

\subsubsection{Myocardial Blood Supply}

\begin{itemize}
  \item The myocardium has a high metabolic rate and extracts a large fraction of the delivered oxygen at rest; increases in demand must therefore be met mainly by increased coronary flow.
  \item The coronary arteries arise from the aortic root just above the aortic valve.
  \item The left coronary artery (LCA) divides into the left anterior descending (LAD) and circumflex (Cx) branches:
    \begin{itemize}
      \item The LAD runs in the anterior interventricular groove and supplies the anterior LV wall, anterior septum and frequently the apex.
      \item The circumflex artery travels in the left atrioventricular groove and supplies the lateral LV wall; in left-dominant coronary anatomy it also gives rise to the posterior descending artery.
    \end{itemize}
  \item The right coronary artery (RCA) runs in the right atrioventricular groove and supplies the right ventricle, the inferior LV, parts of the interventricular septum and important components of the conduction system. In right-dominant anatomy it gives rise to the posterior descending artery.
\end{itemize}

\subsubsection{Phasic Nature of Coronary Flow}

\begin{itemize}
  \item Coronary perfusion, particularly in the LV, is strongly phasic due to the large intramural pressures generated during systole.
  \item During LV systole intramyocardial pressure may exceed aortic pressure, compressing intramural vessels and reducing or reversing forward flow, especially in subendocardial layers.
  \item Most LV perfusion therefore occurs during diastole, when myocardial relaxation lowers intramural pressures below aortic diastolic pressure.
  \item RV perfusion is more continuous, as RV systolic pressures are lower and extravascular compression is less.
  \item Subendocardial regions are most vulnerable to ischaemia due to higher wall stress and shorter perfusion time, particularly during tachycardia, hypotension or elevated LV end-diastolic pressure (LVEDP).
\end{itemize}

\subsubsection{Determinants of Coronary Blood Flow}

Coronary flow is determined by:
\begin{itemize}
  \item Coronary perfusion pressure (CPP), approximated by
        \[
        \text{CPP} \approx P_{\text{aortic, diastolic}} - \text{LVEDP}.
        \]
  \item Coronary vascular resistance, influenced by metabolic demand, autonomic tone, endothelial factors and mechanical compression.
  \item Myocardial oxygen demand, which depends mainly on heart rate, wall tension and contractility.
\end{itemize}

\subsubsection{Neural Supply of the Heart}

\paragraph{Sympathetic innervation}
\begin{itemize}
  \item Preganglionic fibres arise from T1--T5 segments of the spinal cord and synapse in cervical and upper thoracic ganglia.
  \item Postganglionic fibres innervate atria, ventricles and coronary vessels.
  \item Effects: increased heart rate (positive chronotropy), increased conduction velocity (dromotropy), increased contractility (inotropy) and enhanced relaxation (lusitropy).
\end{itemize}

\paragraph{Parasympathetic innervation}
\begin{itemize}
  \item Supplied primarily by the vagus nerve.
  \item Mainly innervates the SA node, AV node and atria; ventricular parasympathetic innervation is sparse.
  \item Effects: reduced heart rate, slowed AV nodal conduction and modest reduction in atrial contractility.
\end{itemize}

\paragraph{Reflex control}
Baroreceptor, Bainbridge and Bezold--Jarisch reflexes modulate autonomic outflow, adjusting heart rate, contractility and vascular tone in response to changes in venous return and arterial pressure.

%----------------------------------------------
\subsection{Cardiac Output and Venous Return}

Cardiac output (CO) is given by
\[
\text{CO} = \text{HR} \times \text{SV},
\]
where HR is heart rate and SV is stroke volume.

\begin{itemize}
  \item Stroke volume is influenced by preload, afterload and contractility.
  \item In the steady state, CO must equal venous return.
  \item The venous system, as a high-capacitance reservoir, is central in regulating venous return and hence preload.
\end{itemize}

%----------------------------------------------
\subsection{Distribution of Blood Flow}

\subsubsection{Resting Distribution}

At rest:
\begin{itemize}
  \item Heart, brain and kidneys receive high blood flow relative to their mass.
  \item Skeletal muscle and skin receive more modest flow but can undergo large increases during exercise or thermal stress.
\end{itemize}

\subsubsection{Control of Regional Flow}

Regional flow is governed by changes in arteriolar resistance:
\begin{itemize}
  \item Local metabolic control is the dominant mechanism matching flow to tissue metabolic demand (e.g. adenosine, CO$_2$, H$^+$, K$^+$).
  \item The autonomic nervous system provides global adjustments and redistribution during stress, exercise or haemorrhage.
  \item Endothelial factors such as nitric oxide, prostacyclin and endothelin modulate tone.
  \item Hormones (angiotensin II, vasopressin, adrenaline) influence systemic vascular resistance and volume regulation.
\end{itemize}

%----------------------------------------------
\subsection{Regulation of Arterial Pressure}

Mean arterial pressure (MAP) can be approximated by:
\[
\text{MAP} \approx \text{CO} \times \text{TPR},
\]
where TPR is total peripheral resistance.

\subsubsection{Short-Term Regulation}

\begin{itemize}
  \item The arterial baroreceptor reflex provides rapid feedback control via changes in sympathetic and parasympathetic tone, adjusting heart rate, contractility, arteriolar resistance and venous tone.
\end{itemize}

\subsubsection{Intermediate and Long-Term Regulation}

\begin{itemize}
  \item Hormonal systems (renin--angiotensin--aldosterone, catecholamines, vasopressin) adjust vascular tone and circulating volume.
  \item The kidneys provide long-term control of arterial pressure by regulating extracellular fluid volume and sodium balance.
\end{itemize}

%----------------------------------------------
\subsection{Microcirculation and Capillary Exchange}

\begin{itemize}
  \item Capillaries, with thin walls and large surface area, are the primary site for diffusion of gases, nutrients and waste products.
  \item Fluid movement across capillary walls is governed by Starling forces:
    \begin{itemize}
      \item Capillary hydrostatic pressure.
      \item Interstitial hydrostatic pressure.
      \item Capillary oncotic pressure.
      \item Interstitial oncotic pressure.
    \end{itemize}
  \item The lymphatic system returns filtered fluid and proteins to the circulation, preventing oedema.
\end{itemize}

%----------------------------------------------
\subsection{Autoregulation and Integrated Control}

\subsubsection{Local Autoregulation}

Many organs (e.g. brain, heart, kidneys) exhibit autoregulation, maintaining relatively constant flow despite changes in perfusion pressure through:
\begin{itemize}
  \item Myogenic responses of vascular smooth muscle.
  \item Metabolic vasodilation in response to local changes in metabolites.
\end{itemize}

\subsubsection{Integrated Control}

Effective cardiovascular function requires coordinated interaction between:
\begin{itemize}
  \item The heart as a pressure and flow generator.
  \item The vasculature as a resistance and capacitance network.
  \item Neural and hormonal systems that adjust cardiac performance, vascular tone and volume status.
\end{itemize}

%----------------------------------------------
\subsection{Summary of Chapter 23}

Chapter 23 sets out the foundational concepts of cardiovascular function: generation of pressure, production and distribution of flow, regulation of arterial pressure, microcirculatory exchange and coronary perfusion. These principles underpin the more detailed cardiac and vascular physiology discussed in later chapters.

%%%%%%%%%%%%%%%%%%%%%%%%%%%%%%%%%%%%%%%%%%%%%%%%
% Chapter 24 – Electrical Properties of the Heart
%%%%%%%%%%%%%%%%%%%%%%%%%%%%%%%%%%%%%%%%%%%%%%%%

\section{Chapter 24 – Electrical Properties of the Heart}

\subsection{Introduction}

The heart's ability to function as an effective pump depends fundamentally on its electrical properties. Cardiac tissue generates its own rhythmic impulses, conducts them rapidly and in a coordinated manner, and links electrical activation directly to mechanical contraction. Kam \& Powers emphasise that cardiac electrical behaviour is heterogeneous, with distinct cell types and regions specialised for impulse generation, conduction or contraction.

This chapter explores the ionic basis of cardiac action potentials, mechanisms of automaticity, the structure and function of the conduction system, excitation--contraction coupling and the principles underlying the electrocardiogram (ECG).

%----------------------------------------------
\subsection{Types of Cardiac Cells}

Cardiac electrical responses depend on two major categories of cells: fast-response (non-pacemaker) cells and slow-response (pacemaker) cells.

\subsubsection{Fast-Response Cells (Atrial, Ventricular, Purkinje)}

Fast-response cells generate action potentials with a rapid upstroke and a stable resting potential.

\paragraph{Key characteristics}
\begin{itemize}
  \item Resting membrane potential around $-90\,\text{mV}$, strongly negative due to high K$^+$ conductance.
  \item Phase 0 depolarisation mediated by rapid opening of voltage-gated Na$^+$ channels.
  \item High conduction velocity, particularly in Purkinje fibres (2--4 m/s).
  \item Long refractory period due to the plateau phase, preventing tetanus.
\end{itemize}

\paragraph{Functional role} Synchronous activation of large regions of myocardium to produce coordinated contraction.

\subsubsection{Slow-Response Cells (SA and AV Nodal Cells)}

Slow-response cells are specialised for automaticity and rate regulation.

\paragraph{Key characteristics}
\begin{itemize}
  \item No true resting membrane potential; instead, spontaneous Phase 4 depolarisation.
  \item Phase 0 upstroke mediated predominantly by Ca$^{2+}$ influx through L-type channels.
  \item Very low conduction velocity, especially in the AV node ($\sim 0.02$--$0.05$ m/s).
  \item Higher intrinsic rhythmicity than fast-response tissue.
\end{itemize}

\paragraph{Functional role} Generation of rhythmic impulses, control of heart rate and provision of a physiological delay at the AV node.

%----------------------------------------------
\subsection{Cardiac Action Potentials}

\subsubsection{Fast-Response Action Potential (Myocytes, Purkinje)}

\begin{description}
  \item[Phase 0 -- Rapid depolarisation:] opening of fast Na$^+$ channels produces a steep upstroke; the rate of rise determines conduction velocity.
  \item[Phase 1 -- Early repolarisation:] transient outward K$^+$ current ($I_{\text{to}}$).
  \item[Phase 2 -- Plateau phase:] inward Ca$^{2+}$ current (L-type channels) balanced by outward K$^+$ currents; critical for Ca$^{2+}$-induced Ca$^{2+}$ release.
  \item[Phase 3 -- Repolarisation:] increased K$^+$ efflux through delayed rectifier channels.
  \item[Phase 4 -- Resting membrane potential:] stable, maintained by inward rectifier K$^+$ current.
\end{description}

The prolonged plateau prevents tetanus, ensures uniform contraction and couples electrical activity to mechanical systole.

\subsubsection{Slow-Response Action Potential (Nodal Tissue)}

\begin{description}
  \item[Phase 4 -- Pacemaker depolarisation:] driven by the funny current ($I_f$; mixed Na$^+$/K$^+$), T-type Ca$^{2+}$ channels and declining K$^+$ efflux.
  \item[Phase 0 -- Upstroke:] Ca$^{2+}$-mediated depolarisation via L-type Ca$^{2+}$ channels.
  \item[Phase 3 -- Repolarisation:] K$^+$ efflux through delayed rectifier channels.
\end{description}

This pattern underlies intrinsic rhythmicity and allows modulation of heart rate via changes in the slope of Phase 4.

%----------------------------------------------
\subsection{Refractory Periods}

\subsubsection{Effective (Absolute) Refractory Period}

\begin{itemize}
  \item During the effective refractory period (ERP), no new propagated action potential can be generated.
  \item Ensures unidirectional propagation and prevents sustained tetanic contraction.
\end{itemize}

\subsubsection{Relative Refractory Period}

\begin{itemize}
  \item During the relative refractory period (RRP), a stronger-than-normal stimulus can elicit an action potential.
  \item Conduction during this period may be slowed or abnormal, predisposing to re-entry.
\end{itemize}

\subsubsection{Functional Consequences}

\begin{itemize}
  \item Prevention of tetanus.
  \item Assurance of adequate filling time between beats.
  \item Determination of vulnerability to re-entrant arrhythmias.
\end{itemize}

%----------------------------------------------
\subsection{Automaticity and Pacemaker Activity}

\subsubsection{Hierarchy of Pacemakers}

\begin{itemize}
  \item SA node: primary pacemaker (intrinsic rate $\sim 60$--$100$ beats/min).
  \item AV node: intrinsic rate $40$--$60$ beats/min.
  \item His--Purkinje system: intrinsic rate $20$--$40$ beats/min.
\end{itemize}

\subsubsection{Determinants of Pacemaker Rate}

Pacemaker rate is determined chiefly by the slope of Phase 4 depolarisation:
\begin{itemize}
  \item Increased Phase 4 slope $\rightarrow$ increased heart rate.
  \item Decreased Phase 4 slope $\rightarrow$ bradycardia.
\end{itemize}

\subsubsection{Autonomic Effects}

\paragraph{Sympathetic stimulation}
\begin{itemize}
  \item Noradrenaline acting on $\beta_1$ receptors increases cAMP.
  \item Enhanced $I_f$ and Ca$^{2+}$ currents steepen Phase 4 and increase heart rate.
\end{itemize}

\paragraph{Parasympathetic stimulation}
\begin{itemize}
  \item Acetylcholine acting on M$_2$ receptors decreases cAMP and increases K$^+$ conductance.
  \item Membrane hyperpolarisation and reduced Phase 4 slope slow the heart rate.
\end{itemize}

%----------------------------------------------
\subsection{Conduction Pathways}

\subsubsection{Sequence of Activation}

\begin{enumerate}
  \item SA node initiates the impulse.
  \item Impulse spreads through the right atrium and via Bachmann's bundle to the left atrium.
  \item AV node delays conduction by about $0.1$\,s.
  \item His bundle conducts impulses to the interventricular septum.
  \item Right and left bundle branches distribute the signal to their respective ventricles.
  \item Purkinje fibres rapidly activate the ventricular myocardium.
\end{enumerate}

\subsubsection{Conduction Velocities}

\begin{itemize}
  \item SA node: slow.
  \item Atrial muscle: $0.5$--$1$\,m/s.
  \item AV node: $0.02$--$0.05$\,m/s.
  \item His--Purkinje system: $2$--$4$\,m/s (fastest).
  \item Ventricular muscle: $0.3$--$0.5$\,m/s.
\end{itemize}

\subsubsection{AV Nodal Delay}

The AV nodal delay:
\begin{itemize}
  \item Allows completion of atrial contraction before ventricular systole, augmenting ventricular filling.
  \item Acts as a filter, preventing excessively rapid atrial rhythms from being transmitted one-to-one to the ventricles.
\end{itemize}

%----------------------------------------------
\subsection{Excitation--Contraction Coupling in Cardiac Muscle}

Excitation--contraction (EC) coupling describes how an electrical impulse is converted into mechanical shortening.

\subsubsection{Initiation and Calcium-Induced Calcium Release}

\begin{itemize}
  \item The plateau phase of the action potential opens L-type Ca$^{2+}$ channels in the sarcolemma and T-tubules.
  \item Ca$^{2+}$ influx triggers Ca$^{2+}$ release from the sarcoplasmic reticulum (SR) via ryanodine receptors (Ca$^{2+}$-induced Ca$^{2+}$ release).
  \item The amount of Ca$^{2+}$ released is proportional to the trigger Ca$^{2+}$ current, so contraction strength is graded.
\end{itemize}

\subsubsection{Role of Intracellular Ca$^{2+}$ and Relaxation}

\begin{itemize}
  \item Cytosolic Ca$^{2+}$ binds to troponin C to initiate cross-bridge cycling and force generation.
  \item Relaxation (lusitropy) requires efficient removal of Ca$^{2+}$:
    \begin{itemize}
      \item SR uptake via SERCA2a (regulated by phospholamban).
      \item Extrusion via the Na$^+$--Ca$^{2+}$ exchanger.
      \item Minor contributions from sarcolemmal Ca$^{2+}$-ATPase and mitochondrial buffering.
    \end{itemize}
  \item Impaired lusitropy contributes to diastolic dysfunction and elevated filling pressures.
\end{itemize}

%----------------------------------------------
\subsection{Autonomic Regulation of Electrical Properties}

\subsubsection{Sympathetic Effects ($\beta_1$ Receptors)}

\begin{itemize}
  \item Increase heart rate (SA node).
  \item Increase conduction velocity (AV node).
  \item Enhance Ca$^{2+}$ influx and SR Ca$^{2+}$ uptake, increasing contractility and speeding relaxation.
\end{itemize}

\subsubsection{Parasympathetic Effects (M$_2$ Receptors)}

\begin{itemize}
  \item Decrease heart rate.
  \item Slow AV nodal conduction.
  \item Increase ACh-sensitive K$^+$ conductance, hyperpolarising nodal cells.
\end{itemize}

%----------------------------------------------
\subsection{Abnormal Electrical Activity}

Kam \& Powers describe three major mechanisms of arrhythmogenesis.

\subsubsection{Enhanced Automaticity}

\begin{itemize}
  \item Increased slope of pacemaker depolarisation in normal pacemaker tissue.
  \item Abnormal automaticity in non-pacemaker cells (e.g. ischaemic myocardium).
\end{itemize}

\subsubsection{Triggered Activity}

\begin{itemize}
  \item Early afterdepolarisations (EADs) arising during Phases 2 or 3, often when action potentials are prolonged (e.g. long QT states).
  \item Delayed afterdepolarisations (DADs) occurring after repolarisation, usually due to intracellular Ca$^{2+}$ overload (e.g. digoxin toxicity).
\end{itemize}

\subsubsection{Re-Entry Circuits}

\begin{itemize}
  \item Require an anatomical or functional circuit, a unidirectional block and slowed conduction.
  \item Represent the mechanism underlying many tachyarrhythmias, both supraventricular and ventricular.
\end{itemize}

%----------------------------------------------
\subsection{Electrocardiography (ECG)}

The ECG is a surface recording of the summed electrical activity of the heart. It reflects coordinated depolarisation and repolarisation of large numbers of myocardial cells and provides essential information about rhythm, conduction, chamber enlargement and metabolic or pharmacological disturbances.

\subsubsection{Basic Components of the ECG}

\begin{itemize}
  \item P wave: atrial depolarisation.
  \item PR interval: atrial conduction plus AV nodal delay.
  \item QRS complex: rapid ventricular depolarisation.
  \item ST segment: period during which the ventricles are fully depolarised (plateau phase).
  \item T wave: ventricular repolarisation.
  \item QT interval: total duration of ventricular depolarisation and repolarisation.
\end{itemize}

\subsubsection{Physiological Basis of ECG Deflections}

\begin{itemize}
  \item A wave of depolarisation moving towards a positive electrode produces an upward deflection; moving away produces a downward deflection.
  \item Repolarisation produces deflections of opposite polarity because it restores negativity.
  \item QRS amplitude and morphology depend on ventricular mass, activation sequence and conduction velocity.
\end{itemize}

\subsubsection{Cardiac Vectors and Mean Electrical Axis}

\paragraph{Instantaneous and mean vectors}
\begin{itemize}
  \item At any instant, the heart's electrical activity can be represented as a vector with magnitude and direction.
  \item Summation over time yields the mean electrical axis of depolarisation in the frontal plane (typically around $+30^\circ$ to $+90^\circ$ in adults).
\end{itemize}

\paragraph{Factors affecting axis}
\begin{itemize}
  \item Left axis deviation: LV hypertrophy, left anterior fascicular block, inferior myocardial infarction.
  \item Right axis deviation: RV hypertrophy, pulmonary hypertension, left posterior fascicular block.
  \item Extreme axis deviation: some ventricular rhythms and severe conduction disturbances.
\end{itemize}

\paragraph{Clinical usefulness}
Axis analysis helps identify chamber enlargement, conduction block and some tachyarrhythmias, and is influenced by anatomical orientation of the heart in the thorax.

\subsubsection{Effects of Electrolyte Disturbances}

Electrolyte abnormalities alter action potential shape and conduction and produce characteristic ECG changes.

\paragraph{Hyperkalaemia}
\begin{itemize}
  \item Tall, peaked T waves with narrow base.
  \item Flattened or absent P waves.
  \item Widened QRS complexes, which may progress to a sine-wave pattern.
  \item Mechanism: depolarisation of resting membrane potential, Na$^+$ channel inactivation and slowed conduction.
\end{itemize}

\paragraph{Hypokalaemia}
\begin{itemize}
  \item Flattened T waves.
  \item ST segment depression.
  \item Prominent U waves.
  \item Mechanism: hyperpolarisation and prolonged repolarisation, increasing arrhythmia risk.
\end{itemize}

\paragraph{Hypercalcaemia}
\begin{itemize}
  \item Shortened QT interval due to abbreviated plateau phase.
\end{itemize}

\paragraph{Hypocalcaemia}
\begin{itemize}
  \item Prolonged QT interval with risk of torsades de pointes.
\end{itemize}

\paragraph{Magnesium disturbances}
\begin{itemize}
  \item Low Mg$^{2+}$ predisposes to torsades de pointes and may mimic hypokalaemic/hypocalcaemic repolarisation changes.
  \item High Mg$^{2+}$ can prolong PR and QRS intervals and reduce myocardial excitability.
\end{itemize}

\subsubsection{Effects of Temperature and Drugs}

\paragraph{Hypothermia}
\begin{itemize}
  \item Osborn (J) waves at the junction of QRS and ST.
  \item Bradycardia with prolonged PR, QRS and QT intervals.
  \item Increased susceptibility to atrial and ventricular arrhythmias due to slowed ion channel kinetics and heterogeneous repolarisation.
\end{itemize}

\paragraph{Drug effects}
\begin{itemize}
  \item Digoxin: ``scooped'' ST depression, shortened QT and a range of arrhythmias mediated by delayed afterdepolarisations.
  \item Class I Na$^+$ channel blockers: QRS widening (especially Class Ic) and possible AV block.
  \item Class III K$^+$ channel blockers: QT prolongation with torsades risk.
  \item $\beta$-blockers: PR prolongation and reduced heart rate.
  \item Non-dihydropyridine Ca$^{2+}$ channel blockers (verapamil, diltiazem): PR prolongation and slowed nodal conduction.
  \item Volatile anaesthetics: dose-dependent QT prolongation, suppression of SA/AV nodal automaticity and increased arrhythmogenic risk in susceptible patients.
\end{itemize}

\subsubsection{Clinical Relevance}

\begin{itemize}
  \item ECG changes often reflect alterations in conduction velocity, action potential duration or tissue heterogeneity.
  \item Electrolyte- and drug-induced modification of ionic currents map directly onto characteristic ECG patterns.
  \item Understanding vector orientation and repolarisation dynamics is essential for recognising ischaemia, hypertrophy and conduction disorders.
\end{itemize}

The ECG provides a non-invasive window into cardiac electrical behaviour, integrating depolarisation and repolarisation patterns across the myocardium.

%----------------------------------------------
\subsection{Summary of Chapter 24}

Chapter 24 provides the electrophysiological foundation for cardiac function, describing ionic mechanisms of action potentials, regional specialisation of cardiac tissue, pacemaker activity, conduction pathways, excitation--contraction coupling and the principles underlying the ECG. These concepts explain how the heart generates rhythmic impulses, conducts them rapidly and directionally, and coordinates mechanical contraction under a wide range of physiological and pathological conditions.

\section{Chapter 25 --- Mechanical Events of the Cardiac Cycle}

\subsection{Overview and timing}
\begin{itemize}
\item \textbf{Two phases (ventricular mechanics):}
  \begin{itemize}
  \item \textbf{Systole:} ventricular contraction $\rightarrow$ AV valves close (\textbf{S1}) $\rightarrow$ ventricular pressure rises $\rightarrow$ semilunar valves open $\rightarrow$ ejection.
  \item \textbf{Diastole:} ventricular relaxation $\rightarrow$ semilunar valves close (\textbf{S2}, may be split with A2 before P2) $\rightarrow$ ventricular pressure falls $\rightarrow$ AV valves open $\rightarrow$ filling.
  \end{itemize}
\item \textbf{At rest (HR $\approx 72$/min):} cycle $\approx 0.8\,\mathrm{s}$ $\rightarrow$ \textbf{systole $\approx 0.3\,\mathrm{s}$}, \textbf{diastole $\approx 0.5\,\mathrm{s}$}.
\item \textbf{At very high HR ($\approx 200$/min):} cycle $\approx 0.3\,\mathrm{s}$ $\rightarrow$ \textbf{systole $\approx 0.15\,\mathrm{s}$}, \textbf{diastole $\approx 0.15\,\mathrm{s}$} (filling time becomes limiting).
\end{itemize}

\subsection{Reference volumes and pressures used in this chapter}
\begin{itemize}
\item \textbf{End-diastolic volume (EDV):} $\approx \textbf{130 mL}$ (standing); $\approx \textbf{160 mL}$ (lying).
\item \textbf{Stroke volume (SV):} $\approx \textbf{70 mL}$.
\item \textbf{End-systolic volume (ESV):} $\approx \textbf{60 mL}$.
\item \textbf{Typical systemic (left-sided) pressures:} aorta \textbf{80 $\rightarrow$ 120 mmHg} during ejection; LA rises to $\approx \textbf{10 mmHg}$ during early systole.
\item \textbf{Typical pulmonary (right-sided) pressures:} PA \textbf{8 $\rightarrow$ 25 mmHg} during ejection (also quoted as \textbf{24/8 mmHg}); RA rises to $\approx \textbf{5 mmHg}$ during early systole.
\end{itemize}

\subsection{Cardiac cycle phases (mechanical sequence)}

\subsubsection{A. Mid-diastole (starting point)}
\begin{itemize}
\item AV valves \textbf{open}, semilunar valves \textbf{closed}.
\item Atrial pressure slightly exceeds ventricular pressure $\rightarrow$ \textbf{slow filling}.
\end{itemize}

\subsubsection{B. Late diastole --- atrial systole}
\begin{itemize}
\item \textbf{ECG:} SA node discharge (\textbf{P wave}) $\rightarrow$ atrial contraction.
\item Atrial contraction contributes about \textbf{1/5 of EDV}.
\item Importance increases at high HR as passive filling time shortens.
\item \textbf{Adequate ventricular filling depends on:}
  \begin{enumerate}
  \item filling pressure from venous return,
  \item full opening of AV valves,
  \item \textbf{high ventricular compliance} (expands with minimal resistance).
  \end{enumerate}
\end{itemize}

\subsubsection{C. Early systole --- isovolumetric ventricular contraction}
\begin{itemize}
\item Ventricles contract $\rightarrow$ ventricular pressure rises rapidly.
\item \textbf{AV valves close $\rightarrow$ S1.}
\item Semilunar valves remain \textbf{closed} until ventricular pressure exceeds aortic/PA pressure.
\item AV valves bulge into atria $\rightarrow$ atrial \textbf{c wave}.
\end{itemize}

\subsubsection{D. Systole --- ventricular ejection}
\begin{itemize}
\item Semilunar valves \textbf{open} once ventricular pressure exceeds outflow pressure.
\item Two components:
  \begin{itemize}
  \item \textbf{Rapid ejection} (early, brief).
  \item \textbf{Reduced (slow) ejection} (late systole).
  \end{itemize}
\item During reduced ejection, ventricular pressure may fall slightly below aortic pressure, but forward flow continues due to \textbf{momentum} acquired during rapid ejection.
\item Aortic pressure is maintained by \textbf{arterial elastic recoil} and \textbf{peripheral resistance}.
\item \textbf{Ejection fraction (EF):} $\mathrm{EF}=\mathrm{SV}/\mathrm{EDV}$ ($\approx \textbf{60\%}$ in this description).
\item Atrial pressure falls sharply (to zero/negative) during rapid ejection as the AV fibrous ring is pulled downward (atrial volume increases), then rises during systole as venous return continues.
\end{itemize}

\subsubsection{E. Beginning of diastole --- isovolumetric ventricular relaxation}
\begin{itemize}
\item Ventricles relax with \textbf{both} valve sets closed.
\item Semilunar valve closure marks end-systole $\rightarrow$ \textbf{S2} and produces the \textbf{incisura (dicrotic notch)} on the aortic pressure trace.
\item During this phase atrial pressure rises (LA $\approx \textbf{5 mmHg}$, RA $\approx \textbf{2 mmHg}$ in the described values).
\end{itemize}

\subsubsection{F. Early diastole --- rapid ventricular filling}
\begin{itemize}
\item AV valves open when ventricular pressure falls below atrial pressure.
\item Ventricles fill rapidly; atrial and ventricular pressures fall.
\item At very high HR (e.g.\ \textbf{$>200$/min}), filling time can become inadequate.
\end{itemize}

\subsubsection{G. Mid-diastole --- slow filling (diastasis)}
\begin{itemize}
\item Atrial and ventricular pressures are low; atrial pressure remains slightly higher $\rightarrow$ slow filling continues.
\item Ventricles are already $\approx \textbf{80\%}$ full due to early rapid filling.
\item During diastole, aortic/PA pressures fall as blood runs off into the vasculature; ventricular pressures rise slightly as filling proceeds.
\end{itemize}

\subsection{CVP/JVP waveform (a, c, v + descents)}
\begin{itemize}
\item Right and left heart events are synchronous; valves open/close in unison.
\item Right and left ventricles eject \textbf{identical SV} (pumps in series), but the right heart works at much lower pressures due to low pulmonary vascular resistance.
\item \textbf{a wave:} atrial systole.
\item \textbf{c wave:} AV valve bulges into atrium during early ventricular systole.
\item \textbf{x descent:} downward displacement of the AV septum/ring during ventricular systole (atrial volume increases).
\item \textbf{v wave:} venous filling of atrium while AV valve is closed.
\item \textbf{y descent:} atrium empties into ventricle when AV valve opens.
\end{itemize}

\subsection{Determinants of cardiac muscle contraction (mechanical framework)}

\subsubsection{Mechanical model}
\begin{itemize}
\item Cardiac muscle is described using:
  \begin{itemize}
  \item \textbf{Contractile element (CE)}
  \item \textbf{Series elastic element (SE)}
  \item \textbf{Parallel elastic element (PE)}
  \end{itemize}
\end{itemize}

\subsubsection{Isometric contraction and length--tension}
\begin{itemize}
\item Increasing resting muscle length improves actin--myosin overlap $\rightarrow$ \textbf{greater isometric twitch tension}.
\item Excessive stretch reduces overlap $\rightarrow$ twitch tension falls despite high resting tension.
\item Lengthening also increases Ca$^{2+}$ sensitivity of troponin and intracellular free Ca$^{2+}$ $\rightarrow$ increased force.
\item The normal heart operates on the \textbf{ascending limb}: increased length $\rightarrow$ increased force.
\end{itemize}

\subsubsection{Isotonic contractions and afterloaded isotonic contraction}
\begin{itemize}
\item \textbf{Isotonic:} muscle shortens; velocity is maximal at zero load (\textbf{$V_{\max}$}).
\item Increasing load decreases velocity and extent of shortening.
\item \textbf{Afterloaded isotonic contraction:} preload (rest length/EDV) differs from load encountered during ejection.
  \begin{itemize}
  \item \textbf{Preload} $\approx$ end-diastolic fibre length (end-diastolic volume).
  \item \textbf{Afterload (LV)} relates to the pressure against which the ventricle ejects once the aortic valve is open.
  \end{itemize}
\end{itemize}

\subsubsection{Catecholamines}
\begin{itemize}
\item Norepinephrine increases strength and speed of isometric twitch $\rightarrow$ \textbf{positive inotropy}.
\item Catecholamines accelerate both contraction and relaxation (via Ca$^{2+}$ handling/contractile proteins) $\rightarrow$ sympathetic stimulation speeds \textbf{emptying and filling}.
\item In isotonic terms: catecholamines increase \textbf{$V_{\max}$}, increase velocity at other loads, and increase \textbf{$P_0$}.
\end{itemize}

\subsubsection{Whole-heart performance (Starling heart--lung preparation concept)}
\begin{itemize}
\item Raising the venous reservoir $\rightarrow$ increased right atrial pressure $\rightarrow$ increased EDV $\rightarrow$ increased SV.
\item \textbf{Starling’s law:} force of contraction is proportional to initial resting length.
\item The normal heart sits on the ascending part of the Starling curve: increased EDV $\rightarrow$ increased SV.
\end{itemize}

% --------------------------------------------------------------------

\section{Chapter 26 --- Pressure--Volume Loop of the Left Ventricle}

\subsection{Why the PV loop is useful}
PV analysis links \textbf{pressure generation} and \textbf{volume change} over a beat, allowing separation of systolic (active) from diastolic (passive) properties.

\subsubsection{Systolic function (PV framework)}
\begin{itemize}
\item Systolic performance is captured by the \textbf{end-systolic pressure--volume relationship (ESPVR)}.
\item The \textbf{slope of ESPVR} (end-systolic elastance, \textbf{$E_{es}$}) is used as an index of \textbf{contractility} (described as relatively independent of preload and afterload over a wide range).
\item Changes in contractility primarily alter \textbf{end-systolic pressure at a given volume} and \textbf{ESV}, changing stroke volume.
\end{itemize}

\subsubsection{Diastolic function (PV framework)}
\begin{itemize}
\item Diastolic properties are captured by the \textbf{end-diastolic pressure--volume relationship (EDPVR)}.
\item \textbf{Compliance} describes how much volume increases for a given rise in pressure (conceptually $\Delta V/\Delta P$); \textbf{stiffness} is the inverse.
\item A steeper EDPVR implies \textbf{reduced compliance / increased stiffness}, so a given filling volume requires a higher diastolic pressure.
\end{itemize}

\subsubsection{PV loop area as an index of mechanical work}
\begin{itemize}
\item The \textbf{area enclosed by the LV PV loop} represents \textbf{external mechanical work (stroke work)} per beat.
\item Higher pressures and/or larger stroke volumes expand loop area, giving a visual index of how preload, afterload, and contractility alter mechanical workload.
\end{itemize}

\subsubsection{Box: ``What area is what?'' on the LV PV diagram}
\begin{itemize}
\item \textbf{External work (stroke work)}
  \begin{itemize}
  \item \textbf{Definition:} useful mechanical work of ejection.
  \item \textbf{On the PV plot:} \textbf{area inside the PV loop} (integral of pressure with respect to volume over the loop).
  \end{itemize}
\item \textbf{Potential energy / internal work component}
  \begin{itemize}
  \item \textbf{Definition:} energy expended to develop pressure \textbf{without ejection} (especially during \textbf{isovolumetric contraction}) and dissipated as heat during diastole.
  \item \textbf{On the PV plot:} the \textbf{triangular region} bounded by \textbf{ESPVR}, \textbf{EDPVR}, and the line representing \textbf{isovolumic relaxation}.
  \end{itemize}
\item \textbf{Pressure--volume area (PVA)}
  \begin{itemize}
  \item \textbf{Definition:} \textbf{stroke work + internal work (potential energy component)} for a single beat.
  \item \textbf{On the PV plot:} \textbf{area under the ESPVR down to the EDPVR}, partitioned into the \textbf{loop} (stroke work) plus the \textbf{triangle} (internal work).
  \end{itemize}
\end{itemize}

\subsection{The four loop segments (one beat)}
\begin{enumerate}
\item \textbf{Isovolumetric contraction:} pressure rises at constant volume.
\item \textbf{Ejection:} volume falls; pressure rises then falls.
\item \textbf{Isovolumetric relaxation:} pressure falls at constant volume.
\item \textbf{Ventricular filling:} volume rises at low pressure.
\end{enumerate}
\begin{itemize}
\item Normal LV loop is approximately \textbf{rectangular} (RV more triangular).
\item LV shape changes:
  \begin{itemize}
  \item during isovolumetric contraction $\rightarrow$ more spherical (endocardial surface area decreases),
  \item during isovolumetric relaxation $\rightarrow$ more ellipsoidal (surface area increases).
  \end{itemize}
\item RV contraction is not synchronous: starts in inflow tract and reaches outflow tract with $\approx \textbf{50 ms}$ delay (peristaltic).
\end{itemize}

\subsection{End-systolic and end-diastolic pressure--volume relationships}

\subsubsection{ESPVR}
\begin{itemize}
\item Relationship between LV pressure and volume at \textbf{end-systole}.
\item Described as \textbf{linear over a wide range} and \textbf{independent of preload and afterload}.
\item \textbf{Slope = $E_{es}$} $\rightarrow$ index of \textbf{contractility}.
\item \textbf{Increased contractility (positive inotropy):} ESPVR shifts \textbf{left} and becomes \textbf{steeper} (anticlockwise rotation).
  \begin{itemize}
  \item $\mathrm{d}p/\mathrm{d}t$ increases $\rightarrow$ increased ejection velocity and SV $\rightarrow$ decreased ESV.
  \end{itemize}
\item \textbf{Decreased contractility (negative inotropy):} ESPVR slope decreases (clockwise rotation) $\rightarrow$ decreased ejection and SV $\rightarrow$ increased ESV.
\end{itemize}

\subsubsection{EDPVR}
\begin{itemize}
\item Describes passive filling properties at end-diastole.
\item \textbf{Compliance:} $\Delta V/\Delta P$ (conceptual).
\item \textbf{Stiffness:} inverse of compliance.
\item Increased steepness of the diastolic PV relationship indicates \textbf{reduced compliance / increased stiffness}.
\item The LV does \textbf{not} exhibit constant compliance.
\end{itemize}

\subsection{How preload, afterload, and contractility reshape the loop}

\subsubsection{Preload changes}
\begin{itemize}
\item Increased preload $\rightarrow$ increased EDV (loop shifts right along EDPVR) $\rightarrow$ typically larger SV.
\item Decreased preload $\rightarrow$ decreased EDV $\rightarrow$ smaller SV.
\end{itemize}

\subsubsection{Afterload changes (arterial elastance line)}
\begin{itemize}
\item Afterload is represented by the \textbf{arterial elastance line ($E_a$)}:
  \begin{itemize}
  \item a line from the \textbf{EDV point} on the x-axis to the \textbf{end-systolic pressure} point,
  \item a \textbf{steeper $E_a$} (clockwise rotation) indicates \textbf{increased afterload}.
  \end{itemize}
\item Increased afterload tends to increase generated pressure and can reduce SV; developing high pressures increases oxygen requirement.
\end{itemize}

\subsubsection{Contractility changes}
\begin{itemize}
\item Contractility changes shift/rotate the ESPVR (via changes in $E_{es}$).
\end{itemize}

\subsection{Pressure--volume area and oxygen demand}
\begin{itemize}
\item When afterload rises and the ventricle develops higher pressures, \textbf{mechanical workload} and \textbf{oxygen requirement} rise substantially.
\end{itemize}

% --------------------------------------------------------------------

\section{Chapter 27 --- Physical Factors Governing Blood Flow}

\subsection{Core relationships}
\begin{itemize}
\item Flow is driven by a pressure gradient and opposed by resistance:
  \begin{itemize}
  \item \textbf{Hydraulic resistance ($R$) $=\Delta P/Q$}
  \item equivalently \textbf{$Q=\Delta P/R$}
  \end{itemize}
\end{itemize}

\subsubsection{Clinical unit note (SVR/TPR)}
\begin{itemize}
\item Systemic vascular resistance (SVR/TPR) is often expressed in \textbf{dyne$\cdot$s/cm$^5$}.
\item \textbf{Conversion:} Wood units $=$ (dyne$\cdot$s/cm$^5$) / \textbf{80}.
\end{itemize}

\subsection{Laminar flow and Poiseuille’s law (rigid tube, Newtonian fluid)}
\begin{itemize}
\item For laminar flow in a rigid tube:
  \begin{itemize}
  \item \textbf{$Q=(\Delta P \cdot \pi \cdot r^4)/(8\cdot \eta \cdot l)$}
  \item therefore \textbf{$R=(8\cdot \eta \cdot l)/(\pi \cdot r^4)$}
  \end{itemize}
\item Key implications:
  \begin{itemize}
  \item \textbf{Radius dominates} ($r^4$): small changes in $r$ cause large changes in flow/resistance.
  \item Resistance increases with \textbf{viscosity ($\eta$)} and \textbf{length ($l$)}.
  \end{itemize}
\end{itemize}

\subsection{Viscosity ($\eta$)}
\begin{itemize}
\item \textbf{Definition:} internal friction within a fluid.
\item \textbf{Units:} poise (P) or centipoise (cP).
  \begin{itemize}
  \item $1\,\mathrm{P}=0.1\,\mathrm{Pa\cdot s}$; $1\,\mathrm{cP}=0.001\,\mathrm{Pa\cdot s}$.
  \end{itemize}
\item \textbf{Relative viscosity} (vs water at the same temperature):
  \begin{itemize}
  \item Water at $20^{\circ}\mathrm{C}$: viscosity $\approx \textbf{0.01 P}$.
  \item Whole blood at $37^{\circ}\mathrm{C}$: $\approx \textbf{0.04--0.05 P}$ (relative viscosity $\approx \textbf{4--5}$).
  \end{itemize}
\item Blood is \textbf{non-Newtonian} (viscosity varies with shear rate).
\end{itemize}

\subsection{Resistance in series and parallel}
\begin{itemize}
\item \textbf{Series:} total resistance is the sum of individual resistances:
  \[
  R_{\text{total}} = R_1 + R_2 + \cdots + R_n
  \]
\item \textbf{Parallel:} reciprocal of total resistance equals the sum of reciprocals:
  \[
  \frac{1}{R_{\text{total}}} = \frac{1}{R_1} + \frac{1}{R_2} + \cdots + \frac{1}{R_n}
  \]
  therefore $R_{\text{total}}$ is \textbf{less than} any individual branch resistance.
\end{itemize}

\subsection{Turbulent flow and Reynolds number}
\begin{itemize}
\item Turbulent flow increases energy loss and increases resistance beyond Poiseuille predictions.
\item \textbf{Reynolds number:}
  \[
  \mathrm{Re} = \frac{\rho \cdot v \cdot d}{\eta}
  \]
  where $\rho$ = density, $v$ = mean velocity, $d$ = diameter, $\eta$ = viscosity.
\item Typical thresholds:
  \begin{itemize}
  \item laminar when $\mathrm{Re} < \approx 2000$,
  \item turbulence becomes more likely when $\mathrm{Re} > \approx 2000$--3000.
  \end{itemize}
\item Example stated: in the aortic root, $\mathrm{Re}$ can exceed $\approx \textbf{3000}$.
\end{itemize}

\subsection{Total fluid energy and the pressure--velocity trade-off}
\begin{itemize}
\item Total energy per unit volume includes:
  \begin{itemize}
  \item \textbf{lateral (hydrostatic) pressure energy},
  \item \textbf{kinetic energy} (depends on $v^2$),
  \item \textbf{potential energy} (height in a gravitational field).
  \end{itemize}
\item In a narrowing tube, velocity increases; some energy shifts between pressure and kinetic forms.
\end{itemize}

\subsection{Distensible tubes (real vessels)}
\begin{itemize}
\item Blood vessels are distensible, so the pressure--flow relationship can deviate from rigid-tube predictions.
\item Vessel calibre may change with pressure (and physiological control), altering resistance dynamically.
\end{itemize}

% --------------------------------------------------------------------

\section{Chapter 28 --- The Systemic Circulation}

\subsection{Functional overview (what each vessel type is for)}
\begin{itemize}
\item The systemic circulation is a \textbf{high-pressure, low-volume} system enabling rapid distribution and redistribution of cardiac output.
\item Functional roles:
  \begin{itemize}
  \item \textbf{Aorta/large arteries:} elastic reservoir (Windkessel / hydraulic filter) converting intermittent LV output to more continuous peripheral flow.
  \item \textbf{Arterioles:} principal \textbf{resistance} vessels; regulate organ blood flow and mean arterial pressure; protect capillaries from high upstream pressures.
  \item \textbf{Capillaries:} exchange surface (single endothelial layer).
  \item \textbf{Veins:} low-resistance conduits and \textbf{capacitance} vessels (large volume reservoir; capacitance is sympathetically modulated).
  \end{itemize}
\end{itemize}

\subsection{Vessel wall structure (structure--function)}
\begin{itemize}
\item Vessel walls (except capillaries) have three layers:
  \begin{itemize}
  \item \textbf{Intima:} endothelium.
  \item \textbf{Media:} smooth muscle in an elastin/collagen matrix.
  \item \textbf{Adventitia:} connective tissue sheath.
  \end{itemize}
\item Diameter, wall thickness, and proportions of elastic tissue/smooth muscle/connective tissue vary across the arterial and venous tree.
\item (Details are summarised in Table 28.1 in the source; not reproduced here.)
\end{itemize}

\subsection{Aorta: Windkessel (hydraulic filter) effect}
\begin{itemize}
\item The aorta and large arteries act as an elastic reservoir (``pressure storage'') smoothing pulsatile ventricular ejection.
\item During systole:
  \begin{itemize}
  \item About \textbf{one-third} of ejected blood moves through arteries to tissues.
  \item Peripheral resistance/impedance causes much of the remainder to \textbf{distend} the aorta/large arteries.
  \item \textbf{Kinetic energy} of ejected blood is stored as \textbf{potential energy} in stretched elastic tissues.
  \end{itemize}
\item During diastole:
  \begin{itemize}
  \item The \textbf{aortic valve closes}, preventing retrograde flow.
  \item Elastic recoil converts stored \textbf{potential energy back to kinetic energy}, maintaining forward flow.
  \end{itemize}
\item Consequences include more constant peripheral flow and reduced cardiac workload; reduced arterial elasticity with ageing diminishes this effect.
\end{itemize}

\subsection{Arteries and arterial blood pressure}

\subsubsection{Key arterial properties}
\begin{itemize}
\item Arteries are \textbf{low-resistance conduits} whose elastic walls contribute to the Windkessel effect.
\item The arterial system contains $\approx \textbf{15\%}$ of total blood volume ($\approx \textbf{750 mL}$) at mean arterial pressure $\approx \textbf{100 mmHg}$.
\end{itemize}

\subsubsection{Arterial pulse and waveform}
\begin{itemize}
\item The palpable arterial pulse is a \textbf{pressure wave} travelling along arteries much faster (m/s) than bulk blood flow (cm/s).
  \begin{itemize}
  \item With arterial stiffening (age), pressure wave velocity increases.
  \end{itemize}
\item Typical aortic pressure waveform:
  \begin{itemize}
  \item systolic peak $\approx \textbf{120 mmHg}$,
  \item \textbf{incisura} at aortic valve closure,
  \item diastolic pressure $\approx \textbf{80 mmHg}$.
  \end{itemize}
\item Peripherally:
  \begin{itemize}
  \item diastolic and mean pressures fall gradually,
  \item waveform shape changes (young adults: higher/narrower systolic peak; damping of high-frequency components; a dicrotic wave appears on the diastolic portion),
  \item in elderly individuals with stiffer arteries, peripheral and aortic waveforms become more similar.
  \end{itemize}
\end{itemize}

\subsubsection{Mean arterial pressure (MAP)}
\begin{itemize}
\item MAP reflects the \textbf{average pressure driving flow} through the systemic circulation.
\item Clinical approximation:
  \[
  \mathrm{MAP} \approx P_{\text{diastolic}} + \frac{1}{3}\left(P_{\text{systolic}}-P_{\text{diastolic}}\right)
  \]
\end{itemize}

\subsection{Determinants of mean arterial pressure}
\begin{itemize}
\item Mean arterial pressure is determined by the \textbf{amount of blood in the arterial system} at a given time.
\item Arterial volume depends on:
  \begin{itemize}
  \item \textbf{inflow} (cardiac output) and
  \item \textbf{outflow} (peripheral runoff into capillaries).
  \end{itemize}
\item MAP is stable when \textbf{cardiac output equals peripheral runoff}.
\item If cardiac output rises, MAP rises until runoff increases to match inflow.
\item If total peripheral resistance rises, runoff initially falls, arterial volume rises, and MAP rises until runoff again matches cardiac output.
\end{itemize}

\subsubsection{Key relationship (as used in the chapter examples)}
\begin{itemize}
\item \textbf{$\mathrm{MAP} = \mathrm{CO} \times \mathrm{TPR}$} (when runoff equals cardiac output).
\item Example values given:
  \begin{itemize}
  \item CO \textbf{6000 mL/min} and TPR \textbf{0.02 mmHg/(mL/min)} $\rightarrow$ MAP \textbf{120 mmHg}.
  \item CO \textbf{5000 mL/min} and TPR \textbf{0.03 mmHg/(mL/min)} $\rightarrow$ MAP \textbf{150 mmHg}.
  \end{itemize}
\end{itemize}

\subsubsection{Note on arterial compliance}
\begin{itemize}
\item Arterial compliance is not described as a primary determinant of MAP, but it strongly influences pulsatile behaviour (e.g., pulse pressure).
\end{itemize}

\subsection{Determinants of pulse pressure}
\begin{itemize}
\item \textbf{Stroke volume} and \textbf{arterial compliance} are the main determinants of pulse pressure.
\item Pulse pressure increases when stroke volume increases and/or arterial compliance decreases.
\end{itemize}

\subsection{Arterioles: controllers of blood flow}
\begin{itemize}
\item Arterioles contain abundant smooth muscle and provide the \textbf{main site of resistance}.
\item After blood passes through arterioles, pulsatile arterial flow at mean pressure $\approx \textbf{100 mmHg}$ becomes relatively steady capillary flow at $\approx \textbf{35 mmHg}$.
\end{itemize}

\subsubsection{Key formula: arteriolar resistance (dominant control variable = radius)}
\begin{itemize}
\item Using the resistance form of Poiseuille’s relationship:
  \[
  R = \frac{8\cdot \eta \cdot L}{\pi \cdot r^4}
  \]
\item Therefore, for an arteriole:
  \[
  R_{\text{arteriole}} = \frac{8\cdot \eta \cdot L_{\text{arteriole}}}{\pi \cdot r_{\text{arteriole}}^4}
  \]
\item In this framework, $L$ and $\eta$ are relatively fixed, whereas $r$ varies with smooth muscle tone:
\[
R \propto \frac{1}{r^4}
\]
\end{itemize}

\subsubsection{Main arteriolar functions}
\begin{enumerate}
\item Alter total peripheral resistance (and therefore MAP).
\item Alter resistance within individual organs (distribution of cardiac output).
\item Alter capillary hydrostatic pressure (fluid shifts between compartments).
\end{enumerate}

\subsubsection{Arterioles and total peripheral resistance}
\begin{itemize}
\item General arteriolar constriction $\rightarrow$ increased TPR $\rightarrow$ increased MAP.
\item General arteriolar dilation $\rightarrow$ decreased TPR $\rightarrow$ decreased MAP.
\end{itemize}

\subsubsection{Arterioles and organ blood flow}
\begin{itemize}
\item Organ blood flow depends on perfusion pressure and organ vascular resistance:
  \begin{itemize}
  \item \textbf{Flow = Pressure / Resistance}
  \item \textbf{Organ blood flow = MAP / organ vascular resistance}
  \end{itemize}
\end{itemize}

\subsubsection{Arterioles and capillary hydrostatic pressure}
\begin{itemize}
\item Arteriolar dilation reduces the upstream pressure drop $\rightarrow$ arterial pressure decreases, capillary pressure increases.
\item Arteriolar constriction increases the upstream pressure drop $\rightarrow$ arterial pressure increases, capillary pressure decreases.
\end{itemize}

\subsection{Control of arteriolar smooth muscle tone}
\begin{itemize}
\item Arteriolar tone is regulated by \textbf{local} and \textbf{systemic} factors; importance varies by organ.
\end{itemize}

\subsubsection{Local factors}

\paragraph{A. Myogenic control}
\begin{itemize}
\item Increased intraluminal pressure triggers constriction; decreased pressure triggers dilation.
\item Dominant in \textbf{brain and kidneys} (autoregulation over a wide MAP range); less important in skeletal muscle and skin.
\end{itemize}

\paragraph{B. Metabolic control}
\begin{itemize}
\item Increased activity $\rightarrow$ metabolite accumulation and reduced tissue O$_2$ tension $\rightarrow$ arteriolar dilation $\rightarrow$ increased flow.
\item Proposed mediators include: decreased O$_2$, increased CO$_2$, increased temperature, H$^+$, K$^+$, lactate, pyruvate, inorganic phosphate, interstitial osmolarity, adenosine, ATP/ADP/AMP.
\item Explains \textbf{reactive hyperaemia} and \textbf{active hyperaemia}; especially important in \textbf{heart, skeletal muscle, and brain}.
\end{itemize}

\paragraph{C. Local vasoactive chemicals and endothelial factors}
\begin{itemize}
\item Some glands produce \textbf{kallikrein} $\rightarrow$ kinins (e.g.\ \textbf{bradykinin}) $\rightarrow$ arteriolar relaxation (increased flow with gland activity).
\item Endothelium-derived substances:
  \begin{itemize}
  \item \textbf{prostacyclin} and \textbf{NO}: vasodilators,
  \item \textbf{endothelin}: vasoconstrictor.
  \end{itemize}
\item Vessel injury $\rightarrow$ platelet aggregation releases \textbf{thromboxane A$_2$} (vasoconstrictor).
\end{itemize}

\subsubsection{Systemic factors}

\paragraph{A. Sympathetic nervous control}
\begin{itemize}
\item Norepinephrine acts mainly at \textbf{$\alpha$ receptors} $\rightarrow$ vasoconstriction ($\beta_2$-mediated vasodilation is described as weaker).
\item Resting sympathetic outflow maintains partial vasoconstriction.
\item Particularly important in \textbf{skin, kidneys, and gut}; less important in \textbf{brain and heart}.
\end{itemize}

\paragraph{B. Sympathetic cholinergic supply to skeletal muscle}
\begin{itemize}
\item A second sympathetic supply releases acetylcholine $\rightarrow$ arteriolar dilation.
\item May increase skeletal muscle flow at the onset of exercise or during anger/fear.
\end{itemize}

\paragraph{C. Parasympathetic control}
\begin{itemize}
\item Generally much less important.
\item External genitalia: parasympathetic dilator nerves plus sympathetic constrictor supply.
\end{itemize}

\paragraph{D. Circulating hormones}
\begin{itemize}
\item Epinephrine acts at both \textbf{$\alpha$ (constrictor)} and \textbf{$\beta_2$ (dilator)} receptors; net effect depends on receptor distribution.
  \begin{itemize}
  \item Heart and skeletal muscle: relatively more $\beta_2$ $\rightarrow$ dilation.
  \item Gut and skin: relatively more $\alpha$ $\rightarrow$ constriction.
  \end{itemize}
\item Other hormones:
  \begin{itemize}
  \item \textbf{angiotensin II} and \textbf{vasopressin}: constrictors,
  \item \textbf{atrial natriuretic peptide (ANP)}: dilator.
  \end{itemize}
\end{itemize}

% --------------------------------------------------------------------

\section{Chapter 29 --- Microcirculation}

\subsection{What the microcirculation does}
Microcirculation is where the cardiovascular system delivers its two end-point functions:
\begin{enumerate}
\item \textbf{nutrient/metabolite exchange} between blood and tissues (mainly by \textbf{diffusion}), and
\item \textbf{fluid distribution} between intravascular and extravascular compartments (by \textbf{bulk flow / filtration} across a semipermeable barrier).
\end{enumerate}

\subsection{Capillaries: structure, scale, and recruitment}
\begin{itemize}
\item There are $\approx \textbf{25{,}000 million}$ capillaries in the body.
\item Blood volume in capillaries:
  \begin{itemize}
  \item systemic capillaries $\approx \textbf{6\%}$ of total blood volume,
  \item pulmonary capillaries $\approx \textbf{3\%}$.
  \end{itemize}
\item \textbf{Structure:} thin-walled tubes of endothelial cells on a basement membrane.
  \begin{itemize}
  \item Channels connect lumen to interstitium.
  \item \textbf{Intercellular clefts}; \textbf{fused-vesicle channels} (from endocytotic/exocytotic vesicles).
  \item Exception: \textbf{brain} (tight junctions; blood--brain barrier).
  \end{itemize}
\item Dimensions:
  \begin{itemize}
  \item diameter $\approx \textbf{5--10 }\mu\mathrm{m}$, length $\approx \textbf{1 mm}$,
  \item RBCs ($\approx \textbf{7 }\mu\mathrm{m}$) deform in narrower capillaries.
  \end{itemize}
\item Large surface area and short diffusion distance (max $\approx \textbf{50 }\mu\mathrm{m}$).
\item \textbf{Recruitment:} only $\approx \textbf{1/4}$ of capillaries are open at rest; increased activity opens more capillaries $\rightarrow$ reduced diffusion distance.
\item \textbf{Velocity / transit time:} cross-sectional area is greatest at capillaries $\rightarrow$ flow slows to $\approx \textbf{0.5 mm/s}$.
  \begin{itemize}
  \item At rest: RBC transit time $\approx \textbf{2 s}$; during activity can fall to $\approx \textbf{1 s}$ (still sufficient for diffusion).
  \end{itemize}
\end{itemize}

\subsection{Microvascular ``plumbing''}
\begin{itemize}
\item Capillary flow is governed mainly by tone of \textbf{feeding arterioles}:
  \begin{itemize}
  \item constriction $\rightarrow$ decreased capillary flow,
  \item dilation $\rightarrow$ increased capillary flow.
  \end{itemize}
\item In some tissues, capillaries arise from \textbf{metarterioles} connecting arterioles and venules.
  \begin{itemize}
  \item Metarterioles can supply capillaries or act as \textbf{bypass channels} to venules.
  \end{itemize}
\item \textbf{Precapillary sphincter:} smooth muscle band at capillary origin; controlled by local metabolites.
\item \textbf{Vasomotion:} metarterioles and precapillary sphincters open/close cyclically about \textbf{every minute}.
  \begin{itemize}
  \item Open-phase duration is directly proportional to tissue oxygen demand.
  \end{itemize}
\item \textbf{Arteriovenous shunts} (arteriole $\rightarrow$ venule) exist in some tissues; under autonomic control; often serve functions not directly related to tissue nutrition (e.g.\ skin thermoregulation).
\end{itemize}

\subsection{Capillary wall tension (Laplace)}
\begin{itemize}
\item Despite being only $\approx \textbf{0.5 }\mu\mathrm{m}$ thick, capillary walls tolerate hydrostatic pressures $\approx \textbf{35 mmHg}$.
\item \textbf{Laplace (as presented):} \textbf{Wall tension = Pressure $\times$ Radius}.
\item Small radius keeps required wall tension low.
\end{itemize}

\subsection{Types of capillaries}
\begin{enumerate}
\item \textbf{Continuous:} flattened endothelial cells + luminal glycocalyx + external basement membrane.
\item \textbf{Fenestrated:} endothelial ``windows'' (fenestrations).
\item \textbf{Discontinuous:} large gaps between endothelial cells ($>\textbf{100 nm}$).
\end{enumerate}

\subsection{Transport across capillaries}

\subsubsection{Diffusion (nutrient/metabolite exchange)}
\begin{itemize}
\item Diffusion is the main mechanism for movement of gases, nutrients, metabolites, and water between blood, interstitium, and cells.
\item Lipid-soluble substances (O$_2$, CO$_2$) diffuse readily through endothelial lipid membranes.
  \begin{itemize}
  \item Some exchange can occur in arterioles (high lipid solubility).
  \item Tissue metabolic state affects gradients (increased activity $\rightarrow$ decreased intracellular $P\!O_2$ and increased $P\!CO_2$).
  \end{itemize}
\item Water-soluble substances use water-filled channels in endothelial clefts (size-limited):
  \begin{itemize}
  \item small solutes (water, Na$^+$/Cl$^-$, glucose, urea) diffuse easily,
  \item large molecules $\geq \approx \textbf{60{,}000 Da}$ (e.g.\ albumin) do not cross intercellular clefts (some proteins may traverse larger fused-vesicle channels).
  \end{itemize}
\item Quantitative comparison (as stated):
  \begin{itemize}
  \item $\approx \textbf{300 mL water/100 g tissue/min}$ crosses by diffusion,
  \item $< \textbf{1 mL water/100 g tissue/min}$ crosses by filtration.
  \end{itemize}
\end{itemize}

\subsubsection{Bulk flow by filtration (Starling forces)}
\begin{itemize}
\item Capillary wall acts as a semipermeable membrane: permeable to water/solutes, relatively impermeable to large proteins.
\item Bulk flow primarily governs \textbf{fluid distribution}, not nutrition.
\item \textbf{Four Starling forces:}
  \begin{itemize}
  \item capillary hydrostatic pressure (\textbf{$P_c$}),
  \item interstitial hydrostatic pressure (\textbf{$P_{if}$}),
  \item plasma oncotic pressure (\textbf{$\pi_p$}),
  \item interstitial oncotic pressure (\textbf{$\pi_{if}$}).
  \end{itemize}
\end{itemize}

\paragraph{Net filtration pressure (NFP)}
\begin{itemize}
\item \textbf{$\mathrm{NFP} = (P_c - P_{if}) - (\pi_p - \pi_{if})$}
\item Worked example values:
  \begin{itemize}
  \item $P_c$ (arterial end) \textbf{35 mmHg}; $P_c$ (venous end) \textbf{15 mmHg}
  \item $P_{if}$ \textbf{0 mmHg}
  \item $\pi_p$ \textbf{28 mmHg}
  \item $\pi_{if}$ \textbf{3 mmHg}
  \end{itemize}
\item Therefore:
  \begin{itemize}
  \item NFP (arterial end) \textbf{+10 mmHg} (net filtration)
  \item NFP (venous end) \textbf{--10 mmHg} (net absorption)
  \end{itemize}
\end{itemize}

\paragraph{Magnitude of bulk flow}
\begin{itemize}
\item \textbf{Bulk flow = $k \times \mathrm{NFP}$}, where \textbf{$k$} is the capillary membrane filtration constant.
\item Net fluid loss from systemic capillaries is about \textbf{4 L/day}, returned via lymphatics.
\end{itemize}

\paragraph{Important pressure-dependent exceptions}
\begin{itemize}
\item Capillaries may show net filtration or absorption along their whole length depending on hydrostatic pressure.
  \begin{itemize}
  \item Glomerular capillaries: net filtration (high hydrostatic pressure).
  \item Lung capillaries: mean $P_c \approx \textbf{8 mmHg}$ $\rightarrow$ absorption is favoured.
  \end{itemize}
\end{itemize}

\subsection{Determinants of capillary hydrostatic pressure ($P_c$)}
\begin{itemize}
\item $P_c$ varies with the ratio of postcapillary to precapillary resistance:
  \[
  P_c \propto \frac{R_{\text{postcapillary}}}{R_{\text{precapillary}}}
  \]
\item Changes in arterial or venous pressure influence $P_c$, but a given change in \textbf{venous pressure} has a greater effect than the same change in arterial pressure.
  \begin{itemize}
  \item $P_c$ rises with elevated venous pressure (e.g.\ legs on standing; cardiac failure).
  \end{itemize}
\item Control elements include postcapillary smooth muscle (venules), precapillary sphincters, and (as noted) possible distinct postcapillary sphincters.
\end{itemize}

\subsection{Endothelium as an active organ}
\begin{itemize}
\item The cardiovascular system is lined by a single endothelial layer.
\item Beyond diffusion/filtration, endothelium produces vasoactive substances:
  \begin{itemize}
  \item \textbf{prostacyclin} (from arachidonic acid): inhibits platelet adhesion/aggregation and inhibits vasoconstriction,
  \item \textbf{nitric oxide (NO)} from L-arginine: increases vascular smooth muscle cGMP $\rightarrow$ decreased intracellular Ca$^{2+}$ $\rightarrow$ relaxation/vasodilation,
    \begin{itemize}
    \item stimulated by acetylcholine, ATP, bradykinin, serotonin, substance P, and histamine,
    \item may be enhanced by shear stress,
    \end{itemize}
  \item \textbf{endothelin:} potent vasoconstrictor; increases peripheral resistance and arterial pressure.
  \end{itemize}
\item Endothelium can form new capillary networks (angiogenesis) under angiogenic stimuli.
\end{itemize}

\subsection{Endothelial glycocalyx and the revised Starling hypothesis}
\begin{itemize}
\item The \textbf{endothelial glycocalyx} is a luminal layer of glycoproteins and glycosaminoglycans forming the interface between blood and capillary wall.
  \begin{itemize}
  \item It covers endothelial clefts and separates plasma from the \textbf{subglycocalyx space}, described as protein-free.
  \end{itemize}
\item In the revised Starling view, \textbf{subglycocalyx colloid osmotic pressure ($\pi_{sg}$)} replaces interstitial oncotic pressure as a determinant of transcapillary flow.
\end{itemize}

\subsubsection{Revised filtration expression (as shown)}
\begin{itemize}
\item \textbf{Filtration force = $(P_c - P_i) - \sigma(\pi_p - \pi_{sg})$}
  \begin{itemize}
  \item $\sigma$ = reflection coefficient term (as shown).
  \end{itemize}
\end{itemize}

\subsubsection{Functions of the glycocalyx (Table 29.1)}
\begin{itemize}
\item Interface between vessel wall and blood.
\item Barrier to vascular exchange of water/solutes and leukocyte--endothelium adhesion.
\item Sieve for plasma proteins.
\item Maintains colloid osmotic gradient of the vascular barrier.
\item Shear sensor; regulates mechanotransduction.
\item Binds biologically active molecules (e.g.\ antithrombin III, tissue factor inhibitor pathway, growth factors, extracellular SOD).
\end{itemize}

\subsubsection{Glycocalyx injury and potential protection/restoration (as stated)}
\begin{itemize}
\item Reduced glycocalyx thickness reported with hypervolaemia (via ANP), diabetes, hyperglycaemia, and inflammatory mediators during surgery and sepsis (CRP, TNF, bradykinin, mast cell tryptase).
\item Potential restoration/protection described with antithrombin III, hydrocortisone, N-acetyl cysteine, and sevoflurane anaesthesia.
\end{itemize}

\subsection{Lymphatics (return pathway for fluid and protein)}
\begin{itemize}
\item Lymphatic capillaries drain lymph (from interstitial fluid) through lymph nodes and larger vessels to the right and left subclavian veins.
\item Tissues lacking lymph vessels: \textbf{CNS}, cartilage, bone, and epithelium.
\item Lymphatic capillaries:
  \begin{itemize}
  \item thin-walled, blind-ended,
  \item contain valves ensuring one-way flow from interstitium back to the circulation,
  \item are permeable to fluid and protein,
  \item contain some contractile actin--myosin filaments.
  \end{itemize}
\item \textbf{Key principle:} lymphatics are the only route by which protein lost from vessels can return to the circulation.
\item Lymph flow is promoted by lymphatic wall contraction, skeletal muscle contraction, and one-way valves.
\item Protein concentrations (examples):
  \begin{itemize}
  \item interstitial fluid $\approx \textbf{2 g/dL}$,
  \item liver lymph up to \textbf{6 g/dL},
  \item intestinal lymph $\approx \textbf{3--4 g/dL}$.
  \end{itemize}
\item Lymphatics return the net $\approx \textbf{4 L/day}$ of interstitial fluid filtered from capillaries.
\end{itemize}

\section{Section V --- Vascular Function and Regulation (Kam \& Power)}

\subsection{Chapter 30 --- Venous Return and Vascular Function}

\subsubsection{Veins as a capacitance system}
The venous system is a \textbf{low-pressure, high-volume} component of the circulation. Functionally, it is the major \textbf{capacitance (storage) compartment}, in contrast to the arterial system, which is \textbf{high-pressure} and \textbf{low-volume}.

Key quantitative features:
\begin{itemize}
\item Systemic veins contain \textbf{$\sim$60\%} of total blood volume.
\item Veins are \textbf{$\sim$25--30$\times$} more compliant than arteries.
\item Consequently, for a given change in intravascular volume, the volume change is distributed roughly \textbf{25:1 to 30:1 (venous:arterial)}.
\end{itemize}

Why this matters:
\begin{itemize}
\item Veins provide \textbf{low-resistance pathways} back to the heart.
\item Venous capacitance strongly influences \textbf{cardiac filling} and therefore \textbf{cardiac output}.
\end{itemize}

\subsubsection{Venous pathway, pressures, and flow velocity}

\paragraph{Pathway.}
Capillaries $\rightarrow$ venules $\rightarrow$ larger veins $\rightarrow$ venae cavae $\rightarrow$ right atrium.
\begin{itemize}
\item Some \textbf{metabolic exchange} continues in venules.
\item As veins enlarge, their walls contain more \textbf{smooth muscle}, \textbf{elastic tissue}, and \textbf{connective tissue}.
\end{itemize}

\paragraph{Typical mean pressures.}
\begin{itemize}
\item Venules: \textbf{10--15 mmHg}
\item Larger veins: \textbf{4--8 mmHg}
\item Venae cavae: \textbf{0--2 mmHg}
\end{itemize}

\paragraph{Flow velocity and venous pulsations.}
\begin{itemize}
\item As venous tributaries converge, total cross-sectional area falls and \textbf{velocity increases}.
  \begin{itemize}
  \item Venae cavae: \textbf{$\sim$12 cm/s}
  \item (Aorta quoted: \textbf{$\sim$20 cm/s})
  \end{itemize}
\item \textbf{Right atrial contraction} generates \textbf{pressure pulsations} in the venae cavae.
\end{itemize}

\subsubsection{Venous distensibility and configurational change}
A defining property of veins is that \textbf{shape and compliance vary markedly with transmural pressure}.

\paragraph{Pressure--configuration relationship.}
\begin{itemize}
\item At \textbf{low internal pressure}, veins partially collapse and become \textbf{elliptical}.
\item A small pressure rise converts an elliptical lumen into a \textbf{circular} lumen, with a \textbf{large increase in volume}.
\item At \textbf{higher pressures}, circular veins stretch and \textbf{compliance falls}.
\end{itemize}

\paragraph{Consequences.}
\begin{enumerate}
\item \textbf{Large volume accommodation with little pressure rise}, particularly during the elliptical $\rightarrow$ circular transition.
\item When transmural pressure \textbf{approaches zero}, veins \textbf{collapse}, cross-sectional area falls, and \textbf{flow resistance increases}.
\end{enumerate}

Key quantitative point:
\begin{itemize}
\item When transmural pressure falls below \textbf{$\sim$6 cmH$_2$O}, veins tend to collapse and become elliptical.
\end{itemize}

Clinical illustration:
\begin{itemize}
\item In sitting/standing, \textbf{neck veins} often collapse because they are \textbf{$\sim$5--10 cm above the heart}, making transmural pressure close to \textbf{zero}.
\end{itemize}

\subsubsection{Venomotor tone and control of venous capacitance}

\paragraph{Innervation and mediators.}
\begin{itemize}
\item Venous smooth muscle is innervated by \textbf{sympathetic fibres} releasing \textbf{norepinephrine}.
\item Sympathetic stimulation (via \textbf{$\alpha$-adrenoreceptors}) produces:
  \begin{itemize}
  \item \textbf{decreased venous compliance}
  \item \textbf{increased venous pressure}
  \item a \textbf{shift of blood} from capacitance vessels toward the right heart
  \end{itemize}
\item Circulating vasoactive hormones described as producing similar venous effects include \textbf{epinephrine}, \textbf{angiotensin}, and \textbf{vasopressin}.
\end{itemize}

\paragraph{Venomotor tone and reflex control.}
\begin{itemize}
\item Basal sympathetic outflow produces tonic venous smooth muscle contraction: \textbf{venomotor tone}.
\item Reflexes, particularly \textbf{arterial baroreceptors}, adjust venomotor tone to support \textbf{arterial pressure}.
\end{itemize}

\subsubsection{Determinants of venous return}
Venous return is determined by:
\begin{itemize}
\item the \textbf{driving pressure gradient}
\item \textbf{venous valves}
\item the \textbf{skeletal muscle pump}
\item the \textbf{respiratory pump}
\item effects of \textbf{ventricular contraction and relaxation}
\item \textbf{venomotor tone}
\end{itemize}

\subsubsection{Mean systemic filling pressure and the driving gradient}

\paragraph{Definition.}
Guyton's \textbf{mean systemic filling pressure (MSFP)} is the average pressure within the systemic circulation, \textbf{weighted by the relative compliances} of its components.
\begin{itemize}
\item If the heart is arrested experimentally, arterial and venous pressures rapidly equilibrate; the resulting static pressure is the \textbf{MSFP}.
\item In the functioning circulation, MSFP approximates the \textbf{mean venous pressure}.
\end{itemize}

Typical values (quoted):
\begin{itemize}
\item MSFP: \textbf{$\sim$7 mmHg} (range \textbf{0--20 mmHg})
\item Mean pulmonary filling pressure: \textbf{$\sim$2 mmHg}
\end{itemize}
\paragraph{What changes MSFP.}
\begin{itemize}
\item MSFP \textbf{increases} with \textbf{blood volume} and \textbf{venomotor tone}.
\item MSFP \textbf{falls} with \textbf{venodilation} or \textbf{blood loss}.
\item MSFP is stated to be \textbf{unaffected by changes in total peripheral resistance}.
\end{itemize}

\paragraph{Pressure gradient for venous return.}
Define mean right atrial pressure as \textbf{RAP}.
\[
\textbf{Pressure gradient for venous return} = \textbf{MSFP} - \textbf{RAP}
\]
Example (quoted): MSFP 7 mmHg, RAP 1 mmHg $\Rightarrow$ gradient $= \textbf{6 mmHg}$.

\subsubsection{Venous valves and one-way flow}
\begin{itemize}
\item Venous valves are \textbf{thin bicuspid} structures and occur even in small limb veins.
\item They are most frequent \textbf{peripherally} and are absent in:
  \begin{itemize}
  \item central abdominal veins
  \item the venae cavae
  \item cerebral veins
  \item portal veins
  \end{itemize}
\end{itemize}
Functional role:
\begin{itemize}
\item Valves divide the lower-limb blood column into segments, facilitating \textbf{upward propulsion} toward the heart.
\end{itemize}

\subsubsection{Mechanisms that augment venous return}

\paragraph{Skeletal muscle pump.}
\begin{itemize}
\item Alternating muscle contraction and relaxation squeezes veins:
  \begin{itemize}
  \item contraction $\rightarrow$ expels blood toward the heart
  \item relaxation $\rightarrow$ allows veins to refill
  \end{itemize}
\item \textbf{Venous valves} ensure unidirectional flow.
\end{itemize}
Clinical relevance:
\begin{itemize}
\item During standing, rhythmic leg muscle activity reduces venous pressure and postural pooling.
\item During exercise, the muscle pump contributes to increased venous return.
\end{itemize}

\paragraph{Respiratory pump.}
During inspiration:
\begin{itemize}
\item intrapleural pressure falls from \textbf{$-5$ to $-8$ cmH$_2$O}
\item diaphragm descent increases \textbf{intra-abdominal pressure}
\end{itemize}
Net effect: promotes movement of blood from extrathoracic veins into the thorax and right atrium.

Quantified effects (quoted):
\begin{itemize}
\item thoracic blood volume increases by \textbf{$\sim$250 mL}
\item right ventricular stroke volume increases by \textbf{$\sim$20 mL}
\end{itemize}

Left-sided effects:
\begin{itemize}
\item inspiration increases pulmonary vascular capacity $\rightarrow$ \textbf{decreases LV stroke volume}
\item respiratory variation in LV stroke volume quoted as \textbf{$\sim$5\%}
\end{itemize}

Limit:
\begin{itemize}
\item Excessively negative intrathoracic pressure can cause \textbf{collapse of veins} as they enter the chest.
\end{itemize}

\paragraph{Effects of ventricular contraction and relaxation.}
\begin{itemize}
\item During rapid ejection (systole), atrial pressure may fall to \textbf{zero or negative} as the atrioventricular fibrous ring is pulled downward; atrial volume increases, which augments inflow from the venae cavae and pulmonary veins.
\item In early diastole, rapid ventricular filling is associated with falling atrial and ventricular pressures (atrial \textbf{v} wave), which may facilitate atrial inflow.
\end{itemize}

\subsubsection{Posture and venous return}

\paragraph{Supine.}
Approximate mean pressures in foot vessels:
\begin{itemize}
\item arterial: \textbf{$\sim$100 mmHg}
\item venous: \textbf{$\sim$15 mmHg}
\end{itemize}
With no leg muscle activity, leg venous valves are open and venous flow back to the heart is continuous.

\paragraph{Standing.}
\begin{itemize}
\item The foot is \textbf{$\sim$120 cm} below the heart, so hydrostatic effects increase both arterial and venous pressures by \textbf{$\sim$85--90 mmHg} ($\approx$ \textbf{120 cmH$_2$O}).
\item RAP is unchanged.
\end{itemize}
If vessels were rigid, the relevant driving gradients would not change. In reality:
\begin{itemize}
\item increased venous transmural pressure distends leg veins $\rightarrow$ increases venous volume $\rightarrow$ tends to reduce venous return, cardiac output, and arterial pressure.
\end{itemize}
Compensation:
\begin{itemize}
\item arterial baroreceptors increase sympathetic outflow to:
  \begin{itemize}
  \item veins (reduced compliance to help maintain filling pressure)
  \item the heart and resistance vessels
  \end{itemize}
\item skeletal muscle activity plus venous valves reduces pooling
\end{itemize}
Brain:
\begin{itemize}
\item standing reduces arterial and venous pressures in the brain by \textbf{$\sim$50 mmHg}, but the arterial-to-venous perfusion gradient is stated to be unchanged.
\end{itemize}

\paragraph{Cerebral venous sinuses and the sitting position.}
\begin{itemize}
\item On standing, cerebral venous sinus pressure may fall to \textbf{$\sim-40$ mmHg}.
\item Sinuses do not collapse because CSF pressure falls simultaneously and surrounding tissues support them.
\end{itemize}
Clinical implication:
\begin{itemize}
\item In the sitting position for neurosurgery, opened cerebral venous sinuses may admit \textbf{large air emboli}.
\end{itemize}

\subsubsection{The venous return curve (Guyton)}

\paragraph{Experimental basis: right heart bypass.}
\begin{itemize}
\item A bypass pump replaces the right ventricle.
\item Blood enters the pump via a collapsible tube and is returned to the pulmonary artery.
\item The pump maintains \textbf{$-10$ to $-20$ mmHg}.
\item RAP is adjusted by raising or lowering the collapsible tubing.
\end{itemize}
Key observation:
\begin{itemize}
\item When RAP is raised to \textbf{$\sim$7 mmHg}, venous return falls to zero and pressures equalize; this pressure is the \textbf{MSFP}.
\end{itemize}

\paragraph{Linear range and governing equation.}
\begin{itemize}
\item Between RAP \textbf{0} and MSFP (\textbf{$\sim$7 mmHg}), venous return decreases \textbf{linearly} as RAP rises.
\end{itemize}
\[
\textbf{Venous return} = \frac{\textbf{MSFP} - \textbf{RAP}}{\textbf{resistance to venous return}}
\]
(“Resistance to venous return” reflects the resistance and capacitance properties of the venous circulation.)

\paragraph{Negative RAP: plateau (Starling resistor behaviour).}
The linear relationship does not hold when RAP becomes negative:
\begin{itemize}
\item venous return increases only to a \textbf{plateau}, about \textbf{20\% above} the value at RAP $=0$
\item extrathoracic veins collapse and behave as \textbf{Starling resistors}, limiting further increases in flow
\item the pressure in extrathoracic veins cannot be reduced below \textbf{$-4$ mmHg}
\end{itemize}
Exam pitfall:
\begin{itemize}
\item Increasing pump function by lowering RAP alone cannot generate a large increase in cardiac output unless venous return also increases (for example, by raising MSFP).
\end{itemize}

\subsubsection{Changing MSFP: parallel shifts of the venous return curve}
Changes in MSFP produce an approximately \textbf{parallel shift} of the venous return curve:
\begin{itemize}
\item increased MSFP shifts the curve \textbf{right} $\rightarrow$ higher venous return at any given RAP
\item decreased MSFP shifts the curve \textbf{left} $\rightarrow$ lower venous return at any given RAP
\end{itemize}

Blood volume sensitivity (quoted):
\begin{itemize}
\item \textbf{+15\%} blood volume doubles MSFP
\item \textbf{--15\%} blood volume reduces MSFP to \textbf{zero}
\end{itemize}

Sympathetic contribution:
\begin{itemize}
\item Sympathetic constriction of capacitance vessels increases MSFP.
\item Maximal sympathetic discharge increases MSFP to \textbf{$\sim$17 mmHg}, an important compensatory mechanism in hypovolaemia.
\end{itemize}

\subsubsection{Changing resistance to venous return: altering slope}
Changing the resistance to venous return alters the \textbf{slope} of the venous return curve:
\begin{itemize}
\item decreased resistance $\rightarrow$ curve rotates \textbf{clockwise} $\rightarrow$ increased venous return
\item increased resistance $\rightarrow$ curve rotates \textbf{counterclockwise} $\rightarrow$ decreased venous return
\end{itemize}
The text attributes changes in resistance to venous return to changes in peripheral vascular resistance via autoregulatory constriction/dilation.

Example:
\begin{itemize}
\item During exercise, muscle metabolites produce marked vasodilatation $\rightarrow$ reduced venous resistance $\rightarrow$ enhanced venous return.
\end{itemize}

\subsubsection{Venous return and cardiac output: cardiac function curve and equilibrium point}

\paragraph{Cardiac output curve (Guyton).}
The cardiac output (cardiac function) curve relates cardiac output to RAP and reflects \textbf{Frank--Starling} behaviour.
\begin{itemize}
\item The curve has a \textbf{steep upstroke}: small RAP changes can produce large changes in cardiac output.
\item At higher RAP, the curve approaches a \textbf{plateau}, determined by the heart's \textbf{maximum pumping capacity}.
\end{itemize}
Quoted values:
\begin{itemize}
\item Normal resting sympathetic tone: cardiac output is \textbf{$\sim$5 L/min at RAP 0 mmHg}, with a maximum of \textbf{$\sim$10--15 L/min at RAP $\sim$4 mmHg}.
\item Increased sympathetic stimulation (e.g.\ exercise): maximum pumping capacity may rise to \textbf{$\sim$20--30 L/min}.
\end{itemize}
Key interpretation:
\begin{itemize}
\item Under most conditions, venous return is well below maximum pumping capacity, so the heart adjusts output to \textbf{match venous return}.
\end{itemize}

\paragraph{Equilibrium point (venous return = cardiac output).}
A stable circulation requires \textbf{venous return = cardiac output}.
\begin{itemize}
\item If they differ, pressures and volumes change over a few beats until a new steady state is reached.
\item The intersection of the venous return and cardiac output curves is the \textbf{equilibrium point}.
\end{itemize}
Normal example (quoted): RAP $\approx$ \textbf{0 mmHg}, cardiac output $=$ venous return $\approx$ \textbf{5 L/min}.

\paragraph{Sympathetic tone shifts both curves.}
\begin{itemize}
\item Total sympathectomy (e.g.\ high spinal anaesthesia):
  \begin{itemize}
  \item reduced contractility shifts the cardiac function curve to the \textbf{right}
  \item venodilation lowers MSFP and shifts the venous return curve to the \textbf{left}
  \item the new equilibrium is associated with a \textbf{$\sim$40\% decrease} in cardiac output
  \end{itemize}
\item Sympathetic stimulation:
  \begin{itemize}
  \item enhances cardiac function
  \item constricts capacitance vessels $\rightarrow$ increases MSFP $\rightarrow$ enhances venous return
  \item the new equilibrium has \textbf{higher cardiac output} with \textbf{lower RAP}
  \item maximal sympathetic stimulation can \textbf{double cardiac output} while reducing RAP
  \end{itemize}
\end{itemize}
Key interpretation:
\begin{itemize}
\item Most of the increase in cardiac output during sympathetic stimulation is attributed to the \textbf{shift of the venous return curve} (via capacitance vessel constriction), rather than the cardiac function curve alone.
\end{itemize}

\subsubsection{High-yield takeaways and common traps}
\begin{itemize}
\item Venous return is driven by \textbf{MSFP -- RAP}, not by RAP alone.
\item Total peripheral resistance is stated not to change MSFP, although it can change the \textbf{slope} of the venous return curve.
\item Negative RAP does not produce unlimited venous return; flow plateaus due to extrathoracic venous collapse (Starling resistor behaviour).
\item Venomotor tone is powerful because it changes venous compliance and therefore MSFP, redistributing blood centrally.
\item Posture-related reductions in venous return are primarily a \textbf{capacitance/compliance} problem (venous distension), not simply a change in hydrostatic gradients.
\end{itemize}

\subsection{Chapter 31 --- Regulation of Arterial Blood Pressure}

\subsubsection{Objectives and overall framework}
The overall regulation of the circulation serves two linked aims:
\begin{enumerate}
\item \textbf{Maintain an adequate and reasonably constant arterial blood pressure (ABP).}
\item \textbf{Match tissue blood flow to metabolic requirements}, supporting capillary exchange and temperature regulation.
\end{enumerate}

ABP is described as the force exerted by blood per unit area of the arterial wall. Functionally, ABP is the product of \textbf{cardiac output (CO)} and \textbf{total peripheral resistance (TPR)}. The medullary cardiovascular centres regulate ABP by adjusting both variables.

Regulation operates on different time scales:
\begin{itemize}
\item \textbf{Short-term regulation (minutes to hours):} rapid autonomic responses to posture change, exercise, stress, and blood loss.
\item \textbf{Long-term regulation (weeks to months):} dominated by \textbf{renal control of body fluid volume} (balance between intake and excretion).
\end{itemize}

\subsubsection{Functional organization of cardiovascular regulation}
Control of ABP can be described as a closed-loop system with:
\begin{itemize}
\item \textbf{sensors (receptors)} detecting the controlled variables
\item \textbf{afferent pathways} carrying information to integrating centres
\item \textbf{central integration} within the brainstem (modulated by higher centres)
\item \textbf{efferent pathways} (autonomic nerves and hormones)
\item \textbf{effectors} (heart, vessels, kidneys, thirst/water intake)
\end{itemize}

\paragraph{Medullary and spinal components.}
Neuronal elements involved in cardiovascular control include:
\begin{itemize}
\item premotor sympathetic and preganglionic parasympathetic neurones in the medulla
\item preganglionic sympathetic neurones in the thoracolumbar spinal cord (intermediolateral column, \textbf{T1--L2})
\item afferent fibres in the glossopharyngeal and vagus nerves
\item interneurones in the medulla
\end{itemize}

Efferent sympathetic activity originates from preganglionic neurones in the intermediolateral columns (T1--L2), with postganglionic fibres supplying the \textbf{heart, vessels, and kidneys}. The \textbf{adrenal medulla} is innervated by preganglionic sympathetic fibres.

Afferent fibres from cardiovascular receptors synapse in the \textbf{nucleus tractus solitarius (NTS)}. Interneurones link afferents with higher centres and with sympathetic premotor and vagal preganglionic pathways.

\subsubsection{The arterial baroreceptor reflex}
The \textbf{arterial baroreceptor reflex} is described as the most important mechanism for \textbf{short-term regulation} of ABP.

\paragraph{Sensors and location.}
Arterial baroreceptors are mechanoreceptors activated by arterial wall stretch as ABP rises.
\begin{itemize}
\item \textbf{Carotid sinus baroreceptors:} in a thin-walled dilatation at the origin of the internal carotid arteries.
\item \textbf{Aortic baroreceptors:} in the transverse aortic arch.
\end{itemize}
The receptors are spray-like free nerve endings; they sense pressure indirectly via stretch of elastic arterial walls.

\paragraph{Afferent pathways.}
\begin{itemize}
\item Carotid sinus afferents travel via \textbf{Hering's nerve} to the \textbf{glossopharyngeal nerve}.
\item Aortic baroreceptor afferents travel via the \textbf{vagus nerve}.
\end{itemize}
Both project to the \textbf{NTS} in the medulla.

\paragraph{What is sensed: mean pressure and pulsatility.}
Baroreceptors sense:
\begin{itemize}
\item \textbf{absolute stretch} (static response), and
\item \textbf{rate of change of stretch} (dynamic response).
\end{itemize}
Therefore, both \textbf{mean arterial pressure} and \textbf{pulse pressure} influence firing.

A key observation is that \textbf{pulsatile pressure increases firing rate} at a given mean ABP compared with constant pressure. Carotid sinus baroreceptors are described as \textbf{more sensitive} than aortic baroreceptors.

\paragraph{Central integration and efferent response.}
\begin{itemize}
\item \textbf{Rise in ABP} $\rightarrow$ increased baroreceptor discharge $\rightarrow$ reflex \textbf{decrease in sympathetic} and \textbf{increase in parasympathetic} activity:
  \begin{itemize}
  \item decreased heart rate and contractility
  \item dilation of resistance vessels
  \end{itemize}
\item \textbf{Fall in ABP} $\rightarrow$ reduced baroreceptor discharge $\rightarrow$ reflex \textbf{increase in sympathetic} and \textbf{decrease in parasympathetic} activity:
  \begin{itemize}
  \item increased heart rate and contractility
  \item constriction of resistance vessels
  \end{itemize}
\end{itemize}

\paragraph{Operating range and saturation.}
In isolated carotid sinus studies (other baroreceptors denervated):
\begin{itemize}
\item no afferent discharge at carotid pressures \textbf{below $\sim$30 mmHg}
\item linear response with carotid perfusion pressures \textbf{$\sim$70--110 mmHg}
\item no further increase in response when carotid perfusion pressure is \textbf{above $\sim$150 mmHg}
\end{itemize}

\paragraph{Tonic activity and beat-to-beat modulation.}
At normal mean ABP, some carotid and aortic baroreceptor fibres fire continuously and:
\begin{itemize}
\item inhibit tonic sympathetic output
\item stimulate vagal activity
\end{itemize}
Additional fibres fire during systole, producing bursts of vagal discharge and further inhibition of sympathetic outflow.

\paragraph{Resetting and the role of the reflex.}
Arterial baroreceptors can be \textbf{reset} to higher or lower pressures.
\begin{itemize}
\item In isolated preparations, if distending pressure is changed and held constant, discharge initially rises then returns toward baseline (accommodation). A suggested mechanism is opening of potassium channels, returning membrane potential toward a new resting value at the new pressure.
\item Resetting can also occur via central mechanisms (e.g.\ during exercise), and sympathetic activity to the carotid sinus can increase firing at a given pressure.
\end{itemize}
Overall, the arterial baroreceptor reflex primarily regulates ABP in the \textbf{short term}, minimizing fluctuations during rapid changes in posture, cardiac output, or peripheral resistance.

\subsubsection{Cardiopulmonary (low-pressure) baroreceptors}
Cardiopulmonary receptors are described as important contributors to cardiovascular control, with heterogeneous effects. Three main groups are identified:
\begin{itemize}
\item myelinated vagal venoatrial stretch receptors
\item unmyelinated vagal and sympathetic cardiac mechanoreceptors
\item vagal and sympathetic chemosensitive fibres
\end{itemize}

\paragraph{Venoatrial stretch receptors (A and B fibres).}
These myelinated vagal fibres lie in the endocardium at the junction of the vena cava with the atria, and the pulmonary veins with the atria. Two types are described:
\begin{itemize}
\item \textbf{Type A:} fire during atrial contraction (the \textbf{a} wave).
\item \textbf{Type B:} fire during atrial filling (the \textbf{v} wave) and convey information about \textbf{central venous pressure} and cardiac distension.
\end{itemize}
Stimulation produces:
\begin{itemize}
\item \textbf{tachycardia} (Bainbridge effect), due to a selective increase in sympathetic activity to the SA node
\item increased urine volume and salt excretion, attributed to reduced renal sympathetic activity, inhibition of ADH secretion, and increased atrial ANP production
\end{itemize}
Proposed functions include regulation of cardiac size when venous pressure is high and adjustment of blood volume.

\paragraph{Cardiac mechanoreceptors and the Bezold--Jarisch reflex.}
Unmyelinated vagal and sympathetic mechanoreceptors form a fine network in both atria and (mainly) the left ventricle; myelinated vagal mechanoreceptors also occur around coronary arteries.

The combined input of atrial and ventricular mechanoreceptors is described as producing a reflex of \textbf{bradycardia} and \textbf{vasodilatation} (Bezold--Jarisch reflex).

Left ventricular mechanoreceptors may contribute to \textbf{vasovagal syncope} when stimulated during orthostatic hypotension by vigorous ventricular contraction at reduced filling volume (e.g.\ following spinal anaesthesia in a hypovolaemic patient).

\paragraph{Chemosensitive cardiac afferents.}
Vagal and sympathetic chemosensitive fibres in the heart are stimulated by products released from ischaemic myocardium.
\begin{itemize}
\item Sympathetic chemosensitive afferents are described as mediating the pain of myocardial ischaemia and infarction.
\item Convergence of sympathetic afferents with somatic pathways in the spinothalamic tracts is described as the basis of referred pain to the arms, neck, and chest wall.
\end{itemize}

\subsubsection{Peripheral chemoreceptors}
Peripheral chemoreceptors in the carotid and aortic bodies are stimulated by \textbf{hypoxia} and \textbf{hypercapnia}. Direct cardiovascular effects are described as \textbf{hypertension and bradycardia}, but bradycardia is offset by chemoreceptor stimulation of inspiratory neurones and lung stretch receptors, both of which produce tachycardia.

Net effect of chemoreceptor stimulation:
\begin{itemize}
\item increased \textbf{peripheral resistance}
\item increased \textbf{heart rate}
\end{itemize}
Chemoreceptors are also stimulated by stagnant hypoxia and metabolic acidosis during \textbf{very low ABP}, and this is described as important in the cardiovascular response to \textbf{severe hypotension}.

\subsubsection{Short-term control: central integration and higher-centre modulation}

\paragraph{Central sympathetic cells.}
\begin{itemize}
\item \textbf{Rostral ventrolateral medulla (RVLM):} major source of excitatory input to sympathetic nerves controlling peripheral blood vessels; tonically active and \textbf{inhibited by the baroreceptor reflex}. Neurotransmitter described: \textbf{glutamate}.
\item \textbf{Caudal ventrolateral medulla (CVLM):} described as having a depressor effect (reducing peripheral resistance and cardiac contractility), thought to tonically inhibit RVLM via \textbf{GABA}.
\end{itemize}

\paragraph{Central parasympathetic neurones.}
Cell bodies of efferent vagal parasympathetic cardiac nerves lie in the \textbf{nucleus ambiguus} and the \textbf{dorsal vagal nucleus}.
\begin{itemize}
\item They receive baroreceptor input from the NTS and discharge synchronously with each cardiac cycle.
\item Medullary inspiratory neurones directly inhibit parasympathetic activity, contributing to tachycardia during inspiration (sinus arrhythmia).
\end{itemize}

\paragraph{Other CNS influences.}
\begin{itemize}
\item The \textbf{hypothalamus} has a central role in regulating visceral function; different regions have predominantly pressor or depressor actions when stimulated.
  \begin{itemize}
  \item Dorsal anterior hypothalamus: suppresses sympathetic activity and activates vagal inhibition of the heart (baroreceptor-like).
  \item Anterior perifornical region: involved in pressor responses in defence/alerting responses.
  \item Activated by the limbic system, particularly the amygdala, in rage and fear reactions.
  \end{itemize}
\item The \textbf{cerebellum} mediates cardiovascular responses at the onset of exercise via proprioceptive input from muscles, joints, and skin.
\end{itemize}

\subsubsection{Efferent pathways and effectors}
Efferent control is mediated by parasympathetic (vagal) nerves, sympathetic nerves, and hormones including epinephrine, norepinephrine, ADH, renin, angiotensin, aldosterone, and ANP.

Effectors include the heart, blood vessels, kidneys, and thirst/water intake.

Key efferent actions described:
\begin{itemize}
\item Sympathetic nerves and catecholamines: increase force and rate of cardiac contraction; constrict arteriolar resistance vessels; reduce venous capacitance.
\item Vagal effects: largely limited to the heart (especially atria and AV node), reducing SA node discharge rate and AV node conduction.
\item ADH: increases water reabsorption in collecting ducts and produces arteriolar vasoconstriction.
\end{itemize}

\subsubsection{Negative feedback control of arterial pressure}
Mean ABP is described as being determined by the amount of blood in the arterial system at any point in time.
\begin{itemize}
\item Arterial blood volume is determined by inflow to the aorta (cardiac output) and runoff into capillaries.
\item Cardiac output $=$ heart rate $\times$ stroke volume.
\item Peripheral runoff is determined by mean ABP acting across the total peripheral resistance of arterioles.
\end{itemize}
The arterial baroreceptor reflex influences determinants of cardiac output, peripheral resistance, and ABP within a negative feedback loop.

\subsubsection{Long-term regulation of arterial blood pressure}
Long-term regulation involves the kidneys, renal handling of sodium, and regulation of blood volume.

The baroreceptor reflex influences longer-term body fluid balance, including renal blood flow, RAAS, ADH, and thirst/water intake. Other cardiopulmonary receptors also contribute to reflex modulation of the factors determining ABP, and arterial chemoreceptors stimulate sympathetic drive in severe hypotension.

\paragraph{Renin--angiotensin--aldosterone system.}
The renin--angiotensin system is described as the most important renal--body fluid mechanism for long-term blood pressure control.

Trigger:
\begin{itemize}
\item a decrease in ABP $\rightarrow$ decreased renal perfusion, sensed by mechanoreceptors in afferent arterioles $\rightarrow$ renin secretion from juxtaglomerular cells
\end{itemize}

Pathway:
\begin{itemize}
\item renin converts angiotensinogen $\rightarrow$ angiotensin I (plasma)
\item in lungs and kidneys, angiotensin I $\rightarrow$ \textbf{angiotensin II}
\end{itemize}

Actions of angiotensin II (via AT$_1$ receptors in adrenal cortex, vascular smooth muscle, kidneys, and brain):
\begin{itemize}
\item stimulates zona glomerulosa to synthesize and secrete \textbf{aldosterone}
\item increases proximal tubular Na$^+$/H$^+$ exchanger activity and enhances reabsorption of Na$^+$ and HCO$_3^-$
\item acts on hypothalamus to increase \textbf{thirst} and stimulates \textbf{ADH} release $\rightarrow$ increased water reabsorption in collecting ducts
\item causes arteriolar vasoconstriction via IP$_3$/Ca$^{2+}$ signalling $\rightarrow$ increased TPR and ABP
\end{itemize}

\paragraph{Cardiopulmonary receptors, ANP, and volume control.}
Low-pressure cardiopulmonary baroreceptors are described as sensing the ``fullness'' of the circulation (blood volume). Responses to increased blood volume include:
\begin{enumerate}
\item increased \textbf{ANP} secretion from the atria $\rightarrow$ vasodilation and increased Na$^+$/water excretion
\item decreased \textbf{ADH} secretion $\rightarrow$ increased water excretion
\item renal vasodilation $\rightarrow$ increased Na$^+$ and water excretion
\end{enumerate}

\subsubsection{High-yield takeaways and common traps}
\begin{itemize}
\item Short-term ABP control is dominated by the \textbf{arterial baroreceptor reflex}; long-term control is dominated by \textbf{renal--body fluid mechanisms}.
\item Baroreceptors respond to \textbf{mean pressure and pulse pressure}; \textbf{pulsatility increases firing} at a given mean ABP.
\item The baroreceptor reflex has an operating range: below \textbf{$\sim$30 mmHg} there may be no discharge; responses saturate at higher pressures (no further response above \textbf{$\sim$150 mmHg} in isolated carotid sinus studies).
\item Baroreceptors \textbf{reset} with sustained pressure changes and during conditions such as exercise.
\item Cardiopulmonary receptors (especially venoatrial stretch receptors) link ``fullness'' of the circulation to \textbf{heart rate} (Bainbridge effect) and \textbf{renal salt/water handling} (via renal sympathetic activity, ADH, and ANP).
\end{itemize}

\subsection{Chapter 32 --- Integrated Cardiovascular Responses}

\subsubsection{Haemorrhage}
Haemorrhage decreases \textbf{mean systemic filling pressure (MSFP)}, reducing venous return so that \textbf{cardiac output falls}. The physiological effects depend on the \textbf{rate and degree} of blood loss; multiple compensatory mechanisms can maintain arterial pressure until they are overwhelmed.

\paragraph{Immediate responses.}
A reduction in blood volume reduces venous return, cardiac output, and arterial blood pressure. The fall in arterial pressure initiates powerful sympathetic activation via arterial baroreceptor pathways.

Key effects of the sympathetic response include:
\begin{itemize}
\item \textbf{arteriolar vasoconstriction} in most vascular beds $\rightarrow$ $\uparrow$ TPR
\item \textbf{venoconstriction} (reduced venous compliance), predominantly in \textbf{splanchnic and cutaneous} capacitance vessels $\rightarrow$ central blood shift
\item \textbf{tachycardia} and \textbf{increased contractility}
\item redistribution of blood flow away from tissues with strong sympathetic innervation
\end{itemize}

Compensation thresholds described:
\begin{itemize}
\item blood loss \textbf{$<$20\%}: arterial pressure can be maintained (i.e.\ does not fall below \textbf{$\sim$70 mmHg})
\item blood loss \textbf{$>$20\%}: arterial pressure and cardiac output decrease rapidly as compensation becomes inadequate
\item ABP \textbf{$<$50 mmHg}: \textbf{CNS ischaemic response} with powerful sympathetic stimulation (maximal within \textbf{$\sim$30 s})
\item blood loss \textbf{$>$30\%}: \textbf{irreversible hypotension} may occur
\end{itemize}

Microcirculatory contribution:
\begin{itemize}
\item precapillary vasoconstriction reduces capillary hydrostatic pressure and promotes \textbf{absorption of interstitial fluid} into the vascular compartment (quoted: as much as \textbf{$\sim$1 L})
\end{itemize}

Metabolic consequence (as described):
\begin{itemize}
\item inadequate tissue perfusion increases metabolic acid production and PaCO$_2$, reducing peripheral vascular responsiveness to catecholamines
\end{itemize}

\paragraph{Hormonal responses.}
Endocrine responses augment haemodynamic compensation and volume conservation:
\begin{itemize}
\item reduced renal perfusion and reduced NaCl delivery to the macula densa activate the \textbf{renin--angiotensin pathway}, increasing \textbf{aldosterone} release
\item \textbf{angiotensin} and \textbf{vasopressin (ADH)} take \textbf{$\sim$10 min} to become effective; they raise arterial pressure and MSFP, thereby increasing venous return
\item increased sympathetic activity promotes adrenal release of \textbf{catecholamines} and \textbf{cortisol}
\end{itemize}

\paragraph{Haematological responses.}
\begin{itemize}
\item reabsorption of interstitial fluid causes \textbf{haemodilution}
\item decreased haematocrit stimulates \textbf{bone marrow} red cell production
\end{itemize}

\paragraph{Summary of cardiovascular responses to haemorrhage (Table 32.1).}
Direction of change described:
\begin{itemize}
\item heart rate: $\uparrow$
\item contractility: $\uparrow$
\item carotid sinus nerve firing rate: $\downarrow$
\item total peripheral resistance: $\uparrow$
\item renin: $\uparrow$
\item angiotensin II: $\uparrow$
\item aldosterone: $\uparrow$
\item ADH (stimulated by hypovolaemia): $\uparrow$
\item circulating epinephrine (adrenaline; adrenal medulla): $\uparrow$
\end{itemize}

\paragraph{Guyton framework: shifts in venous return and cardiac output curves.}
Haemorrhage reduces MSFP (a leftward shift of the venous return curve). Increased sympathetic activity partially compensates by reducing venous compliance (raising MSFP relative to the haemorrhage-alone state) and enhancing cardiac function. A new equilibrium point is established at the intersection of the altered venous return and cardiac output curves.

\subsubsection{ Circulatory shock}
Shock is described as inadequate tissue perfusion and oxygenation. Causes are classified as:
\begin{itemize}
\item \textbf{hypovolaemic shock:} inadequate circulating blood volume
\item \textbf{cardiogenic shock:} reduced cardiac pumping ability with significant reduction in cardiac output
\item \textbf{obstructive shock:} reduced cardiac filling not related to hypovolaemia
\item \textbf{distributive shock:} profound vasodilatation with inappropriate distribution of blood despite an adequate cardiac output
\end{itemize}

\paragraph{Common pathophysiology in severe shock.}
In severe shock (any cause), pulmonary function may be impaired with hypoxaemia (V/Q mismatch). Cellular hypoxia leads to impaired oxidative metabolism, falling ATP, and failure of membrane ionic pumps:
\begin{itemize}
\item Na$^+$, Ca$^{2+}$, and water enter cells; K$^+$ leaves $\rightarrow$ swelling and eventual cell death
\item mediators released from disrupted cells contribute to further dysfunction
\end{itemize}

\paragraph{ Responses to shock: neural and endocrine.}
Integrated nervous and endocrine responses limit adverse effects, including fluid and salt conservation, increased cardiac output, and redistribution of tissue perfusion.

Neural mechanisms include arterial baroreceptors, atrial stretch receptors, and chemoreceptors:
\begin{itemize}
\item sympathetic activity increases, causing generalized \textbf{vasoconstriction} and \textbf{venoconstriction}, with increased activity in most regions except the coronary and cerebral circulations
\end{itemize}

Endocrine responses (pituitary, kidney, adrenal) include:
\begin{itemize}
\item adrenal medullary catecholamines (vasoconstriction, inotropy/chronotropy, metabolic substrates)
\item ACTH release and adrenal cortical hormone responses
\item ADH secretion (water conservation; vasoconstrictor action)
\item RAAS activation (angiotensin II and aldosterone)
\end{itemize}

\subsubsection{Hypovolaemic and distributive shock}

\paragraph{Hypovolaemic shock and clinical staging of haemorrhagic shock (Table 32.3).}
Hypovolaemic shock results from inadequate circulating volume (e.g.\ haemorrhage and other causes of fluid loss). Haemorrhagic shock is divided clinically into four stages:

\begin{center}
\small
\begin{tabular}{|l|c|c|c|c|}
\hline
\textbf{Parameter} & \textbf{Class I} & \textbf{Class II} & \textbf{Class III} & \textbf{Class IV} \\
\hline
Blood loss (mL) & $<750$ & 750--1500 & 1500--2000 & $>2000$ \\
\hline
\% blood volume & $<15$\% & 15--30\% & 30--40\% & $>40$\% \\
\hline
Pulse (beats/min) & $<100$ & 100--120 & 120--140 & $>140$ \\
\hline
Blood pressure & Normal & Normal & Decreased & Decreased \\
\hline
Pulse pressure & Normal & Decreased & Decreased & Decreased \\
\hline
Resp.\ rate (breaths/min) & 14--20 & 20--30 & 30--40 & $>40$ \\
\hline
Urine output (mL/h) & $>30$ & 20--30 & 5--15 & Negligible \\
\hline
Mental status & Slightly anxious & Anxious & Confused & Lethargic \\
\hline
\end{tabular}
\end{center}

\paragraph{Distributive shock (sepsis and anaphylaxis).}
Distributive shock is characterised by profound systemic vasodilatation with inadequate perfusion pressure and inappropriate distribution of blood, despite an adequate cardiac output.

In septic shock, a severe vascular disturbance occurs in response to bacterial infection. The text describes generalized vasodilatation, increased vascular permeability, and myocardial depression, with multiple mediators implicated (e.g.\ prostaglandins/leukotrienes, cytokines, nitric oxide, and others released from immune cells and endothelium), contributing to hypotension, poor tissue perfusion, and hypoxia.

\subsubsection{ Cardiogenic and obstructive shock}

\paragraph{Cardiogenic shock.}
Cardiogenic shock is due to impaired myocardial pumping function and a significant fall in cardiac output (described as clinically apparent if cardiac output is \textbf{$<2.5$ L/min}). As myocardial function fails:
\begin{itemize}
\item left atrial pressure rises (pulmonary congestion)
\item hypotension triggers reflex vasoconstriction
\item neurohumoral responses drive reflex tachycardia and vasoconstriction
\end{itemize}
Myocardial ischaemia can occur and worsen the degree of circulatory failure.

\paragraph{Obstructive shock.}
Obstructive shock results from factors that physically interfere with diastolic filling and/or mechanically restrict cardiac pumping. Examples described include:
\begin{itemize}
\item \textbf{cardiac tamponade} (pericardial fluid compresses the heart)
\item \textbf{pulmonary embolism} (marked reduction in venous return to the heart)
\end{itemize}

\subsubsection{The Valsalva manoeuvre}
Forced expiration against a closed airway is termed the \textbf{Valsalva manoeuvre}.
\begin{itemize}
\item During the manoeuvre, intrathoracic pressure increases; arterial pressure rises by about \textbf{7 mmHg} for a \textbf{10 mmHg} rise in mouth pressure.
\end{itemize}

The cardiovascular changes are divided into four phases:
\begin{itemize}
\item \textbf{Phase I (onset):} transient rise in arterial pressure due to transmission of increased intrathoracic pressure to the aorta and a transient increase in LV stroke volume (autotransfusion from pulmonary vessels).
\item \textbf{Phase II (strain):} reduced venous return $\rightarrow$ reduced stroke volume and arterial pressure; reflex tachycardia and vasoconstriction tend to restore blood pressure.
\item \textbf{Phase III (release):} transient fall in arterial pressure at release of the positive pressure.
\item \textbf{Phase IV (recovery):} venous return rises as intrathoracic pressure returns to baseline; arterial pressure overshoots, triggering reflex bradycardia via vagal activation; vascular relaxation restores pressure toward baseline.
\end{itemize}

Abnormal responses:
\begin{itemize}
\item diminished baroreceptor reflex (e.g.\ quadriplegia, diabetic autonomic neuropathy): absent tachycardia in phase II and absent overshoot/bradycardia in phase IV
\item congestive cardiac failure: ``square-wave'' response with sustained elevation of blood pressure
\end{itemize}

Clinical use:
\begin{itemize}
\item The manoeuvre can assess autonomic function. The \textbf{Valsalva ratio} is the minimum heart rate (longest R--R interval) in phase IV divided by the maximum heart rate (shortest R--R interval) in phase II; normal \textbf{$>1.5$}.
\end{itemize}

\subsubsection{Physiology of cardiopulmonary exercise testing}
Cardiopulmonary exercise testing (CPET) is described as a non-invasive, dynamic, integrated test of cardiovascular, respiratory, and skeletal muscle function.
\begin{itemize}
\item Resting oxygen consumption in a sitting position is \textbf{3.5 mL/kg/min}.
\end{itemize}

\paragraph{Cardiovascular response to exercise.}
The increase in oxygen uptake during exercise is achieved by:
\begin{itemize}
\item increased \textbf{cardiac output} (via heart rate and stroke volume)
\item redistribution of blood flow toward skeletal muscle
\item increased oxygen extraction, widening the arteriovenous oxygen difference
\item increased pulmonary blood flow (increased cardiac output and pulmonary vasodilation)
\end{itemize}
Maximal aerobic capacity:
\begin{itemize}
\item maximal ability to take in and utilize oxygen is termed maximal oxygen capacity (VO$_2$max) and is expressed via the Fick principle as the product of maximal cardiac output and maximal arteriovenous oxygen difference
\end{itemize}

\paragraph{Ventilatory response to exercise.}
Minute ventilation increases in proportion to work during the initial aerobic phase. The text relates PaCO$_2$ to CO$_2$ production and alveolar ventilation via the alveolar ventilation equation.

\paragraph{Anaerobic threshold.}
Anaerobic metabolism is described as occurring when \textbf{$\sim$47--64\% of VO$_2$max} is reached and aerobic metabolism cannot meet increasing demands.
\begin{itemize}
\item blood lactate rises; buffering generates additional CO$_2$ which stimulates ventilation
\item oxygen consumption at onset of blood lactate accumulation is termed the \textbf{ventilatory anaerobic threshold} (lactate threshold)
\item anaerobic threshold can be estimated from CO$_2$ output and ventilation patterns (including the \textbf{V-slope method})
\end{itemize}

\paragraph{Respiratory exchange ratio.}
\[
\textbf{RER} = \frac{VCO_2}{VO_2}
\]
RER is used to quantify effort:
\begin{itemize}
\item RER $<1$ suggests poor effort or limitation preventing sustained exercise
\item RER $>1.2$ indicates an excellent effort
\end{itemize}

\paragraph{Oxygen pulse.}
Oxygen pulse is calculated as VO$_2$ divided by heart rate and is used as a surrogate for stroke volume.
\begin{itemize}
\item depends on stroke volume and the arteriovenous oxygen difference
\item increases at the beginning of exercise and plateaus at maximal oxygen consumption
\end{itemize}

\paragraph{Ventilatory equivalent for CO$_2$ (VE/VCO$_2$).}
\begin{itemize}
\item VE/VCO$_2$ indicates how many litres of air are breathed to eliminate \textbf{1 L of CO$_2$}
\item increased VE/VCO$_2$ ratios are observed in \textbf{heart disease} and \textbf{pulmonary disease}
\end{itemize}

\paragraph{CPET for perioperative risk stratification.}
CPET can risk-stratify patients for major non-cardiac surgery using variables including anaerobic threshold, maximal oxygen consumption, and VE/VCO$_2$.
\begin{itemize}
\item In general (as stated), anaerobic threshold $<\textbf{11 mL/kg/min}$ and VE/VCO$_2$ $>\textbf{34}$ suggest \textbf{high risk}.
\end{itemize}

\subsubsection{Physiological consequences of a pneumoperitoneum}
During laparoscopic surgery, CO$_2$ is insufflated into the peritoneal cavity at \textbf{$\sim$4 L/min} until intra-abdominal pressure reaches \textbf{$\sim$10--20 mmHg}, followed by a constant flow of \textbf{$\sim$200--400 mL/min} to maintain the pneumoperitoneum.

Physiological effects relate to:
\begin{enumerate}
\item raised intra-abdominal pressure, and
\item CO$_2$ absorption,
\end{enumerate}
in addition to general anaesthesia and positioning.

\paragraph{Effects of increased intra-abdominal pressure.}
Cardiovascular:
\begin{itemize}
\item initial autotransfusion from the splanchnic circulation may transiently increase venous return and cardiac output
\item increasing intra-abdominal pressure compresses the inferior vena cava $\rightarrow$ reduces venous return and cardiac output $\rightarrow$ activates sympathetic drive
\item systemic vascular resistance rises more than the fall in cardiac output, tending to maintain arterial pressure
\item if intra-abdominal pressure exceeds \textbf{$\sim$20 mmHg}, marked fall in cardiac output may produce hypotension
\item arrhythmias (nodal rhythm, sinus bradycardia, asystole) may be initiated by rapid peritoneal stretch
\end{itemize}

Respiratory:
\begin{itemize}
\item cephalad diaphragm displacement further reduces FRC
\item reduced compliance, basal atelectasis, increased V/Q mismatch, hypoxaemia, and hypercarbia may occur
\item pneumothorax may occur if congenital diaphragmatic defects are present
\end{itemize}

Gastrointestinal:
\begin{itemize}
\item raised intra-abdominal pressure predisposes to regurgitation
\item splanchnic blood flow is reduced due to raised intra-abdominal pressure, reduced cardiac output, and catecholamine release
\end{itemize}

Renal:
\begin{itemize}
\item renal function is reduced during pneumoperitoneum
\item raised intra-abdominal pressure and increased sympathetic activity may reduce renal blood flow, GFR, and urine output
\item ADH release (associated with hypotension and reduced cardiac output) contributes to oliguria
\end{itemize}

Neurological:
\begin{itemize}
\item intracranial pressure may increase due to diminished cerebral venous drainage and hyperperfusion induced by hypercarbia
\end{itemize}

\paragraph{Effects of CO$_2$ absorption.}
\begin{itemize}
\item CO$_2$ is readily absorbed $\rightarrow$ rise in PaCO$_2$
\item hypercapnia can cause tachycardia and reduced diastolic filling, both of which may reduce cardiac output
\item hypercapnia increases sympathetic activity
\item venous gas embolism is described as rare but may occur
\end{itemize}

\subsubsection{High-yield takeaways and common traps}
\begin{itemize}
\item In haemorrhage, the primary haemodynamic insult is a fall in \textbf{MSFP} with reduced venous return; sympathetic activation partially restores venous return by reducing venous compliance.
\item Compensation has limits: marked deterioration occurs beyond about \textbf{20\%} blood loss; \textbf{CNS ischaemic response} occurs when arterial pressure falls below \textbf{$\sim$50 mmHg}.
\item Shock classification is mechanistic (hypovolaemic, cardiogenic, obstructive, distributive) but converges on cellular hypoxia, mediator release, and progressive organ dysfunction.
\item The Valsalva manoeuvre is a structured probe of autonomic function; absence of phase IV overshoot/bradycardia suggests impaired baroreflexes.
\item In CPET, anaerobic threshold, VO$_2$-derived indices, and VE/VCO$_2$ integrate cardiovascular and ventilatory reserve; as stated, \textbf{AT $<11$ mL/kg/min} and \textbf{VE/VCO$_2 >34$} suggest higher perioperative risk.
\item Pneumoperitoneum can initially increase venous return but later reduces it via IVC compression; adverse respiratory effects are amplified by anaesthesia and supine positioning.
\end{itemize}

\end{document}
