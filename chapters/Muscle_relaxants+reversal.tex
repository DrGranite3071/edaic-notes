\documentclass[11pt,a4paper]{article}

\usepackage[a4paper,margin=2.2cm]{geometry}
\usepackage[T1]{fontenc}
\usepackage[utf8]{inputenc} % pdfLaTeX-safe
\usepackage{lmodern}
\usepackage{microtype}
\usepackage{hyperref}
\usepackage{booktabs}
\usepackage{tabularx}
\usepackage{array}
\usepackage{enumitem}
\usepackage{amsmath}

\hypersetup{
  colorlinks=true,
  linkcolor=black,
  urlcolor=black
}

\setlist[itemize]{leftmargin=*, itemsep=2pt, topsep=4pt}
\setlength{\parskip}{6pt}
\setlength{\parindent}{0pt}

\newcolumntype{Y}{>{\raggedright\arraybackslash}X}

\title{Chapter 12 --- Muscle Relaxants and Reversal Agents\\
\large Peck \& Harris: \textit{Pharmacology for Anaesthesia and Intensive Care} (5th ed.)}
\date{}

\begin{document}
\maketitle

% -------------------- SUMMARY --------------------
\section*{Summary (High yield)}
\addcontentsline{toc}{section}{Summary (High yield)}
\begin{itemize}
  \item \textbf{Fade} on TOF/tetanus indicates \textbf{non-depolarising} block (or Phase II features), not Phase I depolarising block.
  \item \textbf{Phase I depolarising block}: reduced twitch height with \textbf{no fade} and \textbf{no post-tetanic potentiation}; anticholinesterases \textbf{worsen} it.
  \item \textbf{Suxamethonium}: major exam hazards include \textbf{hyperkalaemia} in receptor upregulation states, \textbf{bradyarrhythmias} (esp.\ repeat dosing/children), transient \textbf{IOP rise}, and \textbf{MH triggering}.
  \item \textbf{NDMRs}: competitive antagonists at post-junctional nAChR; large safety margin (\textbf{$>$70\%} occupancy before monitoring detects block); volatile agents and several antibiotics \textbf{potentiate} block.
  \item \textbf{Reversal}: anticholinesterases increase ACh but cause muscarinic effects (use antimuscarinic cover); \textbf{sugammadex} binds rocuronium/vecuronium and is dosed by block depth.
\end{itemize}

\hrule
\vspace{0.8em}

% -------------------- TOC --------------------
\tableofcontents
\newpage

% =================================================
\section*{Part 1 --- Physiology, Monitoring, Depolarising Muscle Relaxants}
\addcontentsline{toc}{section}{Part 1 --- Physiology, Monitoring, Depolarising Muscle Relaxants}

\subsection*{1) Physiology of neuromuscular transmission (NMJ)}

\subsubsection*{1.1 Anatomy of the NMJ}
\begin{itemize}
  \item The \textbf{motor nerve terminal} lies adjacent to a specialised region of skeletal muscle membrane (the \textbf{motor end-plate}).
  \item The post-junctional membrane forms \textbf{junctional folds}:
  \begin{itemize}
    \item \textbf{Nicotinic ACh receptors (nAChRs)} are concentrated at the \textbf{crests}.
    \item \textbf{Voltage-gated Na$^{+}$} channels are denser in peri-/extra-junctional membrane.
  \end{itemize}
  \item Two end-plate morphologies:
  \begin{itemize}
    \item \textbf{En plaque} (most limb/respiratory skeletal muscle): single, discrete end-plate.
    \item \textbf{En grappe} (some cranial muscles): multiple smaller end-plates.
  \end{itemize}
\end{itemize}

\subsubsection*{1.2 Acetylcholine (ACh): synthesis, storage, release}
\begin{itemize}
  \item \textbf{Synthesis}: choline + acetyl-CoA $\rightarrow$ ACh (via \textbf{choline acetyltransferase}).
  \item \textbf{Choline source}: diet plus \textbf{recycling} following ACh breakdown.
  \item \textbf{Storage}: packaged into synaptic vesicles.
  \item \textbf{Release}:
  \begin{itemize}
    \item Nerve action potential $\rightarrow$ \textbf{Ca$^{2+}$} influx $\rightarrow$ vesicle fusion $\rightarrow$ \textbf{quantal} ACh release into the synaptic cleft.
    \item Quantal release underpins \textbf{mini end-plate potentials}; summation can reach threshold to trigger a muscle action potential.
  \end{itemize}
\end{itemize}

\subsubsection*{1.3 Nicotinic ACh receptor (nAChR)}
\begin{itemize}
  \item \textbf{Pentameric ligand-gated ion channel}.
  \item \textbf{Adult receptor subunits}: $(\alpha1)_2\beta1\delta\varepsilon$.
  \item \textbf{Foetal/extrajunctional form}: $\gamma$ replaces $\varepsilon$ (clinically relevant to suxamethonium hyperkalaemia risk states).
  \item \textbf{Activation requirement}: ACh must bind to \textbf{both $\alpha$ subunits} to open the channel.
  \item Conductance is mainly \textbf{Na$^{+}$} influx (with K$^{+}$ efflux): net \textbf{end-plate depolarisation}.
\end{itemize}

\subsubsection*{1.4 Termination of ACh action}
\begin{itemize}
  \item ACh is rapidly hydrolysed by \textbf{acetylcholinesterase (AChE)} in the synaptic cleft.
  \item \textbf{Choline} is released and recycled into the presynaptic terminal.
\end{itemize}

\subsubsection*{1.5 Where ``fade'' comes from (key concept)}
\begin{itemize}
  \item During repetitive stimulation, \textbf{presynaptic mechanisms} normally support mobilisation of ACh.
  \item \textbf{NDMRs} block \textbf{presynaptic ACh receptors}, reducing ACh mobilisation $\rightarrow$ \textbf{fade} on TOF/tetanus.
  \item \textbf{Phase I depolarising block}: twitch height is reduced \textbf{without} fade.
\end{itemize}

\subsection*{2) Monitoring neuromuscular block (peripheral nerve stimulation)}

\subsubsection*{2.1 Principles and setup}
\begin{itemize}
  \item Monitoring assesses \textbf{muscle contraction} after \textbf{peripheral nerve stimulation}.
  \item \textbf{Supramaximal stimulus}: typically \textbf{60--80 mA}.
  \item \textbf{Pulse duration}: \textbf{0.1 ms}.
  \item \textbf{Electrode placement}:
  \begin{itemize}
    \item \textbf{Negative electrode (cathode)} directly over the nerve.
    \item \textbf{Positive electrode (anode)} positioned so it does \textbf{not} directly stimulate muscle.
  \end{itemize}
\end{itemize}

\subsubsection*{2.2 Common stimulation patterns (what they tell you)}

\textbf{A) Single twitch}
\begin{itemize}
  \item Requires a \textbf{baseline twitch height} for best interpretation.
  \item No reduction in twitch height until $\sim$\textbf{75\%} of NMJ receptors are occupied (large safety margin).
  \item Both partial depolarising and partial non-depolarising block \textbf{reduce} twitch height.
\end{itemize}

\textbf{B) Tetanic stimulation}
\begin{itemize}
  \item Frequencies \textbf{$>$30 Hz} fuse twitches into a sustained contraction (\textbf{tetany}).
  \item Commonly delivered as \textbf{0.1 ms} stimuli at \textbf{50 Hz} (max sensitivity).
  \item \textbf{Partial NDMR block}: tetanus \textbf{fades}.
  \item \textbf{Partial Phase I depolarising block}: tetanus reduced but \textbf{no fade}.
\end{itemize}

\textbf{C) Post-tetanic potentiation and post-tetanic count (PTC)}
\begin{itemize}
  \item After tetanus, subsequent twitches may be \textbf{larger} (post-tetanic potentiation).
  \item \textbf{PTC}: start \textbf{1 Hz} stimuli \textbf{3 seconds} after tetanus.
  \item Most useful when receptor blockade is \textbf{$>$95\%} (TOF/single twitch may be absent).
  \item \textbf{Pitfall}: tetanic effects may persist for \textbf{up to $\sim$6 minutes}.
  \item Phase I depolarising block shows \textbf{no} post-tetanic potentiation.
\end{itemize}

\textbf{D) Train-of-four (TOF)}
\begin{itemize}
  \item \textbf{Four} stimuli, each \textbf{0.1 ms}, delivered at \textbf{2 Hz}.
  \item Measures: \textbf{TOF ratio} (T4:T1) and \textbf{TOF count}.
  \item \textbf{NDMR block}: \textbf{fade} (T4 $<$ T1).
  \item \textbf{Phase I depolarising block}: twitches reduced but \textbf{equal} $\rightarrow$ TOF ratio $\approx$ 1.
  \item Occupancy relationships (approx):
  \begin{itemize}
    \item $>$70\%: T4 begins to decrease.
    \item T4 down by $\sim$25\%: T1 begins to fall ($\sim$75--80\% occupancy).
    \item T4 disappears when T1 is $\sim$25\% of baseline.
    \item At $\sim$95\% occupancy only T1 remains.
  \end{itemize}
\end{itemize}

\textbf{E) Double-burst stimulation (DBS)}
\begin{itemize}
  \item Two bursts separated by \textbf{0.75 s}.
  \item Each burst: \textbf{3} stimuli of \textbf{0.2 ms}, separated by \textbf{20 ms} (50 Hz within burst).
  \item Improves \textbf{manual detection} of small residual block; mechanically not more sensitive than TOF.
\end{itemize}

\subsection*{3) Depolarising muscle relaxants}

\subsubsection*{3.1 Core mechanism (Phase I depolarising block)}
\begin{itemize}
  \item Depolarising agents \textbf{mimic ACh} at nAChR and \textbf{depolarise} the end-plate.
  \item \textbf{Plasma/pseudo-cholinesterase is not present at the NMJ}.
  \item Persistent depolarisation inactivates nearby \textbf{voltage-sensitive Na$^{+}$} channels (within $\sim$1--2 mm) $\rightarrow$ electrically inexcitable zone $\rightarrow$ paralysis.
\end{itemize}

\subsubsection*{3.2 Phase I vs Phase II block (monitoring signatures)}
\begin{center}
\renewcommand{\arraystretch}{1.15}
\begin{tabularx}{\textwidth}{@{}lYY@{}}
\toprule
\textbf{Feature} & \textbf{Partial depolarising (Phase I)} & \textbf{Partial non-depolarising / Phase II} \\
\midrule
Single twitch & Reduced & Reduced \\
TOF ratio (T4:T1) & $>$0.7 (often $\sim$1; no fade) & $<$0.7 (fade) \\
1 Hz stimulation & Sustained & Fade \\
Post-tetanic potentiation & No & Yes \\
Anticholinesterase effect & Block \textbf{augmented} & Block \textbf{antagonised} \\
\bottomrule
\end{tabularx}
\end{center}

\textbf{Exam trap}: anticholinesterases (e.g.\ neostigmine) \textbf{worsen/prolong Phase I} depolarising block.

\subsection*{4) Suxamethonium (succinylcholine)}

\subsubsection*{4.1 Structure, presentation, typical use}
\begin{itemize}
  \item Two ACh molecules joined ``back-to-back'' through their acetyl groups.
  \item Solution \textbf{50 mg\,mL$^{-1}$}; store at \textbf{4$^\circ$C}.
  \item Used for rapid muscle relaxation (e.g.\ RSI).
\end{itemize}

\subsubsection*{4.2 Kinetics}
\begin{itemize}
  \item Rapid hydrolysis by plasma/pseudo-cholinesterase (none at NMJ).
  \item About \textbf{20\%} of IV dose reaches NMJ.
  \item Metabolism: sux $\rightarrow$ choline + succinylmonocholine (weak) $\rightarrow$ succinic acid + choline.
  \item $<$10\% excreted unchanged in urine.
\end{itemize}

\subsubsection*{4.3 Adverse effects (high yield)}
\begin{itemize}
  \item \textbf{Bradyarrhythmias}: muscarinic effects, worse after second dose; atropine may prevent; more pronounced in children.
  \item \textbf{Hyperkalaemia}: major risk in receptor upregulation states (burns, denervation/immobility, neuromuscular disease).
  \item Burns: risk from $\sim$24 h and up to $\sim$18 months.
  \item \textbf{Myalgia}: often reported; precurarisation etc.\ described with limited success.
\end{itemize}

\subsubsection*{4.4 Other clinically important effects and reactions (including prolonged block)}

\paragraph{4.4.1 Intra-ocular pressure (IOP)}
\begin{itemize}
  \item IOP rises by about \textbf{10 mmHg} for a few minutes.
  \item Important in globe perforation/open-globe injury.
  \item Thiopental offsets the rise.
\end{itemize}

\paragraph{4.4.2 Intragastric pressure and reflux risk}
\begin{itemize}
  \item Intragastric pressure rises by about \textbf{10 cmH$_2$O}.
  \item Lower oesophageal sphincter tone increases simultaneously, so reflux risk is not increased.
\end{itemize}

\paragraph{4.4.3 Anaphylaxis}
\begin{itemize}
  \item About twice as likely as non-depolarising NM blockers.
  \item Rate $\sim$1 per 10{,}000.
\end{itemize}

\paragraph{4.4.4 Prolonged block (``suxamethonium apnoea'') --- plasma cholinesterase issues}
\begin{itemize}
  \item Reduced plasma cholinesterase activity (genetic/acquired) $\rightarrow$ prolonged block.
  \item Alleles at locus on chromosome 3: usual, atypical (dibucaine-resistant), silent, fluoride-resistant.
  \item Dibucaine number reflects genotype/function, not enzyme quantity.
  \item Management: sedate/ventilate; FFP can supply cholinesterase (risk--benefit).
\end{itemize}

\subsubsection*{4.5 Malignant hyperthermia (MH) and treatment}
\begin{itemize}
  \item Autosomal dominant; can occur after uneventful anaesthetics.
  \item Trigger agents include suxamethonium and volatile agents.
  \item Mechanism: abnormal RYR1 (chromosome 19) $\rightarrow$ excess SR Ca$^{2+}$ release $\rightarrow$ rigidity, hypermetabolism, heat, CO$_2$, lactate, rhabdomyolysis with myoglobinaemia and hyperkalaemia.
  \item Treatment: stop triggers; ICU; aggressive cooling; IV dantrolene 2.5 mg\,kg$^{-1}$ then 1 mg\,kg$^{-1}$ every 5 min until metabolic signs resolve; little benefit above $\sim$10 mg\,kg$^{-1}$ total.
\end{itemize}

\subsection*{5) MCQ anchors (Part 1)}
\begin{itemize}
  \item TOF = 4 $\times$ 0.1 ms at 2 Hz.
  \item DBS = (3 pulses at 50 Hz) + 0.75 s + (3 pulses at 50 Hz).
  \item Fade = non-depolarising (or Phase II), not Phase I.
  \item Anticholinesterases worsen Phase I depolarising block.
\end{itemize}

\newpage

% =================================================
\section*{Part 2 --- Non-depolarising Muscle Relaxants (NDMRs)}
\addcontentsline{toc}{section}{Part 2 --- Non-depolarising Muscle Relaxants (NDMRs)}

\subsection*{6) NDMRs --- overview}

\subsubsection*{6.1 Core mechanism and margin of safety}
\begin{itemize}
  \item Competitive antagonism at post-junctional nAChR (binding at $\alpha$ subunit via quaternary ammonium group).
  \item Bisquaternary compounds are more potent than monoqaternary compounds.
  \item Safety margin: $>$70\% receptor occupancy required before block is detectable on monitoring.
  \item Pattern resembles Phase II block (fade, post-tetanic potentiation).
\end{itemize}

\subsubsection*{6.2 Classification}
\textbf{By chemical class}
\begin{itemize}
  \item Aminosteroidal: vecuronium, rocuronium, pancuronium
  \item Benzylisoquinolinium: atracurium, mivacurium, tubocurarine
\end{itemize}

\textbf{By duration}
\begin{itemize}
  \item Short: mivacurium
  \item Intermediate: atracurium
  \item Long: pancuronium
\end{itemize}

\subsubsection*{6.3 Potency vs onset (exam favourite)}
\begin{itemize}
  \item Lower potency $\rightarrow$ higher dose $\rightarrow$ larger gradient $\rightarrow$ faster onset (e.g.\ rocuronium).
\end{itemize}

\subsubsection*{6.4 Interactions (selected high yield)}
\textbf{Pharmacological}
\begin{itemize}
  \item Volatile anaesthetics prolong blockade.
  \item Aminoglycosides/polymyxins/tetracyclines can prolong blockade.
  \item Ca$^{2+}$ channel blockers and lithium can prolong blockade.
\end{itemize}

\textbf{Physiological}
\begin{itemize}
  \item Hypothermia prolongs block.
  \item Acute hypokalaemia potentiates; hyperkalaemia tends to antagonise.
  \item Hypermagnesaemia prolongs block (e.g.\ magnesium therapy in pre-eclampsia).
\end{itemize}

\subsubsection*{6.5 Snapshot properties (Table 12.4 style)}
\begin{center}
\renewcommand{\arraystretch}{1.15}
\begin{tabularx}{\textwidth}{@{}l c c c Y c@{}}
\toprule
\textbf{Drug} & \textbf{Dose} & \textbf{Onset} & \textbf{Duration} & \textbf{CVS effects} & \textbf{Histamine} \\
 & \textbf{(mg\,kg$^{-1}$)} &  &  &  & \textbf{release} \\
\midrule
Vecuronium & 0.1 & medium & medium & none / bradycardia & rare \\
Rocuronium & 0.6 & rapid & medium & none & rare \\
Pancuronium & 0.1 & medium & long & tachycardia & rare \\
Atracurium & 0.5 & medium & medium & none & slight \\
Cisatracurium & 0.2 & medium & medium & none & rare \\
Mivacurium & 0.2 & medium & short & none & slight \\
\bottomrule
\end{tabularx}
\end{center}

\subsection*{7) Individual NDMRs (Peck \& Harris focus)}
% (kept concise here; include your full existing subsections if you want verbatim expansion)
\begin{itemize}
  \item Vecuronium: minimal CVS effects; reconstitution required; intermediate duration.
  \item Rocuronium: low potency, rapid onset; biliary elimination; reversible with sugammadex.
  \item Pancuronium: long duration; tachycardia (vagolysis).
  \item Atracurium: Hofmann + ester hydrolysis; histamine with rapid bolus; slowed by acidosis/hypothermia.
  \item Cisatracurium: potent atracurium isomer; minimal histamine; Hofmann-based elimination.
  \item Mivacurium: short duration; metabolised by plasma cholinesterase; neostigmine can prolong.
\end{itemize}

\subsection*{8) MCQ anchors (Part 2)}
\begin{itemize}
  \item Competitive antagonists at post-junctional $\alpha$ subunit.
  \item $>$70\% occupancy before detectable block (margin of safety).
  \item Atracurium/cisatracurium: Hofmann elimination slowed by acidosis/hypothermia.
  \item Neostigmine can prolong mivacurium (plasma cholinesterase inhibition).
\end{itemize}

\newpage

% =================================================
\section*{Part 3 --- Reversal Agents}
\addcontentsline{toc}{section}{Part 3 --- Reversal Agents}

\subsection*{9) Principles}
\begin{itemize}
  \item Reverse NDMR block by increasing ACh (anticholinesterases) and/or removing relaxant (sugammadex for aminosteroids).
  \item Anticholinesterases have a ceiling effect; unreliable in very deep block.
  \item Co-administer antimuscarinic (glycopyrrolate/atropine) to reduce muscarinic adverse effects.
\end{itemize}

\subsection*{10) Anticholinesterases}
\begin{itemize}
  \item Edrophonium: reversible binding; rapid onset; does not cross BBB.
  \item Neostigmine: carbamylates AChE; inhibits plasma cholinesterase (can prolong suxamethonium/mivacurium); excess can cause paradoxical weakness.
  \item Pyridostigmine: longer duration; more renal elimination; no BBB penetration.
  \item Physostigmine: tertiary amine; crosses BBB; anticholinergic toxicity context.
\end{itemize}

\subsection*{11) Sugammadex}
\begin{itemize}
  \item $\gamma$-cyclodextrin binding agent for rocuronium/vecuronium.
  \item Doses by depth: 2 mg\,kg$^{-1}$ (T2 present), 4 mg\,kg$^{-1}$ (PTC activity), 16 mg\,kg$^{-1}$ (emergency after RSI-dose rocuronium).
  \item Renal excretion unchanged; contraceptive advice: extra contraception for 7 days.
\end{itemize}

\subsection*{12) MCQ anchors (Part 3)}
\begin{itemize}
  \item Anticholinesterases reverse NDMR block but cause muscarinic effects $\rightarrow$ antimuscarinic cover.
  \item Sugammadex reverses aminosteroids (rocuronium/vecuronium) and is dosed by depth.
\end{itemize}

\newpage

% =================================================
\section*{Appendix --- Additions from other references (Non-Peck)}
\addcontentsline{toc}{section}{Appendix --- Additions from other references (Non-Peck)}

\subsection*{A1) Practical monitoring sites and muscle-group differences (Non-Peck)}
\textbf{Monitoring sites}
\begin{itemize}
  \item Ulnar nerve $\rightarrow$ adductor pollicis. \textit{(Source: Morgan \& Mikhail, 5th ed.)}
  \item Facial nerve $\rightarrow$ orbicularis oculi. \textit{(Source: Morgan \& Mikhail, 5th ed.)}
\end{itemize}

\textbf{Muscle recovery differences}
\begin{itemize}
  \item Diaphragm recovers earlier than adductor pollicis. \textit{(Source: Morgan \& Mikhail, 5th ed.)}
  \item Rectus abdominis recovers earlier than adductor pollicis. \textit{(Source: Morgan \& Mikhail, 5th ed.)}
  \item Laryngeal adductors recover earlier than adductor pollicis. \textit{(Source: Morgan \& Mikhail, 5th ed.)}
  \item Orbicularis oculi recovers earlier than adductor pollicis. \textit{(Source: Morgan \& Mikhail, 5th ed.)}
  \item Thumb monitoring can overestimate block at diaphragm/upper airway muscles during recovery. \textit{(Source: Morgan \& Mikhail, 5th ed.)}
\end{itemize}

\textbf{Misleading sites}
\begin{itemize}
  \item Atrophied muscle may appear relatively resistant due to receptor proliferation. \textit{(Source: Morgan \& Mikhail, 5th ed.)}
  \item Hemiplegic/denervated limb may appear relatively resistant due to receptor proliferation. \textit{(Source: Morgan \& Mikhail, 5th ed.)}
  \item Monitoring such a limb can lead to overdosing. \textit{(Source: Morgan \& Mikhail, 5th ed.)}
\end{itemize}

\subsection*{A2) Quantitative neuromuscular monitoring methods (Non-Peck)}
\begin{itemize}
  \item Tactile/visual assessment is inaccurate for small residual blocks. \textit{(Source: Morgan \& Mikhail, 5th ed.)}
  \item MMG measures force against a preload. \textit{(Source: Fundamentals of Anaesthesia, 4th ed.)}
  \item MMG is accurate but awkward to set up/maintain. \textit{(Source: Fundamentals of Anaesthesia, 4th ed.)}
  \item AMG measures acceleration of a moving digit (thumb) using a sensor. \textit{(Source: Fundamentals of Anaesthesia, 4th ed.)}
  \item AMG is sensitive to joint position. \textit{(Source: Fundamentals of Anaesthesia, 4th ed.)}
  \item EMG records evoked CMAPs. \textit{(Source: Fundamentals of Anaesthesia, 4th ed.)}
  \item EMG depends on electrode geometry/position. \textit{(Source: Fundamentals of Anaesthesia, 4th ed.)}
\end{itemize}

\subsection*{A3) Reversal/extubation thresholds emphasised elsewhere (Non-Peck)}
\begin{itemize}
  \item Suggested adequacy threshold: TOFR $>$ 0.90. \textit{(Source: Cross \& Plunkett, 2008)}
  \item Suggested: $\geq$3 twitches before anticholinesterase reversal. \textit{(Source: Cross \& Plunkett, 2008)}
  \item Head lift $\geq$5 s as clinical indicator. \textit{(Source: Morgan \& Mikhail, 5th ed.)}
  \item Inspiratory pressure at least $-25$ cmH$_2$O as indicator. \textit{(Source: Morgan \& Mikhail, 5th ed.)}
  \item Forceful hand grip as indicator. \textit{(Source: Morgan \& Mikhail, 5th ed.)}
  \item Twitch tension falls with local hypothermia ($\sim$6\% per $^\circ$C). \textit{(Source: Morgan \& Mikhail, 5th ed.)}
\end{itemize}

\subsection*{A4) Additional suxamethonium points in other texts (Non-Peck)}
\begin{itemize}
  \item Fasciculations commonly described. \textit{(Source: Primary FRCA in a Box, 2nd ed.)}
  \item Repeated dosing/infusion may progress to Phase II features. \textit{(Source: Primary FRCA in a Box, 2nd ed.)}
  \item Some sources describe increased ICP with suxamethonium. \textit{(Source: Primary FRCA in a Box, 2nd ed.)}
  \item One text describes increased intragastric pressure with decreased LOS pressure. \textit{(Source: Primary FRCA in a Box, 2nd ed.)}
  \item This conflicts with the Peck-based LOS statement earlier. \textit{(Source: Primary FRCA in a Box, 2nd ed.; Peck \& Harris 5th ed., Ch 12)}
  \item No evidence of worsened outcomes in open-globe injuries despite IOP rise. \textit{(Source: Morgan \& Mikhail, 5th ed.)}
  \item NDMR pretreatment does not reliably prevent the IOP rise. \textit{(Source: Morgan \& Mikhail, 5th ed.)}
  \item Transient masseter tone increase can occur. \textit{(Source: Morgan \& Mikhail, 5th ed.)}
  \item Marked masseter rigidity may warn of MH. \textit{(Source: Morgan \& Mikhail, 5th ed.)}
\end{itemize}

\subsection*{A5) Receptor upregulation timing after denervation/burns (Non-Peck)}
\begin{itemize}
  \item Receptor upregulation takes time after denervation-type injury. \textit{(Source: Fundamentals of Anaesthesia, 4th ed.)}
  \item Suggested avoidance after 48--72 hours in these settings. \textit{(Source: Fundamentals of Anaesthesia, 4th ed.)}
  \item Differs from Peck-based burn timing (risk from $\sim$24 h). \textit{(Source: Fundamentals of Anaesthesia, 4th ed.; Peck \& Harris 5th ed., Ch 12)}
\end{itemize}

\subsection*{A6) Additional sugammadex points from another revision text (Non-Peck)}
\begin{itemize}
  \item Quoted reversal times by depth. \textit{(Source: Primary FRCA in a Box, 2nd ed.)}
  \item 2 mg\,kg$^{-1}$ at T2: $\sim$2 min. \textit{(Source: Primary FRCA in a Box, 2nd ed.)}
  \item 4 mg\,kg$^{-1}$ deep block (1--2 PTC): $\sim$3 min. \textit{(Source: Primary FRCA in a Box, 2nd ed.)}
  \item 16 mg\,kg$^{-1}$ rescue: $\sim$1.5 min. \textit{(Source: Primary FRCA in a Box, 2nd ed.)}
  \item Suggested 24 h delay before re-using rocuronium/vecuronium after rescue reversal. \textit{(Source: Primary FRCA in a Box, 2nd ed.)}
  \item Bitter/metallic taste listed. \textit{(Source: Primary FRCA in a Box, 2nd ed.)}
  \item Bronchospasm listed. \textit{(Source: Primary FRCA in a Box, 2nd ed.)}
\end{itemize}

\end{document}
