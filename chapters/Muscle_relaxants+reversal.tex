\documentclass[11pt,a4paper]{article}

% preamble.tex
\usepackage[T1]{fontenc}
\usepackage[utf8]{inputenc}
\usepackage[english]{babel}
\usepackage[a4paper,margin=2cm]{geometry}
\usepackage{lmodern}
\usepackage{amsmath,amssymb}
\usepackage{graphicx}
\usepackage{booktabs}
\usepackage{hyperref}

\usepackage[T1]{fontenc}
\usepackage{lmodern}
\usepackage{microtype}
\usepackage{hyperref}
\usepackage{booktabs}
\usepackage{tabularx}
\usepackage{array}
\usepackage{enumitem}
\usepackage{amsmath}

\setlist[itemize]{leftmargin=*, itemsep=2pt, topsep=4pt}
\setlength{\parskip}{6pt}
\setlength{\parindent}{0pt}

\newcolumntype{Y}{>{\raggedright\arraybackslash}X}

\title{Chapter 12 --- Muscle Relaxants and Reversal Agents\\
\large Peck \& Harris: \textit{Pharmacology for Anaesthesia and Intensive Care} (5th ed.)}
\date{}

\begin{document}
\maketitle

\hrule
\vspace{0.8em}

\section*{Part 1 --- Physiology, Monitoring, Depolarising Muscle Relaxants}

\subsection*{1) Physiology of neuromuscular transmission (NMJ)}

\subsubsection*{1.1 Anatomy of the NMJ}
\begin{itemize}
  \item The \textbf{motor nerve terminal} lies adjacent to a specialised region of skeletal muscle membrane (the \textbf{motor end-plate}).
  \item The post-junctional membrane forms \textbf{junctional folds}:
  \begin{itemize}
    \item \textbf{Nicotinic ACh receptors (nAChRs)} are concentrated at the \textbf{crests}.
    \item \textbf{Voltage-gated Na$^{+}$} channels are denser in peri-/extra-junctional membrane.
  \end{itemize}
  \item Two end-plate morphologies:
  \begin{itemize}
    \item \textbf{En plaque} (most limb/respiratory skeletal muscle): single, discrete end-plate.
    \item \textbf{En grappe} (some cranial muscles): multiple smaller end-plates.
  \end{itemize}
\end{itemize}

\subsubsection*{1.2 Acetylcholine (ACh): synthesis, storage, release}
\begin{itemize}
  \item \textbf{Synthesis}: choline + acetyl-CoA $\rightarrow$ ACh (via \textbf{choline acetyltransferase}).
  \item \textbf{Choline source}: diet plus \textbf{recycling} following ACh breakdown.
  \item \textbf{Storage}: packaged into synaptic vesicles.
  \item \textbf{Release}:
  \begin{itemize}
    \item Nerve action potential $\rightarrow$ \textbf{Ca$^{2+}$} influx $\rightarrow$ vesicle fusion $\rightarrow$ \textbf{quantal} ACh release into the synaptic cleft.
    \item Quantal release underpins \textbf{mini end-plate potentials}; summation can reach threshold to trigger a muscle action potential.
  \end{itemize}
\end{itemize}

\subsubsection*{1.3 Nicotinic ACh receptor (nAChR)}
\begin{itemize}
  \item \textbf{Pentameric ligand-gated ion channel}.
  \item \textbf{Adult receptor subunits}: $(\alpha1)_2\beta1\delta\varepsilon$.
  \item \textbf{Foetal/extrajunctional form}: $\gamma$ replaces $\varepsilon$ (clinically relevant to suxamethonium hyperkalaemia risk states).
  \item \textbf{Activation requirement}: ACh must bind to \textbf{both $\alpha$ subunits} to open the channel.
  \item Conductance is mainly \textbf{Na$^{+}$} influx (with K$^{+}$ efflux): net \textbf{end-plate depolarisation}.
\end{itemize}

\subsubsection*{1.4 Termination of ACh action}
\begin{itemize}
  \item ACh is rapidly hydrolysed by \textbf{acetylcholinesterase (AChE)} in the synaptic cleft.
  \item \textbf{Choline} is released and recycled into the presynaptic terminal.
\end{itemize}

\subsubsection*{1.5 Where ``fade'' comes from (key concept)}
\begin{itemize}
  \item During repetitive stimulation, \textbf{presynaptic mechanisms} normally support mobilisation of ACh.
  \item \textbf{Non-depolarising muscle relaxants (NDMRs)} block \textbf{presynaptic ACh receptors}, reducing ACh mobilisation $\rightarrow$ \textbf{fade} on TOF/tetanus.
  \item \textbf{Phase I depolarising block}: twitch height is reduced \textbf{without} fade.
\end{itemize}

\hrule
\vspace{0.8em}

\subsection*{2) Monitoring neuromuscular block (peripheral nerve stimulation)}

\subsubsection*{2.1 Principles and setup}
\begin{itemize}
  \item Monitoring assesses \textbf{muscle contraction} after \textbf{peripheral nerve stimulation}.
  \item \textbf{Supramaximal stimulus} is required to depolarise all composite fibres: typically \textbf{60--80 mA}.
  \item \textbf{Pulse duration}: \textbf{0.1 ms}.
  \item \textbf{Electrode placement}:
  \begin{itemize}
    \item \textbf{Negative electrode (cathode)} directly over the nerve.
    \item \textbf{Positive electrode (anode)} positioned so it does \textbf{not} directly stimulate muscle.
  \end{itemize}
\end{itemize}

\subsubsection*{2.2 Common stimulation patterns (what they tell you)}
\textbf{A) Single twitch}
\begin{itemize}
  \item Requires a \textbf{baseline twitch height} for best interpretation.
  \item No reduction in twitch height until $\sim$\textbf{75\%} of NMJ receptors are occupied (large ``margin of safety'').
  \item Both partial depolarising and partial non-depolarising block \textbf{reduce} twitch height.
\end{itemize}

\textbf{B) Tetanic stimulation}
\begin{itemize}
  \item Frequencies \textbf{$>$30 Hz} fuse twitches into a sustained contraction (\textbf{tetany}).
  \item Commonly delivered as \textbf{0.1 ms} stimuli at \textbf{50 Hz} (max sensitivity).
  \item \textbf{Partial NDMR block}: tetanus \textbf{fades} with time (presynaptic effect).
  \item \textbf{Partial depolarising block}: tetanus is reduced but shows \textbf{no fade}.
\end{itemize}

\textbf{C) Post-tetanic potentiation and post-tetanic count (PTC)}
\begin{itemize}
  \item After tetanus, subsequent twitches may be \textbf{larger} (post-tetanic potentiation).
  \item \textbf{PTC}: start \textbf{1 Hz} stimuli \textbf{3 seconds} after tetanus.
  \item PTC is inversely related to depth of block; most useful when receptor blockade is \textbf{$>$95\%} (TOF/single twitch may be absent).
  \item \textbf{Pitfall}: tetanic effects may persist for \textbf{up to $\sim$6 minutes}, creating a \textbf{false impression} of inadequate block on subsequent TOF/single twitch.
  \item Partial depolarising (Phase I) block shows \textbf{no} post-tetanic potentiation.
\end{itemize}

\textbf{D) Train-of-four (TOF)}
\begin{itemize}
  \item \textbf{Four} stimuli, each \textbf{0.1 ms}, delivered at \textbf{2 Hz}.
  \item Measures:
  \begin{itemize}
    \item \textbf{TOF ratio} = T4:T1
    \item \textbf{TOF count} = number of visible/palpable twitches
  \end{itemize}
  \item \textbf{NDMR block}: T4 $<$ T1 (\textbf{fade}).
  \item \textbf{Depolarising Phase I block}: twitch heights reduced but \textbf{equal} $\rightarrow$ TOF ratio $\approx$ \textbf{1}.
  \item Approximate receptor-occupancy relationships:
  \begin{itemize}
    \item Occupancy \textbf{$>$70\%}: \textbf{T4} begins to decrease.
    \item When T4 has fallen by $\sim$25\%, \textbf{T1} begins to fall ($\approx$ \textbf{75--80\%} occupancy).
    \item \textbf{T4 disappears} when \textbf{T1} is $\sim$25\% of baseline.
    \item \textbf{TOF count} becomes useful for deeper block: at $\sim$\textbf{95\%} occupancy only \textbf{T1} remains.
  \end{itemize}
  \item Practical note: TOF ratio is \textbf{hard to judge reliably} by eye.
\end{itemize}

\textbf{E) Double-burst stimulation (DBS)}
\begin{itemize}
  \item Two bursts separated by \textbf{0.75 s}.
  \item Each burst: \textbf{3} stimuli of \textbf{0.2 ms}, separated by \textbf{20 ms} (i.e. \textbf{50 Hz} within each burst).
  \item Designed to improve \textbf{manual detection} of small residual block.
  \item If both bursts feel/appear equal, clinically significant residual block is unlikely; mechanically it is \textbf{no more sensitive} than TOF.
\end{itemize}

\hrule
\vspace{0.8em}

\subsection*{3) Depolarising muscle relaxants}

\subsubsection*{3.1 Core mechanism (Phase I depolarising block)}
\begin{itemize}
  \item Depolarising muscle relaxants \textbf{mimic ACh} at the post-junctional nAChR and \textbf{depolarise} the end-plate.
  \item \textbf{Plasma/pseudo-cholinesterase is not present at the NMJ}, so the depolarising effect outlasts ACh.
  \item Persistent depolarisation inactivates nearby \textbf{voltage-sensitive Na$^{+}$} channels (within $\sim$1--2 mm), creating an electrically inexcitable zone $\rightarrow$ \textbf{no further action potentials} $\rightarrow$ relaxation.
  \item As plasma concentration falls, diffusion away from the NMJ plus hydrolysis in plasma reverses the concentration gradient and recovery occurs.
\end{itemize}

\subsubsection*{3.2 Phase I vs Phase II block (monitoring signatures)}
\begin{center}
\renewcommand{\arraystretch}{1.15}
\begin{tabularx}{\textwidth}{@{}lYY@{}}
\toprule
\textbf{Feature} & \textbf{Partial depolarising (Phase I)} & \textbf{Partial non-depolarising / Phase II} \\
\midrule
Single twitch & Reduced & Reduced \\
TOF ratio (T4:T1) & $>$0.7 (often $\sim$1; no fade) & $<$0.7 (fade) \\
1 Hz stimulation & Sustained & Fade \\
Post-tetanic potentiation & No & Yes \\
Anticholinesterase effect & Block \textbf{augmented} & Block \textbf{antagonised} \\
\bottomrule
\end{tabularx}
\end{center}

\textbf{Exam trap}: anticholinesterases (e.g. neostigmine) \textbf{worsen/prolong Phase I} depolarising block.

\hrule
\vspace{0.8em}

\subsection*{4) Suxamethonium (succinylcholine)}

\subsubsection*{4.1 Structure, presentation, typical use}
\begin{itemize}
  \item Structurally: \textbf{two ACh molecules joined ``back-to-back''} through their acetyl groups.
  \item \textbf{Presentation}: colourless solution \textbf{50 mg\,mL$^{-1}$}; store at \textbf{4$^\circ$C}.
  \item Used for \textbf{rapid muscle relaxation} (e.g. rapid sequence induction); has also been used by infusion for short procedures.
\end{itemize}

\subsubsection*{4.2 Kinetics}
\begin{itemize}
  \item Rapid hydrolysis by \textbf{plasma/pseudo-cholinesterase} (liver and plasma enzyme; none at the NMJ).
  \item Only about \textbf{20\%} of an IV dose reaches the NMJ.
  \item Metabolism: suxamethonium $\rightarrow$ \textbf{choline + succinylmonocholine} (weakly active) $\rightarrow$ \textbf{succinic acid + choline}.
  \item Because metabolism is rapid, \textbf{$<$10\%} is excreted unchanged in urine.
\end{itemize}

\subsubsection*{4.3 Adverse effects (high yield)}
\begin{itemize}
  \item \textbf{Arrhythmias}: sinus or nodal \textbf{bradycardia} and ventricular arrhythmias via muscarinic receptor stimulation in the sinus node.
  \begin{itemize}
    \item Bradycardia is often worse after a \textbf{second dose}; may be prevented with \textbf{atropine}; more pronounced in \textbf{children}.
  \end{itemize}
  \item \textbf{Hyperkalaemia}:
  \begin{itemize}
    \item Normal subjects: small expected rise (K$^{+}$ efflux during depolarisation).
    \item High-risk states: \textbf{burns $>$10\%}, neuromuscular disorders, denervation/immobility states $\rightarrow$ large K$^{+}$ release $\rightarrow$ arrhythmias/cardiac arrest risk.
    \item Burns: risk from about \textbf{24 hours} post-injury and up to \textbf{18 months}.
    \item Mechanism: proliferation of \textbf{extrajunctional (foetal $\gamma$-subunit) receptors} across the muscle surface.
    \item \textbf{Renal failure}: not inherently higher risk of the sudden hyperkalaemic response \emph{per se}, but baseline K$^{+}$ may already be deranged.
  \end{itemize}
  \item \textbf{Myalgia}: muscle pains (reported commonest in young females mobilising early post-op).
  \begin{itemize}
    \item Strategies described with limited success: small ``precurarising'' dose of an NDMR (e.g. vecuronium), diazepam, or dantrolene.
  \end{itemize}
\end{itemize}

\subsubsection*{4.4 Other clinically important effects and reactions (including prolonged block)}

\paragraph{4.4.1 Intra-ocular pressure (IOP)}
\begin{itemize}
  \item IOP rises by about \textbf{10 mmHg} for \textbf{a few minutes} following suxamethonium.
  \item Clinically important in \textbf{globe perforation / open-globe injury}.
  \item Concurrent \textbf{thiopental} offsets this rise (IOP may remain static or fall).
  \item Proposed mechanisms include contraction of tonic myofibrils and transient dilation of choroidal blood vessels.
\end{itemize}

\paragraph{4.4.2 Intragastric pressure and reflux risk}
\begin{itemize}
  \item Intragastric pressure rises by about \textbf{10 cmH$_2$O}.
  \item \textbf{Lower oesophageal sphincter tone increases simultaneously}, so there is \textbf{no increased reflux risk}.
\end{itemize}

\paragraph{4.4.3 Anaphylaxis}
\begin{itemize}
  \item Suxamethonium is about \textbf{twice as likely} to cause anaphylaxis as non-depolarising neuromuscular blockers.
  \item Reported rate approximately \textbf{1 per 10,000} administrations.
\end{itemize}

\paragraph{4.4.4 Prolonged block (``suxamethonium apnoea'') --- plasma cholinesterase issues}
\begin{itemize}
  \item Reduced plasma cholinesterase activity (genetic or acquired) $\rightarrow$ prolonged neuromuscular block.
  \item Genetics:
  \begin{itemize}
    \item Four alleles at a locus on \textbf{chromosome 3}: \textbf{usual (normal)}, \textbf{atypical (dibucaine-resistant)}, \textbf{silent (absent)}, \textbf{fluoride-resistant}.
    \item Most people ($\sim$\textbf{96\%}) are \textbf{Eu:Eu} and metabolise suxamethonium rapidly.
    \item Heterozygotes may have mildly prolonged block (up to $\sim$10 min); rare genotypes can produce block lasting \textbf{hours}.
  \end{itemize}
  \item Testing:
  \begin{itemize}
    \item \textbf{Dibucaine number} = \% inhibition of plasma cholinesterase by dibucaine under standard conditions.
    \item Reflects \textbf{genotype/function}, not the \textbf{quantity} of enzyme.
  \end{itemize}
  \item Management options described:
  \begin{itemize}
    \item \textbf{Sedate and ventilate} until recovery.
    \item \textbf{Fresh frozen plasma} can supply plasma cholinesterase and may reverse prolonged block (balance against transfusion risks).
  \end{itemize}
\end{itemize}

\subsubsection*{4.5 Malignant hyperthermia (MH) and treatment}

\paragraph{4.5.1 Key facts}
\begin{itemize}
  \item MH is a rare \textbf{autosomal dominant} condition; it may occur after \textbf{many previous uneventful anaesthetics}.
  \item Trigger agents include \textbf{suxamethonium} (and volatile anaesthetics).
\end{itemize}

\paragraph{4.5.2 Mechanism}
\begin{itemize}
  \item Abnormal \textbf{ryanodine receptor} function (RYR1 in skeletal muscle, encoded on \textbf{chromosome 19}).
  \item Excess Ca$^{2+}$ release from sarcoplasmic reticulum $\rightarrow$ \textbf{generalised muscle rigidity}, very high ATP consumption, and heat production.
  \item Consequences: increased CO$_2$ and lactate; subsequent cell breakdown $\rightarrow$ \textbf{myoglobinaemia} and \textbf{hyperkalaemia}.
\end{itemize}

\paragraph{4.5.3 Treatment (core actions)}
\begin{itemize}
  \item \textbf{Stop triggers} and manage on \textbf{ICU}; recurrence can occur if treatment is stopped too early.
  \item \textbf{Dantrolene (IV)}:
  \begin{itemize}
    \item Start \textbf{2.5 mg\,kg$^{-1}$}, then \textbf{1 mg\,kg$^{-1}$} every \textbf{5 minutes} until metabolic signs resolve.
    \item Little benefit above a cumulative dose of about \textbf{10 mg\,kg$^{-1}$}.
  \end{itemize}
  \item \textbf{Aggressive cooling} (e.g. ice-cold saline lavage of bladder and peritoneum if open).
  \item Correct biochemical/haematological abnormalities; continue treatment until symptoms have completely resolved.
\end{itemize}

\subsubsection*{5) MCQ anchors (Part 1)}
\begin{itemize}
  \item TOF = \textbf{4 $\times$ 0.1 ms} at \textbf{2 Hz}.
  \item DBS = \textbf{(3 pulses at 50 Hz) + 0.75 s + (3 pulses at 50 Hz)}.
  \item \textbf{Fade = non-depolarising (or Phase II), not Phase I}.
  \item \textbf{Anticholinesterase worsens Phase I depolarising block}.
  \item Burns: sux hyperkalaemia risk starts at $\sim$\textbf{24 h} and can persist up to \textbf{18 months}.
\end{itemize}

\newpage
\section*{Part 2 --- Non-depolarising Muscle Relaxants (NDMRs)}

\subsection*{6) NDMRs --- overview}

\subsubsection*{6.1 Core mechanism and ``margin of safety''}
\begin{itemize}
  \item NDMRs \textbf{competitively inhibit ACh} at the NMJ by binding to the \textbf{$\alpha$ subunit} of the post-junctional \textbf{nicotinic ACh receptor}, via their \textbf{quaternary ammonium} group.
  \item \textbf{Bisquaternary} compounds (two quaternary ammonium groups) are \textbf{more potent} than \textbf{monoqaternary} compounds.
  \item The NMJ has a wide \textbf{safety margin}: \textbf{$>$70\%} receptor occupancy is required before block is detectable with peripheral nerve stimulation.
  \item The pattern of NDMR block resembles \textbf{Phase II} depolarising block (fade, post-tetanic potentiation, etc.).
\end{itemize}

\subsubsection*{6.2 Classification (as described in Peck \& Harris)}
\textbf{By chemical class}
\begin{itemize}
  \item \textbf{Aminosteroidal}: vecuronium, rocuronium, pancuronium
  \item \textbf{Benzylisoquinolinium}: atracurium, mivacurium, tubocurarine
\end{itemize}

\textbf{By duration of action}
\begin{itemize}
  \item \textbf{Short}: mivacurium
  \item \textbf{Intermediate}: atracurium
  \item \textbf{Long}: pancuronium
\end{itemize}

\subsubsection*{6.3 Physicochemical / kinetic theme}
\begin{itemize}
  \item NDMRs are relatively \textbf{polar}, cross lipid membranes poorly $\rightarrow$ \textbf{small volume of distribution}.
  \item Some are hydrolysed in plasma (atracurium, mivacurium), while others undergo hepatic metabolism to varying degrees (e.g. pancuronium, vecuronium), with unmetabolised drug excreted via \textbf{urine} and/or \textbf{bile}.
\end{itemize}

\subsubsection*{6.4 Potency vs onset (exam favourite)}
\begin{itemize}
  \item \textbf{Lower potency} relaxants must be given in \textbf{higher doses}.
  \item Higher dose $\rightarrow$ larger concentration gradient from plasma $\rightarrow$ NMJ $\rightarrow$ faster diffusion to the NMJ $\rightarrow$ \textbf{more rapid onset} (classic example: rocuronium).
\end{itemize}

\subsubsection*{6.5 Important interactions (selected high yield)}
\textbf{Pharmacological}
\begin{itemize}
  \item \textbf{Volatile anaesthetics}: prolong blockade.
  \item \textbf{Aminoglycosides / polymyxins / tetracyclines} (notably large intraperitoneal doses): prolong blockade.
  \item \textbf{Ca$^{2+}$ channel antagonists}: prolong blockade.
  \item \textbf{Lithium}: prolong blockade.
  \item \textbf{Local anaesthetics}: variable; low doses may enhance.
  \item \textbf{Diuretics}: variable.
\end{itemize}

\textbf{Physiological}
\begin{itemize}
  \item \textbf{Hypothermia}: prolongs blockade.
  \item \textbf{Acidosis}: variable; usually prolongs.
  \item \textbf{K$^{+}$}: acute \textbf{hypokalaemia} potentiates NDMRs; \textbf{hyperkalaemia} tends to antagonise.
  \item \textbf{Hypermagnesaemia}: prolongs blockade; can cause apnoea at supranormal levels (e.g. pre-eclampsia).
\end{itemize}

\subsubsection*{6.6 Snapshot properties (Table 12.4 style)}
\begin{center}
\renewcommand{\arraystretch}{1.15}
\begin{tabularx}{\textwidth}{@{}l c c c Y c@{}}
\toprule
\textbf{Drug} & \textbf{Intubating dose} & \textbf{Onset} & \textbf{Duration} & \textbf{CVS effects} & \textbf{Histamine} \\
 & \textbf{(mg\,kg$^{-1}$)} &  &  &  & \textbf{release} \\
\midrule
Vecuronium & 0.1 & medium & medium & none / bradycardia & rare \\
Rocuronium & 0.6 & rapid & medium & none & rare \\
Pancuronium & 0.1 & medium & long & tachycardia & rare \\
Atracurium & 0.5 & medium & medium & none & slight \\
Cisatracurium & 0.2 & medium & medium & none & rare \\
Mivacurium & 0.2 & medium & short & none & slight \\
\bottomrule
\end{tabularx}
\end{center}

\hrule
\vspace{0.8em}

\subsection*{7) Individual NDMRs (Peck \& Harris focus)}

\subsubsection*{7.1 Vecuronium}
\begin{itemize}
  \item \textbf{Concept}: ``clean'' profile (minimal CVS effects; does not precipitate histamine release).
  \item \textbf{Structure}: differs from pancuronium by one methyl group $\rightarrow$ \textbf{monoqaternary} analogue.
\end{itemize}

\textbf{Presentation and use}
\begin{itemize}
  \item \textbf{10 mg} freeze-dried powder (with mannitol + NaOH); dissolve in \textbf{5 mL} water before administration.
  \item \textbf{0.1 mg\,kg$^{-1}$}: intubating conditions in \textbf{$\sim$90--120 s}; \textbf{intermediate} duration.
\end{itemize}

\textbf{Other effects}
\begin{itemize}
  \item CVS: no direct cardiac effects, but may leave fentanyl/propofol bradycardia ``unchecked'' (vs pancuronium).
  \item \textbf{Critical illness myopathy}: described association (often with corticosteroids and/or relaxants) with potentially prolonged recovery.
\end{itemize}

\textbf{Kinetics (summary; Table 12.5 values)}
\begin{itemize}
  \item Hepatic de-acetylation to \textbf{3- and 17-hydroxy} metabolites; \textbf{3-hydroxy} has NM-blocking activity but usually limited clinical impact with normal renal function.
  \item Approximate kinetics: protein bound \textbf{$\sim$10\%}; $V_d$ \textbf{$\sim$0.23 L\,kg$^{-1}$}; metabolised \textbf{$\sim$20\%}; elimination \textbf{bile $\sim$70\% / urine $\sim$30\%}.
\end{itemize}

\subsubsection*{7.2 Rocuronium}
\begin{itemize}
  \item Developed from vecuronium; key advantage is \textbf{rapid onset} due to \textbf{low potency}.
\end{itemize}

\textbf{Presentation and use}
\begin{itemize}
  \item Colourless solution \textbf{50 mg in 5 mL}.
  \item \textbf{0.6 mg\,kg$^{-1}$}: intubating conditions \textbf{$\sim$100--120 s}.
  \item Higher dose \textbf{0.9--1.2 mg\,kg$^{-1}$}: intubating conditions may be as fast as \textbf{$\sim$60 s} (with longer duration).
\end{itemize}

\textbf{Other effects}
\begin{itemize}
  \item CVS: minimal; at high RSI-style doses may increase heart rate.
  \item Reversal: antagonised by anticholinesterases and also by \textbf{sugammadex}.
\end{itemize}

\textbf{Kinetics (summary; Table 12.5 values)}
\begin{itemize}
  \item Mainly excreted unchanged in \textbf{bile} and less in urine; some de-acetylated metabolites.
  \item Duration may be prolonged in hepatic and renal failure.
  \item Approximate kinetics: protein bound \textbf{$\sim$10\%}; $V_d$ \textbf{$\sim$0.20 L\,kg$^{-1}$}; metabolised \textbf{$<$5\%}; elimination \textbf{bile $\sim$60\% / urine $\sim$40\%}.
\end{itemize}

\subsubsection*{7.3 Pancuronium}
\begin{itemize}
  \item \textbf{Bisquaternary} aminosteroidal compound.
\end{itemize}

\textbf{Presentation and use}
\begin{itemize}
  \item Colourless solution \textbf{4 mg in 2 mL}; store at \textbf{4$^\circ$C}.
  \item \textbf{0.1 mg\,kg$^{-1}$}: intubating conditions \textbf{$\sim$90--150 s}.
  \item Duration about \textbf{$\sim$45 min}.
\end{itemize}

\textbf{Other effects}
\begin{itemize}
  \item CVS: \textbf{tachycardia} (cardiac muscarinic receptor blockade).
  \item Also described: \textbf{indirect sympathomimetic} effect (reduced noradrenaline uptake at postganglionic nerve endings).
\end{itemize}

\textbf{Kinetics (summary; Table 12.5 values)}
\begin{itemize}
  \item Protein binding described variably ($\sim$10--40\%); low $V_d$.
  \item $\sim$\textbf{35\%} metabolised in liver by de-acetylation to \textbf{3- and 17-hydroxy} metabolites; \textbf{3-hydroxy} metabolite $\sim$\textbf{half as potent} as pancuronium.
  \item Unchanged drug eliminated mainly in \textbf{urine}; metabolites excreted in \textbf{bile}.
  \item Approximate kinetics: protein bound \textbf{$\sim$20--60\%}; $V_d$ \textbf{$\sim$0.27 L\,kg$^{-1}$}; metabolised \textbf{$\sim$30\%}; elimination \textbf{bile $\sim$20\% / urine $\sim$80\%}.
\end{itemize}

\subsubsection*{7.4 Atracurium}
\begin{itemize}
  \item Benzylisoquinolinium compound formulated as a mixture of \textbf{10 stereoisomers}.
\end{itemize}

\textbf{Presentation and use}
\begin{itemize}
  \item Colourless solution \textbf{10 mg\,mL$^{-1}$} in 2.5, 5, and 25 mL vials; store at \textbf{4$^\circ$C}.
  \item \textbf{0.5 mg\,kg$^{-1}$}: intubating conditions \textbf{$\sim$90--120 s}.
\end{itemize}

\textbf{Other effects}
\begin{itemize}
  \item Rapid IV administration may precipitate \textbf{histamine release}: local (injection site) or generalised $\rightarrow$ \textbf{bronchospasm} and \textbf{hypotension}.
  \item Slow IV injection minimises these effects.
  \item \textbf{Critical illness myopathy}: described association (similar to vecuronium).
\end{itemize}

\textbf{Kinetics (core exam points)}
\begin{itemize}
  \item Metabolism via:
  \begin{itemize}
    \item \textbf{Ester hydrolysis} by non-specific esterases (not plasma cholinesterase) ($\sim$60\% of metabolism); products include quaternary alcohol, quaternary acid, and \textbf{laudanosine}.
    \item \textbf{Hofmann elimination} (spontaneous breakdown at body temperature and pH) $\rightarrow$ laudanosine + quaternary monoacrylate.
  \end{itemize}
  \item \textbf{Acidosis and hypothermia slow} Hofmann elimination.
  \item Laudanosine has no NM-blocking activity; described as a glycine antagonist and associated with seizures only at concentrations above those encountered clinically.
  \item Net effect: elimination relatively independent of hepatic/renal function in many settings.
  \item Approximate kinetics: protein bound \textbf{$\sim$15\%}; $V_d$ \textbf{$\sim$0.15 L\,kg$^{-1}$}; metabolised \textbf{$\sim$90\%}; elimination \textbf{urine $\sim$10\%}.
\end{itemize}

\subsubsection*{7.5 Cisatracurium}
\begin{itemize}
  \item One stereoisomer of atracurium; \textbf{3--4$\times$ more potent} $\rightarrow$ tends to have \textbf{slower onset} (can be improved by increasing dose because histamine release potential is extremely low).
\end{itemize}

\textbf{Presentation}
\begin{itemize}
  \item Colourless solution \textbf{2 or 5 mg\,mL$^{-1}$}; store at \textbf{4$^\circ$C}.
\end{itemize}

\textbf{Kinetics}
\begin{itemize}
  \item Similar overall profile to atracurium but \textbf{does not} undergo direct hydrolysis by plasma esterases.
  \item Predominant pathway: \textbf{Hofmann elimination} $\rightarrow$ laudanosine + monoquaternary acrylate, then hydrolysis by non-specific plasma esterases.
  \item Metabolites have \textbf{no} NM-blocking properties.
  \item Minimal kinetic alteration in elderly; no change in end-stage renal or hepatic impairment.
  \item Approximate kinetics: protein bound \textbf{$\sim$15\%}; $V_d$ \textbf{$\sim$0.15 L\,kg$^{-1}$}; metabolised \textbf{$\sim$95\%}; elimination \textbf{urine $\sim$5\%}.
\end{itemize}

\subsubsection*{7.6 Mivacurium}
\begin{itemize}
  \item Benzylisoquinolinium ester; mixture of \textbf{three stereoisomers}:
  \begin{itemize}
    \item \textbf{36\%} cis-trans
    \item \textbf{58\%} trans-trans
    \item \textbf{6\%} cis-cis
  \end{itemize}
  \item The \textbf{cis-cis} isomer has $\sim$\textbf{10\%} of the potency of the other two and is not metabolised enzymatically; its half-life is \textbf{$\sim$10$\times$} that of the other isomers.
\end{itemize}

\textbf{Clinical niche}
\begin{itemize}
  \item Key advantage: \textbf{short duration}.
  \item Routine reversal with neostigmine may be unnecessary due to rapid enzymatic metabolism.
  \item \textbf{Neostigmine inhibits plasma cholinesterase} and may prevent metabolism of mivacurium; \textbf{edrophonium} may be more suitable for reversal of mivacurium-related block.
\end{itemize}

\textbf{Presentation}
\begin{itemize}
  \item Acidic solution (pH \textbf{3.5--5.0}) \textbf{2 mg\,mL$^{-1}$} in 5 and 10 mL ampoules.
  \item Shelf-life \textbf{18 months} when stored below \textbf{25$^\circ$C}.
\end{itemize}

\textbf{Effects}
\begin{itemize}
  \item Higher doses may cause \textbf{histamine release} $\rightarrow$ fall in BP and bronchospasm.
\end{itemize}

\textbf{Kinetics (summary; Table 12.5 values)}
\begin{itemize}
  \item Approximate kinetics: protein bound \textbf{$\sim$10\%}; $V_d$ \textbf{$\sim$0.21--0.32 L\,kg$^{-1}$} (isomer-specific); metabolised \textbf{$\sim$90\%}; elimination \textbf{urine $\sim$5\%}.
\end{itemize}

\subsubsection*{8) MCQ anchors (Part 2)}
\begin{itemize}
  \item \textbf{Mechanism}: competitive antagonism at \textbf{post-junctional $\alpha$ subunit}.
  \item \textbf{$>$70\% receptor occupancy} needed before block is detectable (safety margin).
  \item \textbf{Low potency $\rightarrow$ higher dose $\rightarrow$ faster onset} (concentration-gradient argument; rocuronium).
  \item \textbf{Atracurium/cisatracurium}: organ-independent elimination via \textbf{Hofmann} (slowed by \textbf{acidosis} and \textbf{hypothermia}).
  \item \textbf{Mivacurium + neostigmine}: neostigmine can \textbf{prolong} mivacurium by inhibiting plasma cholinesterase.
\end{itemize}

\newpage
\section*{Part 3 --- Reversal Agents}

\subsection*{9) Reversal of neuromuscular block: principles}
\begin{itemize}
  \item For \textbf{non-depolarising blockade}, recovery is achieved by:
  \begin{itemize}
    \item increasing \textbf{ACh at the NMJ} (anticholinesterases), and/or
    \item \textbf{removing relaxant} from plasma/NMJ (sugammadex).
  \end{itemize}
  \item \textbf{Anticholinesterases have a ceiling effect}: once AChE is maximally inhibited, further dosing does \textbf{not} proportionally increase reversal and may increase adverse effects.
  \item \textbf{Depth matters}: anticholinesterases are unreliable in very deep block. Practical reversal is performed once there is evidence of recovery (e.g. return of twitches on TOF).
\end{itemize}

\subsection*{10) Anticholinesterases}

\subsubsection*{10.1 Overview}
\begin{itemize}
  \item AChE hydrolyses ACh. \textbf{Anticholinesterases inhibit AChE}, slowing breakdown of ACh $\rightarrow$ \textbf{more ACh} to compete with NDMRs.
  \item Because AChE is widespread, anticholinesterases produce predictable \textbf{muscarinic effects}.
  \item An \textbf{anticholinergic} (typically \textbf{glycopyrrolate} or \textbf{atropine}) is commonly co-administered to reduce \textbf{bradycardia} and \textbf{secretions}.
\end{itemize}

\subsubsection*{10.2 Mechanistic classes (as described in Peck)}
\begin{itemize}
  \item \textbf{Easily reversible} inhibition (non-covalent): \textbf{edrophonium}.
  \item \textbf{Carbamylated enzyme complex} (slowly hydrolysed): \textbf{neostigmine, pyridostigmine, physostigmine}.
  \item \textbf{Irreversible} inactivation (organophosphorous compounds): relevant for toxicity and for prolonging suxamethonium/mivacurium via plasma cholinesterase inhibition.
\end{itemize}

\subsubsection*{10.3 Edrophonium}
\begin{itemize}
  \item \textbf{Class}: easily reversible AChE inhibition; \textbf{phenolic quaternary amine}.
\end{itemize}

\textbf{Uses}
\begin{itemize}
  \item IV \textbf{2--10 mg}: distinguishes \textbf{myasthenic crisis} (improves power) from \textbf{cholinergic crisis} (worsens).
\end{itemize}

\textbf{Mechanism}
\begin{itemize}
  \item Quaternary amine binds the \textbf{anionic} site of AChE; hydroxyl group forms a hydrogen bond at the \textbf{esteratic} site $\rightarrow$ stabilises the complex.
  \item No covalent bond $\rightarrow$ ACh competes with edrophonium.
  \item Also increases \textbf{ACh release}.
\end{itemize}

\textbf{Kinetics / effects}
\begin{itemize}
  \item Low lipid solubility $\rightarrow$ not orally absorbed; does \textbf{not} cross BBB or placenta.
  \item Faster onset than neostigmine.
  \item Up to \textbf{65\%} excreted unchanged in urine; remainder glucuronidated (liver) $\rightarrow$ bile.
  \item Muscarinic side effects are usually slight, but \textbf{bradycardia} and \textbf{salivation} can still occur.
\end{itemize}

\subsubsection*{10.4 Neostigmine}
\begin{itemize}
  \item \textbf{Class}: carbamate ester; \textbf{quaternary amine}.
\end{itemize}

\textbf{Presentation and uses}
\begin{itemize}
  \item Tablets; IV solution (also available combined with \textbf{glycopyrrolate}).
  \item Reversal of NDMR block: \textbf{0.05 mg\,kg$^{-1}$ IV}.
  \item Also used for myasthenia gravis (oral) and urinary retention.
\end{itemize}

\textbf{Mechanism}
\begin{itemize}
  \item Forms a \textbf{carbamylated AChE complex} $\rightarrow$ hydrolysed more slowly than the acetylated enzyme $\rightarrow$ prolonged AChE inhibition.
  \item Inhibits \textbf{plasma cholinesterase} $\rightarrow$ can \textbf{prolong suxamethonium} (and may affect ester-metabolised drugs).
\end{itemize}

\textbf{Effects (reason for pairing with an anticholinergic)}
\begin{itemize}
  \item \textbf{Cardiovascular}: bradycardia if given alone.
  \item \textbf{Respiratory}: can precipitate bronchospasm (important in asthma).
  \item \textbf{GI}: increased salivation and intestinal motility $\rightarrow$ cramps.
\end{itemize}

\textbf{Kinetics}
\begin{itemize}
  \item Poor oral absorption; minimally protein-bound; low $V_d$.
  \item Partly metabolised in liver; $\sim$\textbf{55\%} excreted unchanged in urine.
\end{itemize}

\textbf{Exam pitfall}
\begin{itemize}
  \item Excess dosing can itself cause a \textbf{depolarising-type neuromuscular block}.
\end{itemize}

\subsubsection*{10.5 Pyridostigmine}
\begin{itemize}
  \item \textbf{Class}: carbamate ester; \textbf{quaternary amine}.
  \item Used mainly in \textbf{myasthenia gravis}; described as preferred to neostigmine for longer duration and fewer autonomic effects.
\end{itemize}

\textbf{Kinetics}
\begin{itemize}
  \item Slower onset than neostigmine; longer duration.
  \item Greater dependence on renal elimination: $\sim$\textbf{75\%} excreted unchanged.
  \item Does not cross BBB.
\end{itemize}

\subsubsection*{10.6 Physostigmine (context)}
\begin{itemize}
  \item \textbf{Tertiary amine} $\rightarrow$ well absorbed orally and \textbf{crosses the BBB}.
  \item Historically used for \textbf{anticholinergic poisoning} (central effects), rather than routine NDMR reversal.
\end{itemize}

\subsection*{11) Sugammadex (selective relaxant binding agent; SRBA)}
\begin{itemize}
  \item A \textbf{$\gamma$-cyclodextrin} that encapsulates aminosteroidal NDMRs.
  \item Approved for Europe in \textbf{2008}; US approval was later due to concerns about \textbf{hypersensitivity reactions}.
\end{itemize}

\textbf{Licensed indication}
\begin{itemize}
  \item Reversal of neuromuscular blockade induced by \textbf{rocuronium} or \textbf{vecuronium}.
\end{itemize}

\textbf{Mechanism}
\begin{itemize}
  \item Encapsulates rocuronium/vecuronium $\rightarrow$ removes them from plasma and NMJ.
  \item Creates a concentration gradient favouring movement \textbf{away from the NMJ}.
  \item No intrinsic cholinergic activity $\rightarrow$ avoids predictable muscarinic effects.
\end{itemize}

\textbf{Presentation}
\begin{itemize}
  \item \textbf{100 mg\,mL$^{-1}$} solution in \textbf{2 mL} or \textbf{5 mL} vials (shelf-life $\sim$3 years).
\end{itemize}

\textbf{Dose (tied to depth of block --- high yield)}
\begin{itemize}
  \item \textbf{Routine reversal}:
  \begin{itemize}
    \item \textbf{4 mg\,kg$^{-1}$} when there is \textbf{some post-tetanic twitch activity}.
    \item \textbf{2 mg\,kg$^{-1}$} when, during TOF, \textbf{T2 is visible}.
  \end{itemize}
  \item \textbf{Emergency reversal}:
  \begin{itemize}
    \item \textbf{16 mg\,kg$^{-1}$} immediately following \textbf{1.2 mg\,kg$^{-1}$} rocuronium (requires multiple vials).
  \end{itemize}
\end{itemize}

\textbf{Interactions (displacement vs capture)}
\begin{itemize}
  \item \textbf{Displacement} of rocuronium/vecuronium from sugammadex (risk of recurarisation) reported with: \textbf{flucloxacillin, diclofenac, fusidic acid, toremifene}.
  \item \textbf{Capture} of other drugs may reduce efficacy; most clinically significant is reduced efficacy of \textbf{oral contraceptives} $\rightarrow$ advise \textbf{additional contraception for 7 days}.
\end{itemize}

\textbf{Kinetics}
\begin{itemize}
  \item $V_d$ $\sim$\textbf{11--14 L}; linear kinetics at therapeutic doses.
  \item Not metabolised; eliminated \textbf{unchanged by the kidneys}; effective half-life $\sim$\textbf{2.5 h}.
\end{itemize}

\textbf{Anaphylaxis note (exam nuance)}
\begin{itemize}
  \item Has been used to treat suspected \textbf{rocuronium-induced anaphylaxis}, but does \textbf{not} currently form part of any approved anaphylaxis guideline.
\end{itemize}

\subsubsection*{12) MCQ anchors (Part 3)}
\begin{itemize}
  \item \textbf{Anticholinesterases} increase \textbf{ACh} at NMJ $\rightarrow$ reverse \textbf{NDMR} block but cause \textbf{muscarinic} effects $\rightarrow$ pair with \textbf{glycopyrrolate/atropine}.
  \item \textbf{Neostigmine} inhibits \textbf{plasma cholinesterase} $\rightarrow$ can \textbf{prolong suxamethonium} and complicate \textbf{mivacurium} metabolism.
  \item \textbf{Sugammadex} is specific for \textbf{aminosteroids} (rocuronium/vecuronium) and is dosed by \textbf{depth}: \textbf{2 mg\,kg$^{-1}$} (T2 present), \textbf{4 mg\,kg$^{-1}$} (PTC activity), \textbf{16 mg\,kg$^{-1}$} (emergency after RSI-dose rocuronium).
\end{itemize}

\end{document}
