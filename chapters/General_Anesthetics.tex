\documentclass[11pt,a4paper]{article}

% preamble.tex
\usepackage[T1]{fontenc}
\usepackage[utf8]{inputenc}
\usepackage[english]{babel}
\usepackage[a4paper,margin=2cm]{geometry}
\usepackage{lmodern}
\usepackage{amsmath,amssymb}
\usepackage{graphicx}
\usepackage{booktabs}
\usepackage{hyperref}

\usepackage[T1]{fontenc}
\usepackage{setspace}
\usepackage{geometry}
\usepackage{titlesec}
\usepackage{enumitem}
\usepackage{hyperref}
\usepackage{booktabs}


\geometry{margin=2.5cm}
\setstretch{1.18}

\titleformat{\section}{\large\bfseries}{\thesection}{0.5em}{}
\titleformat{\subsection}{\bfseries}{\thesubsection}{0.5em}{}

\title{\textbf{Chapter 9 --- General Anaesthetic Agents}\\[4pt]
\large Complete Expanded Summary (Introduction up to IV Agents)}
\date{}
\author{}

\begin{document}

\maketitle

\section{Overview of General Anaesthetic Effects}

According to Peck \cite{peck-ch9}, a valid mechanism of general anaesthesia must explain:
\begin{itemize}[noitemsep]
    \item loss of consciousness,
    \item loss of response to noxious stimuli (immobility),
    \item amnesia,
    \item and must be reversible.
\end{itemize}

These arise from a coordinated but heterogeneous depression of CNS function.

\section{Anatomical Sites of Action}

Peck identifies the following structures as critical \cite{peck-ch9}:

\subsection{Thalamus}
Disruption of thalamocortical connectivity correlates strongly with loss of consciousness.

\subsection{Cortex}
Cortical regions mediate higher‐order awareness and conscious perception.

\subsection{Limbic System}
Structures in the limbic system (e.g.\ hippocampus) contribute to amnesia.

\subsection{Spinal Cord}
Volatile agents exert a stronger effect on the spinal cord than IV agents, producing immobility (basis of MAC).

\subsubsection*{Additional detail (NOT in Peck)}
Fundamentals of Anaesthesia notes that sedation involves GABA\textsubscript{A} modulation in the \emph{tuberomammillary nucleus}, a histaminergic arousal centre \cite{fa-tmn}.

\section{Historical Theories of Anaesthetic Mechanism}

\subsection{Meyer--Overton Rule}
Potency correlates with lipid solubility across many agents \cite{peck-ch9}.  
However, lipid solubility predicts potency, not mechanism.

\subsection{Limitations of Lipid Theories}

\paragraph{Stereoselectivity}
R-etomidate is active while S-etomidate is inactive despite identical lipid solubility \cite{peck-ch9}.

\paragraph{Ketamine outlier}
Ketamine has poor fit with lipid potency predictions.

\paragraph{Temperature paradox}
(Not in Peck.) Fundamentals shows membrane fluidisation at higher temperature reduces anaesthetic effect \cite{fa-lipid}.

\paragraph{Annular lipid perturbation}
(Not in Peck.) Anaesthetics alter lipids around channel proteins rather than whole-membrane expansion \cite{fa-lipid}.

\paragraph{Critical volume \& pressure reversal}
(Not in Peck.) Morgan \& Mikhail describes reversal of anaesthesia at high atmospheric pressure, contradicting simple volume expansion theories \cite{mm-pressure}.

\section{Modern Protein-Based Mechanisms}

Peck emphasises that ligand-gated ion channels, not lipid bilayers, are the primary anaesthetic targets \cite{peck-ch9}.

\subsection*{Additional evidence for protein binding (NOT in Peck)}
\begin{itemize}
    \item Anaesthetic inhibition of \emph{luciferase} at clinically relevant potencies \cite{fa-luciferase}.
    \item Saturable binding of halothane to synaptosomal proteins \cite{fa-luciferase}.
\end{itemize}

\section{Primary Ion Channel Targets (Peck + Table 9.1)}

Peck identifies four major ligand-gated ion channel systems \cite{peck-table}:

\begin{itemize}
    \item GABA\textsubscript{A} (inhibitory),
    \item Glycine (inhibitory),
    \item NMDA (excitatory),
    \item Neuronal nicotinic acetylcholine (nACh) receptors (excitatory).
\end{itemize}

\subsection*{Additional structural detail (NOT in Peck)}
All belong to the \textbf{cys-loop pentameric receptor family} with four transmembrane domains per subunit (M1--M4) \cite{fa-cysloop}.

\section{Detailed Receptor Mechanisms}

\subsection{GABA\textsubscript{A} Receptors}

Peck: Pentameric chloride channel; activation causes hyperpolarisation.  
Etomidate, propofol, barbiturates, volatiles potentiate GABA\textsubscript{A} via $\beta$ subunits \cite{peck-ch9}.

\subsubsection*{Additional mechanistic details (NOT in Peck)}
\begin{itemize}
    \item R- vs S-etomidate differs by $\sim$30-fold potency at GABA\textsubscript{A} \cite{fa-gaba}.
    \item Propofol also inhibits 5-HT\textsubscript{3} and neuronal nACh receptors \cite{fa-gaba}.
    \item Barbiturate enantiomers exhibit potency differences \cite{fa-gaba}.
\end{itemize}

\subsection{Glycine Receptors}

Peck: Potentiated strongly by volatile anaesthetics; correlates with immobility \cite{peck-ch9}.

\subsubsection*{Additional detail (NOT in Peck)}
Volatile stereoselectivity at glycine receptors is minimal \cite{fa-nmda}.

\subsection{NMDA Receptors}

Peck: Inhibited non-competitively by ketamine, N$_2$O, xenon \cite{peck-ch9}.

\subsubsection*{Additional physiology (NOT in Peck)}
NMDA channels:
\begin{itemize}
    \item are blocked by Mg$^{2+}$ at rest, 
    \item mediate long-term potentiation (memory formation) \cite{cp-nmda}.
\end{itemize}

\subsection{Neuronal Nicotinic ACh Receptors}

Peck: Inhibited by propofol, thiopental, volatile agents \cite{peck-ch9}.

\subsubsection*{Additional detail (NOT in Peck)}
Part of the cys-loop receptor family \cite{fa-cysloop}.

\section{Secondary Mechanisms (NOT in Peck)}

Fundamentals and Morgan \& Mikhail describe:

\begin{itemize}
    \item Mild inhibition of voltage-gated sodium channels at supraclinical doses \cite{fa-vgsc}.
    \item Mild inhibition of voltage-gated calcium channels \cite{fa-vgsc}.
\end{itemize}

These are \emph{not} primary mechanisms.

\section{Summary of Table 9.1 (Peck)}

Table 9.1 shows receptor actions of the main agents \cite{peck-table}:
\begin{itemize}
    \item Volatiles: GABA\textsubscript{A}$\uparrow$, glycine$\uparrow$, nACh$\downarrow$.
    \item Ketamine/N$_2$O/xenon: primarily NMDA$\downarrow$.
    \item R-etomidate: highly selective GABA\textsubscript{A} potentiation.
    \item S-etomidate: no measurable activity.
\end{itemize}

\section{Ideal Intravenous Anaesthetic}

Peck lists ideal properties:
\begin{itemize}
    \item rapid onset (mainly unionised at physiological pH) 
    \item high lipid solubility 
    \item rapid recovery, no accumulation during prolonged infusion 
    \item analgesic at sub-anaesthetic concentrations 
    \item minimal cardiovascular and respiratory depression 
    \item no emetic effects
    \item no pain on injection 
    \item no excitation or emergence phenomena 
    \item no interaction with other agents 
    \item safe following inadvertent intra-arterial injection 
    \item no toxic effects 
    \item no histamine release 
    \item no hypersensitivity reactions 
    \item water-soluble formulation 
    \item long shelf-life at room temperature 
    \item minimal environmental impact.y \cite{peck-ideal}
\end{itemize}

% ============================
% CHAPTER 9 — IV AGENTS SECTION
% ============================

\section{Intravenous General Anaesthetic Agents}

This section summarises the intravenous induction agents described in Peck \& Harris, 
including structural class, mechanism of action, kinetics, physiological effects, toxicity, 
and environmental considerations.

% -------------------------------------------------
% 1. THIOPENTAL
% -------------------------------------------------

\subsection{Thiopental}

\begin{figure}[h!]
\centering
\includegraphics[width=0.4\linewidth]{thiopental_structure.png}
\caption{Chemical structure of Thiopental (placeholder).}
\end{figure}

\textbf{Biochemical class and structure.}  
Thiopental is a thiobarbiturate in which the oxygen at C2 is replaced by sulphur, increasing lipid solubility.
It is supplied as thiopental sodium in a strongly alkaline solution (pH 10--11).  
Keto–enol tautomerism exists, but the keto form predominates at physiological pH.

\textbf{Mechanism of action.}  
According to Table~9.1 in Peck \& Harris, thiopental:
\begin{itemize}
    \item strongly potentiates GABA$_A$ receptors (+++ ++),
    \item moderately potentiates glycine receptors (++),
    \item inhibits neuronal nicotinic ACh receptors (–),
    \item has no NMDA receptor action.
\end{itemize}

\textbf{Kinetics.}  
Thiopental has an extremely rapid onset because of its high lipid solubility and large unionised fraction.
Its duration of action is short due to redistribution rather than metabolism.
It undergoes slow hepatic metabolism and accumulates with repeated doses or infusions.

\textbf{Physiological effects.}  
Thiopental decreases cerebral metabolic rate, CBF, and ICP.
It causes vasodilation and a fall in arterial blood pressure, with reflex tachycardia.
It produces dose-dependent respiratory depression.

\textbf{Toxicity.}  
Intra-arterial injection can cause severe vasospasm and tissue necrosis.
Thiopental is contraindicated in porphyria because it induces $\delta$-ALA synthetase.

\textbf{Environmental impact.}  
Peck \& Harris do not describe environmental effects for thiopental.

% -------------------------------------------------
% 2. METHOHEXITAL
% -------------------------------------------------

\subsection{Methohexital}

\begin{figure}[h!]
\centering
\includegraphics[width=0.4\linewidth]{methohexital_structure.png}
\caption{Chemical structure of Methohexital (placeholder).}
\end{figure}

\textbf{Biochemical structure.}  
Methohexital is an oxybarbiturate (oxygen rather than sulphur at C2).
Its structural substitutions increase excitatory and pro-convulsant potential.

\textbf{Kinetics.}  
Rapid onset with faster redistribution and metabolism than thiopental, leading to quicker recovery.

\textbf{Physiological effects.}  
It decreases CBF and ICP and lowers blood pressure.  
Myoclonus and seizure-like activity occur more frequently than with thiopental.

\textbf{Clinical relevance.}  
Preferred for electroconvulsive therapy because it lowers seizure threshold.

\textbf{Environmental impact.}  
Not described.

% -------------------------------------------------
% 3. ETOMIDATE
% -------------------------------------------------

\subsection{Etomidate}

\begin{figure}[h!]
\centering
\includegraphics[width=0.4\linewidth]{etomidate_structure.png}
\caption{Chemical structure of Etomidate (placeholder).}
\end{figure}

\textbf{Biochemical structure.}  
Etomidate is an imidazole ester and exists as R- and S-enantiomers.
Only R-etomidate is clinically active.
Commercial preparation in propylene glycol causes pain on injection.

\textbf{Mechanism of action (Table 9.1).}
\begin{itemize}
    \item GABA$_A$ potentiation: +++++ (very high)
    \item Glycine: 0
    \item NMDA: 0
    \item nACh: 0
\end{itemize}

\textbf{Kinetics.}  
Rapid onset; redistribution leads to short duration.
Hepatic metabolism with minimal accumulation.

\textbf{Physiological effects.}  
Etomidate is the most haemodynamically stable IV anaesthetic described by Peck.
It minimally alters BP, HR, or cardiac output.
It decreases CBF, CMRO$_2$, and ICP.
Respiratory depression is mild.

\textbf{Toxicity.}  
Etomidate inhibits 11-$\beta$-hydroxylase, suppressing cortisol and aldosterone synthesis.
It produces myoclonus, pain on injection, and postoperative nausea/vomiting.

\textbf{Environmental impact.}  
Not described.

% -------------------------------------------------
% 4. PROPOFOL
% -------------------------------------------------

\subsection{Propofol}

\begin{figure}[h!]
\centering
\includegraphics[width=0.4\linewidth]{propofol_structure.png}
\caption{Chemical structure of Propofol (placeholder).}
\end{figure}

\textbf{Biochemical structure.}  
Propofol is 2,6-diisopropylphenol, an alkylphenol with very high lipid solubility.
It requires a lipid emulsion containing soybean oil, egg phosphatide, and glycerol.

\textbf{Mechanism of action (Table 9.1).}
\begin{itemize}
    \item GABA$_A$: ++++
    \item Glycine: +
    \item nACh: –
    \item NMDA: 0
\end{itemize}

\textbf{Kinetics.}  
Propofol has extremely rapid onset and recovery due to redistribution and high hepatic \emph{and extrahepatic} clearance.  
It accumulates minimally, making it suitable for TIVA.

\textbf{Physiological effects.}  
\emph{CNS:} decreases CBF, CMRO$_2$, and ICP; possesses intrinsic antiemetic activity.  
\emph{Cardiovascular:} marked hypotension from vasodilation and negative inotropy; blunts baroreflex.  
\emph{Respiratory:} profound respiratory depression and apnoea; suppresses airway reflexes, facilitating LMA insertion.

\textbf{Toxicity.}  
\begin{itemize}
    \item Propofol infusion syndrome (PRIS): metabolic acidosis, rhabdomyolysis, cardiac failure.
    \item Pain on injection.
    \item Supports bacterial growth: must be discarded within 6 hours.
\end{itemize}

\textbf{Environmental impact (Peck p.~104).}  
\begin{itemize}
    \item No direct greenhouse or ozone-depleting effect.
    \item Environmental footprint arises from production and the electricity used to run infusion pumps.
    \item Global warming potential approximately four orders of magnitude lower than volatile agents.
    \item Moderate environmental persistence.
    \item Toxic to aquatic organisms.
    \item Unused propofol must not be poured into drains; must be disposed of in containers destined for incineration.
\end{itemize}

% -------------------------------------------------
% 5. KETAMINE
% -------------------------------------------------

\subsection{Ketamine}

\begin{figure}[h!]
\centering
\includegraphics[width=0.4\linewidth]{ketamine_structure.png}
\caption{Chemical structure of Ketamine (placeholder).}
\end{figure}

\textbf{Biochemical structure.}  
Ketamine is an arylcyclohexylamine and a derivative of phencyclidine (PCP).  
It is chiral; S-ketamine is more potent.

\textbf{Mechanism of action (Table 9.1).}
\begin{itemize}
    \item NMDA receptor antagonism (–)
    \item GABA$_A$: 0
    \item Glycine: 0
    \item nACh: 0
\end{itemize}

\textbf{Kinetics.}  
Rapid onset.
Redistribution is slower than with thiopental or propofol.
Metabolized hepatically to norketamine (active metabolite).
Can accumulate with repeated administration.

\textbf{Physiological effects.}  
Ketamine produces a \emph{dissociative} state: eyes open, nystagmus, preserved reflexes, analgesia, amnesia.
\emph{Cardiovascular:} increases BP, HR, and CO.
\emph{Respiratory:} minimal depression; causes bronchodilation.
\emph{CNS:} increases CBF and ICP.

\textbf{Toxicity.}  
Emergence delirium, hallucinations, nightmares; increased salivation; nausea.

\textbf{Environmental impact.}  
Not described.

%insert IV table here

% ============================
% CHAPTER 9 — INHALED (VOLATILE) AGENTS
% ============================

\section{Inhalational (Volatile) Anaesthetic Agents}

Inhalational agents include nitrous oxide and a number of volatile liquids
(isoflurane, sevoflurane, desflurane, with xenon discussed as a useful but
limited option). They are delivered via the lungs and their clinical effect
is determined primarily by their partial pressure in the brain, which at
steady state equilibrates with the alveolar partial pressure.

% -------------------------------------------------
% 1. MINIMUM ALVEOLAR CONCENTRATION (MAC)
% -------------------------------------------------

\subsection{Minimum Alveolar Concentration (MAC)}

\textbf{Definition.}
Minimum alveolar concentration (MAC) is the steady-state alveolar concentration
of an inhalational anaesthetic at 1~atm that prevents movement in response
to a standard surgical skin incision in 50\% of subjects.
It is therefore an ED$_{50}$ for immobility.

\textbf{Related MAC concepts.}
\begin{itemize}
  \item \textbf{MAC-awake:} alveolar concentration at which 50\% of patients
  will respond to verbal command (usually $\approx 0.3$--$0.5$~MAC).
  \item \textbf{MAC-BAR:} alveolar concentration that blocks autonomic (adrenergic)
  responses to surgical stimulation in 50\% of patients (often $\approx 1.7$--$2$~MAC).
  \item MAC values are \textbf{additive}:
  e.g. 0.5~MAC sevoflurane + 0.5~MAC nitrous oxide $\approx 1$~MAC for movement.
\end{itemize}

\textbf{Factors that increase MAC (higher requirement).}
\begin{itemize}
  \item Young age (beyond the neonatal period)
  \item Hyperthermia
  \item Hyperthyroidism, raised catecholamine states
  \item Chronic alcohol or opioid use; acute amphetamine use
  \item Hypernatraemia
\end{itemize}

\textbf{Factors that decrease MAC (lower requirement).}
\begin{itemize}
  \item Increasing age (approximately $-6\%$ per decade after young adulthood)
  \item Neonatal period
  \item Pregnancy
  \item Hypothermia, hypotension
  \item Sedative drugs, opioids, $\alpha_2$-agonists
  \item Acute alcohol use; lithium therapy
  \item Hypoxia, hypercarbia, severe anaemia, metabolic acidosis
\end{itemize}

At equilibrium the clinically relevant variable is the \emph{partial pressure}
of anaesthetic in the brain, which is reflected by the end-tidal (alveolar)
partial pressure.

% -------------------------------------------------
% 2. IDEAL INHALED ANAESTHETIC AGENT
% -------------------------------------------------

\subsection{The Ideal Inhaled Anaesthetic Agent}

The agents in current use represent compromises. An ideal inhalational agent
would have the following properties.

\textbf{Physical and equipment-related properties.}
\begin{itemize}
  \item Non-flammable and non-explosive.
  \item Chemically stable to light, heat and storage.
  \item Non-reactive with metals, plastics, rubber and CO$_2$ absorbents.
  \item Suitable vapour pressure and boiling point for accurate vaporisation.
  \item Odourless or pleasant, non-irritant to the airway (suitable for mask induction).
\end{itemize}

\textbf{Pharmacokinetic and pharmacodynamic properties.}
\begin{itemize}
  \item High potency (low MAC) with a wide therapeutic index.
  \item Low blood:gas partition coefficient (rapid uptake and elimination).
  \item Predictable context-insensitive offset.
  \item Minimal metabolism, no toxic metabolites.
  \item Produces reliable hypnosis, amnesia and some analgesia.
  \item Minimal effect on myocardium, systemic vascular resistance, pulmonary vasculature,
        intracranial pressure, hepatic and renal blood flow and neuromuscular transmission.
  \item Not epileptogenic, no triggering of malignant hyperthermia.
\end{itemize}

\textbf{Environmental and practical properties.}
\begin{itemize}
  \item Very low global warming potential and no ozone-depleting effect.
  \item Inexpensive, easy to manufacture, store and transport safely.
\end{itemize}

No currently available volatile agent fully satisfies these criteria.

% -------------------------------------------------
% 3. ENVIRONMENTAL IMPACT OF INHALED AGENTS
% -------------------------------------------------

\subsection{Environmental Impact of Inhalational Anaesthetics}

Inhalational anaesthetic agents function as greenhouse gases because they
absorb infrared radiation in spectral bands through which the Earth normally
radiates heat (the ``atmospheric window''). Their climate impact is
characterised by:

\begin{itemize}
  \item \textbf{Atmospheric lifetime} (years): how long a molecule persists before breakdown.
  \item \textbf{Global Warming Potential (GWP)}: warming effect over a specified time horizon
        (commonly 100~years, GWP$_{100}$), expressed relative to CO$_2$ (CO$_2$ = 1).
\end{itemize}

Typical values quoted in anaesthetic sustainability literature are:

\begin{center}
\begin{tabular}{lcc}
\hline
\textbf{Agent} & \textbf{Atmospheric lifetime (years)} & \textbf{GWP$_{100}$} \\
\hline
Sevoflurane & $\approx 1.1$ & $\approx 130$ \\
Isoflurane & $\approx 3.2$ & $\approx 510$ \\
Desflurane & $\approx 14$  & $\approx 2500$ \\
Nitrous oxide (N$_2$O) & 114 & 298 \\
\hline
\end{tabular}
\end{center}

\textbf{Key points.}
\begin{itemize}
  \item Desflurane has both a relatively long atmospheric lifetime and a high GWP,
        and is the least potent volatile; it therefore has the highest CO$_2$-equivalent
        footprint per MAC-hour.
  \item Sevoflurane has the lowest GWP among the commonly used halogenated volatiles,
        but is still far more impactful than intravenous agents.
  \item Nitrous oxide has a long atmospheric lifetime and substantial GWP; global
        emissions are dominated by non-medical sources (mainly agriculture), but
        medical use can be a major contributor to a hospital's anaesthetic footprint.
\end{itemize}

Compounds containing chlorine or bromine may deplete stratospheric ozone.
Nitrous oxide also contributes to ozone depletion via photolytic reactions in
the upper atmosphere. Overall, desflurane and nitrous oxide are the most
environmentally damaging anaesthetic gases.

% -------------------------------------------------
% 4. INDIVIDUAL VOLATILE AGENTS
% -------------------------------------------------

\subsection{Sevoflurane}

\begin{figure}[h!]
\centering
\includegraphics[width=0.4\linewidth]{sevoflurane_structure.png}
\caption{Chemical structure of sevoflurane (placeholder).}
\end{figure}

\textbf{Class and structure.}
Sevoflurane is a fluorinated methyl isopropyl ether. It is non-pungent with
a relatively pleasant smell, which makes it well suited to inhalational
induction, particularly in children.

\textbf{Physical properties.}
\begin{itemize}
  \item Blood:gas partition coefficient $\approx 0.65$ (relatively insoluble).
  \item Oil:gas partition coefficient $\approx 50$ (moderate potency).
  \item Vapour pressure compatible with conventional variable-bypass vaporisers.
\end{itemize}

\textbf{Potency.}
\begin{itemize}
  \item Adult MAC $\approx 2.0\%$ (range 1.8--2.1\% depending on source).
\end{itemize}

\textbf{Kinetics.}
\begin{itemize}
  \item Relatively low blood solubility gives rapid rise and fall of alveolar
        concentration, and therefore rapid induction and recovery.
  \item Approximately 3--5\% is metabolised by hepatic CYP\,2E1 to inorganic
        fluoride and hexafluoroisopropanol; the remainder is exhaled unchanged.
\end{itemize}

\textbf{Central nervous system effects.}
\begin{itemize}
  \item Decreases CMRO$_2$ with dose-dependent EEG slowing.
  \item Cerebral vasodilation leads to a moderate increase in CBF and ICP,
        although less than with desflurane at equivalent MAC.
\end{itemize}

\textbf{Cardiovascular effects.}
\begin{itemize}
  \item Dose-dependent fall in arterial blood pressure, predominantly via
        reduction in systemic vascular resistance.
  \item Heart rate is usually maintained or slightly increased; cardiac
        output is typically preserved at clinical concentrations.
\end{itemize}

\textbf{Respiratory effects.}
\begin{itemize}
  \item Non-irritant, facilitating smooth mask induction.
  \item Produces dose-dependent depression of tidal volume with some
        compensatory tachypnoea, resulting in reduced minute ventilation.
  \item Acts as a bronchodilator, advantageous in reactive airway disease.
\end{itemize}

\textbf{Other effects and toxicity.}
\begin{itemize}
  \item Some uterine relaxation at higher concentrations.
  \item Interaction with desiccated strong-base CO$_2$ absorbents may produce
        vinyl ether degradation products (``Compound~A'') that have shown
        nephrotoxicity in animals; clinically significant renal injury in humans
        has not been convincingly demonstrated at usual flows, but extremely
        low fresh-gas flows with sevoflurane are generally avoided with
        older absorbents.
  \item Inorganic fluoride levels rise but typically remain below levels
        associated with methoxyflurane nephrotoxicity.
\end{itemize}

\textbf{Environmental profile.}
\begin{itemize}
  \item Atmospheric lifetime $\approx 1.1$~years; GWP$_{100} \approx 130$.
  \item Among currently used halogenated volatiles, sevoflurane has the most
        favourable global-warming profile, but its impact is still far higher
        than intravenous techniques.
\end{itemize}

% -------------------------------------------------

\subsection{Desflurane}

\begin{figure}[h!]
\centering
\includegraphics[width=0.4\linewidth]{desflurane_structure.png}
\caption{Chemical structure of desflurane (placeholder).}
\end{figure}

\textbf{Class and structure.}
Desflurane is a fully fluorinated methyl ethyl ether, structurally related
to isoflurane. It has a pungent odour and is irritant to the airway.

\textbf{Physical properties.}
\begin{itemize}
  \item Blood:gas partition coefficient $\approx 0.42$ (least soluble of the
        commonly used volatiles).
  \item Oil:gas partition coefficient $\approx 18$--19 (lower potency).
  \item Boiling point around 23$^\circ$C and high vapour pressure, requiring
        a heated, pressurised vaporiser.
\end{itemize}

\textbf{Potency.}
\begin{itemize}
  \item Adult MAC $\approx 6\%$.
\end{itemize}

\textbf{Kinetics.}
\begin{itemize}
  \item Very low blood solubility leads to extremely rapid uptake and elimination,
        with quick changes in depth and very rapid emergence.
  \item Metabolism is negligible ($\approx 0.02\%$), so virtually all is exhaled unchanged.
\end{itemize}

\textbf{Central nervous system effects.}
\begin{itemize}
  \item Decreases CMRO$_2$.
  \item Potent cerebral vasodilation leads to increased CBF and ICP; desflurane
        is therefore relatively undesirable in patients with raised ICP.
\end{itemize}

\textbf{Cardiovascular effects.}
\begin{itemize}
  \item At steady concentrations, desflurane produces vasodilation with a fall
        in arterial blood pressure, similar to other volatiles.
  \item Rapid increases in inspired concentration can cause sympathetic activation
        with marked tachycardia and hypertension.
\end{itemize}

\textbf{Respiratory effects.}
\begin{itemize}
  \item Pungent and irritant: can provoke coughing, breath-holding and laryngospasm,
        making it unsuitable for inhalational induction.
  \item Produces dose-dependent respiratory depression.
\end{itemize}

\textbf{Toxicity.}
\begin{itemize}
  \item No clinically significant fluoride-related nephrotoxicity.
  \item A trigger of malignant hyperthermia, as with all potent volatile agents.
\end{itemize}

\textbf{Environmental profile.}
\begin{itemize}
  \item Atmospheric lifetime $\approx 14$~years; GWP$_{100} \approx 2500$.
  \item Because of its high GWP and relatively low potency, desflurane has the
        highest CO$_2$-equivalent emissions per MAC-hour among the common
        inhalational agents.
\end{itemize}

% -------------------------------------------------

\subsection{Isoflurane}

\begin{figure}[h!]
\centering
\includegraphics[width=0.4\linewidth]{isoflurane_structure.png}
\caption{Chemical structure of isoflurane (placeholder).}
\end{figure}

\textbf{Class and structure.}
Isoflurane is a chlorinated, fluorinated methyl isopropyl ether with a pungent
ether smell, which makes it unsuitable for inhalational induction in most patients.

\textbf{Physical properties.}
\begin{itemize}
  \item Blood:gas partition coefficient $\approx 1.4$ (more soluble than sevoflurane
        or desflurane).
  \item Oil:gas partition coefficient $\approx 90$--100 (high potency).
\end{itemize}

\textbf{Potency.}
\begin{itemize}
  \item Adult MAC $\approx 1.15\%$.
\end{itemize}

\textbf{Kinetics.}
\begin{itemize}
  \item Intermediate onset and recovery: slower than sevoflurane and desflurane,
        but acceptable for many longer procedures.
  \item Metabolism $<0.2\%$; the vast majority is exhaled unchanged.
\end{itemize}

\textbf{Central nervous system effects.}
\begin{itemize}
  \item Decreases CMRO$_2$.
  \item Cerebral vasodilation can produce significant increases in CBF and ICP,
        limiting its usefulness in patients with intracranial hypertension.
\end{itemize}

\textbf{Cardiovascular effects.}
\begin{itemize}
  \item Marked reduction in systemic vascular resistance with a fall in arterial
        blood pressure.
  \item Cardiac output is often maintained by reflex tachycardia and relatively
        preserved contractility.
  \item Coronary steal has been proposed but is of uncertain clinical importance
        at routine concentrations.
\end{itemize}

\textbf{Respiratory effects.}
\begin{itemize}
  \item Dose-dependent respiratory depression.
  \item Pungency makes isoflurane unsuitable for mask induction.
\end{itemize}

\textbf{Toxicity and environmental profile.}
\begin{itemize}
  \item Minimal metabolism and no characteristic intrinsic organ toxicity.
  \item Contains chlorine and therefore has some ozone-depleting potential,
        although its relatively short atmospheric lifetime limits overall impact.
  \item Atmospheric lifetime $\approx 3.2$~years; GWP$_{100} \approx 510$.
\end{itemize}

% -------------------------------------------------

\subsection{Nitrous Oxide (N$_2$O)}

\begin{figure}[h!]
\centering
\includegraphics[width=0.4\linewidth]{nitrous_oxide_cylinder.png}
\caption{Nitrous oxide cylinder (placeholder).}
\end{figure}

\textbf{Class and structure.}
Nitrous oxide is a simple inorganic gas with a faintly sweet odour.

\textbf{Physical properties and potency.}
\begin{itemize}
  \item Blood:gas partition coefficient $\approx 0.47$.
  \item Adult MAC $\approx 105\%$ at 1~atm; it cannot produce surgical anaesthesia
        alone in air and is used as an adjunct for analgesia and MAC-sparing.
\end{itemize}

\textbf{Central nervous system effects.}
\begin{itemize}
  \item Provides significant analgesia at sub-anaesthetic concentrations.
  \item Increases CBF and may increase ICP.
\end{itemize}

\textbf{Cardiorespiratory effects.}
\begin{itemize}
  \item Mild sympathetic stimulation; heart rate and blood pressure are usually
        stable or slightly increased in healthy patients.
  \item Mild respiratory depression at higher concentrations.
\end{itemize}

\textbf{Specific hazards.}
\begin{itemize}
  \item \emph{Diffusion hypoxia}: at the end of anaesthesia, rapid diffusion of N$_2$O
        from blood into alveoli can transiently dilute alveolar oxygen; this is
        prevented by administering 100\% O$_2$ for several minutes.
  \item \emph{Expansion of closed gas spaces}: N$_2$O equilibrates more rapidly
        than N$_2$ and can expand pneumothoraces, bowel gas, air emboli, intracranial
        air and middle ear cavities.
  \item \emph{Vitamin B$_{12}$ inactivation}: prolonged or high exposure can oxidise
        the cobalt atom in vitamin~B$_{12}$, inhibiting methionine synthase and
        leading to megaloblastic changes and neuropathy.
\end{itemize}

\textbf{Environmental and ozone effects.}
\begin{itemize}
  \item Atmospheric lifetime $\approx 114$~years; GWP$_{100} \approx 298$.
  \item N$_2$O is now considered a major anthropogenic ozone-depleting gas via
        stratospheric photolysis and subsequent reactions.
  \item Medical use represents a small proportion of global N$_2$O emissions,
        but within hospitals it can be a major contributor to greenhouse-gas output.
\end{itemize}

% -------------------------------------------------

\subsection{Xenon}

\begin{figure}[h!]
\centering
\includegraphics[width=0.4\linewidth]{xenon_cylinder.png}
\caption{Xenon gas cylinder (placeholder).}
\end{figure}

\textbf{Class and structure.}
Xenon is an inert noble gas present in trace amounts in the atmosphere.

\textbf{Physical properties and potency.}
\begin{itemize}
  \item Blood:gas partition coefficient $\approx 0.115$, giving extremely rapid
        uptake and elimination.
  \item Adult MAC $\approx 71\%$.
\end{itemize}

\textbf{Clinical profile.}
\begin{itemize}
  \item Provides hypnosis and analgesia with minimal haemodynamic disturbance.
  \item Appears neuroprotective in experimental models and preserves cerebral
        autoregulation.
\end{itemize}

\textbf{Limitations and environmental considerations.}
\begin{itemize}
  \item Very expensive and relatively scarce; requires specialised closed-circuit
        delivery systems to limit wastage.
  \item Xenon is inert and does not contribute meaningfully to global warming
        or ozone depletion.
\end{itemize}




\bibliographystyle{plain}
\begin{thebibliography}{9}

\bibitem{peck-ch9}
Peck \& Harris. \emph{Pharmacology for Anaesthesia and Intensive Care}, Chapter 9 (provided PDF).

\bibitem{peck-table}
Peck Table 9.1, \emph{General Anaesthetic Effects at Central Receptors}.

\bibitem{peck-ideal}
Peck: Section ``Ideal Intravenous Agent''.

\bibitem{fa-tmn}
\emph{Fundamentals of Anaesthesia}, sedation via TMN GABA\textsubscript{A} modulation.

\bibitem{fa-lipid}
\emph{Fundamentals of Anaesthesia}, chapter on lipid vs protein theories.

\bibitem{fa-luciferase}
\emph{Fundamentals of Anaesthesia}, biochemical evidence for protein binding.

\bibitem{fa-cysloop}
\emph{Fundamentals of Anaesthesia}, ligand-gated ion channel structure.

\bibitem{fa-gaba}
\emph{Fundamentals of Anaesthesia}, GABA\textsubscript{A} stereoselectivity.

\bibitem{fa-nmda}
\emph{Fundamentals of Anaesthesia}, glycine and NMDA mechanisms.

\bibitem{fa-vgsc}
\emph{Fundamentals of Anaesthesia}, voltage-gated ion channel involvement.

\bibitem{cp-nmda}
\emph{Pharmacology for Anaesthesia and Intensive Care}, NMDA physiology.

\bibitem{cp-ideal}
\emph{Pharmacology for Anaesthesia and Intensive Care}, ideal IV agent.

\bibitem{mm-pressure}
\emph{Morgan \& Mikhail’s Clinical Anesthesiology}, pressure-reversal of anaesthesia.

\end{thebibliography}

\end{document}
