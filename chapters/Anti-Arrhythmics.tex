\documentclass[11pt,a4paper]{article}

% ---------- Core packages ----------
\usepackage[T1]{fontenc}
\usepackage[utf8]{inputenc} % pdfLaTeX
\usepackage{lmodern}
\usepackage{microtype}
\usepackage{amsmath,amssymb}

% ---------- Layout / lists ----------
\usepackage[a4paper,margin=2.2cm]{geometry}
\usepackage{enumitem}

% ---------- PDF / links ----------
\usepackage[hidelinks]{hyperref}

% Keep headings unnumbered (so titles like "1. ..." don't get double-numbered)
\setcounter{secnumdepth}{-1}

\title{Anti-arrhythmics}
\author{}
\date{}

\begin{document}
\maketitle

\noindent\emph{Peck \& Harris —} \textbf{Pharmacology for Anaesthesia and Intensive Care}, \textbf{Chapter 15}

\tableofcontents

\bigskip\noindent\rule{\linewidth}{0.4pt}\bigskip

\section{At a glance}
\begin{itemize}[leftmargin=*]
\item \textbf{Most SVTs} are treated by targeting the \textbf{AV node} (adenosine, verapamil, \(\beta\)-blockers, digoxin).
\item \textbf{Ventricular arrhythmias} often require membrane-active drugs (e.g.\ lidocaine, amiodarone) and careful attention to \textbf{pro-arrhythmia}.
\item The \textbf{Vaughan–Williams} system is useful for exams, but many agents have \textbf{mixed actions}.
\end{itemize}

\bigskip\noindent\rule{\linewidth}{0.4pt}\bigskip

\section{1. Electrophysiology needed for anti-arrhythmic pharmacology}

\subsection{1.1 Cardiac action potentials (high-yield)}

\textbf{Fast-response tissues} (atria, ventricles, His–Purkinje)
\begin{itemize}[leftmargin=*]
\item \textbf{Phase 0}: fast \textbf{\(\mathrm{Na^+}\)} influx (upstroke \(\rightarrow\) conduction)
\item \textbf{Phase 3}: \textbf{\(\mathrm{K^+}\)} efflux (repolarisation)
\item \textbf{Phase 4}: resting membrane potential
\end{itemize}

\textbf{Slow-response tissues} (SA/AV node)
\begin{itemize}[leftmargin=*]
\item \textbf{Phase 0}: mainly \textbf{\(\mathrm{Ca^{2+}}\)} entry (slower upstroke \(\rightarrow\) slower conduction)
\item \textbf{Phase 4}: spontaneous depolarisation (pacemaker activity)
\end{itemize}

\subsection{1.2 Mechanisms of tachyarrhythmias}
\begin{itemize}[leftmargin=*]
\item \textbf{Enhanced automaticity}
\item \textbf{Triggered activity} (after-depolarisations)
\item \textbf{Re-entry} (key exam framework)
  \begin{itemize}[leftmargin=*]
  \item Requires: a \textbf{circuit}, \textbf{unidirectional block}, and \textbf{sufficiently slow conduction} so refractory tissue can recover and be re-excited.
  \end{itemize}
\end{itemize}

\subsection{1.3 Two classic re-entry examples}
\begin{itemize}[leftmargin=*]
\item \textbf{AV nodal re-entrant tachycardia (AVNRT)}
  \begin{itemize}[leftmargin=*]
  \item Re-entry within/around the AV node involving \textbf{fast} and \textbf{slow} pathways.
  \end{itemize}
\item \textbf{Wolff–Parkinson–White (WPW)}
  \begin{itemize}[leftmargin=*]
  \item Re-entry involving an \textbf{accessory pathway} (bundle of Kent) in addition to the AV node.
  \end{itemize}
\end{itemize}

\subsection{1.4 Bradyarrhythmias (chapter frame)}
\begin{itemize}[leftmargin=*]
\item Typically due to \textbf{failure of impulse generation} (SA node) or \textbf{failure of conduction} (e.g.\ AV block).
\item Management may include \textbf{atropine}, \textbf{\(\beta\)-adrenergic stimulation}, and/or \textbf{pacing} (details beyond atropine/\(\beta\)-stimulation are not the focus here).
\end{itemize}

\bigskip\noindent\rule{\linewidth}{0.4pt}\bigskip

\section{2. Classification systems (and why they’re imperfect)}

\subsection{2.1 Vaughan–Williams classification (exam scaffold)}
\begin{itemize}[leftmargin=*]
\item \textbf{Class I: \(\mathrm{Na^+}\) channel blockers}
  \begin{itemize}[leftmargin=*]
  \item \textbf{Ia}: quinidine, procainamide, disopyramide
  \item \textbf{Ib}: lidocaine, mexiletine
  \item \textbf{Ic}: flecainide, propafenone
  \end{itemize}
\item \textbf{Class II: \(\beta\)-blockers}
\item \textbf{Class III: \(\mathrm{K^+}\) channel blockers / repolarisation prolongers}
  \begin{itemize}[leftmargin=*]
  \item e.g.\ sotalol; amiodarone is often grouped here but has broader activity
  \end{itemize}
\item \textbf{Class IV: \(\mathrm{Ca^{2+}}\) channel blockers}
  \begin{itemize}[leftmargin=*]
  \item verapamil (diltiazem listed but not expanded in this chapter)
  \end{itemize}
\item \textbf{Unclassified / other} (important in anaesthesia/ICU practice)
  \begin{itemize}[leftmargin=*]
  \item digoxin, adenosine, ivabradine
  \end{itemize}
\end{itemize}

\subsection{2.2 Limitations (common exam trap)}
\begin{itemize}[leftmargin=*]
\item Many agents have \textbf{mixed actions} (e.g.\ amiodarone; sotalol has \(\beta\)-blockade plus class I/III features).
\item Vaughan–Williams does \textbf{not} include all clinically relevant drugs (e.g.\ adenosine, digoxin, ivabradine).
\end{itemize}

\bigskip\noindent\rule{\linewidth}{0.4pt}\bigskip

\section{3. Clinical use: supraventricular tachyarrhythmias (SVT)}

\subsection{3.1 Digoxin}

\textbf{Presentation}
\begin{itemize}[leftmargin=*]
\item Tablets, elixir, and IV preparation.
\end{itemize}

\textbf{Uses (SVT focus)}
\begin{itemize}[leftmargin=*]
\item \textbf{Rate control} by slowing AV nodal conduction (classically in atrial fibrillation/flutter where AV nodal slowing is required).
\end{itemize}

\textbf{Mechanism of action (two components)}
\begin{itemize}[leftmargin=*]
\item \textbf{Direct (inotropy):} inhibits \textbf{\(\mathrm{Na^+}/\mathrm{K^+}\)-ATPase} \(\rightarrow\) \(\uparrow\) intracellular \(\mathrm{Na^+}\) \(\rightarrow\) \(\downarrow\) \(\mathrm{Na^+}/\mathrm{Ca^{2+}}\) exchange \(\rightarrow\) \(\uparrow\) intracellular \(\mathrm{Ca^{2+}}\).
\item \textbf{Indirect (anti-arrhythmic):} \(\uparrow\) \textbf{vagal tone} \(\rightarrow\) slows AV nodal conduction and increases AV nodal refractoriness.
\end{itemize}

\textbf{Adverse effects / toxicity (high-yield)}
\begin{itemize}[leftmargin=*]
\item Toxicity is more likely with \textbf{electrolyte disturbances} and renal impairment.
\item \textbf{ECG changes} (can occur without clinical toxicity): PR prolongation, ST-segment “scooping”, T-wave changes.
\item Toxicity may present with:
  \begin{itemize}[leftmargin=*]
  \item \textbf{GI} upset
  \item \textbf{Visual} symptoms
  \item \textbf{Arrhythmias} (including ventricular ectopy and AV block)
  \end{itemize}
\end{itemize}

\textbf{Drug interactions (effect on plasma levels)}
\begin{itemize}[leftmargin=*]
\item \textbf{Increase digoxin plasma levels:} amiodarone, captopril, erythromycin, carbenoxolone.
  \begin{itemize}[leftmargin=*]
  \item \textbf{\(\mathrm{Ca^{2+}}\)} channel antagonists: \textbf{verapamil increases} levels; \textbf{nifedipine} and \textbf{diltiazem} may have \textbf{no effect} or a \textbf{small rise}.
  \end{itemize}
\item \textbf{Reduce digoxin plasma levels:} antacids, cholestyramine, phenytoin, metoclopramide.
\end{itemize}

\textbf{Management principles for digoxin toxicity}
\begin{itemize}[leftmargin=*]
\item Stop digoxin; correct contributing factors (notably \textbf{\(\mathrm{K^+}\)} and \textbf{\(\mathrm{Mg^{2+}}\)} disturbances).
\item Activated charcoal may reduce absorption if appropriate.
\item \textbf{Digoxin-specific antibody fragments (Fab)} for severe toxicity.
\end{itemize}

\textbf{Kinetics}
\begin{itemize}[leftmargin=*]
\item Incomplete oral bioavailability.
\item Large volume of distribution (extensive tissue binding).
\item Elimination mainly \textbf{renal}; half-life prolonged in renal impairment.
\end{itemize}

\textbf{Common exam pitfalls}
\begin{itemize}[leftmargin=*]
\item ECG changes \textbf{do not} equal definite toxicity.
\item Toxicity risk increases with \textbf{hypokalaemia}.
\end{itemize}

\bigskip\noindent\rule{\linewidth}{0.4pt}\bigskip

\subsection{3.2 Adenosine}

\textbf{Presentation}
\begin{itemize}[leftmargin=*]
\item IV preparation for rapid bolus.
\end{itemize}

\textbf{Uses}
\begin{itemize}[leftmargin=*]
\item Acute termination/diagnosis of \textbf{re-entrant SVT} involving the AV node.
\end{itemize}

\textbf{Mechanism of action}
\begin{itemize}[leftmargin=*]
\item Activates \textbf{\(\mathrm{A_1}\)} receptors \(\rightarrow\) \(\uparrow\) \(\mathrm{K^+}\) efflux and \(\downarrow\) \(\mathrm{Ca^{2+}}\) entry in AV nodal tissue \(\rightarrow\) \textbf{hyperpolarisation} and \textbf{transient AV nodal block}.
\end{itemize}

\textbf{Adverse effects}
\begin{itemize}[leftmargin=*]
\item Very common but brief: flushing, chest tightness, dyspnoea, sense of impending doom.
\item May precipitate bronchospasm (caution in reactive airway disease).
\item Effects enhanced by \textbf{dipyridamole} and antagonised by \textbf{methylxanthines} (especially \textbf{aminophylline}).
\end{itemize}

\textbf{Kinetics (key point)}
\begin{itemize}[leftmargin=*]
\item Extremely short half-life (seconds) due to rapid uptake/metabolism.
\end{itemize}

\textbf{Exam pitfall}
\begin{itemize}[leftmargin=*]
\item Must be given as a \textbf{rapid bolus} with immediate flush.
\end{itemize}

\bigskip\noindent\rule{\linewidth}{0.4pt}\bigskip

\subsection{3.3 Verapamil (non-dihydropyridine \(\mathrm{Ca^{2+}}\) channel blocker)}

\textbf{Presentation}
\begin{itemize}[leftmargin=*]
\item Oral preparations (including modified-release) and IV formulation.
\end{itemize}

\textbf{Uses}
\begin{itemize}[leftmargin=*]
\item SVT involving AV nodal conduction; may be used for rate control in atrial fibrillation/flutter.
\end{itemize}

\textbf{Mechanism of action}
\begin{itemize}[leftmargin=*]
\item Blocks L-type \(\mathrm{Ca^{2+}}\) channels (particularly relevant in AV nodal tissue) \(\rightarrow\) slows AV conduction and increases AV nodal refractory period.
\item Also has negative inotropic and vasodilator effects.
\end{itemize}

\textbf{Adverse effects}
\begin{itemize}[leftmargin=*]
\item Hypotension, bradycardia, heart block.
\item Constipation is a notable non-cardiac adverse effect.
\end{itemize}

\textbf{Interactions / cautions (exam-relevant)}
\begin{itemize}[leftmargin=*]
\item Additive AV nodal depression with \textbf{\(\beta\)-blockers}.
\item Can increase digoxin levels and/or enhance digoxin effects.
\end{itemize}

\textbf{Kinetics}
\begin{itemize}[leftmargin=*]
\item Marked first-pass metabolism; hepatic clearance is important.
\end{itemize}

\bigskip\noindent\rule{\linewidth}{0.4pt}\bigskip

\subsection{3.4 \(\beta\)-blockers (SVT)}

\textbf{Core use}
\begin{itemize}[leftmargin=*]
\item Reduce sympathetic drive; useful in SVT prophylaxis and rate control.
\end{itemize}

\subsubsection{Esmolol (the “anaesthesia-friendly” \(\beta\)-blocker)}

\textbf{Presentation}
\begin{itemize}[leftmargin=*]
\item IV formulation.
\end{itemize}

\textbf{Uses}
\begin{itemize}[leftmargin=*]
\item Short-term control of tachycardia and SVT where rapid titration is required.
\end{itemize}

\textbf{Mechanism}
\begin{itemize}[leftmargin=*]
\item \(\beta_1\)-selective antagonist (dose-dependent selectivity).
\end{itemize}

\textbf{Adverse effects}
\begin{itemize}[leftmargin=*]
\item Bradycardia, hypotension, heart block; bronchospasm at higher doses/less selectivity.
\end{itemize}

\textbf{Kinetics (key point)}
\begin{itemize}[leftmargin=*]
\item Rapid metabolism by \textbf{red blood cell esterases} \(\rightarrow\) very short half-life.
\end{itemize}

\bigskip\noindent\rule{\linewidth}{0.4pt}\bigskip

\subsection{3.5 Quinidine (Class Ia)}

\textbf{Why it’s less used}
\begin{itemize}[leftmargin=*]
\item Use has declined due to adverse effects, but it remains examinable.
\end{itemize}

\textbf{Uses}
\begin{itemize}[leftmargin=*]
\item SVT (including atrial fibrillation/flutter) and ventricular ectopy.
\end{itemize}

\textbf{Mechanism of action}
\begin{itemize}[leftmargin=*]
\item Blocks fast \(\mathrm{Na^+}\) channels \(\rightarrow\) slows phase 0 upstroke.
\item Raises threshold potential; prolongs refractory period.
\item Has vagolytic effects (can increase AV conduction if not controlled).
\end{itemize}

\textbf{Adverse effects (high-yield)}
\begin{itemize}[leftmargin=*]
\item \textbf{Cardiac:} pro-arrhythmia; PR/QRS/QT prolongation; risk of ventricular arrhythmias. Hypotension via \(\alpha\)-blockade and myocardial depression.
\item \textbf{Non-cardiac:} cinchonism (tinnitus, blurred vision, impaired hearing, headache, confusion).
\end{itemize}

\textbf{Important clinical warning}
\begin{itemize}[leftmargin=*]
\item In atrial fibrillation/flutter, \textbf{pretreat} with an AV nodal slowing agent (\(\beta\)-blocker, \(\mathrm{Ca^{2+}}\) antagonist, or digoxin) to avoid dangerously rapid ventricular response.
\end{itemize}

\textbf{Kinetics}
\begin{itemize}[leftmargin=*]
\item Well absorbed orally; high protein binding; hepatic metabolism with urinary excretion of metabolites; half-life in hours.
\end{itemize}

\bigskip\noindent\rule{\linewidth}{0.4pt}\bigskip

\section{4. Clinical use: ventricular tachyarrhythmias (VT/VF context)}

\subsection{4.1 Lidocaine (Class Ib)}

\textbf{Presentation}
\begin{itemize}[leftmargin=*]
\item IV 1\% or 2\% solution.
\end{itemize}

\textbf{Uses}
\begin{itemize}[leftmargin=*]
\item Sustained ventricular tachyarrhythmias, particularly associated with \textbf{ischaemia}.
\end{itemize}

\textbf{Mechanism of action}
\begin{itemize}[leftmargin=*]
\item Blocks inactivated \(\mathrm{Na^+}\) channels \(\rightarrow\) reduces phase 0 upstroke in affected tissue.
\item Shortens phase 3 repolarisation \(\rightarrow\) decreases action potential duration and refractory period.
\end{itemize}

\textbf{Adverse effects}
\begin{itemize}[leftmargin=*]
\item \textbf{Cardiac:} AV block, hypotension (notably at higher plasma levels).
\item \textbf{CNS:} circumoral tingling, dizziness, paraesthesia \(\rightarrow\) confusion, coma, seizures with rising levels.
\end{itemize}

\textbf{Kinetics}
\begin{itemize}[leftmargin=*]
\item IV only for arrhythmias.
\item Hepatic metabolism; clearance reduced in \textbf{cardiac failure} (reduced hepatic blood flow) and liver disease.
\item Short elimination half-life (\(\approx\) 90 minutes).
\end{itemize}

\bigskip\noindent\rule{\linewidth}{0.4pt}\bigskip

\subsection{4.2 Mexiletine (Class Ib)}

\textbf{Concept}
\begin{itemize}[leftmargin=*]
\item Lidocaine analogue with similar ventricular effects.
\end{itemize}

\textbf{Presentation}
\begin{itemize}[leftmargin=*]
\item IV solution; oral modified-release also available.
\end{itemize}

\textbf{Uses}
\begin{itemize}[leftmargin=*]
\item Similar to lidocaine, including arrhythmias associated with ischaemia or digoxin.
\end{itemize}

\textbf{Mechanism}
\begin{itemize}[leftmargin=*]
\item \(\mathrm{Na^+}\) channel blockade; shortens action potential duration/refractory period.
\end{itemize}

\textbf{Adverse effects}
\begin{itemize}[leftmargin=*]
\item Low therapeutic ratio; adverse effects are common.
\item \textbf{Cardiac:} can precipitate bradycardia and tachyarrhythmias.
\item \textbf{Non-cardiac:} frequent GI intolerance; neurological symptoms (confusion, diplopia, seizures, tremor, ataxia); rash/jaundice reported.
\end{itemize}

\textbf{Kinetics}
\begin{itemize}[leftmargin=*]
\item High oral bioavailability (\(\sim\)90\%); minimal first-pass metabolism.
\item Hepatic metabolism; some renal excretion unchanged.
\end{itemize}

\bigskip\noindent\rule{\linewidth}{0.4pt}\bigskip

\subsection{4.3 Amiodarone (multi-class actions; often grouped as Class III)}

\textbf{Presentation}
\begin{itemize}[leftmargin=*]
\item Oral tablets; IV ampoules (dilute in 5\% dextrose before administration).
\end{itemize}

\textbf{Uses}
\begin{itemize}[leftmargin=*]
\item SVT, VT, and WPW.
\end{itemize}

\textbf{Mechanism of action}
\begin{itemize}[leftmargin=*]
\item Prolongs repolarisation via \(\mathrm{K^+}\) channel blockade (\(\uparrow\) action potential duration and refractory period).
\item Also exhibits additional class I/II/IV-type effects (mixed channel/receptor actions).
\end{itemize}

\textbf{Adverse effects (high-yield and very testable)}
\begin{itemize}[leftmargin=*]
\item \textbf{Pulmonary:} pneumonitis/fibrosis/pleuritis; important risk and can be fatal.
\item \textbf{Thyroid:} hyper- or hypothyroidism; inhibits peripheral conversion of \(\mathrm{T_4}\) \(\rightarrow\) \(\mathrm{T_3}\).
\item \textbf{Hepatic:} hepatitis/cirrhosis/jaundice (long-term monitoring relevant).
\item \textbf{Cardiac:} bradycardia/hypotension with rapid high-dose IV; QT prolongation but relatively low pro-arrhythmic potential.
\item \textbf{Ophthalmic:} corneal microdeposits (visual haloes/blur), usually reversible.
\item \textbf{Dermatological:} photosensitivity; slate-grey skin discolouration (face) with long-term use.
\item \textbf{Interactions:} increases effects/levels of other highly protein-bound drugs (notably warfarin, phenytoin; digoxin levels may rise). Additive AV nodal slowing with \(\beta\)-blockers/verapamil. Avoid with other QT-prolonging drugs.
\item \textbf{IV irritation:} central vein recommended.
\end{itemize}

\textbf{Kinetics (key point = very long half-life)}
\begin{itemize}[leftmargin=*]
\item Oral absorption variable; highly protein-bound; very large Vd with accumulation in muscle/fat.
\item Elimination half-life can be \textbf{weeks to months}.
\item Hepatic metabolism to an active metabolite; excretion via biliary tract/skin/lacrimal glands.
\end{itemize}

\bigskip\noindent\rule{\linewidth}{0.4pt}\bigskip

\subsection{4.4 Flecainide (Class Ic)}

\textbf{Presentation / dosing (as described in the chapter)}
\begin{itemize}[leftmargin=*]
\item Oral and IV formulations; oral dosing in divided doses; IV loading followed by infusion can be used.
\end{itemize}

\textbf{Uses}
\begin{itemize}[leftmargin=*]
\item Powerful activity against atrial and ventricular tachyarrhythmias, including WPW.
\end{itemize}

\textbf{Mechanism}
\begin{itemize}[leftmargin=*]
\item Blocks fast \(\mathrm{Na^+}\) flux and prolongs phase 0 (marked effects in conducting pathways).
\item No significant effect on action potential duration or refractory period.
\end{itemize}

\textbf{Adverse effects}
\begin{itemize}[leftmargin=*]
\item \textbf{Cardiac:} may worsen conduction disease; caution in SA/AV disease and bundle branch block. Can paradoxically increase ventricular rate in AF/flutter. Associated with increased mortality when used to suppress ventricular ectopy after MI. Negative inotropy; raises pacing threshold.
\item \textbf{Non-cardiac:} dizziness, paraesthesia, headache.
\end{itemize}

\textbf{Kinetics}
\begin{itemize}[leftmargin=*]
\item High oral bioavailability; moderate protein binding; large Vd.
\item Hepatic metabolism to active metabolites; renal excretion of metabolites and unchanged drug.
\end{itemize}

\bigskip\noindent\rule{\linewidth}{0.4pt}\bigskip

\subsection{4.5 Procainamide (Class Ia)}

\textbf{Uses}
\begin{itemize}[leftmargin=*]
\item SVT and ventricular tachyarrhythmias; can terminate VT.
\end{itemize}

\textbf{Mechanism}
\begin{itemize}[leftmargin=*]
\item Similar electrophysiological effects to quinidine (\(\mathrm{Na^+}\) channel blockade; \(\uparrow\) threshold; \(\uparrow\) refractory period) but less vagolytic.
\end{itemize}

\textbf{Adverse effects}
\begin{itemize}[leftmargin=*]
\item \textbf{Cardiac:} hypotension/vasodilatation and reduced cardiac output (especially IV); heart block; may increase ventricular response in SVT; QT prolongation and torsades de pointes.
\item \textbf{Non-cardiac:} drug-induced lupus (notably in slow acetylators); GI upset, fever, rash.
\end{itemize}

\textbf{Kinetics}
\begin{itemize}[leftmargin=*]
\item Well absorbed orally.
\item Short half-life \(\rightarrow\) frequent dosing or slow-release formulations.
\item Hepatic metabolism including acetylation to active metabolite (genetic polymorphism: slow vs fast acetylators).
\end{itemize}

\bigskip\noindent\rule{\linewidth}{0.4pt}\bigskip

\subsection{4.6 Disopyramide (Class Ia)}

\textbf{Presentation / dosing (as described in the chapter)}
\begin{itemize}[leftmargin=*]
\item Oral tablets (including SR) and IV solution.
\end{itemize}

\textbf{Uses}
\begin{itemize}[leftmargin=*]
\item Second-line for SVT and ventricular tachyarrhythmias.
\item In AF/flutter, control ventricular rate first (\(\beta\)-blocker or verapamil).
\end{itemize}

\textbf{Mechanism}
\begin{itemize}[leftmargin=*]
\item \(\mathrm{Na^+}\) channel blockade (phase 0 slowing) with prolongation of action potential and refractory period.
\item Also has \textbf{anticholinergic} effects.
\end{itemize}

\textbf{Adverse effects}
\begin{itemize}[leftmargin=*]
\item \textbf{Cardiac:} QT prolongation and torsades risk; negative inotropy may worsen heart failure.
\item \textbf{Non-cardiac:} anticholinergic effects (blurred vision, dry mouth, urinary retention).
\end{itemize}

\textbf{Kinetics}
\begin{itemize}[leftmargin=*]
\item Good oral absorption.
\item Predominantly \textbf{renal excretion unchanged}; half-life increases in renal or cardiac failure.
\end{itemize}

\bigskip\noindent\rule{\linewidth}{0.4pt}\bigskip

\subsection{4.7 Propafenone (Class Ic-like; similar to flecainide)}

\textbf{Presentation}
\begin{itemize}[leftmargin=*]
\item Oral tablets (IV use described in some settings).
\end{itemize}

\textbf{Uses}
\begin{itemize}[leftmargin=*]
\item Second-line for resistant SVT (including AF/flutter) and for ventricular tachyarrhythmias.
\end{itemize}

\textbf{Mechanism}
\begin{itemize}[leftmargin=*]
\item Blocks fast \(\mathrm{Na^+}\) influx \(\rightarrow\) prolongs phase 0.
\item Prolongs action potential and refractory period particularly in conducting tissue.
\item At higher doses: some \textbf{\(\beta\)-blocking} properties.
\end{itemize}

\textbf{Adverse effects}
\begin{itemize}[leftmargin=*]
\item Generally well tolerated.
\item \textbf{Cardiac:} caution in heart failure (\(\beta\)-blocking effect).
\item \textbf{Other:} neuro and GI effects at higher doses; may worsen myasthenia gravis; can precipitate asthma (\(\beta\)-blockade).
\item Increases plasma levels of digoxin and warfarin.
\end{itemize}

\textbf{Kinetics}
\begin{itemize}[leftmargin=*]
\item Near-complete absorption.
\item Oral bioavailability increases disproportionately as first-pass enzymes saturate.
\item Very high protein binding; extensive hepatic metabolism with genetic polymorphism.
\end{itemize}

\bigskip\noindent\rule{\linewidth}{0.4pt}\bigskip

\subsection{4.8 Sotalol (\(\beta\)-blocker with class I/III anti-arrhythmic activity)}

\textbf{Presentation}
\begin{itemize}[leftmargin=*]
\item Oral tablets; IV solution; racemate with differing isomer contributions.
\end{itemize}

\textbf{Uses}
\begin{itemize}[leftmargin=*]
\item Ventricular tachyarrhythmias.
\item Prophylaxis of paroxysmal SVT.
\end{itemize}

\textbf{Mechanism}
\begin{itemize}[leftmargin=*]
\item \(\beta\)-blockade plus repolarisation prolongation (class III) \(\rightarrow\) can prolong QT.
\end{itemize}

\textbf{Adverse effects (key exam focus)}
\begin{itemize}[leftmargin=*]
\item \(\beta\)-blocker adverse effects (bradycardia, hypotension, bronchospasm).
\item QT prolongation \(\rightarrow\) torsades de pointes risk.
\end{itemize}

\textbf{Kinetics}
\begin{itemize}[leftmargin=*]
\item Predominantly renal elimination (dose adjustment follows from this).
\end{itemize}

\bigskip\noindent\rule{\linewidth}{0.4pt}\bigskip

\section{5. Other / special-case anti-arrhythmics in this chapter}

\subsection{5.1 Phenytoin}
\begin{itemize}[leftmargin=*]
\item Mentioned as an anti-arrhythmic agent (Class Ib-type activity), particularly in the context of \textbf{digoxin-induced arrhythmias}.
\item Detailed dosing/kinetics for anti-arrhythmic use are not fully specified in this chapter.
\end{itemize}

\subsection{5.2 Ivabradine}

\textbf{Uses}
\begin{itemize}[leftmargin=*]
\item Slows heart rate by acting on the sinoatrial node.
\end{itemize}

\textbf{Mechanism}
\begin{itemize}[leftmargin=*]
\item Selective inhibition of the \textbf{\(I_f\)} (“funny”) current in the SA node \(\rightarrow\) reduces the slope of phase 4 depolarisation \(\rightarrow\) lowers heart rate.
\end{itemize}

\textbf{Adverse effects}
\begin{itemize}[leftmargin=*]
\item Bradycardia.
\item Visual phenomena (phosphenes).
\end{itemize}

\textbf{Kinetics}
\begin{itemize}[leftmargin=*]
\item Oral administration; hepatic metabolism (CYP3A4) is relevant; half-life in the order of hours.
\end{itemize}

\bigskip\noindent\rule{\linewidth}{0.4pt}\bigskip

\section{6. Pattern recognition and common exam traps (chapter-aligned)}
\begin{itemize}[leftmargin=*]
\item \textbf{SVT choice hinges on AV nodal involvement:} adenosine/verapamil/\(\beta\)-blockers/digoxin target AV nodal conduction.
\item \textbf{WPW:} think re-entry via an accessory pathway; avoid strategies that risk preferential conduction via the accessory pathway (management nuances beyond listed drug uses are not expanded here).
\item \textbf{Quinidine in AF/flutter:} slow AV nodal conduction first.
\item \textbf{Lidocaine:} hepatic blood flow matters; clearance falls in cardiac failure.
\item \textbf{Amiodarone:} multi-organ adverse effects + \textbf{very long half-life} + high interaction burden.
\item \textbf{Class Ia and class III agents:} QT prolongation \(\rightarrow\) torsades risk.
\end{itemize}

\end{document}
