\documentclass[11pt,a4paper]{article}

\usepackage[a4paper,margin=2.2cm]{geometry}
\usepackage[T1]{fontenc}
\usepackage[utf8]{inputenc} % pdfLaTeX-safe
\usepackage{lmodern}
\usepackage{microtype}
\usepackage{hyperref}
\usepackage{booktabs}
\usepackage{tabularx}
\usepackage{array}
\usepackage{enumitem}
\usepackage{amsmath}

\begin{document}


\section{Sympathomimetics (Peck \& Harris, Ch.\ 13) --- structured notes with cross-check additions}

\subsection{Sources used for this chapter}
\begin{itemize}
  \item \textit{Pharmacology for Anaesthesia and Intensive Care} (Peck \& Harris), Chapter 13 (primary structure/content).
  \item \textit{Clinical Anesthesiology} (Morgan \& Mikhail), adrenergic agonist tables/summary material (selectivity, organ effects, direct vs indirect principles, dosing anchors where stated).
  \item \textit{Fundamentals of Anaesthesia} (physiology linkage: catecholamine synthesis/PNMT context).
  \item \textit{1,000 Practice MTF} (exam-focused question bank with answer explanations; used only to add exam-oriented points, labelled as such and ideally cross-checked in core texts).
\end{itemize}

\subsection{1.\ Overview}
Sympathomimetics \textbf{mimic sympathetic nervous system activity}, largely by reproducing the effects of \textbf{noradrenaline (NA)} and \textbf{adrenaline} at \textbf{adrenoceptors}.

\paragraph{Endogenous autonomic transmitters (orientation)}
\begin{itemize}
  \item \textbf{Sympathetic:} noradrenaline (most postganglionic fibres), adrenaline (circulating hormone), dopamine (selected CNS/autonomic sites).
  \item \textbf{Parasympathetic:} acetylcholine.
  \item \textbf{Non-adrenergic/non-cholinergic:} purines (e.g.\ ATP).
\end{itemize}

\subsection{2.\ Classification (exam-friendly)}
\subsubsection*{A) By mode of action}
\begin{itemize}
  \item \textbf{Direct-acting:} bind and activate adrenoceptors directly.
  \item \textbf{Indirect-acting:} increase synaptic NA (e.g.\ by releasing NA from nerve terminals).
  \item \textbf{Mixed-acting:} both direct and indirect effects.
\end{itemize}

\subsubsection*{B) By predominant receptor profile / typical clinical role}
\begin{itemize}
  \item \textbf{Catecholamines (endogenous and synthetic):} adrenaline, noradrenaline, dopamine, isoprenaline, dobutamine, dopexamine.
  \item \textbf{Selective $\beta_2$ agonists (bronchodilation / uterine relaxation):} salbutamol, salmeterol, ritodrine, terbutaline.
  \item \textbf{Predominantly $\alpha_1$ agonists (raise SVR):} phenylephrine.
  \item \textbf{Mixed $\alpha/\beta$ agents (common perioperative vasopressors):} ephedrine, metaraminol.
  \item \textbf{Other inotropic agents (non-adrenoceptor primary mechanisms):} aminophylline; PDE III inhibitors (enoximone, milrinone); levosimendan.
\end{itemize}

\subsection{3.\ Chemical and kinetic principles (high yield)}
\subsubsection*{Catecholamines}
\begin{itemize}
  \item A \textbf{catechol} group = benzene ring with \textbf{two hydroxyl groups} (3 and 4 positions).
  \item Catecholamines are typically \textbf{not useful orally} (rapid metabolism) and \textbf{short acting} (rapid enzymatic degradation).
\end{itemize}

\subsubsection*{Termination of action: uptake and metabolism}
\begin{itemize}
  \item \textbf{Uptake 1:} active reuptake into nerve terminals (major mechanism for endogenous NA).
  \item \textbf{Uptake 2:} diffusion away from the synapse into tissues/circulation.
  \item Key enzymes: \textbf{MAO} (monoamine oxidase) and \textbf{COMT} (catechol-\textit{O}-methyl transferase).
\end{itemize}

\subsection{4.\ Receptor signalling (core mechanisms)}
\begin{itemize}
  \item \textbf{$\alpha_1$:} Gq $\rightarrow$ phospholipase C $\rightarrow$ PIP$_2$ $\rightarrow$ IP$_3$ $\rightarrow$ \(\uparrow\) intracellular Ca$^{2+}$.
  \item \textbf{$\alpha_2$:} Gi $\rightarrow$ inhibits adenylate cyclase $\rightarrow$ \(\downarrow\) cAMP.
  \item \textbf{$\beta$:} Gs $\rightarrow$ activates adenylate cyclase $\rightarrow$ \(\uparrow\) cAMP.
  \item \textbf{Dopamine receptors (as described):} dopamine acts at \textbf{D1 and D2} receptors via Gs and Gi (adenylate cyclase) \(\rightarrow\) \(\uparrow\) or \(\downarrow\) cAMP.
\end{itemize}

\subsection{5.\ Individual drugs}

\subsubsection{Adrenaline (epinephrine)}
\textbf{Type:} direct-acting sympathomimetic; endogenous catecholamine.
\paragraph{Presentation and uses} Used for \textbf{anaphylaxis}, \textbf{cardiac arrest}, \textbf{bronchospasm}, and \textbf{vasopressor/inotrope} infusions in selected settings (preparation details as per source).
\paragraph{Mechanism} Acts at \(\alpha_1, \alpha_2, \beta_1, \beta_2\). Net haemodynamic effects are \textbf{dose-dependent}.
\paragraph{Effects (organ-based)}
\begin{itemize}
  \item \textbf{Cardiovascular:} \(\beta_1\) \(\uparrow\) HR/contractility; \(\alpha_1\) vasoconstriction; BP response depends on dose and vascular tone.
  \item \textbf{Respiratory:} bronchodilation.
  \item \textbf{Metabolic:} hyperglycaemia; intracellular K$^+$ shift \(\rightarrow\) \textbf{hypokalaemia}.
  \item \textbf{CNS/skeletal muscle:} tremor, anxiety.
  \item \textbf{GU:} increased sphincter tone \(\rightarrow\) difficulty with micturition.
\end{itemize}
\paragraph{Kinetics} Not used orally; SC absorption slower than IM; tracheal absorption erratic. Metabolised by MAO and COMT to inactive metabolites (including VMA and metadrenaline conjugates). \textbf{Half-life \(\sim\)2 min.}

\subsubsection{Noradrenaline (norepinephrine)}
\textbf{Type:} direct-acting sympathomimetic; endogenous catecholamine.
\paragraph{Uses} IV infusion to increase \textbf{systemic vascular resistance}.
\paragraph{Mechanism} Predominantly \(\alpha_1\), with some \(\beta\) activity.
\paragraph{Effects}
\begin{itemize}
  \item Peripheral vasoconstriction \(\rightarrow\) \(\uparrow\) systolic/diastolic BP; \textbf{reflex bradycardia} common; CO may fall; \(\uparrow\) myocardial O$_2$ consumption.
  \item Excessive vasoconstriction \(\rightarrow\) ischaemia/gangrene (extreme cases).
  \item \textbf{Extravasation} risk \(\rightarrow\) tissue necrosis.
  \item \textbf{Splanchnic:} \(\downarrow\) renal/hepatic blood flow.
  \item \textbf{Uterus:} \(\downarrow\) uterine blood flow; potential fetal bradycardia/asphyxia.
\end{itemize}
\paragraph{Interactions} Effects may be exaggerated/prolonged with \textbf{MAOIs}.
\paragraph{Kinetics} Endogenous NA mainly terminated by \textbf{uptake 1} (then MAO metabolism or recycling). Circulating NA metabolised by COMT to VMA and normetadrenaline conjugates. \textbf{Half-life \(\sim\)2 min.} A proportion is taken up during pulmonary passage (as described).

\subsubsection{Dopamine}
\textbf{Type:} direct-acting catecholamine with dopaminergic and adrenoceptor activity.
\paragraph{Uses} IV infusion to improve haemodynamic parameters and urine output (as described).
\paragraph{Mechanism} Dose-dependent activity at D1/D2, \(\beta_1\), then \(\alpha\) with increasing infusion rate (inter-individual variability noted).
\paragraph{Effects} Rate-dependent; renal/splanchnic effects relate to dopaminergic receptor activation. Nausea/vomiting can occur (CTZ-related effect described).
\paragraph{Interactions/cautions} Ensure adequate preload; tachycardia may limit use. Caution with MAOIs.
\paragraph{Kinetics} Metabolised by MAO and COMT; \textbf{half-life \(\sim\)2 min}. A fraction is taken up into nerve terminals and converted to noradrenaline (as described).

\subsection{6.\ Synthetic catecholamines and receptor-selective agonists}

\subsubsection{Phenylephrine (\(\alpha_1\) agonist)}
\textbf{Type:} direct-acting synthetic \(\alpha_1\) agonist.
\paragraph{Uses} Low SVR hypotension (e.g.\ spinal anaesthesia, vasodilating drugs). Also nasal decongestant and mydriatic; limited use in SVT with hypotension.
\paragraph{Effects} \(\uparrow\) SVR and BP with \textbf{reflex bradycardia} \(\rightarrow\) often \(\downarrow\) CO; not arrhythmogenic. \(\downarrow\) renal blood flow. In obstetrics, more favourable cord gases than ephedrine when treating spinal hypotension (as described).
\paragraph{Kinetics} IV rapid onset and short duration; IM/SC slower onset and longer duration. Metabolised by MAO (details as per sources).

\subsubsection{Isoprenaline (isoproterenol)}
\textbf{Type:} potent synthetic catecholamine; \(\beta_1\) and \(\beta_2\) agonist; \textbf{no \(\alpha\) effects}.
\paragraph{Uses} IV for severe bradycardia associated with AV block or \(\beta\)-blocker toxicity (as described).
\paragraph{Effects}
\begin{itemize}
  \item \textbf{Cardiovascular:} \(\uparrow\) HR/contractility/automaticity/CO; \(\beta_2\) vasodilation may \(\downarrow\) SVR \(\rightarrow\) variable BP response.
  \item \textbf{Myocardial O$_2$:} tachycardia + reduced diastolic perfusion pressure may worsen supply--demand.
  \item \textbf{Respiratory:} bronchodilation; may worsen V/Q matching and increase dead space \(\rightarrow\) hypoxaemia.
  \item \textbf{CNS/metabolic/splanchnic:} stimulant effects; \(\uparrow\) glucose and free fatty acids; \(\uparrow\) mesenteric and renal blood flow.
\end{itemize}
\paragraph{Kinetics} Low oral bioavailability (first-pass). Rapid COMT metabolism; urinary excretion of metabolites/unchanged fraction.

\subsubsection{Dobutamine}
\textbf{Type:} synthetic catecholamine (derived from isoprenaline); \(\beta_1\)-predominant with some \(\beta_2\).
\paragraph{Uses} Low cardiac output states (MI, cardiac surgery, cardiogenic shock) and cardiac stress testing (as described).
\paragraph{Effects} \(\uparrow\) contractility and HR; \(\uparrow\) myocardial O$_2$ requirement. BP often rises despite limited \(\beta_2\)-mediated fall in SVR. Arrhythmias may occur; avoid in cardiac outflow obstruction (e.g.\ severe AS, tamponade).
\paragraph{Kinetics} IV only; rapid COMT metabolism to inactive metabolites; \textbf{half-life \(\sim\)2 min}.

\subsubsection{Dopexamine}
\textbf{Type:} synthetic analogue of dopamine.
\paragraph{Use/mechanism} Used to improve cardiac output and mesenteric perfusion (IV). Mechanistic emphasis on \(\beta_2\) and dopaminergic actions (as described).

\subsection{7.\ Selective \texorpdfstring{$\beta_2$}{beta2} agonists}

\subsubsection{Salbutamol}
\textbf{Type:} predominantly \(\beta_2\) agonist.
\paragraph{Uses} Reversible lower airway obstruction; occasionally premature labour (as described). Multiple preparations including inhaled/nebulised/oral and IV infusion (after dilution).
\paragraph{Effects}
\begin{itemize}
  \item \textbf{Respiratory:} bronchodilation; may reverse hypoxic pulmonary vasoconstriction \(\rightarrow\) \(\uparrow\) shunt and potential hypoxaemia (give oxygen with nebulised therapy when appropriate).
  \item \textbf{Cardiovascular:} high doses (especially IV) can cause tachycardia; \(\beta_2\) vasodilation may \(\downarrow\) BP; arrhythmias possible (especially with hypokalaemia).
  \item \textbf{Metabolic:} intracellular K$^+$ shift (Na$^+$/K$^+$-ATPase) \(\rightarrow\) \textbf{hypokalaemia}; \(\uparrow\) blood glucose (worse with steroids, especially in diabetes).
  \item \textbf{Uterus:} relaxes gravid uterus; small placental transfer.
  \item \textbf{Other:} tremor.
\end{itemize}
\paragraph{Kinetics} Incomplete GI absorption with significant first-pass metabolism. Rapid onset after inhaled/IV. \(\sim\)10\% protein bound; half-life 4--6 h. Hepatic metabolism to inactive sulfate; urinary excretion.

\subsubsection{Salmeterol}
\textbf{Type:} long-acting \(\beta_2\) agonist.
\paragraph{Uses} Nocturnal and exercise-induced asthma; not for acute attacks (slow onset).
\paragraph{Key property} Long non-polar side chain binds \(\beta_2\) receptor \(\rightarrow\) \(\sim\)12 h duration; more \(\beta_2\)-selective than \(\beta_1\) (as described).

\subsubsection{Ritodrine}
\begin{itemize}
  \item \(\beta_2\) agonist used for premature labour (as described).
  \item Tachycardia common; crosses placenta \(\rightarrow\) fetal tachycardia.
  \item Severe maternal adverse effects including pulmonary oedema have been described.
\end{itemize}

\subsubsection{Terbutaline}
\begin{itemize}
  \item \(\beta_2\) agonist with some \(\beta_1\) activity.
  \item Used for asthma and uncomplicated pre-term labour; similar class adverse-effect profile.
\end{itemize}

\subsection{8.\ Mixed \texorpdfstring{$\alpha/\beta$}{alpha/beta} agents used as vasopressors}

\subsubsection{Ephedrine}
\textbf{Type:} mixed direct and indirect sympathomimetic; also described as inhibiting MAO action on NA.
\paragraph{Uses} Hypotension with regional anaesthesia. In obstetrics: poorer cord gases vs purer \(\alpha\) agonists, but may be preferred if maternal bradycardia. Other uses include bronchospasm, nocturnal enuresis, narcolepsy.
\paragraph{Mechanism} Direct + indirect actions; prone to \textbf{tachyphylaxis} (NA store depletion).
\paragraph{Effects} \(\uparrow\) CO, HR, BP, coronary flow; \(\uparrow\) myocardial O$_2$ consumption; arrhythmias possible. Respiratory stimulant + bronchodilation. \(\downarrow\) renal blood flow and \(\downarrow\) GFR.
\paragraph{Interactions} Extreme caution with MAOIs.
\paragraph{Kinetics} Well absorbed orally/IM/SC. Not metabolised by MAO/COMT \(\rightarrow\) longer duration; elimination half-life \(\sim\)4 h. Some hepatic metabolism; large fraction excreted unchanged in urine.

\subsubsection{Metaraminol}
\textbf{Type:} synthetic mixed direct and indirect agent; mainly \(\alpha_1\) with some \(\beta\) activity.
\paragraph{Uses} Correct hypotension with spinal/epidural anaesthesia (bolus dosing as described).
\paragraph{Effects} \(\uparrow\) SVR \(\rightarrow\) \(\uparrow\) BP; CO may fall due to increased SVR. \(\uparrow\) coronary flow indirectly. \(\uparrow\) PVR \(\rightarrow\) \(\uparrow\) pulmonary artery pressure.

\subsection{9.\ Other inotropic agents (non-adrenoceptor primary mechanism)}

\subsubsection{Aminophylline (non-selective PDE inhibitor)}
\paragraph{What it is} Theophylline + ethylenediamine complex (improves solubility).
\paragraph{Uses} Asthma (oral or IV in severe attacks) and other uses as described.
\paragraph{Mechanism} Non-selective PDE inhibition \(\rightarrow\) \(\uparrow\) intracellular cAMP (and \(\pm\) cGMP). Additional described actions include NA release, mast cell effects via adenosine receptors, and effects on Ca$^{2+}$ translocation.
\paragraph{Effects}
\begin{itemize}
  \item \textbf{Respiratory:} bronchodilation; \(\uparrow\) diaphragm contractility; \(\uparrow\) CO$_2$ sensitivity of respiratory centre.
  \item \textbf{Cardiovascular:} mild positive inotropy/chronotropy; coronary and peripheral vasodilation; lowers arrhythmia threshold (especially with halothane).
  \item \textbf{CNS:} stimulant; lowers seizure threshold.
  \item \textbf{Renal:} weak diuretic; natriuresis; may precipitate hypokalaemia.
\end{itemize}
\paragraph{Interactions} CYP450 inhibitors may delay elimination (dose reduction may be required). CYP450 inducers may increase clearance (dose may need increasing).
\paragraph{Kinetics} High oral bioavailability; \(\sim\)50\% protein bound. Hepatic CYP450 metabolism; \(\sim\)10\% excreted unchanged in urine. Therapeutic plasma concentration is as stated in the source chapter.
\paragraph{Toxicity} At high concentrations, saturation of hepatic enzymes \(\rightarrow\) shift toward zero-order kinetics; toxicity includes tachyarrhythmias, seizures, GI symptoms, rhabdomyolysis (as described).

\subsubsection{Selective PDE III inhibitors}
\paragraph{Enoximone} ``Inodilator'': positive inotropy + vasodilation via \(\uparrow\) cAMP and Ca$^{2+}$ handling. Used for congestive HF/low CO (including post-cardiac surgery). Not useful orally (first-pass); active metabolite; reduce dose in renal failure (as described).
\paragraph{Milrinone} Similar effects; IV only for short-term cardiac failure management. Half-life 1--2.5 h; largely excreted unchanged in urine; reduce dose in renal failure.

\subsubsection{Levosimendan}
Calcium sensitiser of troponin C; also opens ATP-sensitive K$^+$ channels \(\rightarrow\) smooth muscle relaxation. Described as an ``inodilator''. Used as an IV infusion (may follow a loading dose), as described.

\subsection{10.\ EDAIC/FRCA-style pitfalls (chapter-consistent)}
\begin{itemize}
  \item \textbf{Reflex bradycardia with reduced CO} can occur with pure \(\alpha_1\) agonists (e.g.\ phenylephrine).
  \item \textbf{Extravasation injury} is
\end{itemize}

 \end{document}

