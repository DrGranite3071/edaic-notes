\documentclass[11pt,a4paper]{article}

\usepackage[a4paper,margin=2.2cm]{geometry}
\usepackage[T1]{fontenc}
\usepackage[utf8]{inputenc} % pdfLaTeX-safe
\usepackage{lmodern}
\usepackage{microtype}
\usepackage{hyperref}
\usepackage{booktabs}
\usepackage{tabularx}
\usepackage{array}
\usepackage{enumitem}
\usepackage{amsmath}

\begin{document}


\section{Sympathomimetics — Structured Chapter Notes (Peck \& Harris, Ch.\ 13)}

\subsection{Overview}
Sympathomimetics \textbf{mimic sympathetic nervous system activity}, largely by reproducing the effects of \textbf{noradrenaline (NA)} and \textbf{adrenaline} at \textbf{adrenoceptors}.

\subsubsection{Endogenous autonomic transmitters (orientation)}
\begin{itemize}
  \item \textbf{Sympathetic:} noradrenaline (most postganglionic fibres), adrenaline (circulating hormone), dopamine (selected CNS/autonomic sites)
  \item \textbf{Parasympathetic:} acetylcholine
  \item \textbf{Non-adrenergic / non-cholinergic:} purines (e.g.\ ATP)
\end{itemize}

\subsection{Classification (exam-friendly)}
\subsubsection{A) By mode of action}
\begin{itemize}
  \item \textbf{Direct-acting:} bind and activate adrenoceptors directly
  \item \textbf{Indirect-acting:} increase synaptic NA (e.g.\ by releasing NA from nerve terminals)
  \item \textbf{Mixed-acting:} both direct and indirect effects
\end{itemize}

\subsubsection{B) By predominant receptor profile / typical clinical role}
\begin{itemize}
  \item \textbf{Catecholamines (endogenous and synthetic):} adrenaline, noradrenaline, dopamine, isoprenaline, dobutamine, dopexamine
  \item \textbf{Selective $\beta_2$-agonists (bronchodilation / uterine relaxation):} salbutamol, salmeterol, ritodrine, terbutaline
  \item \textbf{Predominantly $\alpha_1$-agonists (raise SVR):} phenylephrine
  \item \textbf{Mixed $\alpha/\beta$ agents (common perioperative vasopressors):} ephedrine, metaraminol
  \item \textbf{Other inotropic agents (non-adrenoceptor primary mechanisms):} aminophylline; PDE III inhibitors (enoximone, milrinone); levosimendan
\end{itemize}

\subsection{Chemical and kinetic principles (high yield)}
\subsubsection{Catecholamines}
\begin{itemize}
  \item A \textbf{catechol} group = benzene ring with \textbf{two hydroxyl groups} (3 and 4 positions).
  \item Catecholamines are typically:
  \begin{itemize}
    \item \textbf{Not useful orally} (rapid metabolism)
    \item \textbf{Short acting} (rapid enzymatic degradation)
  \end{itemize}
\end{itemize}

\subsubsection{Termination of action: uptake and metabolism}
\begin{itemize}
  \item \textbf{Uptake 1:} active reuptake into nerve terminals (major mechanism for endogenous NA)
  \item \textbf{Uptake 2:} diffusion away from the synapse into tissues/circulation
  \item Key enzymes:
  \begin{itemize}
    \item \textbf{MAO} (monoamine oxidase)
    \item \textbf{COMT} (catechol-O-methyl transferase)
  \end{itemize}
\end{itemize}

\subsection{Receptor signalling (core mechanisms)}
\subsubsection{$\alpha_1$-adrenoceptor}
\begin{itemize}
  \item \textbf{Gq} $\rightarrow$ phospholipase C $\rightarrow$ PIP2 $\rightarrow$ IP3 $\rightarrow$ \textbf{$\uparrow$ intracellular Ca$^{2+}$}
\end{itemize}

\subsubsection{$\alpha_2$-adrenoceptor}
\begin{itemize}
  \item \textbf{Gi} $\rightarrow$ inhibits adenylate cyclase $\rightarrow$ \textbf{$\downarrow$ cAMP}
\end{itemize}

\subsubsection{$\beta$-adrenoceptors}
\begin{itemize}
  \item \textbf{Gs} $\rightarrow$ activates adenylate cyclase $\rightarrow$ \textbf{$\uparrow$ cAMP}
\end{itemize}

\subsubsection{Dopamine receptors (as described in this chapter)}
\begin{itemize}
  \item Dopamine acts at \textbf{D1 and D2} receptors via \textbf{Gs and Gi} (adenylate cyclase) $\rightarrow$ \textbf{$\uparrow$ or $\downarrow$ cAMP}.
\end{itemize}

\bigskip\hrule\bigskip

\subsection{Individual drugs}

\subsubsection{A) Adrenaline (epinephrine)}
\textbf{Type:} direct-acting sympathomimetic; endogenous catecholamine.

\paragraph{Presentation and uses}
\begin{itemize}
  \item Used for \textbf{anaphylaxis}, \textbf{cardiac arrest}, \textbf{bronchospasm}, and \textbf{vasopressor/inotrope} infusions in selected settings.
  \item Specific preparation details are as per the chapter.
\end{itemize}

\paragraph{Mechanism of action}
\begin{itemize}
  \item Acts at \textbf{$\alpha_1$, $\alpha_2$, $\beta_1$, $\beta_2$} receptors.
  \item Net haemodynamic effects are \textbf{dose-dependent}, reflecting the balance of $\alpha$ vs $\beta$ effects.
\end{itemize}

\paragraph{Effects (organ-based)}
\begin{itemize}
  \item \textbf{Cardiovascular:} $\beta_1$-mediated $\uparrow$ HR/contractility; $\alpha_1$-mediated vasoconstriction. Overall BP response depends on dose and vascular tone.
  \item \textbf{Respiratory:} bronchodilation.
  \item \textbf{Metabolic:} hyperglycaemia; intracellular K$^+$ shift can cause \textbf{hypokalaemia}.
  \item \textbf{CNS / skeletal muscle:} may cause tremor and anxiety.
  \item \textbf{GU:} increased sphincter tone may cause \textbf{difficulty with micturition}.
\end{itemize}

\paragraph{Interactions / cautions}
Use caution with drugs that affect catecholamine handling (e.g.\ MAO-related interactions as highlighted in the chapter).

\paragraph{Kinetics}
\begin{itemize}
  \item Not used orally (inactivated).
  \item SC absorption is slower than IM; tracheal absorption is erratic.
  \item Metabolised by \textbf{MAO and COMT} to inactive metabolites (including \textbf{VMA} and \textbf{metadrenaline conjugates}) excreted in urine.
  \item \textbf{Short half-life ($\sim$2 min).}
\end{itemize}

\bigskip\hrule\bigskip

\subsubsection{B) Noradrenaline (norepinephrine)}
\textbf{Type:} direct-acting sympathomimetic; endogenous catecholamine.

\paragraph{Presentation and uses}
Given as an \textbf{IV infusion} to increase \textbf{systemic vascular resistance}.

\paragraph{Mechanism of action}
Predominantly \textbf{$\alpha_1$}, with some \textbf{$\beta$} activity.

\paragraph{Effects}
\begin{itemize}
  \item \textbf{Cardiovascular:} peripheral vasoconstriction $\rightarrow$ $\uparrow$ systolic/diastolic BP; \textbf{reflex bradycardia} is common; CO may fall; $\uparrow$ myocardial O$_2$ consumption.
  \item Excessive peripheral vasoconstriction $\rightarrow$ ischaemia/gangrene in extreme cases.
  \item \textbf{Extravasation} can cause \textbf{tissue necrosis}.
  \item \textbf{Splanchnic:} $\downarrow$ renal and hepatic blood flow.
  \item \textbf{Uterus:} reduced uterine blood flow; potential fetal bradycardia/asphyxia.
\end{itemize}

\paragraph{Interactions / cautions}
Effects may be exaggerated or prolonged in patients taking \textbf{MAOIs}.

\paragraph{Kinetics}
\begin{itemize}
  \item Endogenous NA is inactivated mainly by \textbf{uptake 1} into nerve terminals (followed by MAO metabolism or recycling).
  \item Circulating NA is metabolised by \textbf{COMT} $\rightarrow$ \textbf{VMA} and \textbf{normetadrenaline conjugates} (urinary excretion).
  \item \textbf{Short half-life ($\sim$2 min).} A proportion is taken up during pulmonary passage (as per the chapter).
\end{itemize}

\bigskip\hrule\bigskip

\subsubsection{C) Dopamine}
\textbf{Type:} direct-acting catecholamine with dopaminergic and adrenoceptor activity.

\paragraph{Presentation and uses}
Used to improve \textbf{haemodynamic parameters and urine output} (administered IV).

\paragraph{Mechanism of action}
Dose-dependent activity at:
\begin{itemize}
  \item \textbf{D1/D2} (dopamine receptors)
  \item \textbf{$\beta_1$}
  \item \textbf{$\alpha$}
\end{itemize}

\paragraph{Effects}
\begin{itemize}
  \item \textbf{Cardiovascular:} infusion-rate dependent; varies between patients.
  \begin{itemize}
    \item Lower rates: dopaminergic effects predominate.
    \item Higher rates: increasing $\beta$ then $\alpha$ effects.
  \end{itemize}
  \item \textbf{Renal/splanchnic:} effects relate to dopaminergic receptor activation.
  \item \textbf{CNS/other:} nausea and vomiting can occur (CTZ-related effect is described).
\end{itemize}

\paragraph{Interactions / cautions}
Ensure adequate preload; tachycardia may limit utility. Use caution with \textbf{MAOIs}.

\paragraph{Kinetics}
Metabolised by \textbf{MAO and COMT}; \textbf{half-life $\sim$2 min}. A fraction is taken up into nerve terminals and converted to noradrenaline (as described).

\subsection{Synthetic catecholamines and receptor-selective agonists}

\subsubsection{A) Phenylephrine ($\alpha_1$ agonist)}
\textbf{Type:} direct-acting synthetic \textbf{$\alpha_1$} agonist.

\paragraph{Presentation and uses}
\begin{itemize}
  \item Used for \textbf{low SVR hypotension} (e.g.\ spinal anaesthesia, vasodilating drugs).
  \item Also used as a \textbf{nasal decongestant} and \textbf{mydriatic}; limited use in SVT with hypotension.
\end{itemize}

\paragraph{Mechanism of action}
Predominantly \textbf{$\alpha_1$}.

\paragraph{Effects}
\begin{itemize}
  \item \textbf{Cardiovascular:} $\uparrow$ SVR and BP with \textbf{reflex bradycardia}, often $\downarrow$ CO; not arrhythmogenic.
  \item \textbf{Renal:} reduced renal blood flow.
  \item \textbf{Obstetrics:} associated with more favourable cord gases than ephedrine when treating spinal hypotension.
\end{itemize}

\paragraph{Kinetics}
IV rapid onset with minutes duration; IM/SC slower onset with longer duration. Metabolised by \textbf{MAO} (metabolites/elimination details not identified in the chapter).

\bigskip\hrule\bigskip

\subsubsection{B) Isoprenaline (isoproterenol)}
\textbf{Type:} potent synthetic catecholamine; \textbf{$\beta_1$ and $\beta_2$ agonist} with \textbf{no $\alpha$ effects}.

\paragraph{Presentation and uses}
Used IV for \textbf{severe bradycardia} associated with AV block or $\beta$-blocker toxicity (as described).

\paragraph{Effects}
\begin{itemize}
  \item \textbf{Cardiovascular:} $\beta_1$ $\rightarrow$ $\uparrow$ HR/contractility/automaticity/CO; $\beta_2$ $\rightarrow$ $\downarrow$ SVR (variable BP response).
  \item \textbf{Myocardial oxygen balance:} may worsen when tachycardia plus reduced diastolic perfusion pressure reduce coronary supply.
  \item \textbf{Respiratory:} bronchodilation; may worsen V/Q matching and increase dead space $\rightarrow$ possible hypoxaemia.
  \item \textbf{CNS:} stimulant effects.
  \item \textbf{Splanchnic:} $\uparrow$ mesenteric and renal blood flow.
  \item \textbf{Metabolic:} $\uparrow$ blood glucose and free fatty acids.
\end{itemize}

\paragraph{Kinetics}
Oral: extensive first-pass metabolism $\rightarrow$ low bioavailability. Rapid metabolism by \textbf{COMT}; some excreted unchanged/conjugated in urine.

\bigskip\hrule\bigskip

\subsubsection{C) Dobutamine}
\textbf{Type:} direct-acting synthetic catecholamine (derived from isoprenaline); \textbf{$\beta_1$-predominant} with some $\beta_2$.

\paragraph{Presentation and uses}
Used to augment \textbf{low cardiac output} states (MI, cardiac surgery, cardiogenic shock); also used for \textbf{cardiac stress testing}.

\paragraph{Effects}
\begin{itemize}
  \item \textbf{Cardiovascular:} $\beta_1$ $\rightarrow$ $\uparrow$ contractility and HR; $\uparrow$ myocardial O$_2$ requirement.
  \item BP often rises despite a limited $\beta_2$-mediated fall in SVR.
  \item Can precipitate \textbf{arrhythmias}; avoid in \textbf{cardiac outflow obstruction} (e.g.\ severe aortic stenosis, tamponade).
\end{itemize}

\paragraph{Kinetics}
IV only; rapidly metabolised (COMT) to inactive metabolites $\rightarrow$ urinary excretion. \textbf{Half-life $\sim$2 min.}

\bigskip\hrule\bigskip

\subsubsection{D) Dopexamine}
\textbf{Type:} synthetic analogue of dopamine.

\paragraph{Presentation and uses}
Used to improve \textbf{cardiac output} and \textbf{mesenteric perfusion} (given IV).

\paragraph{Mechanism / effects / kinetics}
As described in the chapter (actions emphasised include $\beta_2$ and dopaminergic effects).

\subsection{Selective $\beta_2$ agonists}

\subsubsection{A) Salbutamol}
\textbf{Type:} synthetic sympathomimetic with mainly \textbf{$\beta_2$} activity.

\paragraph{Presentation and uses}
\begin{itemize}
  \item Multiple preparations (IV infusion after dilution; inhaled/nebulised/oral forms).
  \item Used for \textbf{reversible lower airway obstruction}; occasionally used for \textbf{premature labour}.
\end{itemize}

\paragraph{Effects}
\begin{itemize}
  \item \textbf{Respiratory:} bronchodilation. Can reverse hypoxic pulmonary vasoconstriction $\rightarrow$ $\uparrow$ shunt and potential hypoxaemia (give oxygen with nebulised therapy).
  \item \textbf{Cardiovascular:} high doses (especially IV) can cause $\beta_1$-mediated tachycardia; $\beta_2$ vasodilation may reduce BP; arrhythmias possible (especially with hypokalaemia).
  \item \textbf{Metabolic:} drives K$^+$ into cells (Na$^+$/K$^+$ ATPase) $\rightarrow$ \textbf{hypokalaemia}; $\uparrow$ blood glucose (worse with steroids, especially in diabetes).
  \item \textbf{Uterus:} relaxes the gravid uterus; small placental transfer.
  \item \textbf{Misc:} tremor.
\end{itemize}

\paragraph{Kinetics}
Incomplete GI absorption with significant first-pass metabolism; rapid onset after inhaled/IV; $\sim$10\% protein bound; half-life 4--6 h; hepatic metabolism to inactive sulfate; urinary excretion.

\bigskip\hrule\bigskip

\subsubsection{B) Salmeterol}
\textbf{Type:} long-acting \textbf{$\beta_2$} agonist.

\paragraph{Uses}
Nocturnal and exercise-induced asthma. Not for acute attacks (slow onset).

\paragraph{Key property}
Long non-polar side chain binds the $\beta_2$ receptor $\rightarrow$ $\sim$12 h duration; more $\beta_2$-selective than $\beta_1$ (as described).

\subsubsection{C) Ritodrine}
\begin{itemize}
  \item $\beta_2$ agonist used for \textbf{premature labour}.
  \item Tachycardia is common; crosses the placenta $\rightarrow$ fetal tachycardia.
  \item Associated with severe maternal adverse effects including pulmonary oedema (as described).
\end{itemize}

\subsubsection{D) Terbutaline}
\begin{itemize}
  \item $\beta_2$ agonist with some $\beta_1$ activity.
  \item Used for asthma and uncomplicated pre-term labour; similar class side-effect profile.
\end{itemize}

\subsection{Mixed ($\alpha$ and $\beta$) agents used as vasopressors}

\subsubsection{A) Ephedrine}
\textbf{Type:} mixed direct and indirect sympathomimetic; also inhibits MAO action on NA.

\paragraph{Presentation and uses}
\begin{itemize}
  \item Multiple formulations (including injection).
  \item Used IV for \textbf{hypotension associated with regional anaesthesia}.
  \item In obstetrics: poorer cord gases compared with purer $\alpha$-agonists, but may be preferred if maternal bradycardia.
  \item Also used for bronchospasm, nocturnal enuresis, and narcolepsy.
\end{itemize}

\paragraph{Mechanism of action}
Direct and indirect actions; prone to \textbf{tachyphylaxis} (depletion of NA stores).

\paragraph{Effects}
\begin{itemize}
  \item \textbf{Cardiovascular:} $\uparrow$ CO, HR, BP, coronary flow; $\uparrow$ myocardial O$_2$ consumption; may precipitate arrhythmias.
  \item \textbf{Respiratory:} stimulant effects and bronchodilation.
  \item \textbf{Renal:} $\downarrow$ renal blood flow and $\downarrow$ GFR.
\end{itemize}

\paragraph{Interactions}
Use with extreme caution in patients taking \textbf{MAOIs}.

\paragraph{Kinetics}
Well absorbed orally/IM/SC. Not metabolised by MAO or COMT $\rightarrow$ longer duration; elimination half-life $\sim$4 h. Some hepatic metabolism; a large fraction is excreted unchanged in urine.

\bigskip\hrule\bigskip

\subsubsection{B) Metaraminol}
\textbf{Type:} synthetic mixed direct and indirect agent; mainly \textbf{$\alpha_1$}, with some $\beta$ activity.

\paragraph{Presentation and uses}
Used to correct hypotension associated with spinal/epidural anaesthesia (bolus dosing as described in the chapter).

\paragraph{Effects}
\begin{itemize}
  \item \textbf{Cardiovascular:} $\uparrow$ SVR $\rightarrow$ $\uparrow$ BP; CO may fall due to increased SVR.
  \item $\uparrow$ coronary flow indirectly.
  \item $\uparrow$ pulmonary vascular resistance $\rightarrow$ $\uparrow$ pulmonary artery pressure.
\end{itemize}

\subsection{Other inotropic agents (non-adrenoceptor primary mechanism)}

\subsubsection{A) Aminophylline (non-selective PDE inhibitor)}
\paragraph{What it is}
Complex of theophylline + ethylenediamine (improves solubility).

\paragraph{Uses}
Asthma (oral or IV in severe attacks); other uses are described in the chapter.

\paragraph{Mechanism of action}
\begin{itemize}
  \item Non-selective inhibition of phosphodiesterase isoenzymes $\rightarrow$ \textbf{$\uparrow$ intracellular cAMP} ($\pm$ cGMP).
  \item Additional actions described include NA release, mast cell effects via adenosine receptors, and effects on Ca$^{2+}$ translocation.
\end{itemize}

\paragraph{Effects}
\begin{itemize}
  \item \textbf{Respiratory:} bronchodilation; $\uparrow$ diaphragm contractility; $\uparrow$ CO$_2$ sensitivity of the respiratory centre.
  \item \textbf{Cardiovascular:} mild +inotropy/chronotropy; coronary and peripheral vasodilation; lowers arrhythmia threshold (especially with halothane).
  \item \textbf{CNS:} stimulant; lowers seizure threshold.
  \item \textbf{Renal:} weak diuretic; natriuresis; may precipitate hypokalaemia.
\end{itemize}

\paragraph{Interactions}
CYP450 inhibitors can delay elimination (dose reduction may be required). CYP450 inducers can increase clearance (dose may need increasing).

\paragraph{Kinetics}
High oral bioavailability; $\sim$50\% protein bound. Hepatic CYP450 metabolism; $\sim$10\% excreted unchanged in urine. Therapeutic plasma concentration is stated in the chapter.

\paragraph{Toxicity}
At high concentrations, saturation of hepatic enzymes $\rightarrow$ shift to zero-order kinetics. Toxicity includes tachyarrhythmias, seizures, GI symptoms, and rhabdomyolysis (as described).

\subsubsection{B) Selective PDE III inhibitors}
\paragraph{Enoximone}
\begin{itemize}
  \item ``Inodilator'': positive inotropy + vasodilation via $\uparrow$ cAMP and Ca$^{2+}$ handling.
  \item Used for congestive HF / low CO states (including post-cardiac surgery).
  \item Not useful orally (first-pass metabolism); has an active metabolite.
  \item Dose reduction is required in renal failure (as described).
\end{itemize}

\paragraph{Milrinone}
\begin{itemize}
  \item Selective PDE III inhibitor with similar effects to enoximone.
  \item IV only for short-term management of cardiac failure.
  \item Half-life 1--2.5 h; largely excreted unchanged in urine; reduce dose in renal failure.
\end{itemize}

\subsubsection{C) Levosimendan}
\begin{itemize}
  \item Calcium sensitiser of troponin C; also opens ATP-sensitive K$^+$ channels $\rightarrow$ smooth muscle relaxation.
  \item Described as an ``inodilator''.
  \item Used as an IV infusion (may follow a loading dose).
\end{itemize}

\subsection{EDAIC/FRCA-style pitfalls (chapter-consistent)}
\begin{itemize}
  \item \textbf{Reflex bradycardia with reduced CO} can occur with pure $\alpha_1$ agonists (e.g.\ phenylephrine).
  \item \textbf{Extravasation injury} risk with potent vasoconstrictors (notably noradrenaline).
  \item \textbf{Tachyphylaxis} can occur with indirect/mixed agents (e.g.\ ephedrine).
  \item \textbf{V/Q mismatch and hypoxaemia risk} with $\beta$ agonists due to pulmonary vascular effects (give oxygen when appropriate).
  \item \textbf{Arrhythmia risk} increases with catecholamines and with interacting factors noted (e.g.\ halothane with aminophylline).
\end{itemize}

\bigskip\hrule\bigskip

\subsection{Additions from other core references (cross-checked)}
\textit{These points are added for completeness and exam linkage. Where a detail is source-specific, it is labelled explicitly.}

\subsubsection{Receptor selectivity and ``direction of travel'' (quick exam map)}
\paragraph{Source: Morgan \& Mikhail (Adrenergic agonists tables).}
\begin{itemize}
  \item \textbf{Phenylephrine:} $\alpha_1$ +++ (dominant), $\alpha_2$ +; $\beta_1/\beta_2$/DA = 0.
  \item \textbf{Adrenaline:} $\alpha_1/\alpha_2$ ++, $\beta_1$ +++, $\beta_2$ ++ ($\alpha$ effects become more prominent at high doses).
  \item \textbf{Noradrenaline:} $\alpha_1/\alpha_2$ ++, $\beta_1$ ++, $\beta_2$ 0 ($\alpha$ effects become more prominent at high doses).
  \item \textbf{Dopamine:} $\alpha_1/\alpha_2$ ++, $\beta_1$ ++, $\beta_2$ +; \textbf{DA1/DA2 +++} ($\alpha$ effects become more prominent at high doses).
  \item \textbf{Dopexamine:} $\beta_2$ +++, $\beta_1$ +, DA1 ++, DA2 +.
  \item \textbf{Dobutamine:} $\beta_1$ +++ (dominant), $\beta_2$ +; $\alpha$ minimal/variable.
  \item \textbf{Isoprenaline:} $\beta_1$ +++, $\beta_2$ +++ (no $\alpha$).
  \item \textbf{Terbutaline:} $\beta_2$ +++ (dominant), minor $\beta_1$.
  \item \textbf{Fenoldopam:} \textbf{DA1 +++} (no $\alpha/\beta$ activity).
\end{itemize}

\paragraph{Source: Morgan \& Mikhail (organ-system effects table).}
\begin{itemize}
  \item \textbf{Phenylephrine:} HR $\downarrow$, MAP $\uparrow\uparrow\uparrow$, CO $\downarrow$, PVR $\uparrow\uparrow\uparrow$, bronchodilation 0, RBF $\downarrow\downarrow\downarrow$.
  \item \textbf{Noradrenaline:} HR $\downarrow$, MAP $\uparrow\uparrow\uparrow$, CO $\downarrow/\uparrow$ (variable), PVR $\uparrow\uparrow\uparrow$, bronchodilation 0, RBF $\downarrow\downarrow\downarrow$.
  \item \textbf{Adrenaline:} HR $\uparrow\uparrow$, MAP $\uparrow$, CO $\uparrow\uparrow$, PVR $\uparrow/\downarrow$ (variable), bronchodilation $\uparrow\uparrow$, RBF $\downarrow\downarrow$.
  \item \textbf{Isoprenaline:} HR $\uparrow\uparrow\uparrow$, MAP $\downarrow$, CO $\uparrow\uparrow\uparrow$, PVR $\downarrow\downarrow$, bronchodilation $\uparrow\uparrow\uparrow$, RBF $\downarrow/\uparrow$ (variable).
  \item \textbf{Dobutamine:} HR $\uparrow$, MAP $\uparrow$, CO $\uparrow\uparrow\uparrow$, PVR $\downarrow$, bronchodilation 0, RBF $\uparrow$.
  \item \textbf{Ephedrine:} HR $\uparrow\uparrow$, MAP $\uparrow\uparrow$, CO $\uparrow\uparrow$, PVR $\uparrow$, bronchodilation $\uparrow\uparrow$, RBF $\downarrow\downarrow$.
  \item \textbf{Dopamine:} HR $\uparrow/\uparrow\uparrow$, MAP $\uparrow$, CO $\uparrow\uparrow\uparrow$, PVR $\uparrow$, bronchodilation 0, RBF $\uparrow\uparrow\uparrow$.
  \item \textbf{Dopexamine:} HR $\uparrow/\uparrow\uparrow$, MAP $\downarrow/\uparrow$ (variable), CO $\uparrow\uparrow$, PVR $\uparrow$, bronchodilation 0, RBF $\uparrow$.
  \item \textbf{Fenoldopam:} HR $\uparrow\uparrow$, MAP $\downarrow\downarrow\downarrow$, CO $\downarrow/\uparrow$ (variable), PVR $\downarrow\downarrow$, bronchodilation 0, RBF $\uparrow\uparrow\uparrow$.
\end{itemize}

\subsubsection{Catecholamine synthesis and termination (useful physiology link)}
\paragraph{Sources: Morgan \& Mikhail (adrenoceptor physiology) + Fundamentals of Anaesthesia (adrenal medulla physiology).}
\begin{itemize}
  \item \textbf{Catecholamine synthesis:} hydroxylation of \textbf{tyrosine $\rightarrow$ DOPA} is the \textbf{rate-limiting step}; dopamine is transported into vesicles and converted to noradrenaline; noradrenaline can be converted to adrenaline in the adrenal medulla.
  \item \textbf{PNMT dependence:} conversion of \textbf{noradrenaline $\rightarrow$ adrenaline} requires \textbf{phenylethanolamine-N-methyltransferase (PNMT)}, induced by \textbf{glucocorticoids}.
  \item \textbf{Termination of noradrenaline:} mainly \textbf{reuptake into the postganglionic nerve ending}; also diffusion and metabolism by \textbf{MAO} and \textbf{COMT}.
  \item \textbf{$\beta_3$ receptors (orientation):} described as present in \textbf{gallbladder and brown fat}, with a proposed role in \textbf{lipolysis/thermogenesis}.
\end{itemize}

\subsubsection{Direct vs indirect agonists (when it matters clinically)}
\paragraph{Source: Morgan \& Mikhail (direct vs indirect agonists).}
\begin{itemize}
  \item Direct agonists \textbf{bind the receptor}.
  \item Indirect agonists \textbf{increase endogenous neurotransmitter activity} (e.g.\ by increasing release or decreasing reuptake of NA).
  \item Clinical implication: when \textbf{endogenous NA stores are abnormal} (e.g.\ some antihypertensives, MAOIs), \textbf{treat intraoperative hypotension with direct agonists}, as responses to indirect agonists may be unreliable.
\end{itemize}

\subsubsection{Dosing ``anchors'' commonly cited in anaesthesia texts (not exhaustive)}
\paragraph{Source: Morgan \& Mikhail (phenylephrine; ephedrine).}
\begin{itemize}
  \item \textbf{Phenylephrine:} IV boluses \textbf{50--100 $\mu$g} ($\approx$0.5--1 $\mu$g/kg) for vasodilatory hypotension; infusion \textbf{0.25--1 $\mu$g/kg/min}.
  \item \textbf{Ephedrine:} adult IV bolus \textbf{2.5--10 mg}; paediatric bolus \textbf{0.1 mg/kg}; increasing subsequent doses reflects tachyphylaxis (depletion of NA stores).
\end{itemize}

\subsubsection{Additional adverse effects and exam points}
\textbf{Source notes (important):} Some of the points below come from a \textbf{question bank with an answer key/explanations} (1,000 Practice MTF). These are \textbf{exam-oriented summaries} and may be simplified; where possible they should be cross-checked against a core text (e.g.\ Peck; Morgan \& Mikhail).

\paragraph{Sources: Pharmacology for Anaesthesia \& Intensive Care (dopamine section) + 1,000 Practice MTF (answer explanations).}
\begin{itemize}
  \item \textbf{Dopamine:} can \textbf{attenuate the carotid body response to hypoxaemia} and \textbf{increase pulmonary vascular resistance}; extravasation can cause tissue necrosis; nausea/vomiting via CTZ stimulation.
  \item \textbf{Phenylephrine:} described as metabolised by \textbf{MAO but not COMT}; IV duration described as \textbf{$\sim$6--8 min} in one exam-focused source (consistent with short-lived bolus effect).
  \item \textbf{$\beta_2$ agonists (salbutamol):} may cause \textbf{lactic acidosis} (type B) attributed to \textbf{$\beta_2$-driven glycogenolysis and lipolysis}.
  \item \textbf{Levosimendan (if examined alongside inodilators):} calcium sensitiser (troponin C) + opens ATP-sensitive K$^+$ channels (vasodilation); one source lists contraindications including \textbf{moderate--severe renal impairment, severe hepatic impairment, severe ventricular filling/outflow obstruction, severe hypotension/tachycardia, and a history of torsades de pointes}.
\end{itemize}

\subsubsection{Terminology (common FRCA/EDAIC trap)}
\paragraph{Source: 1,000 Practice MTF (answers).}
\begin{itemize}
  \item \textbf{Inotrope} = any agent that affects myocardial contractility (\textbf{positive or negative}). In everyday speech ``inotrope'' often implies positive inotropy, but in exams the broader definition is safer.
\end{itemize}


 \end{document}

